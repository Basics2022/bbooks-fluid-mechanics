%% Generated by Sphinx.
\def\sphinxdocclass{jupyterBook}
\documentclass[letterpaper,10pt,italian]{jupyterBook}
\ifdefined\pdfpxdimen
   \let\sphinxpxdimen\pdfpxdimen\else\newdimen\sphinxpxdimen
\fi \sphinxpxdimen=.75bp\relax
\ifdefined\pdfimageresolution
    \pdfimageresolution= \numexpr \dimexpr1in\relax/\sphinxpxdimen\relax
\fi
%% let collapsible pdf bookmarks panel have high depth per default
\PassOptionsToPackage{bookmarksdepth=5}{hyperref}
%% turn off hyperref patch of \index as sphinx.xdy xindy module takes care of
%% suitable \hyperpage mark-up, working around hyperref-xindy incompatibility
\PassOptionsToPackage{hyperindex=false}{hyperref}
%% memoir class requires extra handling
\makeatletter\@ifclassloaded{memoir}
{\ifdefined\memhyperindexfalse\memhyperindexfalse\fi}{}\makeatother

\PassOptionsToPackage{warn}{textcomp}

\catcode`^^^^00a0\active\protected\def^^^^00a0{\leavevmode\nobreak\ }
\usepackage{cmap}
\usepackage{fontspec}
\defaultfontfeatures[\rmfamily,\sffamily,\ttfamily]{}
\usepackage{amsmath,amssymb,amstext}
\usepackage{polyglossia}
\setmainlanguage{italian}



\setmainfont{FreeSerif}[
  Extension      = .otf,
  UprightFont    = *,
  ItalicFont     = *Italic,
  BoldFont       = *Bold,
  BoldItalicFont = *BoldItalic
]
\setsansfont{FreeSans}[
  Extension      = .otf,
  UprightFont    = *,
  ItalicFont     = *Oblique,
  BoldFont       = *Bold,
  BoldItalicFont = *BoldOblique,
]
\setmonofont{FreeMono}[
  Extension      = .otf,
  UprightFont    = *,
  ItalicFont     = *Oblique,
  BoldFont       = *Bold,
  BoldItalicFont = *BoldOblique,
]



\usepackage[Sonny]{fncychap}
\ChNameVar{\Large\normalfont\sffamily}
\ChTitleVar{\Large\normalfont\sffamily}
\usepackage[,numfigreset=1,mathnumfig]{sphinx}

\fvset{fontsize=\small}
\usepackage{geometry}


% Include hyperref last.
\usepackage{hyperref}
% Fix anchor placement for figures with captions.
\usepackage{hypcap}% it must be loaded after hyperref.
% Set up styles of URL: it should be placed after hyperref.
\urlstyle{same}

\addto\captionsitalian{\renewcommand{\contentsname}{Incompressible Fluid Mechanics}}

\usepackage{sphinxmessages}



        % Start of preamble defined in sphinx-jupyterbook-latex %
         \usepackage[Latin,Greek]{ucharclasses}
        \usepackage{unicode-math}
        % fixing title of the toc
        \addto\captionsenglish{\renewcommand{\contentsname}{Contents}}
        \hypersetup{
            pdfencoding=auto,
            psdextra
        }
        % End of preamble defined in sphinx-jupyterbook-latex %
        

\title{Fluid Mechanics}
\date{14 apr 2025}
\release{}
\author{basics}
\newcommand{\sphinxlogo}{\vbox{}}
\renewcommand{\releasename}{}
\makeindex
\begin{document}

\pagestyle{empty}
\sphinxmaketitle
\pagestyle{plain}
\sphinxtableofcontents
\pagestyle{normal}
\phantomsection\label{\detokenize{intro::doc}}


\sphinxAtStartPar
If you want ot start a new basics\sphinxhyphen{}book, it could be a good idea to start from this template.

\sphinxAtStartPar
Please check out the Github repo of the project, \sphinxhref{https://github.com/Basics2022}{basics\sphinxhyphen{}book project}.
\begin{itemize}
\item {} 
\sphinxAtStartPar
Incompressible Fluid Mechanics

\begin{itemize}
\item {} 
\sphinxAtStartPar
{\hyperref[\detokenize{polimi/fluidmechanics-ita/template/capitoli/01_statica/02teoria::doc}]{\sphinxcrossref{Statics}}}

\item {} 
\sphinxAtStartPar
{\hyperref[\detokenize{polimi/fluidmechanics-ita/template/capitoli/02_tensSup/03teoria::doc}]{\sphinxcrossref{Surface tension}}}

\item {} 
\sphinxAtStartPar
{\hyperref[\detokenize{polimi/fluidmechanics-ita/template/capitoli/03_cinematica/12teoria::doc}]{\sphinxcrossref{Kinematics}}}

\item {} 
\sphinxAtStartPar
{\hyperref[\detokenize{polimi/fluidmechanics-ita/template/capitoli/04_bilanci/04teoria::doc}]{\sphinxcrossref{Balance equations}}}

\item {} 
\sphinxAtStartPar
{\hyperref[\detokenize{polimi/fluidmechanics-ita/template/capitoli/05_bernoulli/05teoria::doc}]{\sphinxcrossref{Bernoulli theorems and vorticity dynamics}}}

\item {} 
\sphinxAtStartPar
{\hyperref[\detokenize{polimi/fluidmechanics-ita/template/capitoli/06_slnEsatte/0600in::doc}]{\sphinxcrossref{Exact solutions of Navier\sphinxhyphen{}Stokes equations}}}

\item {} 
\sphinxAtStartPar
{\hyperref[\detokenize{polimi/fluidmechanics-ita/template/capitoli/07_similitudine/07teoria::doc}]{\sphinxcrossref{Similitude}}}

\item {} 
\sphinxAtStartPar
{\hyperref[\detokenize{polimi/fluidmechanics-ita/template/capitoli/08_aerodinamica/08teoria::doc}]{\sphinxcrossref{Aerodynamics}}}

\item {} 
\sphinxAtStartPar
{\hyperref[\detokenize{polimi/fluidmechanics-ita/template/capitoli/09_bl/09teoria::doc}]{\sphinxcrossref{Boundary layer}}}

\item {} 
\sphinxAtStartPar
{\hyperref[\detokenize{polimi/fluidmechanics-ita/template/capitoli/10_turbolenza/10teoria::doc}]{\sphinxcrossref{Instability and turbulence}}}

\end{itemize}
\end{itemize}

\sphinxstepscope


\chapter{Statics}
\label{\detokenize{polimi/fluidmechanics-ita/template/capitoli/01_statica/02teoria:statics}}\label{\detokenize{polimi/fluidmechanics-ita/template/capitoli/01_statica/02teoria:fluid-mechanics-statics}}\label{\detokenize{polimi/fluidmechanics-ita/template/capitoli/01_statica/02teoria::doc}}

\section{Definizione di fluido}
\label{\detokenize{polimi/fluidmechanics-ita/template/capitoli/01_statica/02teoria:definizione-di-fluido}}\label{\detokenize{polimi/fluidmechanics-ita/template/capitoli/01_statica/02teoria:fluid-mechanics-statics-def}}
\sphinxAtStartPar
Un fluido è un materiale che non è in grado di sopportare sforzi di
taglio, quando è in quiete o in moto con velocità uniforme in un sistema
di riferimento inerziale (invarianza galileiana). I fluidi «ordinari»
sono isotropi, cioè sono indipendenti dall’orientazione nello spazio. Un
fluido isotropo in quiete è quindi caratterizzato da uno stato di sforzo
idrostatico,
\begin{equation*}
\begin{split}\mathbb{T}^{(s)} = - p \mathbb{I} \ ,\end{split}
\end{equation*}
\sphinxAtStartPar
avendo indicato
con \(\mathbb{T}^{(s)}\) il tensore degli sforzi in quiete, \(p\) la
pressione all’interno del fluido e \(\mathbb{I}\) il tensore identità. Il
vettore sforzo \(\mathbf{t_n}\) \sphinxstyleemphasis{agente su} una superficie di fluido con
normale \(\mathbf{\hat{n}}\) si ottiene tramite il \sphinxstylestrong{teorema di Cauchy} per i
mezzi continui
\begin{equation*}
\begin{split}\mathbf{t_n} = \mathbf{\hat{n}} \cdot \mathbb{T} \ ,\end{split}
\end{equation*}
\sphinxAtStartPar
che lega
il vettore sforzo al tensore degli sforzi tramite il versore normale
alla superficie considerata, e che nel caso di fluido in quiete, diventa
\begin{equation*}
\begin{split}\mathbf{t_n}^{(s)} = \mathbf{\hat{n}} \cdot \mathbb{T}^{(s)}  = - p \mathbf{\hat{n}} \ .\end{split}
\end{equation*}
\sphinxAtStartPar
Per il principio di azione e reazione, lo sforzo agente su un materiale
a contatto con un fluido è di intensità uguale e direzione opposta. La
risultante \(\mathbf{R}\) delle forze agenti su un volume di fluido \(V\) è data
dalla somma dell’integrale su \(V\) delle forze di volume \(\mathbf{f}\) e
dell’integrale sulla superficie \(S\), contorno del volume \(V\), del
vettore sforzo \(\mathbf{t_n}\),
\begin{equation*}
\begin{split}\mathbf{R} = \int_V \mathbf{f} + \oint_S \mathbf{t_n} \ .\end{split}
\end{equation*}

\section{Equazione di equilibrio: forma integrale e differenziale}
\label{\detokenize{polimi/fluidmechanics-ita/template/capitoli/01_statica/02teoria:equazione-di-equilibrio-forma-integrale-e-differenziale}}\label{\detokenize{polimi/fluidmechanics-ita/template/capitoli/01_statica/02teoria:fluid-mechanics-statics-equil}}
\sphinxAtStartPar
Un sistema meccanico è in equilibrio quando la risultante delle forze
esterne e la risultante dei momenti esterni agenti sul fluido sono
nulle,
\begin{equation*}
\begin{split}\begin{cases}
 \mathbf{0} = \mathbf{R}^{ext} \\
 \mathbf{0} = \mathbf{M}^{ext}  \ .
\end{cases}\end{split}
\end{equation*}
\sphinxAtStartPar
Per un mezzo continuo non polare, è possibile dimostrare
che l’equilibrio ai momenti si riduce alla condizione di simmetria del
tensore degli sforzi. L’equilibrio delle forze agenti su un volume di
fluido \(V\) in quiete, delimitato dalla superficie \(\partial V = S\),
soggetto a forze per unità di volume \(\mathbf{f}\) in \(V\) e forze per unità
di superficie \(\mathbf{t_n}=-p \mathbf{\hat{n}}\) su \(S\) diventa
\begin{equation*}
\begin{split}\mathbf{0} = \mathbf{R}^{ext} = \int_V \mathbf{f} + \oint_S \mathbf{t_n} = \int_V \mathbf{f} - \oint_S p \mathbf{\hat{n}} \ .\end{split}
\end{equation*}
\sphinxAtStartPar
La condizione appena ottenuta è una \sphinxstylestrong{condizione di equilibrio
integrale}, per l’intero volume fluido \(V\). Se il campo di pressione
\(p\) è sufficientemente regolare, è possibile applicare il teorema del
gradiente (\DUrole{xref,myst}{{[}thm:grad{]}}\{reference\sphinxhyphen{}type=»ref»
reference=»thm:grad»\}) all’integrale di superficie e raccogliere i
termini a destra dell’uguale sotto un unico integrale di volume \(V\),
\begin{equation*}
\begin{split}\mathbf{0} = \int_V \left( \mathbf{f} - \mathbf{\nabla} p \right) \ .\end{split}
\end{equation*}
\sphinxAtStartPar
Poiché la
condizione di equilibrio deve essere valida indipendentemente dal volume
\(V\) considerato, imponendo che l’integranda sia identicamente nulla, si
ottiene l”\sphinxstylestrong{equazione di equilibrio in forma differenziale}
\begin{equation*}
\begin{split}\label{eqn:statica:diff}
 \mathbf{f}(\mathbf{r}) - \mathbf{\nabla} p (\mathbf{r}) = \mathbf{0} \ ,\end{split}
\end{equation*}
\sphinxAtStartPar
dove è stata
esplicitata la dipendenza dei campi \(\mathbf{f}\), \(p\) dall coordinata
spaziale \(\mathbf{r}\). Nel caso in cui sia noto il campo di forze di volume
\(\mathbf{f}\) all’interno del dominio considerato, l’equazione differenziale
alle derivate parziali
(\DUrole{xref,myst}{{[}eqn:statica:diff{]}}\{reference\sphinxhyphen{}type=»ref»
reference=»eqn:statica:diff»\}), con le opportune condizioni al contorno,
permette di calcolare il campo di pressione \(p(\mathbf{r})\).


\section{Legge di Stevino}
\label{\detokenize{polimi/fluidmechanics-ita/template/capitoli/01_statica/02teoria:legge-di-stevino}}\label{\detokenize{polimi/fluidmechanics-ita/template/capitoli/01_statica/02teoria:fluid-mechanics-statics-stevino}}
\sphinxAtStartPar
La legge di Stevino descrive il campo di pressione come funzione della
quota, nelle vicinanze della superficie terrestre. La legge di Stevino
viene ricavata dall’integrazione dell’equilibrio in forma differenziale
(\DUrole{xref,myst}{{[}eqn:statica:diff{]}}\{reference\sphinxhyphen{}type=»ref»
reference=»eqn:statica:diff»\}), nel caso in cui le forze di volume siano
dovute alla gravità \(\mathbf{f}(\mathbf{r}) = \rho(\mathbf{r}) \mathbf{g}(\mathbf{r})\),
avendo indicato con \(\rho(\mathbf{r})\) la densità del fluido e con \(\mathbf{g}\)
il campo di accelerazione gravitazionale,
\begin{equation*}
\begin{split}\label{eqn:statica:diff:g}
    - \mathbf{\nabla} p(\mathbf{r}) + \rho(\mathbf{r}) \mathbf{g}(\mathbf{r}) = \mathbf{0} \ .\end{split}
\end{equation*}
\sphinxAtStartPar
Nell’ipotesi di essere sufficientemente vicino alla terra da poter
considerare il campo vettoriale \(\mathbf{g}\) uniforme e diretto verso il
basso lungo la normale alla superficie terrestre, è possibile scrivere
l’equazione precedente in un sistema di coordinate cartesiane.
Orientando l’asse \(z\) verso l’alto lungo la normale alla superficie, le
tre componenti cartesiane dell’equazione vettoriale sono
\begin{equation*}
\begin{split}\begin{cases}
 \partial p(x,y,z) / \partial x = 0 \\
 \partial p(x,y,z) / \partial y = 0 \\ 
 \partial p(x,y,z) / \partial z = - \rho(x,y,z) g \ .
\end{cases}\end{split}
\end{equation*}
\sphinxAtStartPar
Dalle prime due equazioni si ricava che il campo di
pressione non può dipendere dalle coordinate \(x\), \(y\) ed è quindi solo
funzione di \(z\). Poiché il campo di pressione dipende solo da \(z\),
\(p = P(z)\), la terza equazione diventa un’equazione differenziale
ordinaria,
\begin{equation*}
\begin{split}\label{eqn:statica:Pz}
  \dfrac{d P}{d z} = -\rho(x,y,z) g \ ,\end{split}
\end{equation*}
\sphinxAtStartPar
alla quale deve essere
aggiunta una condizione al contorno del tipo \(P(z_0) = p_0\).%
\begin{footnote}[1]\sphinxAtStartFootnote
In generale, servono delle condizioni di compabibilità dei dati
affinché il problema sia risolvibile. Ad esempio, non dovrebbe
essere difficile convincersi che il campo di densità deve dipendere
solo dalla coordinata \(z\) nel caso considerato.
%
\end{footnote} Senza
ulteriori ipotesi, il problema composto dall’equazione
(\DUrole{xref,myst}{{[}eqn:statica:Pz{]}}\{reference\sphinxhyphen{}type=»ref»
reference=»eqn:statica:Pz»\}) e dalla condizione al contorno necessaria
ha come incognite il campo di pressione \(P\) e il campo di densità
\(\rho\). In generale, per risolvere il problema è necessario la legge di
stato del fluido che mette in relazione i due campi. Nell’ipotesi che la
densità \(\rho\) e la forza di gravità siano costanti, la soluzione del
problema (\DUrole{xref,myst}{{[}eqn:statica:Pz{]}}\{reference\sphinxhyphen{}type=»ref»
reference=»eqn:statica:Pz»\}) coincide con la \sphinxstyleemphasis{legge di Stevino},
\begin{equation*}
\begin{split}p(z) + \rho g z = p_0 = \text{cost} \ ,\end{split}
\end{equation*}
\sphinxAtStartPar
avendo orientato l’asse \(z\)
verso l’alto e imposto la condizione al contorno in \(z_0 = 0\).
\phantomsection \label{exercise:polimi/fluidmechanics-ita/template/capitoli/01_statica/02teoria-exercise-0}

\begin{sphinxadmonition}{note}{Exercise 1.1}



\sphinxAtStartPar
Utilizzando la legge di stato dei gas perfetti per l’aria,
\(P = \rho R T\), e l’approssimazione lineare dell’andamento della
temperatura con la quota \(z\), con gradiente termico
\(dT/dz=a=-6.5^\circ/km\), si ricavi l’andamento con la quota \(z\) delle
variabili termodinamiche \((P,\rho, T)\) per l’atrmosfera standard. Si
trascuri l’andamento di \(g\) con la quota. Trascurando la curvatura
terrestre, si utilizzi un sistema di coordinate cartesiane per scrivere
le componenti dell’equazione vettoriale
(\DUrole{xref,myst}{{[}eqn:statica:diff:g{]}}\{reference\sphinxhyphen{}type=»ref»
reference=»eqn:statica:diff:g»\}).
\end{sphinxadmonition}
\phantomsection \label{exercise:polimi/fluidmechanics-ita/template/capitoli/01_statica/02teoria-exercise-1}

\begin{sphinxadmonition}{note}{Exercise 1.2}



\sphinxAtStartPar
{[}{[}exe:stdatm:sphe{]}{]}\{\#exe:stdatm:sphe label=»exe:stdatm:sphe»\}
Utilizzando la legge di stato dei gas perfetti per l’aria,
\(P = \rho R T\), e l’approssimazione lineare dell’andamento della
temperatura con la quota \(r\), con gradiente termico
\(dT/dr=a=-6.5^\circ/km\), si ricavi l’andamento con la quota \(r\) delle
variabili termodinamiche \((P,\rho, T)\) per l’atrmosfera standard, senza
trascurare l’effetto della curvatura terrestre. Si utilizzi un sistema
di coordinate sferiche per scrivere le componenti dell’equazione
vettoriale
(\DUrole{xref,myst}{{[}eqn:statica:diff:g{]}}\{reference\sphinxhyphen{}type=»ref»
reference=»eqn:statica:diff:g»\}). Si valuti poi l’errore che si commette
nell’esercizio
\DUrole{xref,myst}{{[}exe:stdatm:cart{]}}\{reference\sphinxhyphen{}type=»ref»
reference=»exe:stdatm:cart»\} trascurando la curvatura terrestre sul
calcolo delle variabili termodinamiche a quota \(z = 10 \ km\).
\end{sphinxadmonition}


\section{Galleggiamento di un corpo immerso in un fluido}
\label{\detokenize{polimi/fluidmechanics-ita/template/capitoli/01_statica/02teoria:galleggiamento-di-un-corpo-immerso-in-un-fluido}}\label{\detokenize{polimi/fluidmechanics-ita/template/capitoli/01_statica/02teoria:fluid-mechanics-statics-buoyancy}}
\sphinxAtStartPar
Un corpo immerso in fluido riceve dal basso verso l’alto una spinta
uguale al peso della massa del fluido spostato. Se un corpo di volume
\(V_s\) immerso in un fluido \(\rho_f\) ne sposta un volume \(\tilde{V}_f\),
su di esso agisce una forza (di Archimede o di galleggiamento)
\begin{equation*}
\begin{split}\mathbf{F}_{Arch} = - \rho_f \tilde{V}_f \mathbf{g} = - \int_{\tilde{V}_f} \rho_f \mathbf{g} \ .\end{split}
\end{equation*}
\sphinxAtStartPar
La legge di Archimede vale per un sistema immerso nel campo di gravità
\(\mathbf{g}\), uniforme in spazio. Forze di galleggimento nascono su un corpo
immerso in un fluido in cui c’è un gradiente di pressione. La legge di
Archimede è solo un caso particolare di galleggiamento, forse il più
evidente, per il quale il campo di gravità è all’origine del gradiente
di pressione. In generale, la forza di galleggiamento su un corpo
immerso completamente in un fluido vale
\begin{equation*}
\begin{split}\mathbf{F}_{gall} = -\int_{S_s} p \mathbf{\hat{n}} = - \int_{V_s} \mathbf{\nabla} p \ .\end{split}
\end{equation*}
\sphinxAtStartPar
Un esempio di galleggiamento di interesse aeronautico si incontra quando
si svolge un esperimento in galleria del vento, se nella camera di prova
è presente un gradiente di pressione diretto nella direzione
\(\mathbf{\hat{x}}\) della corrente. Se in prima approssimazione si considera
un gradiente di pressione \(\mathbf{\nabla}p = - G_P \mathbf{\hat{x}}\), \(G_P>0\) e
costante, si può stimare la forza di galleggiamento
\(\mathbf{F}_{gall} = V_s G_P \mathbf{\hat{x}}\) dovuta al gradiente di pressione
in galleria del vento, assente in condizioni di aria libera. Questa
azione contribuisce al valore misurato della resistenza del modello. La
valutazione di questa azione «spuria» sul corpo e la correzione delle
misure effettuate rientrano nell’ambito delle \sphinxstyleemphasis{correzioni di galleria}.

\sphinxAtStartPar
Si ritorna ora sulla legge di Archimede che descrive le forza di
galleggiamento che un fluido esercita su un corpo immerso. Nel problema
di un corpo immerso in un fluido, la risultante delle forze di
galleggiamento entra nell’equazione di equilibrio del corpo in direzione
verticale (direzione della gravità, \sphinxstylestrong{g}). Il punto di applicazione
della risultante delle forze di galleggiamento e la sua posizione
relativa rispetto al baricentro del corpo influenzano la stabilità delle
condizioni di equilibrio.


\section{Risultante delle forze: legge di Archimede}
\label{\detokenize{polimi/fluidmechanics-ita/template/capitoli/01_statica/02teoria:risultante-delle-forze-legge-di-archimede}}


\sphinxAtStartPar
Si considera il problema di un corpo immerso in un fluido di densità
uniforme \(\rho\) molto maggiore della densità dell’aria: la pressione
agente sulla superficie del corpo esposta all’aria si può considerare
costante, uguale a \(p_a\). La legge di Stevino descrive la distribuzione
di pressione all’interno del fluido. Si sceglie l’asse \(z\) in direzione
verticale, così che il campo di gravità è \(\mathbf{g} = -g \mathbf{\hat{z}}\).
Scegliendo l’origine dell’asse \(z\) in corrispondenza del pelo libero, la
pressione all’interno del fluido vale \(p(z) = p_a - \rho g z\), per
\(z < 0\). Facendo riferimento alla figura
\DUrole{xref,myst}{{[}fig:archimede\_01{]}}\{reference\sphinxhyphen{}type=»ref»
reference=»fig:archimede\_01»\}, si può calcolare la risultante delle
forze \(\mathbf{R}\) come
\begin{equation*}
\begin{split}\begin{aligned}
    \mathbf{R} & = - \oint_{S} p \mathbf{\hat{n}} = 
               - \int_{S_a} p \mathbf{\hat{n}} - \int_{S_f} p \mathbf{\hat{n}} = \\
        & =    - \int_{S_a} p_a \mathbf{\hat{n}} - \int_{S_f} p_a \mathbf{\hat{n}} + 
                 \int_{S_f} \rho g z \mathbf{\hat{n}} = \\
        & =    - \underbrace{\int_{S} p_a \mathbf{\hat{n}}}_{=0} + 
                 \int_{S_f} \rho g z \mathbf{\hat{n}} = \\
        & =      \int_{S_f} \rho g z \mathbf{\hat{n}} + 
                 \underbrace{\int_{S_0} \rho g z \mathbf{\hat{n}}}_{=0, \ z|_{S_0} = 0} =
                 \oint_{\tilde{S}_f} \rho g z \mathbf{\hat{n}} = 
                 \int_{\tilde{V}_f} \rho g \mathbf{\hat{z} } = 
                 \rho \tilde{V}_f g \mathbf{\hat{z}} = 
                 M_{\tilde{V}_f} g \mathbf{\hat{z}} \ ,
\end{aligned}\end{split}
\end{equation*}
\sphinxAtStartPar
avendo sommato l’integrale nullo
\(\int_{S_0} \rho g z \mathbf{\hat{n}}\), per poter ottenere l’integrale di
\(\rho g z\) sulla superficie \(\tilde{S}_f = S_f \cup S_0\) e applicare il
teorema del gradiente (\DUrole{xref,myst}{{[}thm:grad{]}}\{reference\sphinxhyphen{}type=»ref»
reference=»thm:grad»\}). Come stabilito dal principio di Archimede, la
risultante delle forze di galleggiamento \(\mathbf{R}\) agenti sul corpo
agisce dal basso verso l’alto con un’intensità pari al peso del volume
di fluido spostato, \(M_{\tilde{V}_f} g\).


\section{Punto di applicazione}
\label{\detokenize{polimi/fluidmechanics-ita/template/capitoli/01_statica/02teoria:punto-di-applicazione}}
\sphinxAtStartPar
Il punto di applicazione della forza di galleggiamento è il punto dove
bisogna applicare la risultante delle forze per ottenere un sistema di
azioni equivalente al sistema di azioni continuo generato dalla
pressione. Dall’equivalenza ai momenti dei due sistemi di azioni, si
ottiene
\begin{equation*}
\begin{split}\begin{aligned}
\mathbf{r}_O \times \mathbf{R} & = - \oint_{S} p \mathbf{r} \times \mathbf{\hat{n}} = 
    - \int_{S_a} p \mathbf{r} \times \mathbf{\hat{n}}
    - \int_{S_f} p \mathbf{r} \times \mathbf{\hat{n}} = \\
& = - \int_{S_a} p_a \mathbf{r} \times \mathbf{\hat{n}}
    - \int_{S_f} p_a \mathbf{r} \times \mathbf{\hat{n}} 
    + \int_{S_f} \rho g z \mathbf{r} \times \mathbf{\hat{n}} = \\
& = - \underbrace{\int_{S} p_a \mathbf{r} \times \mathbf{\hat{n}}}_{=0} + 
      \int_{S_f} \rho g z \mathbf{r} \times \mathbf{\hat{n}} = \\
& =   \int_{S_f} \rho g z \mathbf{r} \times \mathbf{\hat{n}} + 
      \underbrace{\int_{S_0} \rho g z \mathbf{r} \times \mathbf{\hat{n}}}_{=0, \ z|_{S_0} = 0} =\\
& =  \oint_{\tilde{S}_f} \rho g z \mathbf{r} \times \mathbf{\hat{n}} = 
     \oint_{\tilde{S}_f} \rho g \delta_{\ell z} r_{\ell} \epsilon_{ijk} r_j n_k = 
     \rho g \int_{\tilde{V}_f} \dfrac{\partial}{\partial r_k}( \delta_{\ell z} r_{\ell} \epsilon_{ijk} r_j ) = \\
    & = \rho g \int_{\tilde{V}_f} \delta_{\ell z} \epsilon_{ijk}\left(  \dfrac{\partial r_{\ell}}{\partial r_k} r_j + r_{\ell} \dfrac{\partial r_j}{\partial r_k} \right) = \\ 
    & = \rho g \int_{\tilde{V}_f} \delta_{\ell z} \epsilon_{ijk}\left( \delta_{\ell k}r_j + r_{\ell} \delta_{jk} \right) = \\
    & = \rho g \int_{\tilde{V}_f} \epsilon_{ijz} r_j + \delta_{\ell z} \underbrace{\epsilon_{ijj}}_{ = 0} r_{\ell} = \\
    & = \rho g \int_{\tilde{V}_f}  \mathbf{r} \times \mathbf{\hat{z}} \ .
\end{aligned}\end{split}
\end{equation*}
\sphinxAtStartPar
Usando un sistema di assi catesiani e ricordando che
\(\mathbf{R} = R \mathbf{\hat{z}}\), si può scomprre l’equazione nelle componenti
non nulle, \(x\) e \(y\),
\begin{equation*}
\begin{split}\begin{cases}
   x_0 R = \rho g \displaystyle\int_{\tilde{V}_f} x \\ \\
   y_0 R = \rho g \displaystyle\int_{\tilde{V}_f} y 
\end{cases} \qquad \rightarrow \qquad
\begin{cases}
    x_0 = \dfrac{\rho g}{R}  \displaystyle\int_{\tilde{V}_f} x  
        = \dfrac{1}{\tilde{V}_f}  \displaystyle\int_{\tilde{V}_f} x 
    \\ \\
    y_0 = \dfrac{\rho g}{R}  \displaystyle\int_{\tilde{V}_f} y 
        = \dfrac{1}{\tilde{V}_f}  \displaystyle\int_{\tilde{V}_f} y \ .
\end{cases}\end{split}
\end{equation*}

\section{Stabilità statica dell’equilibrio}
\label{\detokenize{polimi/fluidmechanics-ita/template/capitoli/01_statica/02teoria:stabilita-statica-dell-equilibrio}}

\bigskip\hrule\bigskip


\sphinxstepscope


\section{Exercises}
\label{\detokenize{polimi/fluidmechanics-ita/template/capitoli/01_statica/exercises:exercises}}\label{\detokenize{polimi/fluidmechanics-ita/template/capitoli/01_statica/exercises:fluid-mechanics-statics-exercises}}\label{\detokenize{polimi/fluidmechanics-ita/template/capitoli/01_statica/exercises::doc}}
\begin{sphinxuseclass}{sd-container-fluid}
\begin{sphinxuseclass}{sd-sphinx-override}
\begin{sphinxuseclass}{sd-mb-4}
\begin{sphinxuseclass}{sd-row}
\begin{sphinxuseclass}{sd-g-2}
\begin{sphinxuseclass}{sd-g-xs-2}
\begin{sphinxuseclass}{sd-g-sm-2}
\begin{sphinxuseclass}{sd-g-md-2}
\begin{sphinxuseclass}{sd-g-lg-2}
\begin{sphinxuseclass}{sd-col}
\begin{sphinxuseclass}{sd-d-flex-row}
\begin{sphinxuseclass}{sd-col-8}
\begin{sphinxuseclass}{sd-col-xs-8}
\begin{sphinxuseclass}{sd-col-sm-8}
\begin{sphinxuseclass}{sd-col-md-8}
\begin{sphinxuseclass}{sd-col-lg-8}
\begin{sphinxuseclass}{sd-card}
\begin{sphinxuseclass}{sd-sphinx-override}
\begin{sphinxuseclass}{sd-w-100}
\begin{sphinxuseclass}{sd-shadow-sm}
\begin{sphinxuseclass}{sd-card-body}
\begin{sphinxuseclass}{sd-card-title}
\begin{sphinxuseclass}{sd-font-weight-bold}Exercise 1.1
\end{sphinxuseclass}
\end{sphinxuseclass}
\sphinxAtStartPar
Si consideri, sulla superficie terrestre, un recipiente di diametro \(D=2 \ m\) e profondità \(H=3\  m\) contenente acqua di densità \(\rho = 998\ kg / m^3\). Al suo interno è inserita una sfera di raggio \(a=0.2\, m\) e densità pari a \(\rho_s=842.06\ kg / m^3\).
Determinare in modo univoco la posizione assunta dalla sfera nel liquido. Tale posizione varia se invece che sulla terra ci si trova sulla Luna?

\sphinxAtStartPar
(\(h=0.3\  m\), non varia sulla Luna.)

\end{sphinxuseclass}
\end{sphinxuseclass}
\end{sphinxuseclass}
\end{sphinxuseclass}
\end{sphinxuseclass}
\end{sphinxuseclass}
\end{sphinxuseclass}
\end{sphinxuseclass}
\end{sphinxuseclass}
\end{sphinxuseclass}
\end{sphinxuseclass}
\end{sphinxuseclass}
\begin{sphinxuseclass}{sd-col}
\begin{sphinxuseclass}{sd-d-flex-row}
\begin{sphinxuseclass}{sd-col-4}
\begin{sphinxuseclass}{sd-col-xs-4}
\begin{sphinxuseclass}{sd-col-sm-4}
\begin{sphinxuseclass}{sd-col-md-4}
\begin{sphinxuseclass}{sd-col-lg-4}
\begin{sphinxuseclass}{sd-card}
\begin{sphinxuseclass}{sd-sphinx-override}
\begin{sphinxuseclass}{sd-w-100}
\begin{sphinxuseclass}{sd-shadow-sm}
\begin{sphinxuseclass}{sd-card-body}
\sphinxAtStartPar
\sphinxincludegraphics{{recipientesfera}.png}

\end{sphinxuseclass}
\end{sphinxuseclass}
\end{sphinxuseclass}
\end{sphinxuseclass}
\end{sphinxuseclass}
\end{sphinxuseclass}
\end{sphinxuseclass}
\end{sphinxuseclass}
\end{sphinxuseclass}
\end{sphinxuseclass}
\end{sphinxuseclass}
\end{sphinxuseclass}
\end{sphinxuseclass}
\end{sphinxuseclass}
\end{sphinxuseclass}
\end{sphinxuseclass}
\end{sphinxuseclass}
\end{sphinxuseclass}
\end{sphinxuseclass}
\end{sphinxuseclass}
\end{sphinxuseclass}


\begin{sphinxuseclass}{sd-container-fluid}
\begin{sphinxuseclass}{sd-sphinx-override}
\begin{sphinxuseclass}{sd-mb-4}
\begin{sphinxuseclass}{sd-row}
\begin{sphinxuseclass}{sd-g-2}
\begin{sphinxuseclass}{sd-g-xs-2}
\begin{sphinxuseclass}{sd-g-sm-2}
\begin{sphinxuseclass}{sd-g-md-2}
\begin{sphinxuseclass}{sd-g-lg-2}
\begin{sphinxuseclass}{sd-col}
\begin{sphinxuseclass}{sd-d-flex-row}
\begin{sphinxuseclass}{sd-col-8}
\begin{sphinxuseclass}{sd-col-xs-8}
\begin{sphinxuseclass}{sd-col-sm-8}
\begin{sphinxuseclass}{sd-col-md-8}
\begin{sphinxuseclass}{sd-col-lg-8}
\begin{sphinxuseclass}{sd-card}
\begin{sphinxuseclass}{sd-sphinx-override}
\begin{sphinxuseclass}{sd-w-100}
\begin{sphinxuseclass}{sd-shadow-sm}
\begin{sphinxuseclass}{sd-card-body}
\begin{sphinxuseclass}{sd-card-title}
\begin{sphinxuseclass}{sd-font-weight-bold}Exercise 1.2
\end{sphinxuseclass}
\end{sphinxuseclass}
\sphinxAtStartPar
Si consideri il sistema rappresentato in figura in cui un recipiente
aperto all’atmosfera, contenente olio con densità \(\rho= 800\ kg/m^3\), è
collegato tramite una tubazione a un secondo recipiente, contenente a
sua volta olio e aria non miscelati. Date le due altezze \(h_1=1.5\ m\) e \(h_2= 1.8 \ m\)
del pelo libero nei due recipienti e l’altezza \(H= 2.5\ m\) della tubatura,
determinare il valore della pressione nei punti A e B in figura,
esprimendolo sia in Pascal sia in metri d’acqua. Considerare la
pressione atmosferica standard (\(101325\ Pa\)).

\sphinxAtStartPar
(\(p_A=93477\ Pa = 9.53\ m_{H_2O}\), \(p_B=98970.6\ Pa=10.10\ m_{H_2O}\).)

\end{sphinxuseclass}
\end{sphinxuseclass}
\end{sphinxuseclass}
\end{sphinxuseclass}
\end{sphinxuseclass}
\end{sphinxuseclass}
\end{sphinxuseclass}
\end{sphinxuseclass}
\end{sphinxuseclass}
\end{sphinxuseclass}
\end{sphinxuseclass}
\end{sphinxuseclass}
\begin{sphinxuseclass}{sd-col}
\begin{sphinxuseclass}{sd-d-flex-row}
\begin{sphinxuseclass}{sd-col-4}
\begin{sphinxuseclass}{sd-col-xs-4}
\begin{sphinxuseclass}{sd-col-sm-4}
\begin{sphinxuseclass}{sd-col-md-4}
\begin{sphinxuseclass}{sd-col-lg-4}
\begin{sphinxuseclass}{sd-card}
\begin{sphinxuseclass}{sd-sphinx-override}
\begin{sphinxuseclass}{sd-w-100}
\begin{sphinxuseclass}{sd-shadow-sm}
\begin{sphinxuseclass}{sd-card-body}
\sphinxAtStartPar
\sphinxincludegraphics{{recipientiariaolio}.png}

\end{sphinxuseclass}
\end{sphinxuseclass}
\end{sphinxuseclass}
\end{sphinxuseclass}
\end{sphinxuseclass}
\end{sphinxuseclass}
\end{sphinxuseclass}
\end{sphinxuseclass}
\end{sphinxuseclass}
\end{sphinxuseclass}
\end{sphinxuseclass}
\end{sphinxuseclass}
\end{sphinxuseclass}
\end{sphinxuseclass}
\end{sphinxuseclass}
\end{sphinxuseclass}
\end{sphinxuseclass}
\end{sphinxuseclass}
\end{sphinxuseclass}
\end{sphinxuseclass}
\end{sphinxuseclass}


\begin{sphinxuseclass}{sd-container-fluid}
\begin{sphinxuseclass}{sd-sphinx-override}
\begin{sphinxuseclass}{sd-mb-4}
\begin{sphinxuseclass}{sd-row}
\begin{sphinxuseclass}{sd-g-2}
\begin{sphinxuseclass}{sd-g-xs-2}
\begin{sphinxuseclass}{sd-g-sm-2}
\begin{sphinxuseclass}{sd-g-md-2}
\begin{sphinxuseclass}{sd-g-lg-2}
\begin{sphinxuseclass}{sd-col}
\begin{sphinxuseclass}{sd-d-flex-row}
\begin{sphinxuseclass}{sd-col-8}
\begin{sphinxuseclass}{sd-col-xs-8}
\begin{sphinxuseclass}{sd-col-sm-8}
\begin{sphinxuseclass}{sd-col-md-8}
\begin{sphinxuseclass}{sd-col-lg-8}
\begin{sphinxuseclass}{sd-card}
\begin{sphinxuseclass}{sd-sphinx-override}
\begin{sphinxuseclass}{sd-w-100}
\begin{sphinxuseclass}{sd-shadow-sm}
\begin{sphinxuseclass}{sd-card-body}
\begin{sphinxuseclass}{sd-card-title}
\begin{sphinxuseclass}{sd-font-weight-bold}Exercise 1.3
\end{sphinxuseclass}
\end{sphinxuseclass}
\sphinxAtStartPar
Si consideri la sezione di diga rappresentata in figura.
Si determini il modulo e la direzione del risultante
delle forze per unità di apertura agente sui diversi
tratti rettilinei della diga stessa sapendo che la pressione
atmosferica é di \(1.01 \times 10^5\ Pa\). Dimensioni: \(a=10\ m\),
\(b=2\, m\), \(c=8\ m\), \(d=10\ m\), \(e=5\ m\), \(f=3\ m\).

\sphinxAtStartPar
(\(\mathbf{R}_1=347100\hat{\mathbf{x}}\  N/m\),
\(\ \mathbf{R}_2=- 1043200\hat{\mathbf{z}}\ N/m\),
\(\ \mathbf{R}_3=774500\hat{\mathbf{x}}\ N/m\),
\(\ \mathbf{R}_4=2284000 N/m \mathbf{\hat{x}} + 2284000 N/m \mathbf{\hat{z}}\),
\(\ \mathbf{R}_5=2774000\hat{\mathbf{z}}\ N/m\).)

\end{sphinxuseclass}
\end{sphinxuseclass}
\end{sphinxuseclass}
\end{sphinxuseclass}
\end{sphinxuseclass}
\end{sphinxuseclass}
\end{sphinxuseclass}
\end{sphinxuseclass}
\end{sphinxuseclass}
\end{sphinxuseclass}
\end{sphinxuseclass}
\end{sphinxuseclass}
\begin{sphinxuseclass}{sd-col}
\begin{sphinxuseclass}{sd-d-flex-row}
\begin{sphinxuseclass}{sd-col-4}
\begin{sphinxuseclass}{sd-col-xs-4}
\begin{sphinxuseclass}{sd-col-sm-4}
\begin{sphinxuseclass}{sd-col-md-4}
\begin{sphinxuseclass}{sd-col-lg-4}
\begin{sphinxuseclass}{sd-card}
\begin{sphinxuseclass}{sd-sphinx-override}
\begin{sphinxuseclass}{sd-w-100}
\begin{sphinxuseclass}{sd-shadow-sm}
\begin{sphinxuseclass}{sd-card-body}
\sphinxAtStartPar
\sphinxincludegraphics{{diga2}.png}

\end{sphinxuseclass}
\end{sphinxuseclass}
\end{sphinxuseclass}
\end{sphinxuseclass}
\end{sphinxuseclass}
\end{sphinxuseclass}
\end{sphinxuseclass}
\end{sphinxuseclass}
\end{sphinxuseclass}
\end{sphinxuseclass}
\end{sphinxuseclass}
\end{sphinxuseclass}
\end{sphinxuseclass}
\end{sphinxuseclass}
\end{sphinxuseclass}
\end{sphinxuseclass}
\end{sphinxuseclass}
\end{sphinxuseclass}
\end{sphinxuseclass}
\end{sphinxuseclass}
\end{sphinxuseclass}


\begin{sphinxuseclass}{sd-container-fluid}
\begin{sphinxuseclass}{sd-sphinx-override}
\begin{sphinxuseclass}{sd-mb-4}
\begin{sphinxuseclass}{sd-row}
\begin{sphinxuseclass}{sd-g-2}
\begin{sphinxuseclass}{sd-g-xs-2}
\begin{sphinxuseclass}{sd-g-sm-2}
\begin{sphinxuseclass}{sd-g-md-2}
\begin{sphinxuseclass}{sd-g-lg-2}
\begin{sphinxuseclass}{sd-col}
\begin{sphinxuseclass}{sd-d-flex-row}
\begin{sphinxuseclass}{sd-col-8}
\begin{sphinxuseclass}{sd-col-xs-8}
\begin{sphinxuseclass}{sd-col-sm-8}
\begin{sphinxuseclass}{sd-col-md-8}
\begin{sphinxuseclass}{sd-col-lg-8}
\begin{sphinxuseclass}{sd-card}
\begin{sphinxuseclass}{sd-sphinx-override}
\begin{sphinxuseclass}{sd-w-100}
\begin{sphinxuseclass}{sd-shadow-sm}
\begin{sphinxuseclass}{sd-card-body}
\begin{sphinxuseclass}{sd-card-title}
\begin{sphinxuseclass}{sd-font-weight-bold}Exercise 1.4
\end{sphinxuseclass}
\end{sphinxuseclass}
\sphinxAtStartPar
Si consideri il sistema di recipienti rappresentato in figura, in cui
la zona tratteggiata contiene acqua, di densità pari a \(10^3\ kg/m^3\) mentre nella restante parte é
presente aria di densità pari a \(1.2\ kg/m^3\). Determinare la pressione nei punti \(A\), \(B\), \(C\) e \(D\)
sapendo che le rispettive altezze sono \(h_A=1\ m\), \(h_B=1.4\ m\),
\(h_C=1.2\ m\) e \(h_D=1.6\ m\). Sia inoltre \(h_0=1.3\ m\) e la pressione
esterna \(P_0=101325\ Pa\).

\sphinxAtStartPar
(\(P_A=104262\ Pa\), \(P_B=100346\ Pa\), \(P_C=100348\ Pa\), \(P_D=97424\ Pa\).)

\end{sphinxuseclass}
\end{sphinxuseclass}
\end{sphinxuseclass}
\end{sphinxuseclass}
\end{sphinxuseclass}
\end{sphinxuseclass}
\end{sphinxuseclass}
\end{sphinxuseclass}
\end{sphinxuseclass}
\end{sphinxuseclass}
\end{sphinxuseclass}
\end{sphinxuseclass}
\begin{sphinxuseclass}{sd-col}
\begin{sphinxuseclass}{sd-d-flex-row}
\begin{sphinxuseclass}{sd-col-4}
\begin{sphinxuseclass}{sd-col-xs-4}
\begin{sphinxuseclass}{sd-col-sm-4}
\begin{sphinxuseclass}{sd-col-md-4}
\begin{sphinxuseclass}{sd-col-lg-4}
\begin{sphinxuseclass}{sd-card}
\begin{sphinxuseclass}{sd-sphinx-override}
\begin{sphinxuseclass}{sd-w-100}
\begin{sphinxuseclass}{sd-shadow-sm}
\begin{sphinxuseclass}{sd-card-body}
\sphinxAtStartPar
\sphinxincludegraphics{{tubimultipli}.png}

\end{sphinxuseclass}
\end{sphinxuseclass}
\end{sphinxuseclass}
\end{sphinxuseclass}
\end{sphinxuseclass}
\end{sphinxuseclass}
\end{sphinxuseclass}
\end{sphinxuseclass}
\end{sphinxuseclass}
\end{sphinxuseclass}
\end{sphinxuseclass}
\end{sphinxuseclass}
\end{sphinxuseclass}
\end{sphinxuseclass}
\end{sphinxuseclass}
\end{sphinxuseclass}
\end{sphinxuseclass}
\end{sphinxuseclass}
\end{sphinxuseclass}
\end{sphinxuseclass}
\end{sphinxuseclass}


\begin{sphinxuseclass}{sd-container-fluid}
\begin{sphinxuseclass}{sd-sphinx-override}
\begin{sphinxuseclass}{sd-mb-4}
\begin{sphinxuseclass}{sd-row}
\begin{sphinxuseclass}{sd-g-2}
\begin{sphinxuseclass}{sd-g-xs-2}
\begin{sphinxuseclass}{sd-g-sm-2}
\begin{sphinxuseclass}{sd-g-md-2}
\begin{sphinxuseclass}{sd-g-lg-2}
\begin{sphinxuseclass}{sd-col}
\begin{sphinxuseclass}{sd-d-flex-row}
\begin{sphinxuseclass}{sd-col-8}
\begin{sphinxuseclass}{sd-col-xs-8}
\begin{sphinxuseclass}{sd-col-sm-8}
\begin{sphinxuseclass}{sd-col-md-8}
\begin{sphinxuseclass}{sd-col-lg-8}
\begin{sphinxuseclass}{sd-card}
\begin{sphinxuseclass}{sd-sphinx-override}
\begin{sphinxuseclass}{sd-w-100}
\begin{sphinxuseclass}{sd-shadow-sm}
\begin{sphinxuseclass}{sd-card-body}
\begin{sphinxuseclass}{sd-card-title}
\begin{sphinxuseclass}{sd-font-weight-bold}Exercise 1.5
\end{sphinxuseclass}
\end{sphinxuseclass}
\sphinxAtStartPar
La leva idraulica, rappresentata in figura, é formata da due
sistemi cilindro\sphinxhyphen{}pistone.
Determinare la forza che é necessario applicare al secondo pistone
per mantenere il sistema in equilibrio
quando sul primo agisce una forza \(F_1 = 5000\ N\),
allorch’e i pistoni si trovano nella posizione indicata in figura.

\sphinxAtStartPar
Dati: diametro primo cilindro: \(d_1 = 0.2\ m\); diametro secondo cilindro:
\(d_2 = 0.4\ m\); diametro del condotto che unisce i due cilindri:
\(0.025\ m\);
densità del fluido di lavoro: \(600\ kg/m^3\);
altezza del primo pistone \(h_1 = 1\ m\), altezza del secondo pistone
\(h_2=2\ m\).

\sphinxAtStartPar
(\(p_1=159155\ Pa\), \(p_2=153269\ Pa\), \(\mathbf{F}_2=-19260.3 \hat{\mathbf{z}}\ N\).)

\end{sphinxuseclass}
\end{sphinxuseclass}
\end{sphinxuseclass}
\end{sphinxuseclass}
\end{sphinxuseclass}
\end{sphinxuseclass}
\end{sphinxuseclass}
\end{sphinxuseclass}
\end{sphinxuseclass}
\end{sphinxuseclass}
\end{sphinxuseclass}
\end{sphinxuseclass}
\begin{sphinxuseclass}{sd-col}
\begin{sphinxuseclass}{sd-d-flex-row}
\begin{sphinxuseclass}{sd-col-4}
\begin{sphinxuseclass}{sd-col-xs-4}
\begin{sphinxuseclass}{sd-col-sm-4}
\begin{sphinxuseclass}{sd-col-md-4}
\begin{sphinxuseclass}{sd-col-lg-4}
\begin{sphinxuseclass}{sd-card}
\begin{sphinxuseclass}{sd-sphinx-override}
\begin{sphinxuseclass}{sd-w-100}
\begin{sphinxuseclass}{sd-shadow-sm}
\begin{sphinxuseclass}{sd-card-body}
\sphinxAtStartPar
\sphinxincludegraphics{{leva_idraulica1}.png}

\end{sphinxuseclass}
\end{sphinxuseclass}
\end{sphinxuseclass}
\end{sphinxuseclass}
\end{sphinxuseclass}
\end{sphinxuseclass}
\end{sphinxuseclass}
\end{sphinxuseclass}
\end{sphinxuseclass}
\end{sphinxuseclass}
\end{sphinxuseclass}
\end{sphinxuseclass}
\end{sphinxuseclass}
\end{sphinxuseclass}
\end{sphinxuseclass}
\end{sphinxuseclass}
\end{sphinxuseclass}
\end{sphinxuseclass}
\end{sphinxuseclass}
\end{sphinxuseclass}
\end{sphinxuseclass}


\begin{sphinxuseclass}{sd-container-fluid}
\begin{sphinxuseclass}{sd-sphinx-override}
\begin{sphinxuseclass}{sd-mb-4}
\begin{sphinxuseclass}{sd-row}
\begin{sphinxuseclass}{sd-g-2}
\begin{sphinxuseclass}{sd-g-xs-2}
\begin{sphinxuseclass}{sd-g-sm-2}
\begin{sphinxuseclass}{sd-g-md-2}
\begin{sphinxuseclass}{sd-g-lg-2}
\begin{sphinxuseclass}{sd-col}
\begin{sphinxuseclass}{sd-d-flex-row}
\begin{sphinxuseclass}{sd-col-8}
\begin{sphinxuseclass}{sd-col-xs-8}
\begin{sphinxuseclass}{sd-col-sm-8}
\begin{sphinxuseclass}{sd-col-md-8}
\begin{sphinxuseclass}{sd-col-lg-8}
\begin{sphinxuseclass}{sd-card}
\begin{sphinxuseclass}{sd-sphinx-override}
\begin{sphinxuseclass}{sd-w-100}
\begin{sphinxuseclass}{sd-shadow-sm}
\begin{sphinxuseclass}{sd-card-body}
\begin{sphinxuseclass}{sd-card-title}
\begin{sphinxuseclass}{sd-font-weight-bold}Exercise 1.6
\end{sphinxuseclass}
\end{sphinxuseclass}
\sphinxAtStartPar
Si consideri il manometro riportato in figura utilizzato per misurare la
differenza di pressione esistente fra due sezioni diverse di un condotto.
Determinare la differenza di pressione fra i punti \(A\) e \(B\) riportati
sul disegno sapendo che il liquido manometrico é acqua e ha
una densità di \(998\ kg/m^3\), che il fluido che scorre all’interno
del condotto é aria e ha una densità di \(1.225\ kg/m^3\),
che \(h_A = 1\ m\), che \(h_B = 1.2\ m\), che \(h_0= 0.1\ m\),
che \(h_1 = 0.3\ m\) e che \(h_2 = 0.7\ m\).

\sphinxAtStartPar
(\(p_B-p_A=-3913.75\ Pa\))

\end{sphinxuseclass}
\end{sphinxuseclass}
\end{sphinxuseclass}
\end{sphinxuseclass}
\end{sphinxuseclass}
\end{sphinxuseclass}
\end{sphinxuseclass}
\end{sphinxuseclass}
\end{sphinxuseclass}
\end{sphinxuseclass}
\end{sphinxuseclass}
\end{sphinxuseclass}
\begin{sphinxuseclass}{sd-col}
\begin{sphinxuseclass}{sd-d-flex-row}
\begin{sphinxuseclass}{sd-col-4}
\begin{sphinxuseclass}{sd-col-xs-4}
\begin{sphinxuseclass}{sd-col-sm-4}
\begin{sphinxuseclass}{sd-col-md-4}
\begin{sphinxuseclass}{sd-col-lg-4}
\begin{sphinxuseclass}{sd-card}
\begin{sphinxuseclass}{sd-sphinx-override}
\begin{sphinxuseclass}{sd-w-100}
\begin{sphinxuseclass}{sd-shadow-sm}
\begin{sphinxuseclass}{sd-card-body}
\sphinxAtStartPar
\sphinxincludegraphics{{manometro}.png}

\end{sphinxuseclass}
\end{sphinxuseclass}
\end{sphinxuseclass}
\end{sphinxuseclass}
\end{sphinxuseclass}
\end{sphinxuseclass}
\end{sphinxuseclass}
\end{sphinxuseclass}
\end{sphinxuseclass}
\end{sphinxuseclass}
\end{sphinxuseclass}
\end{sphinxuseclass}
\end{sphinxuseclass}
\end{sphinxuseclass}
\end{sphinxuseclass}
\end{sphinxuseclass}
\end{sphinxuseclass}
\end{sphinxuseclass}
\end{sphinxuseclass}
\end{sphinxuseclass}
\end{sphinxuseclass}
\sphinxstepscope


\subsection{Exercise 1.1 }
\label{\detokenize{polimi/fluidmechanics-ita/template/capitoli/01_statica/0201in:exercise-1-1-archimede-s-law-and-buoyancy}}\label{\detokenize{polimi/fluidmechanics-ita/template/capitoli/01_statica/0201in:fluid-mechanics-statics-ex-01}}\label{\detokenize{polimi/fluidmechanics-ita/template/capitoli/01_statica/0201in::doc}}
\begin{sphinxuseclass}{sd-container-fluid}
\begin{sphinxuseclass}{sd-sphinx-override}
\begin{sphinxuseclass}{sd-mb-4}
\begin{sphinxuseclass}{sd-row}
\begin{sphinxuseclass}{sd-g-2}
\begin{sphinxuseclass}{sd-g-xs-2}
\begin{sphinxuseclass}{sd-g-sm-2}
\begin{sphinxuseclass}{sd-g-md-2}
\begin{sphinxuseclass}{sd-g-lg-2}
\begin{sphinxuseclass}{sd-col}
\begin{sphinxuseclass}{sd-d-flex-row}
\begin{sphinxuseclass}{sd-col-8}
\begin{sphinxuseclass}{sd-col-xs-8}
\begin{sphinxuseclass}{sd-col-sm-8}
\begin{sphinxuseclass}{sd-col-md-8}
\begin{sphinxuseclass}{sd-col-lg-8}
\begin{sphinxuseclass}{sd-card}
\begin{sphinxuseclass}{sd-sphinx-override}
\begin{sphinxuseclass}{sd-w-100}
\begin{sphinxuseclass}{sd-shadow-sm}
\begin{sphinxuseclass}{sd-card-body}
\begin{sphinxuseclass}{sd-card-title}
\begin{sphinxuseclass}{sd-font-weight-bold}Exercise 1.1
\end{sphinxuseclass}
\end{sphinxuseclass}
\sphinxAtStartPar
Si consideri, sulla superficie terrestre, un recipiente di diametro \(D=2 \ m\) e profondità \(H=3\  m\) contenente acqua di densità \(\rho = 998\ kg / m^3\). Al suo interno è inserita una sfera di raggio \(a=0.2\, m\) e densità pari a \(\rho_s=842.06\ kg / m^3\).
Determinare in modo univoco la posizione assunta dalla sfera nel liquido. Tale posizione varia se invece che sulla terra ci si trova sulla Luna?

\sphinxAtStartPar
(\(h=0.3\  m\), non varia sulla Luna.)

\end{sphinxuseclass}
\end{sphinxuseclass}
\end{sphinxuseclass}
\end{sphinxuseclass}
\end{sphinxuseclass}
\end{sphinxuseclass}
\end{sphinxuseclass}
\end{sphinxuseclass}
\end{sphinxuseclass}
\end{sphinxuseclass}
\end{sphinxuseclass}
\end{sphinxuseclass}
\begin{sphinxuseclass}{sd-col}
\begin{sphinxuseclass}{sd-d-flex-row}
\begin{sphinxuseclass}{sd-col-4}
\begin{sphinxuseclass}{sd-col-xs-4}
\begin{sphinxuseclass}{sd-col-sm-4}
\begin{sphinxuseclass}{sd-col-md-4}
\begin{sphinxuseclass}{sd-col-lg-4}
\begin{sphinxuseclass}{sd-card}
\begin{sphinxuseclass}{sd-sphinx-override}
\begin{sphinxuseclass}{sd-w-100}
\begin{sphinxuseclass}{sd-shadow-sm}
\begin{sphinxuseclass}{sd-card-body}
\sphinxAtStartPar
\sphinxincludegraphics{{recipientesfera}.png}

\end{sphinxuseclass}
\end{sphinxuseclass}
\end{sphinxuseclass}
\end{sphinxuseclass}
\end{sphinxuseclass}
\end{sphinxuseclass}
\end{sphinxuseclass}
\end{sphinxuseclass}
\end{sphinxuseclass}
\end{sphinxuseclass}
\end{sphinxuseclass}
\end{sphinxuseclass}
\end{sphinxuseclass}
\end{sphinxuseclass}
\end{sphinxuseclass}
\end{sphinxuseclass}
\end{sphinxuseclass}
\end{sphinxuseclass}
\end{sphinxuseclass}
\end{sphinxuseclass}
\end{sphinxuseclass}
\sphinxAtStartPar
\sphinxstylestrong{Concetti.} Legge di Archimede. Condizione di equilibrio. Calcolo del volume di
solidi (integrali di volume). Adimensionalizzazione. Soluzione di
semplici equazioni non lineari per via grafica (studio di funzione) e/o
numerica.

\sphinxAtStartPar
\sphinxstylestrong{Svolgimento.} Per svolgere l’esercizio bisogna calcolare la condizione di equilibrio
del corpo, soggetto alla propria forza peso e alla forza che il fluido
esercita su di esso (legge di Archimede). Nell’equazione di equilibrio,
l’incognita \(h\) compare nella formula del volume immerso nel fluido.
L’equazione di equilibrio è un’equazione non lineare in \(h\), da
risolvere per via grafica o numerica.
\begin{itemize}
\item {} 
\sphinxAtStartPar
Scrittura dell’equazione di equilibrio del corpo soggetto al proprio
peso e alla forza esercitata su di esso dal fluido, diretta verso
l’alto e pari al peso del volume del fluido spostato (legge di
Archimede). \$\(\label{eqn:equil_archimede}
      \rho_s V_s g = \rho V_c g \quad\Rightarrow\quad \rho_s V_s = \rho V_c\)\$

\sphinxAtStartPar
\sphinxstyleemphasis{Osservazione.} Si trova subito la risposta all’ultimo quesito:
poiché \(g\) non compare nell’equazione di equilibrio, la condizione
di equilibrio sulla Luna è uguale a quella che si ha sulla Terra.

\item {} 
\sphinxAtStartPar
Calcolo del volume della sfera e della calotta sferica:
\begin{itemize}
\item {} 
\sphinxAtStartPar
Volume della sfera: \(V_s = \frac{4}{3}\pi a^3\)

\item {} 
\sphinxAtStartPar
Volume della calotta sferica: \(V_c = \pi h^2 (a - \frac{h}{3})\)

\sphinxAtStartPar
(per credere, verificare casi limite: \(h=0\), \(h=a\), \(h=2a\); alla
fine dell’esercizio è riportato il calcolo, tramite un integrale
di volume)

\end{itemize}

\item {} 
\sphinxAtStartPar
Le formule per i volumi \(V_c\) e \(V_s\) sono inserite nell’eq.
\DUrole{xref,myst}{{[}eqn:equil\_archimede{]}}\{reference\sphinxhyphen{}type=»ref»
reference=»eqn:equil\_archimede»\}. L’equazione viene semplificata e
scritta in forma adimensionale, introducendo la variabile
\(x=\frac{h}{a}\), per mettere in evidenza il parametro che governa il
problema, cioè il rapporto di densità \(\rho_s/\rho\). L’equazione di
terzo grado in x viene risolta, considerando i limiti fisici del
problema (\(0 \le x \le 2\)):
\$\(\rho \pi h^2 \Big(a-\frac{h}{3}\Big) = \rho_s \frac{4}{3}\pi a^3  \quad\Rightarrow\quad
      \frac{3}{4} x^2 \Big(1 - \frac{x}{3}\Big) = \frac{\rho_s}{\rho}\)\$
Alcuni metodi per risolvere equazioni non lineari possono essere ad
esempio:
\begin{itemize}
\item {} 
\sphinxAtStartPar
metodi iterativi. Ad esempio metodo di Newton

\begin{sphinxVerbatim}[commandchars=\\\{\}]
x          res 
1.0000    \PYGZhy{}3.437475e\PYGZhy{}01  
1.4583    \PYGZhy{}2.406993e\PYGZhy{}02  
1.4990    \PYGZhy{}5.841602e\PYGZhy{}04  
1.5000    \PYGZhy{}4.027539e\PYGZhy{}07  
1.5000    \PYGZhy{}1.924017e\PYGZhy{}13
\end{sphinxVerbatim}

\item {} 
\sphinxAtStartPar
metodo grafico (educativo: per problemi più complicati, prima di
calcolare le soluzioni con metodi numerici, è bene avere un’idea
di cosa si sta cercando). Si cercano le intersezioni delle
funzioni \(f_1(x) = \frac{3}{4} x^2 \Big(1 - \frac{x}{3}\Big)\) e
\(f_2(x) = \frac{\rho_s}{\rho}\).

\end{itemize}

\end{itemize}

\sphinxAtStartPar
\sphinxstyleemphasis{Osservazione.} Per valori di \(\frac{\rho_s}{\rho}\) compresi tra 0 e 1,
esiste una e una sola soluzione fisica del problema. Per i valori di
desità «estremi» \(\rho_s = 0\) (la sfera non pesa niente),
\(\rho_s = \rho_f\) (la sfera ha la stessa densità del fluido), esistono
infinite soluzioni: ad esempio, nel caso di \(\rho_s = \rho_f\) la
posizione di equilibrio è indipendente dalla profondità alla quale è
posta la sfera. Nel grafico, la funzione \(f_1(x)\) rappresenta il volume
immerso della sfera (diviso il volume totale della sfera stessa) al
variare della distanza \(h\) del punto più basso dal pelo libero: questa
deve quindi essere rappresentata, come in figura, nulla per valori di
\(x<0\) (sfera completamente fuori dall’acqua), con il ramo di cubica per
\(0<x<2\) (sfera parzialmente immersa), uguale a \(1\) per \(x>2\) (sfera
completamente immersa). La funzione \(f_1(x)\) può quindi essere definita
a tratti:
\begin{equation*}
\begin{split}f_1(x) = 
 \begin{cases}
   0 &    x < 0 \\
   \frac{3}{4} x^2 \Big(1 - \frac{x}{3}\Big) &    0 \leq x \leq 2 \\
   1 &   x > 2 \\
 \end{cases}\end{split}
\end{equation*}
\sphinxAtStartPar
\sphinxstyleemphasis{Discussione dei risultati.} Quando diminuisce la denistà relativa del
solido, la linea rossa si abbassa e la soluzione \(x=\frac{h}{a}\)
diminuisce (la sfera ha una porzione maggiore al di fuori dall’acqua).
Esiste una e una sola soluzione che abbia senso fisico, fino a quando la
densità relativa è compresa tra 0 e 1: non ha senso considerare valori
negativi (la densità è una quantità positiva), mentre per valori di
\(\frac{\rho_s}{\rho}\) maggiori di 1 non può esistere una condizione di
equilibrio statico (la sfera affonda…).

\sphinxAtStartPar
\sphinxstylestrong{Calcolo volume cupola sferica.} É comodo svolgere il calcolo in
coordinate cilindriche \((r,\theta,z)\). Il volume \(V_{im}\) della parte
immersa è uguale a
\begin{equation*}
\begin{split}\begin{aligned}
V_{im} = \iiint_{V_{im}} dV & = \int_{\theta=0}^{2\pi} \int_{z=-a}^{l} \int_{r=0}^{\sqrt{a^2-z^2}} dV \\
                & = \int_{\theta=0}^{2\pi} \int_{z=-a}^{l} \int_{r=0}^{\sqrt{a^2-z^2}} r dr dz d\theta \\
                & = 2\pi \int_{z=-a}^{l} \frac{a^2-z^2}{2} dz \\
                & = \frac{\pi}{3} [2 a^3 + 3 a^2 l - l^3]
\end{aligned}\end{split}
\end{equation*}
\sphinxAtStartPar
Definendo \(h = R+l\) come la quota immersa della sfera, si ottiene:
\$\(V_{im} = \pi h^2 \displaystyle \left( a - \frac{h}{3} \right)\)\$

\sphinxstepscope


\subsection{Exercise 1.2 }
\label{\detokenize{polimi/fluidmechanics-ita/template/capitoli/01_statica/0202in:exercise-1-2-stevino-law-pressure-in-a-vessel}}\label{\detokenize{polimi/fluidmechanics-ita/template/capitoli/01_statica/0202in:fluid-mechanics-statics-ex-01}}\label{\detokenize{polimi/fluidmechanics-ita/template/capitoli/01_statica/0202in::doc}}
\begin{sphinxuseclass}{sd-container-fluid}
\begin{sphinxuseclass}{sd-sphinx-override}
\begin{sphinxuseclass}{sd-mb-4}
\begin{sphinxuseclass}{sd-row}
\begin{sphinxuseclass}{sd-g-2}
\begin{sphinxuseclass}{sd-g-xs-2}
\begin{sphinxuseclass}{sd-g-sm-2}
\begin{sphinxuseclass}{sd-g-md-2}
\begin{sphinxuseclass}{sd-g-lg-2}
\begin{sphinxuseclass}{sd-col}
\begin{sphinxuseclass}{sd-d-flex-row}
\begin{sphinxuseclass}{sd-col-8}
\begin{sphinxuseclass}{sd-col-xs-8}
\begin{sphinxuseclass}{sd-col-sm-8}
\begin{sphinxuseclass}{sd-col-md-8}
\begin{sphinxuseclass}{sd-col-lg-8}
\begin{sphinxuseclass}{sd-card}
\begin{sphinxuseclass}{sd-sphinx-override}
\begin{sphinxuseclass}{sd-w-100}
\begin{sphinxuseclass}{sd-shadow-sm}
\begin{sphinxuseclass}{sd-card-body}
\begin{sphinxuseclass}{sd-card-title}
\begin{sphinxuseclass}{sd-font-weight-bold}Exercise 1.2
\end{sphinxuseclass}
\end{sphinxuseclass}
\sphinxAtStartPar
Si consideri il sistema rappresentato in figura in cui un recipiente
aperto all’atmosfera, contenente olio con densità \(\rho= 800\ kg/m^3\), è
collegato tramite una tubazione a un secondo recipiente, contenente a
sua volta olio e aria non miscelati. Date le due altezze \(h_1=1.5\ m\) e \(h_2= 1.8 \ m\)
del pelo libero nei due recipienti e l’altezza \(H= 2.5\ m\) della tubatura,
determinare il valore della pressione nei punti A e B in figura,
esprimendolo sia in Pascal sia in metri d’acqua. Considerare la
pressione atmosferica standard (\(101325\ Pa\)).

\sphinxAtStartPar
(\(p_A=93477\ Pa = 9.53\ m_{H_2O}\), \(p_B=98970.6\ Pa=10.10\ m_{H_2O}\).)

\end{sphinxuseclass}
\end{sphinxuseclass}
\end{sphinxuseclass}
\end{sphinxuseclass}
\end{sphinxuseclass}
\end{sphinxuseclass}
\end{sphinxuseclass}
\end{sphinxuseclass}
\end{sphinxuseclass}
\end{sphinxuseclass}
\end{sphinxuseclass}
\end{sphinxuseclass}
\begin{sphinxuseclass}{sd-col}
\begin{sphinxuseclass}{sd-d-flex-row}
\begin{sphinxuseclass}{sd-col-4}
\begin{sphinxuseclass}{sd-col-xs-4}
\begin{sphinxuseclass}{sd-col-sm-4}
\begin{sphinxuseclass}{sd-col-md-4}
\begin{sphinxuseclass}{sd-col-lg-4}
\begin{sphinxuseclass}{sd-card}
\begin{sphinxuseclass}{sd-sphinx-override}
\begin{sphinxuseclass}{sd-w-100}
\begin{sphinxuseclass}{sd-shadow-sm}
\begin{sphinxuseclass}{sd-card-body}
\sphinxAtStartPar
\sphinxincludegraphics{{recipientiariaolio}.png}

\end{sphinxuseclass}
\end{sphinxuseclass}
\end{sphinxuseclass}
\end{sphinxuseclass}
\end{sphinxuseclass}
\end{sphinxuseclass}
\end{sphinxuseclass}
\end{sphinxuseclass}
\end{sphinxuseclass}
\end{sphinxuseclass}
\end{sphinxuseclass}
\end{sphinxuseclass}
\end{sphinxuseclass}
\end{sphinxuseclass}
\end{sphinxuseclass}
\end{sphinxuseclass}
\end{sphinxuseclass}
\end{sphinxuseclass}
\end{sphinxuseclass}
\end{sphinxuseclass}
\end{sphinxuseclass}
\sphinxAtStartPar
\sphinxstylestrong{Concetti.} Legge di Stevino, \(P_1 + \rho g h_1 = P_2 + \rho g h_2\). Conversione
\(Pa\) \sphinxhyphen{} metri di \(H_2O\),
\$\(1 m_{{H_2O}} = P[Pa] = \rho_{H_2O} \cdot g \cdot 1 m =
9810 \dfrac{kg}{m^2 s^2} \cdot 1 m = 9810 Pa \ .\)\$

\sphinxAtStartPar
\sphinxstylestrong{Svolgimento.} Il problema si risolve applicando due volte la legge di Stevino e la
conversione da Pascal \(Pa\) a metri d’acqua \(m_{H_2O}\). Sia \(O\) il punto
sul pelo libero nel serbatoio \sphinxstyleemphasis{aperto} di sinistra, sul quale agisce la
pressione ambiente.
\begin{equation*}
\begin{split}\begin{cases}
  P_A = P_O + \rho g (h_1 - H)  = 93477 Pa = \dfrac{93477}{9810} m_{H_2O} = 9.53 m_{H_2O}  & \text{(Stevino O-A)} \\ \\
  P_B = P_O + \rho g (h_1 - h_2) = 98970.6 Pa = \dfrac{98970.6}{9810} m_{H_2O}= 10.10 m_{H_2O} & \text{(Stevino O-B)}
\end{cases}\end{split}
\end{equation*}
\sphinxstepscope


\subsection{Exercise 1.3}
\label{\detokenize{polimi/fluidmechanics-ita/template/capitoli/01_statica/0203in:exercise-1-3}}\label{\detokenize{polimi/fluidmechanics-ita/template/capitoli/01_statica/0203in:fluid-mechanics-statics-ex-03}}\label{\detokenize{polimi/fluidmechanics-ita/template/capitoli/01_statica/0203in::doc}}
\begin{sphinxuseclass}{sd-container-fluid}
\begin{sphinxuseclass}{sd-sphinx-override}
\begin{sphinxuseclass}{sd-mb-4}
\begin{sphinxuseclass}{sd-row}
\begin{sphinxuseclass}{sd-g-2}
\begin{sphinxuseclass}{sd-g-xs-2}
\begin{sphinxuseclass}{sd-g-sm-2}
\begin{sphinxuseclass}{sd-g-md-2}
\begin{sphinxuseclass}{sd-g-lg-2}
\begin{sphinxuseclass}{sd-col}
\begin{sphinxuseclass}{sd-d-flex-row}
\begin{sphinxuseclass}{sd-col-8}
\begin{sphinxuseclass}{sd-col-xs-8}
\begin{sphinxuseclass}{sd-col-sm-8}
\begin{sphinxuseclass}{sd-col-md-8}
\begin{sphinxuseclass}{sd-col-lg-8}
\begin{sphinxuseclass}{sd-card}
\begin{sphinxuseclass}{sd-sphinx-override}
\begin{sphinxuseclass}{sd-w-100}
\begin{sphinxuseclass}{sd-shadow-sm}
\begin{sphinxuseclass}{sd-card-body}
\begin{sphinxuseclass}{sd-card-title}
\begin{sphinxuseclass}{sd-font-weight-bold}Exercise 1.3
\end{sphinxuseclass}
\end{sphinxuseclass}
\sphinxAtStartPar
Si consideri la sezione di diga rappresentata in figura.
Si determini il modulo e la direzione del risultante
delle forze per unità di apertura agente sui diversi
tratti rettilinei della diga stessa sapendo che la pressione
atmosferica é di \(1.01 \times 10^5\ Pa\). Dimensioni: \(a=10\ m\),
\(b=2\, m\), \(c=8\ m\), \(d=10\ m\), \(e=5\ m\), \(f=3\ m\).

\sphinxAtStartPar
(\(\mathbf{R}_1=347100\hat{\mathbf{x}}\  N/m\),
\(\ \mathbf{R}_2=- 1043200\hat{\mathbf{z}}\ N/m\),
\(\ \mathbf{R}_3=774500\hat{\mathbf{x}}\ N/m\),
\(\ \mathbf{R}_4=2284000 N/m \mathbf{\hat{x}} + 2284000 N/m \mathbf{\hat{z}}\),
\(\ \mathbf{R}_5=2774000\hat{\mathbf{z}}\ N/m\).)

\end{sphinxuseclass}
\end{sphinxuseclass}
\end{sphinxuseclass}
\end{sphinxuseclass}
\end{sphinxuseclass}
\end{sphinxuseclass}
\end{sphinxuseclass}
\end{sphinxuseclass}
\end{sphinxuseclass}
\end{sphinxuseclass}
\end{sphinxuseclass}
\end{sphinxuseclass}
\begin{sphinxuseclass}{sd-col}
\begin{sphinxuseclass}{sd-d-flex-row}
\begin{sphinxuseclass}{sd-col-4}
\begin{sphinxuseclass}{sd-col-xs-4}
\begin{sphinxuseclass}{sd-col-sm-4}
\begin{sphinxuseclass}{sd-col-md-4}
\begin{sphinxuseclass}{sd-col-lg-4}
\begin{sphinxuseclass}{sd-card}
\begin{sphinxuseclass}{sd-sphinx-override}
\begin{sphinxuseclass}{sd-w-100}
\begin{sphinxuseclass}{sd-shadow-sm}
\begin{sphinxuseclass}{sd-card-body}
\sphinxAtStartPar
\sphinxincludegraphics{{diga2}.png}

\end{sphinxuseclass}
\end{sphinxuseclass}
\end{sphinxuseclass}
\end{sphinxuseclass}
\end{sphinxuseclass}
\end{sphinxuseclass}
\end{sphinxuseclass}
\end{sphinxuseclass}
\end{sphinxuseclass}
\end{sphinxuseclass}
\end{sphinxuseclass}
\end{sphinxuseclass}
\end{sphinxuseclass}
\end{sphinxuseclass}
\end{sphinxuseclass}
\end{sphinxuseclass}
\end{sphinxuseclass}
\end{sphinxuseclass}
\end{sphinxuseclass}
\end{sphinxuseclass}
\end{sphinxuseclass}
\sphinxAtStartPar
\sphinxstylestrong{Concetti.} Legge di Stevino, \(P_1 + \rho g h_1 = P_2 + \rho g h_2\). Calcolo della
risultante delle azioni statiche, data la distribuzione di pressione e
la normale \(\mathbf{\hat{n}}\) uscente dal volume fluido,
\$\(\mathbf{R} = \int_{S} P \mathbf{\hat{n}} \ .\)\$

\sphinxAtStartPar
\sphinxstylestrong{Svolgimento.} Si risolve il problema bidimensionale, al quale «manca» la dimensione
perpendicolare al piano del disegno. La risultante per unità di apertura
agente sul lato \(\ell\) (unità di misura nel SI, \(N/m\)) sarà quindi il
risultato dell’integrale di linea
\begin{equation*}
\begin{split}\mathbf{R} = \int_{\ell} P \mathbf{\hat{n}} \ .\end{split}
\end{equation*}
\sphinxAtStartPar
Per ogni lato si calcola la
distribuzione di pressione, grazie alla legge di Stevino. Si integra la
distribuzione di pressione per ottenere il modulo della risultante; la
direzione coincide con quella della normale (uscente dal volume occupato
dal fluido). Per lo svolgimento, è stato scelto il sistema di
riferimento rappresentato in figura, con l’asse x diretto verso destra e
l’asse z verso il basso.
\begin{itemize}
\item {} 
\sphinxAtStartPar
Lato 1. Pressione lineare in z,
\(P(z) = P_O + \rho g z , \ z \in [0,f]\). Risultante
\begin{equation*}
\begin{split}\mathbf{R}_1 = \int_{\ell_1} P \mathbf{\hat{n}} = \int_{0}^{f} (P_O + \rho g z) \mathbf{\hat{x}} dz = 
         \displaystyle\left(P_O  f + \frac{1}{2} \rho g f^2 \right) \mathbf{\hat{x}} = 347100 N/m \mathbf{\hat{x}}\end{split}
\end{equation*}
\item {} 
\sphinxAtStartPar
Lato 2. Pressione costante, \(P = P_O + \rho g f\). Risultante
\begin{equation*}
\begin{split}\mathbf{R}_2 = \int_{\ell_2} P \mathbf{\hat{n}} = P\cdot c (-\mathbf{\hat{z}})=(P_O + \rho g f)\cdot c(-\mathbf{\hat{z}}) = - 1043200 N/m \mathbf{\hat{z}}\end{split}
\end{equation*}
\item {} 
\sphinxAtStartPar
Lato 3. Pressione lineare in z,
\(P(z) = P_O + \rho g z , \  z \in [f,f+e]\). Risultante
\begin{equation*}
\begin{split}\mathbf{R}_3 = \int_{\ell_3} P \mathbf{\hat{n}} = \int_{f}^{f+e} (P_O + \rho g z) \mathbf{\hat{x}} dz = 
         \displaystyle\left(P_O e + \frac{1}{2} \rho g \left[(f+e)^2 - f^2\right]\right) \mathbf{\hat{x}} = 774500 N/m \mathbf{\hat{x}}\end{split}
\end{equation*}
\item {} 
\sphinxAtStartPar
Lato 4. Pressione lineare in z,
\(P(z)  = P_O + \rho g z , \ z \in [f+e,f+e+d]\). Poichè il tratto di
parete è rettilineo, il vettore normale è costante e può essere
portato fuori dall’integrale. Si calcola prima il modulo della
risultante e poi lo si moltiplica per il versore normale. Il modulo
della risultante vale
\begin{equation*}
\begin{split}\begin{aligned}
          {R}_4 & = \int_{\ell_4} P d\ell = \int_{f+e}^{f+e+d} P(z) \frac{\sqrt{(b+c)^2+d^2}}{d} dz = \qquad \qquad \text{$\displaystyle\left(d\ell = \frac{\sqrt{(b+c)^2+d^2}}{d} dz \right)$} \\
         & = \int_{f+e}^{f+e+d} (P_O + \rho g z) \frac{\sqrt{(b+c)^2+d^2}}{d} dz = \\
         & = \frac{\sqrt{(b+c)^2+d^2}}{d}\left[ P_O d + \frac{1}{2} \rho g \left((f+e+d)^2-(f+e)^2 \right)  \right] = 
         \sqrt{2} \cdot 2284000 N/m \\
       \end{aligned}\end{split}
\end{equation*}
\sphinxAtStartPar
La forza può essere scritta come
\(\mathbf{R}_4 = R_4 \mathbf{\hat{n}}_4\), con
\(\mathbf{\hat{n}}_4 = 1/\sqrt{2} \ \hat{\mathbf{x}} + 1/\sqrt{2} \ \hat{\mathbf{z}}\).
Proietttando \(\mathbf{R}_4\) lungo gli assi si ottengono le componenti
orizzontali e verticali
\begin{equation*}
\begin{split}\mathbf{R}_4 = 2284000 N/m \mathbf{\hat{x}} + 2284000 N/m \mathbf{\hat{z}}\end{split}
\end{equation*}
\item {} 
\sphinxAtStartPar
Lato 5. Pressione costante, \(P = P_O + \rho g (f+e+d)\). Risultante
\begin{equation*}
\begin{split}\mathbf{R}_5 = P\cdot a \mathbf{\hat{z}} =(P_O + \rho g (f+e+d))\cdot a \mathbf{\hat{z}} =  2774000 N/m \mathbf{\hat{z}}\end{split}
\end{equation*}
\end{itemize}

\sphinxstepscope


\subsection{Exercise 1.4}
\label{\detokenize{polimi/fluidmechanics-ita/template/capitoli/01_statica/0204in:exercise-1-4}}\label{\detokenize{polimi/fluidmechanics-ita/template/capitoli/01_statica/0204in:fluid-mechanics-statics-ex-04}}\label{\detokenize{polimi/fluidmechanics-ita/template/capitoli/01_statica/0204in::doc}}
\begin{sphinxuseclass}{sd-container-fluid}
\begin{sphinxuseclass}{sd-sphinx-override}
\begin{sphinxuseclass}{sd-mb-4}
\begin{sphinxuseclass}{sd-row}
\begin{sphinxuseclass}{sd-g-2}
\begin{sphinxuseclass}{sd-g-xs-2}
\begin{sphinxuseclass}{sd-g-sm-2}
\begin{sphinxuseclass}{sd-g-md-2}
\begin{sphinxuseclass}{sd-g-lg-2}
\begin{sphinxuseclass}{sd-col}
\begin{sphinxuseclass}{sd-d-flex-row}
\begin{sphinxuseclass}{sd-col-8}
\begin{sphinxuseclass}{sd-col-xs-8}
\begin{sphinxuseclass}{sd-col-sm-8}
\begin{sphinxuseclass}{sd-col-md-8}
\begin{sphinxuseclass}{sd-col-lg-8}
\begin{sphinxuseclass}{sd-card}
\begin{sphinxuseclass}{sd-sphinx-override}
\begin{sphinxuseclass}{sd-w-100}
\begin{sphinxuseclass}{sd-shadow-sm}
\begin{sphinxuseclass}{sd-card-body}
\begin{sphinxuseclass}{sd-card-title}
\begin{sphinxuseclass}{sd-font-weight-bold}Exercise 1.4
\end{sphinxuseclass}
\end{sphinxuseclass}
\sphinxAtStartPar
Si consideri il sistema di recipienti rappresentato in figura, in cui
la zona tratteggiata contiene acqua, di densità pari a \(10^3\ kg/m^3\) mentre nella restante parte é
presente aria di densità pari a \(1.2\ kg/m^3\). Determinare la pressione nei punti \(A\), \(B\), \(C\) e \(D\)
sapendo che le rispettive altezze sono \(h_A=1\ m\), \(h_B=1.4\ m\),
\(h_C=1.2\ m\) e \(h_D=1.6\ m\). Sia inoltre \(h_0=1.3\ m\) e la pressione
esterna \(P_0=101325\ Pa\).

\sphinxAtStartPar
(\(P_A=104262\ Pa\), \(P_B=100346\ Pa\), \(P_C=100348\ Pa\), \(P_D=97424\ Pa\).)

\end{sphinxuseclass}
\end{sphinxuseclass}
\end{sphinxuseclass}
\end{sphinxuseclass}
\end{sphinxuseclass}
\end{sphinxuseclass}
\end{sphinxuseclass}
\end{sphinxuseclass}
\end{sphinxuseclass}
\end{sphinxuseclass}
\end{sphinxuseclass}
\end{sphinxuseclass}
\begin{sphinxuseclass}{sd-col}
\begin{sphinxuseclass}{sd-d-flex-row}
\begin{sphinxuseclass}{sd-col-4}
\begin{sphinxuseclass}{sd-col-xs-4}
\begin{sphinxuseclass}{sd-col-sm-4}
\begin{sphinxuseclass}{sd-col-md-4}
\begin{sphinxuseclass}{sd-col-lg-4}
\begin{sphinxuseclass}{sd-card}
\begin{sphinxuseclass}{sd-sphinx-override}
\begin{sphinxuseclass}{sd-w-100}
\begin{sphinxuseclass}{sd-shadow-sm}
\begin{sphinxuseclass}{sd-card-body}
\sphinxAtStartPar
\sphinxincludegraphics{{tubimultipli}.png}

\end{sphinxuseclass}
\end{sphinxuseclass}
\end{sphinxuseclass}
\end{sphinxuseclass}
\end{sphinxuseclass}
\end{sphinxuseclass}
\end{sphinxuseclass}
\end{sphinxuseclass}
\end{sphinxuseclass}
\end{sphinxuseclass}
\end{sphinxuseclass}
\end{sphinxuseclass}
\end{sphinxuseclass}
\end{sphinxuseclass}
\end{sphinxuseclass}
\end{sphinxuseclass}
\end{sphinxuseclass}
\end{sphinxuseclass}
\end{sphinxuseclass}
\end{sphinxuseclass}
\end{sphinxuseclass}
\sphinxAtStartPar
\sphinxstylestrong{Concetti.} Legge di Stevino, \(P_1 + \rho g h_1 = P_2 + \rho g h_2\).

\sphinxAtStartPar
\sphinxstylestrong{Svolgimento.} Il problema viene risolto applicando ripetutamente la legge di Stevino,
a partire dalla superficie \(0\) sulla quale agisce la pressione ambiente
\(P_0\). Nella legge di Stevino è necessario prestare attenzione ad usare
la densità del fluido che mette in collegamento i due punti considerati.
I punti \(A\) e \(B\) sono messi in collegamento con il punto \(0\)
dall’acqua. I punti \(B\) e \(C\) sono messi in collegamento tra di loro
dall’aria. I punti \(C\) e \(D\) di nuovo dall’acqua. La soluzione del
problema è quindi
\begin{equation*}
\begin{split}\begin{aligned}
 & P_0 = 101325 Pa & \text{dato} \\
 & P_A = P_0 + \rho g (h_0 - h_A) = ... \\
 & P_B = P_0 + \rho g (h_0 - h_B) = ... \\
 & P_C = P_B + \rho_a g (h_B - h_C) = ... \\
 & P_D = P_C + \rho g (h_C - h_D) = ... 
\end{aligned}\end{split}
\end{equation*}
\sphinxstepscope


\subsection{Exercise 1.5}
\label{\detokenize{polimi/fluidmechanics-ita/template/capitoli/01_statica/0205in:exercise-1-5}}\label{\detokenize{polimi/fluidmechanics-ita/template/capitoli/01_statica/0205in:fluid-mechanics-statics-ex-05}}\label{\detokenize{polimi/fluidmechanics-ita/template/capitoli/01_statica/0205in::doc}}
\begin{sphinxuseclass}{sd-container-fluid}
\begin{sphinxuseclass}{sd-sphinx-override}
\begin{sphinxuseclass}{sd-mb-4}
\begin{sphinxuseclass}{sd-row}
\begin{sphinxuseclass}{sd-g-2}
\begin{sphinxuseclass}{sd-g-xs-2}
\begin{sphinxuseclass}{sd-g-sm-2}
\begin{sphinxuseclass}{sd-g-md-2}
\begin{sphinxuseclass}{sd-g-lg-2}
\begin{sphinxuseclass}{sd-col}
\begin{sphinxuseclass}{sd-d-flex-row}
\begin{sphinxuseclass}{sd-col-8}
\begin{sphinxuseclass}{sd-col-xs-8}
\begin{sphinxuseclass}{sd-col-sm-8}
\begin{sphinxuseclass}{sd-col-md-8}
\begin{sphinxuseclass}{sd-col-lg-8}
\begin{sphinxuseclass}{sd-card}
\begin{sphinxuseclass}{sd-sphinx-override}
\begin{sphinxuseclass}{sd-w-100}
\begin{sphinxuseclass}{sd-shadow-sm}
\begin{sphinxuseclass}{sd-card-body}
\begin{sphinxuseclass}{sd-card-title}
\begin{sphinxuseclass}{sd-font-weight-bold}Exercise 1.5
\end{sphinxuseclass}
\end{sphinxuseclass}
\sphinxAtStartPar
La leva idraulica, rappresentata in figura, é formata da due
sistemi cilindro\sphinxhyphen{}pistone.
Determinare la forza che é necessario applicare al secondo pistone
per mantenere il sistema in equilibrio
quando sul primo agisce una forza \(F_1 = 5000\ N\),
allorch’e i pistoni si trovano nella posizione indicata in figura.

\sphinxAtStartPar
Dati: diametro primo cilindro: \(d_1 = 0.2\ m\); diametro secondo cilindro:
\(d_2 = 0.4\ m\); diametro del condotto che unisce i due cilindri:
\(0.025\ m\);
densità del fluido di lavoro: \(600\ kg/m^3\);
altezza del primo pistone \(h_1 = 1\ m\), altezza del secondo pistone
\(h_2=2\ m\).

\sphinxAtStartPar
(\(p_1=159155\ Pa\), \(p_2=153269\ Pa\), \(\mathbf{F}_2=-19260.3 \hat{\mathbf{z}}\ N\).)

\end{sphinxuseclass}
\end{sphinxuseclass}
\end{sphinxuseclass}
\end{sphinxuseclass}
\end{sphinxuseclass}
\end{sphinxuseclass}
\end{sphinxuseclass}
\end{sphinxuseclass}
\end{sphinxuseclass}
\end{sphinxuseclass}
\end{sphinxuseclass}
\end{sphinxuseclass}
\begin{sphinxuseclass}{sd-col}
\begin{sphinxuseclass}{sd-d-flex-row}
\begin{sphinxuseclass}{sd-col-4}
\begin{sphinxuseclass}{sd-col-xs-4}
\begin{sphinxuseclass}{sd-col-sm-4}
\begin{sphinxuseclass}{sd-col-md-4}
\begin{sphinxuseclass}{sd-col-lg-4}
\begin{sphinxuseclass}{sd-card}
\begin{sphinxuseclass}{sd-sphinx-override}
\begin{sphinxuseclass}{sd-w-100}
\begin{sphinxuseclass}{sd-shadow-sm}
\begin{sphinxuseclass}{sd-card-body}
\sphinxAtStartPar
\sphinxincludegraphics{{leva_idraulica1}.png}

\end{sphinxuseclass}
\end{sphinxuseclass}
\end{sphinxuseclass}
\end{sphinxuseclass}
\end{sphinxuseclass}
\end{sphinxuseclass}
\end{sphinxuseclass}
\end{sphinxuseclass}
\end{sphinxuseclass}
\end{sphinxuseclass}
\end{sphinxuseclass}
\end{sphinxuseclass}
\end{sphinxuseclass}
\end{sphinxuseclass}
\end{sphinxuseclass}
\end{sphinxuseclass}
\end{sphinxuseclass}
\end{sphinxuseclass}
\end{sphinxuseclass}
\end{sphinxuseclass}
\end{sphinxuseclass}
\sphinxAtStartPar
\sphinxstylestrong{Concetti.} Legge di Stevino. Risultante statica. Leva idraulica.

\sphinxAtStartPar
\sphinxstylestrong{Svolgimento.} Il problema si risolve scrivendo le condizioni di equilibrio tra le
forze esterne e la risultante dello sforzo di pressione sulle facce
opposte dei pistoni e applicando la legge di Stevino tra le due sezioni
\(A_1\) e \(A_2\). Si ottiene un sistema lineare di tre equazioni in tre
incognite \(p_1, p_2, F_2\)),
\begin{equation*}
\begin{split}\begin{cases}
  F_1 = p_1 \pi \dfrac{d_1 ^2}{4} & \text{(Equilibrio pistone 1)} \\
  p_2 = p_1 - \rho g (h_2 - h_1) & \text{(Legge di Stevino)} \\
  F_2 = p_2 \pi \dfrac{d_2 ^2}{4} & \text{(Equilibrio pistone 2)} \ ,
\end{cases}\end{split}
\end{equation*}
\sphinxAtStartPar
la cui soluzione è
\begin{equation*}
\begin{split}\Rightarrow \quad
\begin{cases}
  p_1 = \dfrac{4}{\pi} \dfrac{F_1}{d_1^2}  & = 159155 Pa \\
  p_2 = \dfrac{4}{\pi} \dfrac{F_1}{d_1^2} - \rho g (h_2 - h_1) & = 153269 Pa \\
  F_2 = \dfrac{d_1^2}{d_2^2} F_1 - \dfrac{\pi}{4}d_2^2 \ \rho g (h_2 - h_1) & = 19260.3  N \ .
\end{cases}\end{split}
\end{equation*}
\sphinxAtStartPar
La componente verticale \(F_2\) della forza \(\mathbf{F_2}\) è
positiva diretta verso il basso, come nel disegno. Si può scrivere
quindi \(\mathbf{F_2} = - F_2 \mathbf{\hat{z}}\), se il versore \(\mathbf{\hat{z}}\) è
orientato verso l’alto.

\sphinxstepscope


\subsection{Exercise 1.6}
\label{\detokenize{polimi/fluidmechanics-ita/template/capitoli/01_statica/0206in:exercise-1-6}}\label{\detokenize{polimi/fluidmechanics-ita/template/capitoli/01_statica/0206in:fluid-mechanics-statics-ex-06}}\label{\detokenize{polimi/fluidmechanics-ita/template/capitoli/01_statica/0206in::doc}}
\begin{sphinxuseclass}{sd-container-fluid}
\begin{sphinxuseclass}{sd-sphinx-override}
\begin{sphinxuseclass}{sd-mb-4}
\begin{sphinxuseclass}{sd-row}
\begin{sphinxuseclass}{sd-g-2}
\begin{sphinxuseclass}{sd-g-xs-2}
\begin{sphinxuseclass}{sd-g-sm-2}
\begin{sphinxuseclass}{sd-g-md-2}
\begin{sphinxuseclass}{sd-g-lg-2}
\begin{sphinxuseclass}{sd-col}
\begin{sphinxuseclass}{sd-d-flex-row}
\begin{sphinxuseclass}{sd-col-8}
\begin{sphinxuseclass}{sd-col-xs-8}
\begin{sphinxuseclass}{sd-col-sm-8}
\begin{sphinxuseclass}{sd-col-md-8}
\begin{sphinxuseclass}{sd-col-lg-8}
\begin{sphinxuseclass}{sd-card}
\begin{sphinxuseclass}{sd-sphinx-override}
\begin{sphinxuseclass}{sd-w-100}
\begin{sphinxuseclass}{sd-shadow-sm}
\begin{sphinxuseclass}{sd-card-body}
\begin{sphinxuseclass}{sd-card-title}
\begin{sphinxuseclass}{sd-font-weight-bold}Exercise 1.6
\end{sphinxuseclass}
\end{sphinxuseclass}
\sphinxAtStartPar
Si consideri il manometro riportato in figura utilizzato per misurare la
differenza di pressione esistente fra due sezioni diverse di un condotto.
Determinare la differenza di pressione fra i punti \(A\) e \(B\) riportati
sul disegno sapendo che il liquido manometrico é acqua e ha
una densità di \(998\ kg/m^3\), che il fluido che scorre all’interno
del condotto é aria e ha una densità di \(1.225\ kg/m^3\),
che \(h_A = 1\ m\), che \(h_B = 1.2\ m\), che \(h_0= 0.1\ m\),
che \(h_1 = 0.3\ m\) e che \(h_2 = 0.7\ m\).

\sphinxAtStartPar
(\(p_B-p_A=-3913.75\ Pa\))

\end{sphinxuseclass}
\end{sphinxuseclass}
\end{sphinxuseclass}
\end{sphinxuseclass}
\end{sphinxuseclass}
\end{sphinxuseclass}
\end{sphinxuseclass}
\end{sphinxuseclass}
\end{sphinxuseclass}
\end{sphinxuseclass}
\end{sphinxuseclass}
\end{sphinxuseclass}
\begin{sphinxuseclass}{sd-col}
\begin{sphinxuseclass}{sd-d-flex-row}
\begin{sphinxuseclass}{sd-col-4}
\begin{sphinxuseclass}{sd-col-xs-4}
\begin{sphinxuseclass}{sd-col-sm-4}
\begin{sphinxuseclass}{sd-col-md-4}
\begin{sphinxuseclass}{sd-col-lg-4}
\begin{sphinxuseclass}{sd-card}
\begin{sphinxuseclass}{sd-sphinx-override}
\begin{sphinxuseclass}{sd-w-100}
\begin{sphinxuseclass}{sd-shadow-sm}
\begin{sphinxuseclass}{sd-card-body}
\sphinxAtStartPar
\sphinxincludegraphics{{manometro}.png}

\end{sphinxuseclass}
\end{sphinxuseclass}
\end{sphinxuseclass}
\end{sphinxuseclass}
\end{sphinxuseclass}
\end{sphinxuseclass}
\end{sphinxuseclass}
\end{sphinxuseclass}
\end{sphinxuseclass}
\end{sphinxuseclass}
\end{sphinxuseclass}
\end{sphinxuseclass}
\end{sphinxuseclass}
\end{sphinxuseclass}
\end{sphinxuseclass}
\end{sphinxuseclass}
\end{sphinxuseclass}
\end{sphinxuseclass}
\end{sphinxuseclass}
\end{sphinxuseclass}
\end{sphinxuseclass}
\sphinxAtStartPar
\sphinxstylestrong{Concetti.} Legge di Stevino. Manometro. Venturi.
\$\(P_1 + \rho g h_1 = P_2 + \rho g h_2\)\$

\sphinxAtStartPar
\sphinxstylestrong{Svolgimento.} Si scrive la legge di Stevino tra i punti A e 1, 1 e 2, 2 e B:
\$\(\label{eqn:stevino:underdet}
\begin{cases}
  P_B + \rho_a g z_B = P_2 + \rho_a g z_2  \\
  P_1 + \rho g z_1 = P_2 + \rho g z_2   \\
  P_A + \rho_a g z_A = P_1 + \rho_a g z_1 \\
  \Delta P = P_B - P_A 
\end{cases}\)\( Si risolve il sistema lineare (come più piace). Ad
esempio, partendo dalla terza e inserendo nella seconda e nella prima i
risultati trovati: \)\(\begin{aligned}
 & P_1 = P_A + \rho_a g (z_A - z_1) \\
 & P_2 = P_A + \rho_a g (z_A - z_1) + \rho g (z_1 - z_2) \\
 & P_B = P_A + \rho_a g (z_A - z_1) + \rho g (z_1 - z_2) + \rho_a g (z_2 - z_B)
\end{aligned}\)\( E quindi, portando \)P\_A\( a sinistra:
\)\(\Delta P = -(\rho - \rho_a) g ( z_2-z_1) - \rho_a g (z_B - z_A) = -3909.8 Pa\)\$


\subsubsection{Osservazione.}
\label{\detokenize{polimi/fluidmechanics-ita/template/capitoli/01_statica/0206in:osservazione}}
\sphinxAtStartPar
Il sistema lineare
(\DUrole{xref,myst}{{[}eqn:stevino:underdet{]}}\{reference\sphinxhyphen{}type=»ref»
reference=»eqn:stevino:underdet»\}) è sotto determinato (se esiste una
soluzione, ne esistono infinite), essendo un sistema lineare di 4
equazioni in 5 incognite, \(P_1\), \(P_2\), \(P_A\), \(P_B\), \(\Delta P\). Il
sistema lineare può essere scritto usando il formalismo matriciale come
\(\underline{\underline{A}}\,\underline{x} = \underline{b}\) con
\$\(\underline{\underline{A}} = 
\begin{bmatrix}
  0 &  1 &  0 & -1 &  0 \\
  0 &  0 & -1 &  1 &  0 \\
 -1 &  1 &  1 &  0 &  0 \\
  1 & -1 &  1 &  0 &  0 \\
\end{bmatrix} , \quad
 \underline{x} = \begin{bmatrix}
  P_A \\ P_B \\ P_1 \\ P_2 \\ \Delta P
\end{bmatrix} , \quad
 \underline{b} = \begin{bmatrix}
  \rho_a g (h_2-h_B) \\ \rho g (h_1-h_2) \\ \rho_a g (h_A-h_1) \\ 0 
\end{bmatrix} \ .\)\( Poichè la matrice \)\textbackslash{}underline\{\textbackslash{}underline\{A\}\}\( ha
rango massimo (= 4), esiste una soluzione \)\textbackslash{}underline\{x\}\textasciicircum{}\sphinxstyleemphasis{\( del
problema, tale che
\)\textbackslash{}underline\{\textbackslash{}underline\{A\}\},\textbackslash{}underline\{x\}\textasciicircum{}} = \textbackslash{}underline\{b\}\(. Dal
teorema del rango, si sa che il numero delle colonne (= 5) di una
matrice è uguale alla dimensione del suo rango (= 4) e del suo nucleo
(quindi = 1). Il nucleo della matrice \)\textbackslash{}underline\{\textbackslash{}underline\{A\}\}\(, tutti
i vettori \)\textbackslash{}underline\{v\}\( t.c.
\)\textbackslash{}underline\{\textbackslash{}underline\{A\}\},\textbackslash{}underline\{v\} = \textbackslash{}underline\{0\}\(, è uno spazio
vettoriale di dimensione uno. Se \)\textbackslash{}underline\{x\}\textasciicircum{}\sphinxstyleemphasis{\( è soluzione del
sistema, allora anche tutti i vettori
\)\textbackslash{}underline\{x\}\textasciicircum{}} + a \textbackslash{}underline\{v\}\(, \)a \textbackslash{}in \textbackslash{}mathbb\{R\}\(, sono soluzione
del sistema, poichè
\)\textbackslash{}underline\{\textbackslash{}underline\{A\}\}(\textbackslash{}underline\{x\}\textasciicircum{}* + \textbackslash{}underline\{v\}) = \textbackslash{}underline\{\textbackslash{}underline\{A\}\},\textbackslash{}underline\{x\}\textasciicircum{}* + \textbackslash{}underline\{\textbackslash{}underline\{A\}\},\textbackslash{}underline\{v\} = \textbackslash{}underline\{b\} + \textbackslash{}underline\{0\}\(.
Si può dimostrare il nucleo di \)\textbackslash{}underline\{\textbackslash{}underline\{A\}\}\( è generato
dal vettore \)\textbackslash{}underline\{v\}=(1,1,1,1,0)\textasciicircum{}T\(. Quindi le infinite soluzioni
del problema hanno la forma \)\(\begin{bmatrix}
  P_A \\ P_B \\ P_1 \\ P_2 \\ \Delta P 
 \end{bmatrix} = 
 \begin{bmatrix}
  P_A^* \\ P_B^* \\ P_1^* \\ P_2^* \\ \Delta P 
 \end{bmatrix} + 
 \begin{bmatrix}
   a \\ a \\ a \\ a \\ 0
 \end{bmatrix} \ .\)\( Ora dovrebbe apparire chiaro come non sia possibile
determinare il valore assoluto delle pressioni \)P\_1\(, \)P\_2\(, \)P\_A\(,
\)P\_B\( solamente da una misura di pressione con un manometro
*differenziale*: questi valori sono noti a meno di una costante additiva
\)a\(, indeterminata. Al contrario, la differenza di due di questi valori,
come \)\textbackslash{}Delta P = P\_B \sphinxhyphen{} P\_A\(, è unica (e uguale al risultato ottenuto
nello svolgimento del problema): l'unicità di \)\textbackslash{}Delta P\( dipende dalla
forma dei vettori del nucleo di \)\textbackslash{}underline\{\textbackslash{}underline\{A\}\}\( che hanno
componente \)\textbackslash{}Delta P\$ nulla.

\sphinxstepscope


\chapter{Surface tension}
\label{\detokenize{polimi/fluidmechanics-ita/template/capitoli/02_tensSup/03teoria:surface-tension}}\label{\detokenize{polimi/fluidmechanics-ita/template/capitoli/02_tensSup/03teoria:fluid-mechanics-surface-tension}}\label{\detokenize{polimi/fluidmechanics-ita/template/capitoli/02_tensSup/03teoria::doc}}

\section{Legge di Young\sphinxhyphen{}Laplace.}
\label{\detokenize{polimi/fluidmechanics-ita/template/capitoli/02_tensSup/03teoria:legge-di-young-laplace}}\label{\detokenize{polimi/fluidmechanics-ita/template/capitoli/02_tensSup/03teoria:fluid-mechanics-surface-tension-young-laplace}}
\sphinxAtStartPar
La superficie di interfaccia tra due liquidi può essere modellata come
una membrana, una superficie bidimensionale all’interno della quale
agisce una forza per unità di lunghezza, tangente alla superficie
stessa. La forza per unità di spessore \(\gamma\) agente nella membrana
viene definità \sphinxstyleemphasis{tensione superificiale}. La legge di Young\sphinxhyphen{}Laplace lega
la tensione superficiale, il salto di pressione attraverso la superficie
di interfaccia e la curvatura della superficie stessa. Nel caso di
tensione superficiale costante, vale
\begin{equation*}
\begin{split}p_b - p_a = \gamma \displaystyle\left( \frac{1}{R_1} + \frac{1}{R_2} \right) = 2 \gamma H\end{split}
\end{equation*}
\sphinxAtStartPar
dove con \(R_1\) e \(R_2\) sono stati indicati i due raggi di curvatura
della superficie e con \(H\) si è indicata la curvatura media.

\sphinxAtStartPar
\sphinxincludegraphics{{polimi/fluidmechanics-ita/template/capitoli/02_tensSup/fig/laplaceYoung2Ddim}.eps}\{width=»95\%»\}


\section{Legge di Young\sphinxhyphen{}Laplace in due dimensioni}
\label{\detokenize{polimi/fluidmechanics-ita/template/capitoli/02_tensSup/03teoria:legge-di-young-laplace-in-due-dimensioni}}\label{\detokenize{polimi/fluidmechanics-ita/template/capitoli/02_tensSup/03teoria:fluid-mechanics-surface-tension-young-laplace-2d}}
\sphinxAtStartPar
Viene ricavata la legge di Young\sphinxhyphen{}Laplace in due dimensioni, scrivendo
l’equilibrio di un elemento di membrana (monodimensionale) soggetta agli
sforzi esercitati dai due fluidi su di essa e alla tensione superficiale
al suo interno. L’equazione vettoriale di equilibrio viene proiettata in
direzione normale e tangente alla superficie. La superficie nell’intorno
di un punto, viene approssimata come un arco infinitesimo di una
circonferenza, come in figura.

\sphinxAtStartPar
Si considera un elemento infinitesimo di superficie di dimensioni
\(\Delta x \sim R \Delta \theta\). Anche l’angolo \(\Delta \theta\) è
«piccolo» (\(\cos \Delta \theta \sim 1\),
\(\sin \Delta\theta \sim \Delta\theta\), la dimensione dell’elemento di
superficie è approssimabile con la sua proiezione su un piano normale a
\(\mathbf{\hat{n}}\), …). Con \(R\) viene indicato il raggio di curvatura
della superficie.

\sphinxAtStartPar
Si scrive l’equilibrio.
\$\(\mathbf{t_a} \Delta x + \mathbf{t_b} \Delta x - \mathbf{\gamma}(x) + \mathbf{\gamma}(x) + \Delta \mathbf{\gamma} = 0\)\$

\sphinxAtStartPar
Proiettando nelle direzioni normale e tangente alla superficie,
\begin{equation*}
\begin{split}\begin{aligned}
  ( {t_a}_n + {t_b}_n )\Delta x + \gamma \sin \frac{\Delta\theta}{2}
      + (\gamma + \Delta \gamma) \sin \frac{\Delta\theta}{2} = 0 \\
  ( {t_a}_t + {t_b}_t )\Delta x - \gamma \cos \frac{\Delta\theta}{2}
      + (\gamma + \Delta \gamma) \cos\frac{\Delta \theta}{2} = 0
 \end{aligned}\end{split}
\end{equation*}
\sphinxAtStartPar
Inserendo i valori approssimati di \(\sin \Delta \theta\) e
\(\cos \Delta \theta\), trascurando i termini di ordine superiore
(\(\Delta \gamma \Delta \theta\)):
\begin{equation*}
\begin{split}\begin{aligned}
  & ( {t_a}_n + {t_b}_n )\Delta x + 2 \gamma \frac{\Delta \theta}{2} = 0 \\
  & ( {t_a}_t + {t_b}_t )\Delta x + \Delta \gamma = 0
 \end{aligned}\end{split}
\end{equation*}
\sphinxAtStartPar
Se si può confondere la coordinata che descrive la superficie con la
coordinata \(x\), si può approssimare
\(\Delta \gamma \sim \frac{\partial \gamma}{\partial x} \Delta x\). Usando
la relazione \(\frac{\Delta x}{2} \sim R \frac{\Delta \theta}{2}\) e
semplificando l’elemento \(\Delta x\):
\begin{equation*}
\begin{split}\label{eqn:equil_young_laplace}
\begin{aligned}
 & ( {t_a}_n + {t_b}_n ) + \frac{\gamma}{R} = 0 \\
 & ( {t_a}_t + {t_b}_t ) + \frac{\partial \gamma}{\partial x} = 0
\end{aligned}\end{split}
\end{equation*}
\sphinxAtStartPar
Nel caso in cui si consideri un problema di statica, lo sforzo \sphinxstylestrong{sul}
fluido è dovuto solo al contributo di pressione, che agisce in direzione
normale alla superficie: \(\mathbf{t}_a = -P_a \mathbf{\hat{n}_a}\),
\(\mathbf{t}_b = -P_b \mathbf{\hat{n}_b}\). Lo sforzo che il fluido esercita sulla
superficie di interfaccia è uguale in modulo e opposto in direzione. Le
due normali sono tra di loro opposte: si sceglie di definire la normale
\(\mathbf{\hat{n}} = \mathbf{\hat{n}_a} = -\mathbf{\hat{n}_b}\). Di conseguenza, le
componenti degli sforzi agenti sulla superficie di interfaccia,
proiettati lungo \(\mathbf{\hat{n}}\) e un versore tangente sono:
\({{t_a}_n} = P_a\), \({{t_b}_n} = - P_b\), \({{t_a}_t} = 0\),
\({{t_b}_n} = 0\). Se \(\gamma\) è costante (la tensione superficiale può
avere gradienti non nulli a causa di gradienti di temperatura o di
concentrazione), l’equilibrio in direzione tangente è identicamente
soddisfatto.
\begin{equation*}
\begin{split}P_a - P_b + \frac{\gamma}{R} = 0 \qquad 
  \Rightarrow  \qquad 
   P_b - P_a = \frac{\gamma}{R}\end{split}
\end{equation*}

\subsection{Estensione al caso 3D.}
\label{\detokenize{polimi/fluidmechanics-ita/template/capitoli/02_tensSup/03teoria:estensione-al-caso-3d}}
\sphinxAtStartPar
Per estendere la dimostrazione al caso 3D, nel quale la superficie è 2D,
si procede in modo analogo a quanto nel paragrafo precedente. Va
considerata la curvatura di una superficie e non di una curva (esistono
due raggi di curvatura), …Un utile primo riferimento di \sphinxstyleemphasis{geometria
differenziale} di curve e superfici, è disponibile in rete seguendo il
collegamento

\sphinxAtStartPar
\sphinxhref{http://alpha.math.uga.edu/~shifrin/ShifrinDiffGeo.pdf}{Differential Geometry,
Shiffrin}.

\sphinxAtStartPar
L’esistenza della tensione superificiale spiega il fenomeni della
capillarità, l’esistenza dei menischi formati dalla superficie di
separazione di due fluidi, il galleggiamento di insetti, graffette…
sull’acqua, la formazione di superfici «minimali» di sapone, la
bagnabilità delle superfici e la rottura di getti di piccolo diametro e
la formazione di gocce. Infine, può essere utilizzata anche come mezzo
non convenzionale di propulsione per barchette di carta

\sphinxAtStartPar
\sphinxhref{https://www.youtube.com/watch?v=Oz54Auev9eU}{Boat without a motor \sphinxhyphen{} Marangoni
effect}

\sphinxstepscope


\section{Exercises}
\label{\detokenize{polimi/fluidmechanics-ita/template/capitoli/02_tensSup/exercises:exercises}}\label{\detokenize{polimi/fluidmechanics-ita/template/capitoli/02_tensSup/exercises:fluid-mechanics-surface-tension-exercises}}\label{\detokenize{polimi/fluidmechanics-ita/template/capitoli/02_tensSup/exercises::doc}}
\sphinxstepscope


\subsection{Exercise 2.1}
\label{\detokenize{polimi/fluidmechanics-ita/template/capitoli/02_tensSup/0301in:exercise-2-1}}\label{\detokenize{polimi/fluidmechanics-ita/template/capitoli/02_tensSup/0301in:fluid-mechanics-surface-tension-ex-01}}\label{\detokenize{polimi/fluidmechanics-ita/template/capitoli/02_tensSup/0301in::doc}}
\sphinxAtStartPar
+:———————————:+:———————————:+
| Sia \(\theta\) l’angolo di contatto | \sphinxincludegraphics{{polimi/fluidmechanics-ita/template/capitoli/02_tensSup/fig/cap01}.eps}\{width=» |
| all’interfaccia tra aria, liquido | 80\%»\}                             |
| e solido; sia \(\gamma\) la         |                                   |
| tensione superficiale tra aria e  |                                   |
| liquido; sia \(\rho\) la densità    |                                   |
| del liquido. Determinare          |                                   |
| l’altezza \(h\) dal liquido in una  |                                   |
| colonnina cilindrica di raggio    |                                   |
| \(r = 0.5 \ mm\) rispetto al        |                                   |
| livello nella vasca. Calcolare    |                                   |
| poi la pressione all’interno      |                                   |
| della colonnina. (Si può          |                                   |
| considerare valida                |                                   |
| l’approssimazione che la          |                                   |
| pressione agente sulla vasca e    |                                   |
| sulla superficie superiore del    |                                   |
| liquido all’interno della         |                                   |
| colonnina sia uguale).            |                                   |
|                                   |                                   |
| Si considerino condizioni         |                                   |
| termodinamiche e materiale della  |                                   |
| colonnina tali che: se il liquido |                                   |
| è acqua: \(\rho = 999 \ kg/m^3\),   |                                   |
| \(\theta={1}\degree\),              |                                   |
| \(\gamma=0.073 N/m\). se il liquido |                                   |
| è mercurio:                       |                                   |
| \(\rho = 13579 \ kg/m^3\),          |                                   |
| \(\theta={140}\degree\),            |                                   |
| \(\gamma=0.559 N/m\).               |                                   |
| (\(h_{H_2O} = 2.97 \ cm\),          |                                   |
| \(P_{H_20} - P_0 =  - 291.95 \ Pa\) |                                   |
| ;                                 |                                   |
| \(h_{Hg} = -1.28 \ cm\),            |                                   |
| \(P_{Hg} - P_0 =  1712.87 \ Pa\))   |                                   |
+———————————–+———————————–+

\sphinxAtStartPar
Tensione superficiale. Angolo di contatto. Capillarità. Menisco.

\sphinxAtStartPar
Scrivendo l’equilibrio per il volume di fluido nel capillare si trova
l’altezza \(h\). Successivamente si trova la \(p\) usando la legge di
Stevino. Infine si fanno osservazioni su angolo di contatto, menisco e
salto di pressione all’interfaccia.
\begin{itemize}
\item {} 
\sphinxAtStartPar
Si scrive l’equilibrio del volume di fluido. Il problema è di
statica. Le forze agenti sono la forza dovuta alla tensione
superficiale (che agisce sul perimetro della superficie superiore) e
la forza peso, poichè per ipotesi la pressione agente sulla
superficie superiore è uguale alla pressione ambiente \(P_a\); e
quindi??? Perchè la componente verticale della risultante dovuta
alla pressione esterna è zero??? Vedere immagine…).
\$\(F_{\gamma} = F_g \quad \Rightarrow \quad 2\pi r \gamma  \cos \theta = \pi r^2 h \rho g\)\(
E quindi: \)\(h = \frac{2 \gamma \cos \theta}{\rho g r}
  \quad \Rightarrow \quad
  \begin{cases}
    h_{H_20} = 2.97 \ cm \\
    h_{Hg} = -1.28 \ cm
  \end{cases}\)\( *Commenti sul risultato.* L'effetto della
capillarità è più evidente per tubi stretti (proporzionalità con
\)1/r\(). La quota \)h\( può assumere sia valori positivi, sia valori
negativi, in base al valore dell'angolo di contatto: \)h \textbackslash{}le 0\(, per
\)\textbackslash{}theta \textbackslash{}ge \textbackslash{}pi/2\$.

\item {} 
\sphinxAtStartPar
Si calcola la pressione nel fluido in cima alla colonnina sfruttando
la legge di Stevino.
\$\(P = P_0 - \rho g h = P_0 - \frac{2 \gamma \cos \theta}{r}
  \quad \Rightarrow \quad
  \begin{cases}
    P_{H_20} - P_0 =  - 291.95 \ Pa \\
    P_{Hg}   - P_0 =  1712.87 \ Pa
  \end{cases}\)\( *Commenti sul risultato.* \)P\sphinxhyphen{}P\_0 \textbackslash{}le 0\(, per
\)\textbackslash{}theta \textbackslash{}le \textbackslash{}pi/2\(. Al contrario \)P\sphinxhyphen{}P\_0 \textbackslash{}ge 0\(, per
\)\textbackslash{}theta \textbackslash{}ge \textbackslash{}pi/2\$. Questi risultati sono compatibili (meno male)
con le relazioni tra curvatura (stretta parente del menisco e
dell’angolo di contatto) e il salto di pressione.

\sphinxAtStartPar
\(\theta \le \pi/2\)   \(h \ge 0\)   \(P \le P_a\)


\bigskip\hrule\bigskip


\sphinxAtStartPar
\(\theta \ge \pi/2\)   \(h \le 0\)   \(P \ge P_a\)

\end{itemize}

\sphinxstepscope


\subsection{Exercise 2.2}
\label{\detokenize{polimi/fluidmechanics-ita/template/capitoli/02_tensSup/0302in:exercise-2-2}}\label{\detokenize{polimi/fluidmechanics-ita/template/capitoli/02_tensSup/0302in:fluid-mechanics-surface-tension-ex-02}}\label{\detokenize{polimi/fluidmechanics-ita/template/capitoli/02_tensSup/0302in::doc}}

\bigskip\hrule\bigskip


\sphinxAtStartPar
Due lamine piane uguali parallele sono separate da una distanza \(d\). Tra le lamine è presente un sottile strato di liquido. Sono note l’area della superficie \(A\) e il perimetro \(L\) delle due lamine, la pressione ambiente \(p_a\), la tensione superficiale del liquido \(\gamma\) e l’angolo di contatto \(\theta\). Si chiede di determinare la componente perpendicolare alle lamine della forza agente su ciascuna delle due lamine.   \sphinxincludegraphics{{polimi/fluidmechanics-ita/template/capitoli/02_tensSup/fig/Plates4}}\{width=»85\%»\}


\bigskip\hrule\bigskip


\sphinxAtStartPar
Tensione superficiale. Angolo di contatto.

\sphinxAtStartPar
La condizione descritta nell’esercizio è una condizione equilibrio. La
forza agente su una lamina è dovuta a due fenomeni: la tensione
superficiale sul perimetro del fluido e la differenza di pressione tra
fluido e ambiente. Si consideri positiva la forza se è una forza di
attrazione.
\begin{equation*}
\begin{split}F = F_\gamma + F_p\end{split}
\end{equation*}\begin{itemize}
\item {} 
\sphinxAtStartPar
Calcolo di \(F_\gamma\). \$\(F_\gamma = \gamma L \sin\gamma\)\$

\item {} 
\sphinxAtStartPar
Calcolo di \(F_p\). Il salto di pressione viene calcolato scrivendo
l’equilibrio all’interfaccia. \$\(F_p = (p_a - p) A\)\( con:
\)\((p_a - p) d = 2 \gamma \cos \theta\)\$

\item {} 
\sphinxAtStartPar
La componente totale richiesta risulta quindi:
\$\(F = \frac{2 \gamma A \cos \theta}{d} + L \gamma \sin \theta\)\$

\sphinxAtStartPar
\sphinxincludegraphics{{polimi/fluidmechanics-ita/template/capitoli/02_tensSup/fig/sup01}.eps}\{width=»50\%»\}

\end{itemize}

\sphinxstepscope


\subsection{Exercise 2.3}
\label{\detokenize{polimi/fluidmechanics-ita/template/capitoli/02_tensSup/0303in:exercise-2-3}}\label{\detokenize{polimi/fluidmechanics-ita/template/capitoli/02_tensSup/0303in:fluid-mechanics-surface-tension-ex-03}}\label{\detokenize{polimi/fluidmechanics-ita/template/capitoli/02_tensSup/0303in::doc}}
\sphinxAtStartPar
Si vuole calcolare la forma del pelo libero tra aria ed acqua, di
densità \(\rho\), nelle vicinanze di una parete piana infinita, conoscendo
la tensione superficiale \(\gamma\) e l’angolo di contatto a parete
\(\theta\). La pressione dell’aria è uniforme e uguale a \(P_a\).

\sphinxAtStartPar
\sphinxincludegraphics{{polimi/fluidmechanics-ita/template/capitoli/02_tensSup/fig/freeSurfaceShape}}\{width=»85\%»\}

\sphinxAtStartPar
Poiché si studia il problema nelle vicinanze di una parete piana
infinita, è lecito assumere che la soluzione non dipenda dalla
coordinata che descrive la lunghezza della parete, la coordinata \(z\)
facendo riferimento al disegno (TODO). L’equazione di Young\sphinxhyphen{}Laplace per
una superficie bidmensionale in uno spazio tridimensionale,
\$\(P_1 - P_2 = \gamma \left[ \dfrac{1}{R_1} + \dfrac{1}{R_2} \right] \ ,\)\(
si riduce al caso di una superficie monodimensionale in uno spazio
bidimensionale, la superficie di contatto è piatta nella direzione \)z\(.
Se \)R:=R\_1\( è il raggio della superficie nei piani paralleli al piano
\)x\sphinxhyphen{}y\( e \)R\_2\( il raggio della superficie nei piani paralleli al piano
\)z\sphinxhyphen{}y\(, l'equazione di Young-Laplace si riduce a
\)\(P_1 - P_2 = \dfrac{\gamma}{R} =: \gamma \, k  \ ,\)\( poiché
\)R\_2 \textbackslash{}rightarrow \textbackslash{}infty\(, poiché la quota della superficie di contatto
non varia al variare di \)z\(, mantenendo \)x\( fissato, e quindi il raggio
di curvatura della superficie in quella direzione è infinito. É stata
introdotta la definizione della curvatura \)k := 1/R\$.

\sphinxAtStartPar
La pressione nell’aria è costante e uguale al valore della pressione
ambiente \(P_2 = P_a\). Si può calcolare la pressione \(P_1(x)\) dell’acqua
a contatto con la superficie utilizzando la legge di Stevino,
\$\(P_1(x) - P_a = \rho g y(x) \ ,\)\( essendo \)y(x)\( la quota della
superficie, rispetto alla quota di riferimento \)y=0\(, scelta come la
quota alla quale la pressione dell'acqua è uguale alla pressione \)P\_a\(.
Utilizzando l'espressione di \)P\_1\( e \)P\_2\(, l'equazione di Laplace-Young
diventa \)\(\rho g y(x) = \gamma k(x) \ .\)\( Ricordando che la curvatura di
una superficie rappresentata dalla funzione \)y(x)\( può essere espressa
come (TODO: aggiungere i dettagli ?),
\)\(k(x) = \dfrac{y''(x)}{\left( 1 + y'(x)^2 \right)^{3/2}} \ ,\)\( si
ricava il problema differenziale che descrive la forma della superficie
di contatto, \)\(\begin{cases}
 \rho g y(x) - \dfrac{y''(x)}{\left( 1 + y'(x)^2 \right)^{3/2}} = 0 \\
 y'(x=0) = -\dfrac{1}{\tan\theta} \\
 y(x\rightarrow \infty) \rightarrow 0
\end{cases}\)\( con l'equazione di Young-Laplace accompagnata dalla
condizione al contorno a parete, in \)x=0\(, che lega la derivata della
superficie all'angolo di contatto, e dalla condizione all'infinito che
identifica la quota del pelo libero lontano dalla parete come la quota
alla quale la pressione è uguale alla pressione ambiente. Definendo la
costante \)a\textasciicircum{}2:=\textbackslash{}frac\{2\textbackslash{}gamma\}\{\textbackslash{}rho g\}\( e integrando una volta
l'equazione differenziale si ottiene
\)\(\dfrac{y^2}{a^2} + \dfrac{1}{(1 + y'(x)^2)^{1/2}} = A \ ,\)\( dove \)A\(
rappresenta la costante di integrazione. Affinchè la quota della
superficie tenda alla quota di riferimento a grande distanza della
parete, \)y(x\textbackslash{}rightarrow\textbackslash{}infty) \textbackslash{}rightarrow 0\(, "con una sensata
regolarità", è necessario che anche la sua pendenza si annulli,
\)y”(x\textbackslash{}rightarrow\textbackslash{}infty) \textbackslash{}rightarrow 0\(. Questa condizione impone il
valore della costante di integrazione, \)A = 1\(. Integrando nuovamente
l'equazione,
\)\(\dfrac{y^2}{a^2} + \dfrac{1}{(1 + y'(x)^2)^{1/2}} = 1 \ ,\)\( si ottiene
la soluzione del problema (TODO: aggiungere i dettagli),
\)\(\dfrac{1}{\sqrt{2 a^2 - y(x)^2}} +
 \dfrac{a}{\sqrt{2}} \, \text{Ch}^{-1} \left( \dfrac{\sqrt{2} a}{y(x)} \right) = x + B \ ,\)\(
con la costante di integrazione \)B\$ da calcolare utilizzando la
condizione al contorno a parete.

\sphinxAtStartPar
\sphinxstylestrong{Osservazione.} Questo problema fornisce un esempio di calcolo della
forma della superficie di contatto tra due fluidi. Nonostante il
problema studiato sia uno dei più semplici che si possano immaginare, la
sua soluzione analitica richiede già un notevole impegno. Coloro che
nutrono passione per l’argomento, sono invitati a calcolare la forma
dell’interfaccia tra acqua e aria in un dominio delimitato da due pareti
verticali, risolvendo numericamente (esiste una soluzione analitica
anche per questo problema, ma risulta ancora più «criptica» di quella
ricavata per il problema con una sola parete) il problema differenziale
non lineare, \$\(\begin{cases}
 \rho g y(x) - \dfrac{y''(x)}{\left( 1 + y'(x)^2 \right)^{3/2}} = 0 \quad , \qquad x \in [0,\,L] \\
 y'(x=0) = -\frac{1}{\tan\theta} \\
 y'(x=L) =  \frac{1}{\tan\theta} \\
\end{cases}\)\$

\sphinxstepscope


\chapter{Kinematics}
\label{\detokenize{polimi/fluidmechanics-ita/template/capitoli/03_cinematica/12teoria:kinematics}}\label{\detokenize{polimi/fluidmechanics-ita/template/capitoli/03_cinematica/12teoria:fluid-mechanics-kinematics}}\label{\detokenize{polimi/fluidmechanics-ita/template/capitoli/03_cinematica/12teoria::doc}}
\sphinxAtStartPar
La cinematica è la parte della meccanica che studia il moto di sistemi,
indipendentemente dalle cause che lo generano, a differenza della
dinamica. Prima di ricavare le equazioni che descrivono la dinamica di
un fluido, sembra quindi opportuno concentrarsi sulla sua cinematica.

\sphinxAtStartPar
La cinematica e la dinamica dei mezzi continui, come ad esempio i solidi
o i fluidi, possono essere descritte con un approccio lagrangiano o
euleriano. La \sphinxstylestrong{descrizione lagrangiana}, utilizzata spesso in
meccanica dei solidi, consiste nel seguire nello spazio il moto delle
singole particelle del mezzo continuo. La \sphinxstylestrong{descrizione euleriana},
utilizzata spesso in meccanica dei fluidi, consiste nel descrivere
l’evoluzione del mezzo continuo utilizzando come variabili indipendenti
sia la variabile spaziale \(\mathbf{r}\) sia la variabile temporale \(t\).


\section{Descrizione integrale lagrangiana ed euleriana}
\label{\detokenize{polimi/fluidmechanics-ita/template/capitoli/03_cinematica/12teoria:descrizione-integrale-lagrangiana-ed-euleriana}}\label{\detokenize{polimi/fluidmechanics-ita/template/capitoli/03_cinematica/12teoria:fluid-mechanics-kinematics-integral-lagrange-euler}}
\sphinxAtStartPar
In una descrizione \sphinxstyleemphasis{integrale} del fenomeno, l’approccio lagrangiano
segue l’evoluzione di un \sphinxstylestrong{volume materiale}, i cui punti si muovono in
maniera solidale con il mezzo continuo. In un approccio euleriano invece
viene introdotto un \sphinxstylestrong{volume di controllo}, fisso nello spazio, e i
flussi delle quantità meccaniche (massa, quantità di moto, energia, …)
contribuiscono ai bilancio delle quantità meccaniche relative al volume
di controllo considerato. Queste due descrizioni sono casi particolari
di un approccio generale al problema, definito \sphinxstyleemphasis{ALE} (arbitrario
lagrangiano\sphinxhyphen{}euleriano), che descrive l’evoluzione di un volume in moto
arbitrario. Le tre diverse descrizioni del problema possono essere messe
in relazione tra di loro, tramite le formule di Leibniz, che forniscono
l’espressione della derivata temporale di integrali su domini dipendenti
dal tempo. Si riporta qui, senza dimostrazione, il \sphinxstylestrong{teorema del
trasporto di Reynolds}
\begin{equation*}
\begin{split}\dfrac{d}{d t} \int_{V(t)} f = \int_{V(t)} \dfrac{\partial f}{\partial t} +
  \oint_{S(t)} f\mathbf{v} \cdot \mathbf{\hat{n}} \ ,\end{split}
\end{equation*}
\sphinxAtStartPar
che fornisce l’espressione della derivata temporale dell’integrale della
funzione \(f(\mathbf{x},t)\) (che può essere scalare, vettoriale o in generale
tensoriale) nel volume mobile \(V(t) \ni \mathbf{x}\), la cui frontiera \(S(t)\)
si muove con velocità \(\mathbf{v}(\mathbf{x}_s,t)\), \(\mathbf{x}_s \in S(t)\). La
normale \(\mathbf{\hat{n}}\) alla superficie \(S(t)\) è uscente dal volume
\(V(t)\). Si rimanda all’appendice «Richiami di analisi» per la
dimostrazione del teorema e per le formule della derivata temporale di
flussi e circuitazioni su domini dipendenti dal tempo. Siano ora
\begin{itemize}
\item {} 
\sphinxAtStartPar
\(V(t)\) un volume materiale, la cui frontiera si muove con la
velocità del fluido \(\mathbf{v}=\mathbf{u}\)

\item {} 
\sphinxAtStartPar
\(V_c\) un volume di controllo, la cui frontiera è fissa nello spazio,
\(\mathbf{v}=\mathbf{0}\)

\item {} 
\sphinxAtStartPar
\(v(t)\) un volume in moto arbitrario, la cui frontiera si muove con
velocità generica \(\mathbf{v}\).

\end{itemize}

\sphinxAtStartPar
Come si vedrà nel capitolo sui «Bilanci integrali», il bilancio
integrale di una quantità meccanica \(f\) in un volume materiale \(V(t)\)
descrive la variazione nel tempo dell’integrale \(\int_{V(t)} f\). Il
teorema di Reynolds applicato all’integrale svolto su un volume
materiale \(V(t)\) e all’integrale svolto sul volume in moto generico
\(v(t)\), coincidente con \(V(t)\) all’istante di tempo \(t\) considerato,
\begin{equation*}
\begin{split}\begin{aligned}
  \dfrac{d}{d t} \int_{V(t)} f & = \int_{V(t)} \dfrac{\partial f}{\partial t} +
  \oint_{S(t)} f\mathbf{u} \cdot \mathbf{\hat{n}}  \\
  \dfrac{d}{d t} \int_{v(t)\equiv V(t)} f & = \int_{v(t)\equiv V(t)} \dfrac{\partial f}{\partial t} +
  \oint_{s(t)\equiv S(t)} f\mathbf{v} \cdot \mathbf{\hat{n}}  \ , \\
\end{aligned}\end{split}
\end{equation*}
\sphinxAtStartPar
permette di ricavare il legame tra la descrizione
lagrangiana e una descrizione arbitraria del problema. Confrontando le
ultime due espressioni, si ottiene
\begin{equation*}
\begin{split}\dfrac{d}{d t} \int_{V(t)} f = \dfrac{d}{d t} \int_{v(t)\equiv V(t)} f +
 \oint_{s(t)\equiv S(t)} f (\mathbf{u} - \mathbf{v}) \cdot \mathbf{\hat{n}} \ .\end{split}
\end{equation*}
\sphinxAtStartPar
Dalla formula scritta per il volume arbitrario \(v(t)\), si ricava il
legame tra a descrizione lagrangiana e la descrizione euleriana del
problema, considerando il volume arbitrario coincidente con un volume di
controllo \(V_c\) fisso, per il quale \(\mathbf{v}=\mathbf{0}\),
\begin{equation*}
\begin{split}\dfrac{d}{d t} \int_{V(t)} f = \dfrac{d}{d t} \int_{V_c\equiv V(t)} f +
 \oint_{S_c\equiv S(t)} f \mathbf{u} \cdot \mathbf{\hat{n}} \ .\end{split}
\end{equation*}

\section{Descrizione puntuale lagrangiana ed euleriana}
\label{\detokenize{polimi/fluidmechanics-ita/template/capitoli/03_cinematica/12teoria:descrizione-puntuale-lagrangiana-ed-euleriana}}\label{\detokenize{polimi/fluidmechanics-ita/template/capitoli/03_cinematica/12teoria:fluid-mechanics-kinematics-differential-lagrange-euler}}
\sphinxAtStartPar
In una descrizione \sphinxstyleemphasis{puntuale} del fenomeno, vengono introdotti due
sistemi di coordinate: uno è solidale con il mezzo continuo dipendente
dal tempo, mentre l’altro è fisso. Si può pensare al sistema di
riferimento solidale con il continuo come un” «etichetta» che viene
applicata a ogni \sphinxstylestrong{punto materiale} del mezzo continuo stesso. Un
sistema di riferimento fisso, invece, è indipendente dal moto del mezzo
continuo, come ad esempio un sistema di coordinate cartesiane, la cui
origine e i cui assi sono fissi nel tempo. Mentre il mezzo continuo
evolve nel tempo (trasla, ruota, si deforma …), un punto materiale ha
coordinate costanti \(\mathbf{x_0}\) rispetto al sistema di riferimento
«solidale al volume», cioè che si muove e si deforma insieme al volume:
questa coordinata, detta lagrangiana, può essere pensata come
l“«etichetta» assegnata al punto materiale del continuo. Le coordinate
euleriane \(\mathbf{x}(\mathbf{x_0},t)\) del punto materiale con coordinate
lagrangiane \(\mathbf{x_0}\), ne descrivono il moto nel sistema di riferimento
fisso e in generale sono una funzione del tempo

\sphinxAtStartPar
Il sistema di riferimento solidale al corpo dipende dal tempo, mentre le
coordinate lagrangiane \(\mathbf{x_0}\) di un punto materiale sono costanti.
Il sistema di riferimento fisso è indipendente dal tempo, mentre le
coordinate euleriane \(\mathbf{x}\) di un punto materiale del volume (quindi
con \(\mathbf{x_0}\) costante) sono dipendenti dal tempo.

\sphinxAtStartPar
Assumendo che all’istante \(t=0\) i due sistemi di coordinate coincidano,
e che quindi coincidano anche le coordinate euleriane e lagrangiane
\(\mathbf{x}(\mathbf{x_0},0) = \mathbf{x_0}\), le coordinate lagrangiane \(\mathbf{x_0}\)
rappresentano la configurazione (iniziale) di riferimento della
configurazione attuale \(\mathbf{x}(\mathbf{x_0},t)\). La trasformazione
\(\mathbf{x}(\mathbf{x_0},t)\) descrive l’evoluzione nel tempo \(t\) dei punti
\(\mathbf{x}(0) = \mathbf{x_0}\) appartenenti al volume \(V_0 = V(0)\), all’istante
iniziale. La velocità \(\mathbf{u}(\mathbf{x},t)\) del mezzo continuo nel punto
\(\mathbf{x}(\mathbf{x_0},t)\), per definizione di punto materiale, coincide con
la velocità \(\mathbf{u_0}(\mathbf{x_0},t)\) del punto etichettato con \(\mathbf{x_0}\):
questa è la derivata nel tempo della sua posizione \(\mathbf{x}\), cioè con la
derivata nel tempo della mappa \(\mathbf{x}(\mathbf{x_0},t)\) a coordinata
lagrangiana (che identifica la particella) costante,
\begin{equation*}
\begin{split}\mathbf{u_0}(\mathbf{x_0},t) = \dfrac{\partial \mathbf{x}}{\partial t}\bigg|_{\mathbf{x_0}}(\mathbf{x_0},t) =: \dfrac{d \mathbf{x}}{d t}(\mathbf{x_0},t) =: \dfrac{D\mathbf{x}}{D t}(\mathbf{x_0},t) \ ,\end{split}
\end{equation*}
\sphinxAtStartPar
dove è stato introdotto il simbolo \(D/Dt\) di \sphinxstylestrong{derivata materiale} che
rappresenta l’evoluzione della quantità alla quale è applicata, seguendo
il moto del mezzo continuo: la derivata materiale rappresenta la
variazione nel tempo della quantità «sentita» dalle singole particelle
materiali. Nella descrizione euleriana del problema, i campi sono
funzioni delle variabili indipendenti spazio \(\mathbf{x}\) e tempo \(t\). Data
una funzione \(f(\mathbf{x},t)\) (scalare, vettoriale, tensoriale), viene
indicata con
\begin{equation*}
\begin{split}\dfrac{\partial f}{\partial t} = \dfrac{\partial f}{\partial t}\bigg|_{\mathbf{x}}(\mathbf{x},t) \ ,\end{split}
\end{equation*}
\sphinxAtStartPar
la derivata parziale rispetto al tempo, che rappresenta la variazione
della quantità \(f(\mathbf{x},t)\) nel punto fisso \(\mathbf{x}\) dello spazio, che
coordinata euleriana costante.

\sphinxAtStartPar
É possibile trovare il legame tra le due derivate utilizzando la \sphinxstyleemphasis{regola
di derivazione di funzioni composte} e la funzione \(\mathbf{x}(\mathbf{x_0},t)\)
che descrive il moto dei punti materiali del sistema. Data una funzione
\(f(\mathbf{x},t)\) (rappresentazione euleriana), viene definita
\(f_0(\mathbf{x_0},t)\) come la funzione composta \(f_0 = f \circ \mathbf{x}\)
(descrizione lagrangiana). Ipotizzando poi che si possano esprimere le
coordinate lagrangiane come funzione di quelle euleriane,
\(\mathbf{x_0}(\mathbf{x},t)\), è possibile scrivere
\$\(f(\mathbf{x},t) = f(\mathbf{x}(\mathbf{x_0},t),t) = f_0(\mathbf{x}_0,t) = f_0(\mathbf{x_0}(\mathbf{x},t),t) \ .\)\$
Utilizzando la regola di derivazione per le funzioni composte, si
ottiene il legame cercato,
\begin{equation*}
\begin{split}\label{eqn:cin:lagr-eul}
\begin{aligned}
 \dfrac{D f}{D t}(\mathbf{x},t) & = \dfrac{\partial f}{\partial t}\bigg|_{\mathbf{x_0}} = \dfrac{\partial}{\partial t}\bigg|_{\mathbf{x_0}} f(\mathbf{x}(\mathbf{x_0},t),t) =  \\ 
  & = \dfrac{\partial f}{\partial \mathbf{x}}\bigg|_{t} \cdot \dfrac{\partial \mathbf{x}}{\partial t}\bigg|_{\mathbf{x_0}} 
  + \dfrac{\partial f}{\partial t}\bigg|_{\mathbf{x}} = 
  \dfrac{\partial f}{\partial t} +  
  \dfrac{\partial x_i}{\partial t}\bigg|_{\mathbf{x_0}} \dfrac{\partial f}{\partial x_i}\bigg|_{t}  = 
  \dfrac{\partial f}{\partial t} + \mathbf{u} \cdot \mathbf{\nabla} f \ ,
\end{aligned}\end{split}
\end{equation*}
\sphinxAtStartPar
dove si è indicato con
\(\mathbf{u}(\mathbf{x},t) = \dfrac{\partial \mathbf{x}}{\partial t}\bigg|_{\mathbf{x_0}} (\mathbf{x_0}(\mathbf{x},t),t)\)
il campo di velocità riferito a una descrizione euleriana del problema e
si è riconosciuto l’operatore \(\mathbf{\nabla}\) nell’ultimo passaggio.
Infine è possibile «rimuovere» la funzione \(f\) per ottenere la relazione
tra la forma delle due derivate, valida per funzioni scalari,
vettoriali, tensoriali,
\begin{equation*}
\begin{split}\label{eqn:cin:lagr-eul-2}
 \dfrac{D \rule{1.5ex}{.4pt}}{D t} := \dfrac{d \rule{1.5ex}{.4pt}}{d t} := \dfrac{\partial \rule{1.5ex}{.4pt}}{\partial t}\bigg|_{\mathbf{x_0}} = \dfrac{\partial \rule{1.5ex}{.4pt}}{\partial t} + \mathbf{u} \cdot \mathbf{\nabla} \rule{1.5ex}{.4pt} \ .\end{split}
\end{equation*}
\sphinxAtStartPar
Come esempio, applichiamo la regola
(\DUrole{xref,myst}{{[}eqn:cin:lagr\sphinxhyphen{}eul{]}}\{reference\sphinxhyphen{}type=»ref»
reference=»eqn:cin:lagr\sphinxhyphen{}eul»\}) per ricavare la forma euleriana e
lagrangiana del campo di velocità e di accelerazione delle particelle
del continuo. Il campo di velocità \(\mathbf{u}(\mathbf{x},t)\) si ottiene dalla
derivata materiale della trasformazione \(\mathbf{x}(x_0,t)\),
\begin{equation*}
\begin{split}\mathbf{u}(\mathbf{x},t) = \dfrac{D \mathbf{x}}{D t} = \underbrace{\dfrac{\partial \mathbf{x}}{\partial t}\bigg|_{\mathbf{x}} }_{=0} + \mathbf{u_0}(\mathbf{x_0},t) \cdot \underbrace{ \mathbf{\nabla} \mathbf{x} }_{=\mathbb{I}} =
 \mathbf{u_0}(\mathbf{x_0},t) \ .\end{split}
\end{equation*}
\sphinxAtStartPar
In questo caso, non è stato ottenuto nulla
di nuovo. Il campo di accelerazione nella descrizione euleriana del
fenomeno viene ottenuto calcolando l’accelerazione delle particelle
materiali con la derivata materiale alla velocità. Per componenti,
l’accelerazione della particella materiale identificata con \(\mathbf{x_0}\) è
\begin{equation*}
\begin{split}a_{i}(\mathbf{x},t) = \dfrac{D u_{i}}{D t} = 
 \dfrac{\partial u_i}{\partial t} + u_{k} \dfrac{\partial u_i}{\partial x_k} \ .\end{split}
\end{equation*}
\sphinxAtStartPar
Introducendo l’operatore advettivo \(\mathbf{v}\cdot \mathbf{\nabla}\), è
possibile scrivere il campo di accelerazione (che comparirà nel bilancio
della quantità di moto) in forma vettoriale
\begin{equation*}
\begin{split}\mathbf{a}(\mathbf{x},t) = \dfrac{D \mathbf{u}}{D t}(\mathbf{x},t) = \dfrac{\partial \mathbf{u}}{\partial t}(\mathbf{x},t) + (\mathbf{u}(\mathbf{x},t) \cdot \mathbf{\nabla}) \mathbf{u}(\mathbf{x},t) \ ,\end{split}
\end{equation*}
\sphinxAtStartPar
dove sono stati esplicitati gli argomenti \((\mathbf{x},t)\) delle funzioni,
per evidenziare la rappresentazione euleriana.

\sphinxAtStartPar
Una volta compresa la differenza tra le due descrizioni del problema,
non è necessario esprimere in maniera esplicita gli argomenti delle
funzioni. Da qui in avanti, verrà privilegiata una descrizione
euleriana, per campi, del problema.

\sphinxAtStartPar
In alcuni casi, come ad esempio problemi che riguardano lo studio di
correnti attorno a corpi mobili, può essere conveniente utilizzare una
rappresentazione arbitraria del problema, descrivendo il fenomeno
seguendo l’evoluzione delle grandezza meccaniche su punti, «etichettati»
dalla coordinata arbitraria \(\mathbf{\chi}\), il cui moto è descritto in
coordinate euleriane dalla funzione \(\mathbf{x}(\mathbf{\chi},t)\). Seguendo lo
stesso procedimento svolto per le particelle materiali, la velocità
\(\mathbf{v}\) di questi punti in moto arbitrario è uguale alla derivata
parziale
\begin{equation*}
\begin{split}\mathbf{v} = \dfrac{\partial \mathbf{x}}{\partial t} \bigg|_{\mathbf{\chi}} \ ,\end{split}
\end{equation*}
\sphinxAtStartPar
svolta a coordinata \(\mathbf{\chi}\) costante. Ancora seguendo lo stesso
procedimento svolto in precedenza, è possibile ricavare la relazione tra
la rappresentazione arbitraria e quella euleriana,
\begin{equation*}
\begin{split}\label{eqn:cin:ale-eul}
 \dfrac{\partial \rule{1.5ex}{.4pt}}{\partial t} \bigg|_{\mathbf{\chi}} = \dfrac{\partial \rule{1.5ex}{.4pt}}{\partial t} \bigg|_{\mathbf{x}} + \mathbf{v} \cdot \mathbf{\nabla} \rule{1.5ex}{.4pt} \ .\end{split}
\end{equation*}
\sphinxAtStartPar
e, confrontando la
(\DUrole{xref,myst}{{[}eqn:cin:lagr\sphinxhyphen{}eul{]}}\{reference\sphinxhyphen{}type=»ref»
reference=»eqn:cin:lagr\sphinxhyphen{}eul»\}) e la
(\DUrole{xref,myst}{{[}eqn:cin:ale\sphinxhyphen{}eul{]}}\{reference\sphinxhyphen{}type=»ref»
reference=»eqn:cin:ale\sphinxhyphen{}eul»\}), la relazione tra la rappresentazione
arbitraria e quella lagrangiana,
\begin{equation*}
\begin{split}\dfrac{\partial \rule{1.5ex}{.4pt}}{\partial t} \bigg|_{\mathbf{x_0}} = \dfrac{\partial \rule{1.5ex}{.4pt}}{\partial t} \bigg|_{\mathbf{\chi}} + (\mathbf{u} - \mathbf{v}) \cdot \mathbf{\nabla} \rule{1.5ex}{.4pt} \ .\end{split}
\end{equation*}

\section{Velocità di traslazione, rotazione e deformazione}
\label{\detokenize{polimi/fluidmechanics-ita/template/capitoli/03_cinematica/12teoria:velocita-di-traslazione-rotazione-e-deformazione}}
\sphinxAtStartPar
In questa sezione viene studiato il moto di un segmento materiale, che
segue il moto del mezzo continuo. Viene introdotto il tensore gradiente
di velocità \(\mathbf{\nabla}\mathbf{u}\), con \(\mathbf{u}(\mathbf{x},t)\) il campo di
velocità. Questo tensore viene prima scritto come somma della sua parte
antisimmetrica \(\mathbb{W}\) e della sua parte simmetrica \(\mathbb{D}\),
la quale può essere a sua volta scomposta nella parte idrostatica e
nella parte deviatorica \(\mathbb{D}^d\). Viene infine descritta la natura
di questi tensori grazie alla loro influenza sul moto di segmento
materiale.

\sphinxAtStartPar
Il segmento materiale viene identificato dal vettore
\(\Delta\mathbf{x_{12}}(t) = \mathbf{x_2}(t) - \mathbf{x_1}(t)\), i cui estremi sono i
punti di coordinate \(\mathbf{x_1}(t)\) e \(\mathbf{x_2}(t)\). Indicando con
\(\mathbf{u_1}(t) = \mathbf{u}(\mathbf{x_1}(t),t)\) e
\(\mathbf{u_2}(t) = \mathbf{u}(\mathbf{x_2}(t),t)\) loro velocità, è possibile
ricavare l’evoluzione temporale del segmento materiale,
\$\(\Delta\mathbf{x_{12}}(t+\Delta t) = \Delta\mathbf{x_{12}}(t) + \left( \mathbf{u_2}(t) - \mathbf{u_1}(t) \right) \Delta t + o(\Delta t) \ .\)\$
Tornando alla descrizione euleriana del problema, è possibile scrivere
la differenza di velocità introducendo il tensore gradiente di velocità,
\begin{equation*}
\begin{split}\begin{aligned}
 \mathbf{u_2}(t) - \mathbf{u_1}(t) & = \mathbf{u}(\mathbf{x_2}(t),t) - \mathbf{u}(\mathbf{x_1}(t),t) = \\
 & = \mathbf{u}\left(\mathbf{x_1}(t)+\Delta\mathbf{x_{12}}(t),t\right) - \mathbf{u}\left(\mathbf{x_1}(t),t\right) = \\
 & = \mathbf{u}\left(\mathbf{x_1}(t),t\right) + \mathbf{\nabla}\mathbf{u}\left(\mathbf{x_1}(t),t\right) \cdot \Delta\mathbf{x_{12}}(t) - \mathbf{u}\left(\mathbf{x_1}(t),t\right) + o(|\Delta\mathbf{x_{12}}(t)|) = \\
 & = \mathbf{\nabla}\mathbf{u}\left(\mathbf{x_1}(t),t\right) \cdot \Delta\mathbf{x_{12}}(t) + o(|\Delta\mathbf{x_{12}}(t)|) \ . \\
 \end{aligned}\end{split}
\end{equation*}
\sphinxAtStartPar
Riarrangiando i termini si può scrivere,
\begin{equation*}
\begin{split}\label{eqn:cin:material-segm}
 \Delta\mathbf{x_{12}}(t+\Delta t) = \Delta\mathbf{x_{12}}(t) + 
 \big[ \mathbf{\nabla}\mathbf{u}\left(\mathbf{x_1}(t),t\right) \cdot \Delta\mathbf{x_{12}}(t) + o(|\Delta\mathbf{x_{12}}(t)|) \big] \Delta t + o(\Delta t) \ .\end{split}
\end{equation*}
\sphinxAtStartPar
e facendo tendere a zero \(\Delta t\), si ricava
\begin{equation*}
\begin{split}\dfrac{d \Delta\mathbf{x_{12}}}{d t}(t) = \mathbf{\nabla}\mathbf{u}\left(\mathbf{x_1}(t),t\right) \cdot \Delta\mathbf{x_{12}}(t) + o(|\Delta\mathbf{x_{12}}(t)|) \ .\end{split}
\end{equation*}
\sphinxAtStartPar
Nell’ipotesi che i termini \(o(|\Delta \mathbf{x_{12}}(t)|)\) siano
trascurabili, la velocità \(\mathbf{u_2}\) del punto \(\mathbf{x_2}\) differisce
dalla velocità \(\mathbf{u_1}\) del punto \(\mathbf{x_1}\) del termine
\(d \Delta\mathbf{x_{12}}/d t\) che rappresenta le eventuali rotazioni rigide
e le deformazioni del mezzo continuo,
\begin{equation*}
\begin{split}\label{eqn:cin:relative-vel-1}
 \mathbf{u_2}(t) = \mathbf{u_1}(t) + \mathbf{\nabla}\mathbf{u}\left(\mathbf{x_1}(t),t\right) \cdot \Delta\mathbf{x_{12}}(t) \ .\end{split}
\end{equation*}

\subsection{Tensore gradiente di velocità}
\label{\detokenize{polimi/fluidmechanics-ita/template/capitoli/03_cinematica/12teoria:tensore-gradiente-di-velocita}}
\sphinxAtStartPar
Il tensore gradiente di velocità può essere scritto come somma
\(\mathbf{\nabla}\mathbf{u} = \mathbb{D} + \mathbb{W}\) della sua parte simmetrica
\(\mathbb{D}\), il \sphinxstylestrong{tensore velocità di deformazione}, e della su parte
antisimmetrica \(\mathbb{W}\), il \sphinxstylestrong{tensore di spin},
\begin{equation*}
\begin{split}\mathbb{D} = \dfrac{1}{2}\left(\mathbf{\nabla} \mathbf{u} + \mathbf{\nabla}^T \mathbf{u}\right)
  \quad , \quad 
  \mathbb{W} = \dfrac{1}{2}\left(\mathbf{\nabla} \mathbf{u} - \mathbf{\nabla}^T \mathbf{u}\right) \ ,\end{split}
\end{equation*}
\sphinxAtStartPar
i quali possono essere scritti in componenti, in un sistema di
coordinate cartesiane come
\begin{equation*}
\begin{split}D_{ij} = \dfrac{1}{2}\left[ \dfrac{\partial u_i}{\partial x_j} + \dfrac{\partial u_j}{\partial x_i} \right] \quad , \quad 
  W_{ij} = \dfrac{1}{2}\left[ \dfrac{\partial u_i}{\partial x_j} - \dfrac{\partial u_j}{\partial x_i} \right] \ .\end{split}
\end{equation*}
\sphinxAtStartPar
Il tensore velocità di deformazione può essere poi scomposto nella sua
parte idrostatica e nella sua parte deviatorica \(\mathbb{D}^d\),
\begin{equation*}
\begin{split}\begin{aligned}
  \mathbb{D} & = \dfrac{1}{3} \text{tr}(\mathbb{D}) \mathbb{I} + \mathbb{D}^d \quad , \quad
   \mathbb{D}^d = \mathbb{D} - \dfrac{1}{3} \text{tr}(\mathbb{D}) \mathbb{I} \ ,
 \end{aligned}\end{split}
\end{equation*}
\sphinxAtStartPar
dove la traccia \(\text{tr}(\mathbb{D})\) è uguale alla
divergenza del campo di velocità \(\mathbf{\nabla} \cdot \mathbf{u}\). Il tensore
di spin è un tensore antisimmetrico del secondo ordine. Nello spazio
tridimensionale ha solo tre componenti indipendenti, che contengono le
componenti del vettore vorticità
\(\mathbf{\omega} = \mathbf{\nabla} \times \mathbf{u}\). Ad esempio, utilizzando un
sistema di coordinate cartesiane, è possibile scrivere il tensore di
spin come
\begin{equation*}
\begin{split}\mathbb{W} = \dfrac{1}{2}\begin{bmatrix}
   0 & -\omega_z & \omega_y \\
   \omega_z & 0 & -\omega_x \\
   -\omega_y & \omega_x & 0 \\   
  \end{bmatrix} = \dfrac{1}{2}\text{Spin}(\mathbf{\omega}) \ .\end{split}
\end{equation*}
\sphinxAtStartPar
L’operazione \(\mathbb{W} \cdot \mathbf{v}\) tra il tensore antisimmetrico
\(\mathbb{W}=\text{Spin}(\mathbf{\Omega})\) e un vettore \(\mathbf{v}\) qualsiasi
coincide con l’operazione \(\mathbf{\Omega} \times \mathbf{v}\). Introducendo la
scomposizione di \(\mathbf{\nabla} \mathbf{u}\) nella formula
(\DUrole{xref,myst}{{[}eqn:cin:relative\sphinxhyphen{}vel\sphinxhyphen{}1{]}}\{reference\sphinxhyphen{}type=»ref»
reference=»eqn:cin:relative\sphinxhyphen{}vel\sphinxhyphen{}1»\}), si ricava
\begin{equation*}
\begin{split}\begin{aligned}
 \mathbf{u_2}(t) & = \mathbf{u_1}(t) + \dfrac{1}{2}\mathbf{\omega}(\mathbf{x_1}(t),t) \times (\mathbf{x_2}(t) - \mathbf{x_1}(t) ) +  & \text{(atto di moto rigido)} \\ 
& + \mathbb{D}(\mathbf{x_1}(t),t) \cdot (\mathbf{x_2}(t) - \mathbf{x_1}(t)) \ . & \text{(deformazione)}
 %& + \left[ \dfrac{1}{3} \text{tr}(\mathbb{D}) \mathbb{I} +  \mathbb{D}^d \right] \cdot (\mathbf{x_2}(t) - \mathbf{x_1}(t)) \ . & \text{(deformazione)}
\end{aligned}\end{split}
\end{equation*}
\sphinxAtStartPar
Da questa formula si possono riconoscere i contributi
alla velocità \(\mathbf{u_2}\) di «traslazione» (la velocità del punto
\(\mathbf{x_1}\)), di rotazione con velocità angolare
\(\mathbf{\Omega} = \frac{1}{2} \mathbf{\omega}\) e di deformazione,
\(\mathbb{D} \cdot \Delta\mathbf{x_{12}}\).


\subsection{Derivate temporali di oggetti materiali}
\label{\detokenize{polimi/fluidmechanics-ita/template/capitoli/03_cinematica/12teoria:derivate-temporali-di-oggetti-materiali}}
\sphinxAtStartPar
In questa sezione vengono descritti gli effetti dei singoli termini nei
quali può essere scomposto il gradiente di velocità tramite i loro
effetti sull’evoluzione di un segmento materiale \(\mathbf{v}\) o di una
combinazione di segmenti materiali «elementari» (come ad esempio il
prodotto scalare o il triplo prodotto) , per i quali i termini di ordine
\(o(|\mathbf{v}|)\) sono considerati trascurabili.


\subsubsection{Vettore materiale.}
\label{\detokenize{polimi/fluidmechanics-ita/template/capitoli/03_cinematica/12teoria:vettore-materiale}}
\sphinxAtStartPar
Scrivendo il vettore \(\mathbf{v}\) come prodotto del suo modulo \(v\) per il
versore \(\mathbf{\hat{n}}\) che ne identifica la direzione,
\(\mathbf{v} = v \mathbf{\hat{n}}\), è possibile esprimerne la derivata nel tempo
come,
\begin{equation*}
\begin{split}\label{eqn:cin:dvvec}
 \dfrac{d \mathbf{v}}{dt} = \dfrac{dv}{dt}\mathbf{\hat{n}} + v \dfrac{d \mathbf{\hat{n}}}{d t} \ .\end{split}
\end{equation*}

\subsubsection{Vettore materiale: modulo.}
\label{\detokenize{polimi/fluidmechanics-ita/template/capitoli/03_cinematica/12teoria:vettore-materiale-modulo}}
\sphinxAtStartPar
Utilizzando l’identità \(\mathbf{\dot{\hat{n}}} \cdot \mathbf{\hat{n}} = 0\)%
\begin{footnote}[1]\sphinxAtStartFootnote
Poichè \(\mathbf{\hat{n}}\) è un versore,
\(|\mathbf{\hat{n}}|^2 = \mathbf{\hat{n}}\cdot\mathbf{\hat{n}} = 1\). La derivata
nel tempo di quest’ultima espressione diventa
\(0 = \mathbf{\dot{\hat{n}}} \cdot \mathbf{\hat{n}} + \mathbf{\hat{n}} \cdot \mathbf{\dot{\hat{n}}} = 2 \mathbf{\dot{\hat{n}}} \cdot \mathbf{\hat{n}}\),
da cui si ricava l’identità desiderata.
%
\end{footnote},
moltiplicando scalarmente per \(\mathbf{\hat{n}}\) l’ultima espressione, si
ricava la derivata nel tempo del modulo \(v\) del vettore \(\mathbf{v}\),
\begin{equation*}
\begin{split}\label{eqn:cin:dvmod}
 \dfrac{d v}{d t} = \mathbf{\hat{n}} \cdot \dfrac{d \mathbf{v}}{dt} 
 - \underbrace{v \dfrac{d\mathbf{\hat{n}}}{dt}\cdot\mathbf{\hat{n}}}_{=0} = \mathbf{\hat{n}} \cdot \left[ \mathbb{D} + \mathbb{W} \right] \cdot \mathbf{v} = 
 \mathbf{\hat{n}} \cdot \mathbb{D} \cdot \mathbf{\hat{n}} v \ ,\end{split}
\end{equation*}
\sphinxAtStartPar
avendo
introdotto la scomposizione
\(\mathbf{\nabla} \mathbf{u} = \mathbb{D} + \mathbb{W}\) nella formula
(\DUrole{xref,myst}{{[}eqn:cin:material\sphinxhyphen{}segm{]}}\{reference\sphinxhyphen{}type=»ref»
reference=»eqn:cin:material\sphinxhyphen{}segm»\}) applicata al vettore materiale
\(\mathbf{v}\) e utilizzato l’identità
\(\mathbf{\hat{n}} \cdot \mathbb{W} \cdot \mathbf{\hat{n}} = 0\), poiché
\(\mathbb{W}\) è antisimmetrica. Poichè il tensore velocità di
deformazione è simmetrico, esiste una base di vettori ortonormali
\(\{\mathbf{\hat{p}_1},\mathbf{\hat{p}_2},\mathbf{\hat{p}_3}\}\) che permettono di
scrivere la decomposizione spettrale di \(\mathbb{D}\),
\begin{equation*}
\begin{split}\mathbb{D} = \lambda_1 \mathbf{\hat{p}_1} \otimes \mathbf{\hat{p}_1} +
              \lambda_2 \mathbf{\hat{p}_2} \otimes \mathbf{\hat{p}_2} +
              \lambda_3 \mathbf{\hat{p}_3} \otimes \mathbf{\hat{p}_3} \ .\end{split}
\end{equation*}
\sphinxAtStartPar
I vettori \(\mathbf{\hat{p}_i}\) sono gli autovettori del tensore \(\mathbb{D}\)
che ne rappresentano le \sphinxstyleemphasis{direzioni principali}, mentre gli scalari
\(\lambda_i\) sono gli autovalori associati, tali che
\(\mathbb{D} \cdot \mathbf{\hat{p}_i} = \lambda_i \mathbf{\hat{p_i}}\). É quindi
possibile scrivere la derivata nel tempo del modulo \(v\) del vettore
materiale \(\mathbf{v}\) come
\begin{equation*}
\begin{split}\dfrac{1}{v} \dfrac{d v}{d t} = \lambda_1 n_1^2 +  \lambda_2 n_2^2 +  \lambda_3 n_3^2 \ ,\end{split}
\end{equation*}
\sphinxAtStartPar
avendo indicato con \(n_i = \mathbf{\hat{n}} \cdot \mathbf{\hat{p}_i}\) le
proiezioni del versore \(\mathbf{\hat{n}}\) sugli autovettori del tensore
\(\mathbb{D}\).


\subsubsection{Vettore materiale: direzione.}
\label{\detokenize{polimi/fluidmechanics-ita/template/capitoli/03_cinematica/12teoria:vettore-materiale-direzione}}
\sphinxAtStartPar
Combinando la (\DUrole{xref,myst}{{[}eqn:cin:dvvec{]}}\{reference\sphinxhyphen{}type=»ref»
reference=»eqn:cin:dvvec»\}) e la
(\DUrole{xref,myst}{{[}eqn:cin:dvmod{]}}\{reference\sphinxhyphen{}type=»ref»
reference=»eqn:cin:dvmod»\}), è possibile ricavare la derivata nel tempo
della direzione \(\mathbf{\hat{n}}\) del vettore materiale \(\mathbf{v}\),
\begin{equation*}
\begin{split}\begin{aligned}
 \dfrac{d \mathbf{\hat{n}}}{d t} = \dfrac{1}{v}\dfrac{d\mathbf{v}}{dt} - \dfrac{1}{v} \mathbf{\hat{n}} \dfrac{d v}{d t}  & = [ \mathbb{D} + \mathbb{W} ] \cdot \mathbf{\hat{n}} - \mathbf{\hat{n}} \mathbf{\hat{n}} \cdot \mathbb{D} \cdot \mathbf{\hat{n}} = \\
   & =  [ \mathbb{I} - \mathbf{\hat{n}} \otimes \mathbf{\hat{n}} ] \cdot \mathbb{D} \cdot \mathbf{\hat{n}} + \mathbb{W} \cdot \mathbf{\hat{n}} = \\
   & = [ \mathbb{I} - \mathbf{\hat{n}} \otimes \mathbf{\hat{n}} ] \cdot \mathbb{D} \cdot \mathbf{\hat{n}} + \dfrac{1}{2} \mathbf{\omega} \times \mathbf{\hat{n}} \ .
\end{aligned}\end{split}
\end{equation*}
\sphinxAtStartPar
Il tensore
\(\mathbb{P} := \mathbb{I} - \mathbf{\hat{n}} \otimes \mathbf{\hat{n}}\) è il
proiettore ortogoanle in direzione perpendicolare a \(\mathbf{\hat{n}}\), che
ha nucleo generato da \(\mathbf{\hat{n}}\), cioè
\(\mathbb{P} \cdot \mathbf{\hat{n}} = \mathbf{0}\). Introducendo la scomposizione
del tensore \(\mathbb{D}\) nella sua parte idrostatica e deviatorica, è
possibile dimostrare che la parte idrostatica non influenza la derivata
del versore \(\mathbf{\hat{n}}\)
\begin{equation*}
\begin{split}\dfrac{d \mathbf{\hat{n}}}{d t} = [ \mathbb{I} - \mathbf{\hat{n}} \otimes \mathbf{\hat{n}} ] \cdot \mathbb{D}^d \cdot \mathbf{\hat{n}} + \dfrac{1}{2} \mathbf{\omega} \times \mathbf{\hat{n}} \ ,\end{split}
\end{equation*}
\sphinxAtStartPar
poiché
\(\mathbb{P} \cdot \mathbb{I} \cdot \mathbf{\hat{n}} = \mathbb{P} \cdot \mathbf{\hat{n}} = \mathbf{0}\).
In generale quindi la direzione di un vettore materiale dipende dalle
rotazioni, rappresentate dal termine
\(\frac{1}{2} \mathbf{\omega} \times \mathbf{\hat{n}}\) e dalla parte deviatorica
del tensore velocità di deformazione. Questo ultimo contributo può
essere nullo in alcuni casi, come ad esempio
\begin{itemize}
\item {} 
\sphinxAtStartPar
quando lo stato di deformazione è «idrostatico», per il quale
\(\mathbb{D}^d = 0\),

\item {} 
\sphinxAtStartPar
quando il vettore \(\mathbf{v}\) appartenente al nucleo di \(\mathbb{D}^d\),
\(\mathbb{D}^d \cdot \mathbf{v} = \mathbf{0}\), orientato cioè in una
direzione che non subisce una deformazione deviatorica,

\item {} 
\sphinxAtStartPar
quando il vettore \(\mathbf{v}\) è allineato con una delle direzioni
principali \(\mathbf{\hat{p}_i}\) di \(\mathbb{D}\): in questo caso, il
vettore \(\mathbb{D} \cdot \mathbf{\hat{n}}\) è allineato con
\(\mathbf{\hat{n}}\), poichè
\(\mathbb{D} \cdot \mathbf{\hat{n}} = \lambda_i \mathbf{\hat{n}}\), e quindi
appartiene al nucleo del proiettore \(\mathbb{P}\), cioè
\(\mathbb{P} \cdot (\mathbb{D} \cdot \mathbf{\hat{n}}) = \mathbf{0}\).

\end{itemize}


\subsubsection{Angolo tra vettori materiali.}
\label{\detokenize{polimi/fluidmechanics-ita/template/capitoli/03_cinematica/12teoria:angolo-tra-vettori-materiali}}
\sphinxAtStartPar
Calcolando la derivata materiale del prodotto scalare tra due vettori
materiali \(\mathbf{v}\) e \(\mathbf{w}\), è possibile verificare che il tensore di
spin \(\mathbb{W}\) rappresenta una rotazione rigida, non modificando né i
moduli dei singoli vettori materiali, né l’angolo compreso tra di essi.
Infatti la derivata
\begin{equation*}
\begin{split}\begin{aligned}
 \dfrac{d}{dt} (\mathbf{v} \cdot \mathbf{w}) & = \dfrac{d\mathbf{v}}{dt} \cdot \mathbf{w} + \mathbf{v} \cdot \dfrac{d\mathbf{w}}{dt} = \\
  & = \mathbf{w} \cdot \mathbb{D} \cdot \mathbf{v} + \dfrac{1}{2} \mathbf{w} \cdot \mathbf{\omega} \times \mathbf{v} + 
   \mathbf{v} \cdot \mathbb{D} \cdot \mathbf{w} + \dfrac{1}{2} \mathbf{v} \cdot \mathbf{\omega} \times \mathbf{w} = \\
   & = 2 \mathbf{w} \cdot \mathbb{D} \cdot \mathbf{v} \ ,
\end{aligned}\end{split}
\end{equation*}
\sphinxAtStartPar
avendo utilizzato la simmetria del tensore velocità di
deformazione \(\mathbb{D}\) e l’identità vettoriale
\(\mathbf{c} \cdot \mathbf{a} \times \mathbf{b} = - \mathbf{b} \cdot \mathbf{a} \times \mathbf{c}\).
La derivata del coseno dell’angolo formato dai vettori materiali
\(\mathbf{v} = v \mathbf{\hat{n}_v}\), \(\mathbf{w} = w \mathbf{\hat{n}_w}\) dipende
solamente dalla parte deviatorica del tensore velocità di deformazione,
\begin{equation*}
\begin{split}\begin{aligned}
 \dfrac{d \cos \theta_{vw}}{dt} & = \dfrac{d}{d t} \dfrac{\mathbf{v} \cdot \mathbf{w}}{|\mathbf{v}||\mathbf{w}|} = \\
  & = 2 \mathbf{\hat{n}_w} \cdot \mathbb{D}^d \mathbf{\hat{n}_v} - \mathbf{\hat{n}_v} \cdot \mathbf{\hat{n}_w} (\mathbf{\hat{n}_v} \cdot \mathbb{D}^d \cdot \mathbf{\hat{n}_v} + \mathbf{\hat{n}_w} \cdot \mathbb{D}^d \cdot \mathbf{\hat{n}_w} ) = \\
  & = 2 (1 - \mathbf{\hat{n}_v} \cdot \mathbf{\hat{n}_w}) \mathbf{\hat{n}_w} \cdot \mathbb{D}^d \mathbf{\hat{n}_v} - \mathbf{\hat{n}_v} \cdot \mathbf{\hat{n}_w} (\mathbf{\hat{n}_v} - \mathbf{\hat{n}_w}) \cdot \mathbb{D}^d \cdot (\mathbf{\hat{n}_v} - \mathbf{\hat{n}_w}) \ .
\end{aligned}\end{split}
\end{equation*}

\subsubsection{Volume generato da vettori materiali.}
\label{\detokenize{polimi/fluidmechanics-ita/template/capitoli/03_cinematica/12teoria:volume-generato-da-vettori-materiali}}
\sphinxAtStartPar
Infine, è possibile dimostrare che la derivata del volume materiale
(elementare, per il quale i termini \(o(|\Delta \mathbf{x}|)\) siano
trascurabili) \(V = \mathbf{a} \times \mathbf{b} \cdot \mathbf{c}\) del
parallelepipedo formato dai tre vettori materiali \(\mathbf{a}\), \(\mathbf{b}\),
\(\mathbf{c}\) vale
\begin{equation*}
\begin{split}\dfrac{d V}{d t} = (\mathbf{\nabla} \cdot \mathbf{u}) V \ .\end{split}
\end{equation*}
\sphinxAtStartPar
La
divergenza del campo di velocità rappresenta quindi la derivata nel
tempo di un volume materiale relativa al volume materiale stesso. Il
\sphinxstylestrong{vincolo cinematico di incomprimibilità} impone che l’estensione di un
volume materiale non vari nel tempo, \(dV/dt = 0\), ed è quindi
equivalente alla condizione di solenoidalità del campo di velocità,
\(\mathbf{\nabla} \cdot \mathbf{u} = 0\).


\section{Curve caratteristiche}
\label{\detokenize{polimi/fluidmechanics-ita/template/capitoli/03_cinematica/12teoria:curve-caratteristiche}}
\sphinxAtStartPar
Per descrivere il moto di un fluido vengono definite quattro famiglie di
curve: le linee di corrente, le traiettorie, le curve di emissione (o
linee di fumo) e le tracce. Viene data una definizione matematica di
queste curve, che possono essere ottenute durante le attività
sperimentali tramite delle tecniche di visualizzazione del campo di
moto, come mostrato nel seguente video, \sphinxhref{https://www.youtube.com/watch?v=nuQyKGuXJOs}{Stanford 1963 \sphinxhyphen{} Flow
Visualization}.
\sphinxcode{\sphinxupquote{https://www.youtube.com/watch?v=nuQyKGuXJOs}}, nel caso non funzionasse
il collegamento sopra a uno degli storici video del National Committee.

\sphinxAtStartPar
Come già anticipato, secondo la descrizione euleriana del moto di un
mezzo continuo, il campo di velocità è rappresentato dalla funzione
vettoriale \(\mathbf{u}\) i cui argomenti indipendenti sono la coordinata
spaziale \(\mathbf{r}\) e quella temporale \(t\), \(\mathbf{u}(\mathbf{r},t)\). Vengono
ora definite le quattro curve caratteristiche elencate sopra:
\begin{itemize}
\item {} 
\sphinxAtStartPar
Le \sphinxstylestrong{linee di corrente} sono curve \(\mathbf{S}\) tangenti al campo
vettoriale \(\mathbf{u}(\mathbf{r},t)\) in ogni punto dello spazio \(\mathbf{r}\),
all’istante temporale \(t\) considerato. Essendo curve (dimensione=1),
possono essere espresse in forma parametrica come funzioni di un
parametro scalare \(p\), \(\mathbf{S}(p)\). La «traduzione matematica» della
definizione è quindi
\begin{equation*}
\begin{split}\label{eqn:cinematica:ldc}
     \frac{d\mathbf{S}}{dp}(p) = \lambda(p) \mathbf{u}(\mathbf{S}(p),t) \ ,\end{split}
\end{equation*}
\sphinxAtStartPar
cioè
il vettore tangente \({d\mathbf{S}(p)}/{dp}\) alla curva \(\mathbf{S}(p)\), nel
punto identificato dal valore del parametro \(p\), è parallelo al
vettore velocità \(\mathbf{u}\) calcolato nello stesso punto \(\mathbf{S}(p)\),
al tempo considerato \(t\). La funzione \(\lambda(p)\) dipende dalla
parametrizzazione utilizzata e non influisce sulla forma della linea
di corrente. L’equazione
(\DUrole{xref,myst}{{[}eqn:cinematica:ldc{]}}\{reference\sphinxhyphen{}type=»ref»
reference=»eqn:cinematica:ldc»\}) rappresenta tutte le linee di
corrente: per ottenere la linea di corrente passante per un punto, è
necessario imporre questa condizione come condizione al contorno.

\item {} 
\sphinxAtStartPar
Una \sphinxstylestrong{traiettoria} descrive il moto di una singola particella
materiale, la cui velocità è uguale a quella del fluido, nella
posizione in cui si trova e all’istante di tempo «attuale». La
traiettoria di una particella è descritta dall curva \(\mathbf{R}(t)\),
parametrizzata con il tempo \(t\), che soddisfa il seguente problema
differenziale
\begin{equation*}
\begin{split}\begin{cases}
     \dfrac{d\mathbf{R}}{dt}(t) = \mathbf{u}(\mathbf{R}(t),t) \\
     \mathbf{R}(t_0) = \mathbf{R_0} \ .
    \end{cases}\end{split}
\end{equation*}
\sphinxAtStartPar
L’equazione differenziale traduce la definizione di
particella materiale: la velocità della particella materiale
\(\mathbf{v}(t) = d \mathbf{R} / dt (t)\) è uguale alla velocità del fluido
nello stesso punto allo stesso istante di tempo,
\(\mathbf{u}(\mathbf{R}(t),t)\). La condizione iniziale identifica tra tutte
le traiettorie delle infinite particelle materiali, quella della
particella che all’istante \(t_0\) passa per il punto \(\mathbf{R_0}\).
Fissati i «parametri» \(t_0\) e \(\mathbf{R_0}\) che identificano la
particella desiderata, la sua traiettoria è descritta dalla curva
\(\mathbf{R}(t;t_0,\mathbf{R_0})\), funzione del tempo «attuale» \(t\).

\item {} 
\sphinxAtStartPar
Una \sphinxstylestrong{linea di fumo} è il luogo dei punti descritto dalla posizione
al tempo \(t\) (fissato) di tutte le particelle materiali passate per
un punto (fissato) nello spazio, \(\mathbf{R_0}\), negli istanti di tempo
\(t_0\) precedenti a \(t\), \(t_0 < t\).
\begin{equation*}
\begin{split}\begin{cases}
     \dfrac{d\mathbf{R}}{dt}(t) = \mathbf{u}(\mathbf{R}(t),t) \\
     \mathbf{R}(t_0) = \mathbf{R_0} \ .
    \end{cases}\end{split}
\end{equation*}
\sphinxAtStartPar
Il problema è identico a quello delle traiettorie.
Cambia però il ruolo di \(t\), \(t_0\), \(\mathbf{R_0}\): la linea di fumo al
«tempo di osservazione» \(t\) formata da tutte le particelle passanti
da \(\mathbf{R_0}\) a istanti temporali \(t_0\), con \(t_0<t\), è una
descritta dalla curva \(\mathbf{R}(t_0;t,\mathbf{R_0})\), funzione
dell’istante \(t_0\).

\item {} 
\sphinxAtStartPar
Una \sphinxstylestrong{traccia} è il luogo dei punti descritto dalla posizione al
tempo \(t\) (fissato) di tutte le particelle materiali che si
trovavano su una curva \(\mathbf{R_0}(p)\) al tempo \(t_0\) (fissato).
\begin{equation*}
\begin{split}\begin{cases}
     \dfrac{d\mathbf{R}}{dt}(t) = \mathbf{u}(\mathbf{R}(t),t) \\
     \mathbf{R}(t_0) = \mathbf{R_0} \ .
    \end{cases}\end{split}
\end{equation*}
\sphinxAtStartPar
Ancora una volta il problema è identico a quello delle
traiettorie ma cambia il ruolo di \(t\), \(t_0\), \(\mathbf{R_0}\): fissati i
parametri \(t_0\) e \(t\) che identificano rispettivamente l’istante di
tempo in cui le particelle materiali desiderate si trovano sulla
curva \(\mathbf{R_0}\) e l’istante di tempo in cui la curva viene
osservata, la traccia è una funzione dell luogo dei punti «iniziale»
\(\mathbf{R_0}\), \(\mathbf{R}(\mathbf{R_0};t,t_0)\).

\end{itemize}


\subsection{Osservazione 1.}
\label{\detokenize{polimi/fluidmechanics-ita/template/capitoli/03_cinematica/12teoria:osservazione-1}}
\sphinxAtStartPar
Nel caso di campi stazionari, cioè indipendenti dal tempo,
\(\mathbf{u}(\mathbf{r},t) = \mathbf{u}^{(staz)}(\mathbf{r})\), linee di corrente,
traiettorie e linee di fumo coincidono.


\bigskip\hrule\bigskip


\sphinxstepscope


\section{Exercises}
\label{\detokenize{polimi/fluidmechanics-ita/template/capitoli/03_cinematica/exercises:exercises}}\label{\detokenize{polimi/fluidmechanics-ita/template/capitoli/03_cinematica/exercises:fluid-mechanics-kinematics-exercises}}\label{\detokenize{polimi/fluidmechanics-ita/template/capitoli/03_cinematica/exercises::doc}}
\sphinxstepscope


\subsection{Exercise 3.1}
\label{\detokenize{polimi/fluidmechanics-ita/template/capitoli/03_cinematica/1202in:exercise-3-1}}\label{\detokenize{polimi/fluidmechanics-ita/template/capitoli/03_cinematica/1202in:fluid-mechanics-kinematics-ex-01}}\label{\detokenize{polimi/fluidmechanics-ita/template/capitoli/03_cinematica/1202in::doc}}
\sphinxAtStartPar
Sia dato il campo di moto
\$\(\bm{u}(x,y) = 3 \bm{\hat{x}} + 3t \bm{\hat{y}}\)\( Calcolare l'equazione
delle linee di corrente, delle traiettorie e delle linee di fumo (curve
di emissione) e disegnarle. Infine si determino le tracce generate al
tempo \)t\_0 = 0\( dal segmento che unisce l'origine con il punto
\)(x\_1,y\_1)=(0,1)\$.

\sphinxAtStartPar
Definizione di linee di corrente, traiettorie, linee di fumo, tracce.
Soluzione di sistemi di equazioni differenziali ordinarie (problemi di
Cauchy, ai valori iniziali).

\sphinxAtStartPar
Partendo dalle definizioni, si ricavano le equazioni delle curve
caratteristiche. Il problema per le traiettorie, le linee di fumo e le
tracce viene risolto una volta sola per ottenere il risultato in forma
parametrica in funzione di \(t\), \(t_0\), \(\bm{R_0}(p) = (x_0(p), y_0(p))\).
\begin{itemize}
\item {} 
\sphinxAtStartPar
\sphinxstylestrong{Linee di corrente.} L’equazione vettoriale che definisce una
linea di corrente
\(\bm{S}(p) = X(p) \bm{\hat{x}} + Y(p) \bm{\hat{y}}\) viene scritta
per componenti, \$\$\textbackslash{}begin\{cases\}
\textbackslash{}dfrac\{dX\}\{dp\}(p) = \textbackslash{}lambda(p) 3 \textbackslash{}
\textbackslash{}dfrac\{dY\}\{dp\}(p) = \textbackslash{}lambda(p) 3t  \textbackslash{} . \textbackslash{}
\textbackslash{}end\{cases\}

\sphinxAtStartPar
ricavando dalla prima \(\lambda(p)\) in funzione di \(dX/dp\),
sostituendolo nella seconda, e integrando tra \(p_0\) e \(p\), con \(t\)
fissato
\$\(\int_{p_0}^{p}\dfrac{d Y}{dp}(p') dp' = \int_{p_0}^{p}\dfrac{d X}{dp}(p') \ t \ dp' \quad \rightarrow \quad Y(p) - Y(p_0) = ( X(p) - X(p_0) ) \ t \ .\)\(
Dopo aver fissato una linea di corrente, imponendo il suo passaggio
per un punto, \)(X(p\_0), Y(p\_0)) = (x\_0, y\_0)\(, si ottiene la sua
equazione in *forma cartesiana* \)\(y = y_0 + ( x - x_0 ) t \ .\)\( In
questo problema, le linee di corrente costituiscono una famiglia di
rette parallele nel piano \)x\(-\)y\(, a ogni istante temporale, il cui
coefficiente angolare, \)t\$, aumenta con il tempo.

\item {} 
\sphinxAtStartPar
\sphinxstylestrong{Traiettorie.} Le equazioni di traiettorie, linee di fumo e tracce
vengono ricavate in forma parametrica risolvendo il problema ai
valori iniziali che le definisce. In un secondo momento viene
ricavata la loro equazione in \sphinxstyleemphasis{forma cartesiana}, esplicitando il
parametro in funzione di una delle due coordinate spaziali,
esplicitando il parametro in funzione di una delle due coordinate
spaziali. Per le traiettorie, parametrizzate con \(t\), si ottiene
\$\(\label{eqn:ese:par}
 \begin{cases}
  \dfrac{dx}{dt}(t) = 3 \\
  \dfrac{dy}{dt}(t) = 3t \\
  x(t_0) = x_0 , \quad y(t_0) = y_0
 \end{cases}
 \quad \rightarrow \quad
 \begin{cases}
  x(t;\bm{R}_0,t_0) = x_0 + 3(t-t_0) \\
  y(t;\bm{R}_0,t_0) = y_0 +\frac{3}{2} (t^2 -t_0^2) \ . \\
 \end{cases}\)\( Esplicitando \)t\( in funzione di \)x\(,
\)\(t = t_0 + \dfrac{x-x_0}{3} \ ,\)\( e sostituendo nella coordinata
\)y\( si ottiene l'equazione in forma cartesiana,
\)\(\label{eqn:ese:trai}
 y(x;\bm{R_0},t_0) = \dfrac{1}{6}x^2 + \left[ -\dfrac{1}{3}x_0 +t_0 \right] x +
 y_0 + \dfrac{1}{6}x_0^2 - x_0 t_0 \ ,\)\( all'interno della quale
\)\textbackslash{}bm\{R\}\_0 = (x\_0,y\_0)\( e \)t\_0\$ compaiono ancora come parametri.
Dalla (\DUrole{xref,myst}{{[}eqn:ese:trai{]}}\{reference\sphinxhyphen{}type=»ref»
reference=»eqn:ese:trai»\}), le traiettorie sono parabole con la
concavità rivolta verso l’alto.

\item {} 
\sphinxAtStartPar
\sphinxstylestrong{Linee di fumo (curve di emissione).} La forma parametrica
dell’equazione delle linee di fumo (funzioni di \(t_0)\) è
\$\(\begin{cases}
  x(t_0;t,\bm{R}_0) = x_0 + 3(t-t_0) \\
  y(t_0;t,\bm{R}_0) = y_0 +\frac{3}{2} (t^2 -t_0^2) \ . \\
 \end{cases}\)\$

\sphinxAtStartPar
Esplicitando \(t_0\) in funzione di \(x\),
\$\(t_0 = t - \dfrac{x-x_0}{3} \ ,\)\( e sostituendo nella coordinata
\)y\( si ottiene l'equazione in forma cartesiana,
\)\(\label{eqn:ese:trai}
 y(x;\bm{R_0},t) = -\dfrac{1}{6}x^2 + \left[ \dfrac{1}{3}x_0 +t \right] x +
 y_0 - \dfrac{1}{6}x_0^2 + x_0 t_0 \ ,\)\( all'interno della quale
\)\textbackslash{}bm\{R\}\_0 = (x\_0,y\_0)\( e \)t\$ compaiono ancora come parametri. Dalla
(\DUrole{xref,myst}{{[}eqn:ese:trai{]}}\{reference\sphinxhyphen{}type=»ref»
reference=»eqn:ese:trai»\}), le linee di fumo sono parabole con la
concavità rivolta verso il basso.

\item {} 
\sphinxAtStartPar
\sphinxstylestrong{Tracce.} La forma parametrica dell’equazione delle tracce è
\$\(\begin{cases}
  x(\bm{R}_0;t,t_0) = x_0 + 3(t-t_0) \\
  y(\bm{R}_0;t,t_0) = y_0 +\frac{3}{2} (t^2 -t_0^2) \ . \\
 \end{cases}\)\$

\sphinxAtStartPar
Il segmento che unisce l’origine al punto \((x_1,y_1)=(0,1)\) è
descritto in forma paramterica come \$\(\bm{R_0}(p) = \begin{cases}
 x_0(p) = 0  \\
 y_0(p) = p  \\
\end{cases}  , \quad p \in [0,1] \ .\)\( La forma parametrica delle
tracce (\)p\( è il parametro che descrive la curva, mentre \)t\(, \)t\_0\(
sono parametri fissi) è quindi \)\(\bm{R}(\bm{R_0}(p),t,t_0) = 
 \begin{cases}
  x(p;t,t_0) = 3(t-t_0) \\
  y(p;t,t_0) = p +\frac{3}{2} (t^2 -t_0^2) \\
 \end{cases}  , \quad p \in [0,1] \ .\)\( Queste sono segmenti
verticali di lunghezza uguale a 1, con il punto più basso di
coordinate \)\textbackslash{}left(3(t\sphinxhyphen{}t\_0),\textbackslash{}frac\{3\}\{2\}(t\textasciicircum{}2\sphinxhyphen{}t\_0\textasciicircum{}2)\textbackslash{}right)\$.

\end{itemize}

\sphinxstepscope


\subsection{Exercise 3.2}
\label{\detokenize{polimi/fluidmechanics-ita/template/capitoli/03_cinematica/1201in:exercise-3-2}}\label{\detokenize{polimi/fluidmechanics-ita/template/capitoli/03_cinematica/1201in:fluid-mechanics-kinematics-ex-02}}\label{\detokenize{polimi/fluidmechanics-ita/template/capitoli/03_cinematica/1201in::doc}}
\sphinxAtStartPar
Sia dato il campo di moto
\$\(\bm{u}(x,y) = 2Ax \bm{\hat{x}} - 2Ay \bm{\hat{y}}\)\$ Calcolare
l’equazione delle linee di corrente, delle traiettorie e delle linee di
fumo (curve di emissione) e disegnarle.

\sphinxAtStartPar
Definizione di linee di corrente, traiettorie, linee di fumo, tracce.
Soluzione di sistemi di equazioni differenziali ordinarie (problemi di
Cauchy, ai valori iniziali).

\sphinxAtStartPar
Partendo dalle definizioni, si ricavano le equazioni delle curve
caratteristiche.
\begin{itemize}
\item {} 
\sphinxAtStartPar
\sphinxstylestrong{Linee di corrente.} Dalla scrittura in componenti della
definizione di linee di corrente si ottiene il sistema
\$\$\textbackslash{}begin\{cases\}
\textbackslash{}dfrac\{dX\}\{dp\} = \textbackslash{}lambda(p) 2 A X \textbackslash{}
\textbackslash{}dfrac\{dY\}\{dp\} = \sphinxhyphen{} \textbackslash{}lambda(p) 2 A Y \textbackslash{} , \textbackslash{}
\textbackslash{}end\{cases\}

\sphinxAtStartPar
\(\lambda(p) = \frac{X'(p)}{2 A X(p)}\) dalla prima equazione e
inserendolo nella seconda. Integrando tra \(p_0\) e \(p\), dopo aver
semplificato i fattori \(2 A\), si ottiene (derivare per credere)
\$\(0 = \int_{p_0}^{p} \left( \dfrac{X'(p')}{X(p')} + \dfrac{Y'(p')}{Y(p')} \right) dp' =
 \ln{\dfrac{X(p)}{X(p_0)}} + \ln{\dfrac{Y(p)}{Y(p_0)}}\)\(
\)\(\quad \rightarrow \quad
 X(p)Y(p) = X(p_0)Y(p_0)\)\( Le linee di corrente appena ricavate sono
delle iperboli equilatere con gli asintoti coincidenti con gli assi.
Nel procedimento svolto, per poter dividere per \)X(p)\( e \)Y(p)\(
dobbiamo imporre la condizione che \)X(p)\(, \)Y(p)\$ siano diversi da
zero. Nella ricerca degli equilibri del sistema, si nota che
\begin{itemize}
\item {} 
\sphinxAtStartPar
il punto \((x,y) = (0,0)\) è l’unico punto di equilibrio del
sistema, punto di ristagno del campo di velocità;

\item {} 
\sphinxAtStartPar
gli assi coordinati coincidono con linee di corrente: la
derivata \(dX/dp\) è nulla quando \(X=0\) (se la parametrizzazione
della curva è regolare, cioè \(\lambda(p) \ne 0\)); la derivata
\(dY/dp\) è nulla quando \(Y=0\) (se la parametrizzazione della
curva è regolare, cioè \(\lambda(p) \ne 0\)). Nel primo caso, la
linea di corrente coincide con l’asse \(y\), avendo coordinata
\(X=0\) costante e coordinata \(Y(p)\) descritta dalla seconda
equazione; nel secondo caso, la linea di corrente coincide con
l’asse \(x\), avendo coordinata \(Y=0\) costante e coordinata \(X(p)\)
descritta dalla prima equazione.

\end{itemize}

\item {} 
\sphinxAtStartPar
\sphinxstylestrong{Traiettorie.} \$\(\begin{cases}
  \dfrac{dx}{dt} = 2 A x(t) \\
  \dfrac{dy}{dt} = -  2 A y(t) \\
  x(t_0) = x_0 , \quad y(t_0) = y_0
 \end{cases}
 \quad\rightarrow\quad
 \begin{cases}
  x(t;\bm{r_0},t_0) = x_0 e^{2A(t-t_0)} \\
  y(t;\bm{r_0},t_0) = y_0 e^{-2A(t-t_0)} \\
 \end{cases}\)\$

\item {} 
\sphinxAtStartPar
\sphinxstylestrong{Linee di fumo.} Da quanto riportato nel punto e nell’osservazione
precedenti, è immediato ricavare sia la forma parametrica delle
linee di fumo, \$\(\begin{cases}
  x(t_0;t,\bm{r_0}) = x_0 e^{2A(t-t_0)} \\
  y(t_0;t,\bm{r_0}) = y_0 e^{-2A(t-t_0)} \\
 \end{cases}\)\( sia la forma cartesiana, \)x y = x\_0 y\_0\$.

\end{itemize}


\subsubsection{Osservazione.}
\label{\detokenize{polimi/fluidmechanics-ita/template/capitoli/03_cinematica/1201in:osservazione}}
\sphinxAtStartPar
Per ricavare la forma cartesiana dell’equazione delle traiettorie
bisogna esplicitare il parametro \(t\) in funzione di una delle due
coordinate e inserire la formula ottenuta nell’equazione delle altre
componenti. In questo caso è possibile eliminare la dipendenza da
\(t\), moltiplicando tra di loro le componenti delle traiettorie e
ottenendo \(x y = x_0 y_0\): si osserva l’equazione delle traiettorie
coincide con l’equazione delle linee di corrente per il campo di
velocità considerato. Le linee di corrente coincidono con le linee
di corrente e le linee di fumo nel caso in cui il \sphinxstylestrong{campo di
veloictà} è \sphinxstylestrong{stazionario}: in questo caso, il sistema
differenziale con il quale si ricavano linee di corrente e linee di
fumo è \sphinxstylestrong{autonomo}, cioè il termine forzante non dipende
esplicitamente dal tempo. La soluzione di un problema differenziale
di un sistema autonomo non dipende dal tempo \(t\) in sè, ma dalla
differenza tra il tempo \(t\) e il tempo al quale viene imposta la
condizione iniziale \(t_0\): nella formula parametrica delle
traiettorie, \(t\) e \(t_0\) compaiono sempre come differenza \(t-t_0\) e
mai «in altre forme», come ad esempio nell’esercizio precedente, nel
quale il campo di moto non è stazionario. Per questo motivo si
arriva alla stessa equazione in forma cartesiana per le traiettorie
e le linee di fumo, dopo aver esplicitato rispettivamente \(t\) e
\(t_0\) in funzione di una coordinata e aver inserito questa
espressione nelle formule delle altre componenti.

\sphinxstepscope


\subsection{Exercise 3.3}
\label{\detokenize{polimi/fluidmechanics-ita/template/capitoli/03_cinematica/1204in_hints:exercise-3-3}}\label{\detokenize{polimi/fluidmechanics-ita/template/capitoli/03_cinematica/1204in_hints:fluid-mechanics-kinematics-ex-03}}\label{\detokenize{polimi/fluidmechanics-ita/template/capitoli/03_cinematica/1204in_hints::doc}}
\sphinxAtStartPar
Sia dato il campo di moto
\$\(\bm{u}(x,y,z) = 3y \bm{\hat{x}} - 3x \bm{\hat{y}} +t\bm{\hat{z}}\)\$
Calcolare l’equazione delle linee di corrente, delle traiettorie e delle
linee di fumo (curve di emissione) e disegnarle.

\sphinxAtStartPar
\sphinxstylestrong{Suggerimento.} Le componenti \(x\) e \(y\) del sistema sono accoppiate
tra di loro. Risolvendo il sistema per le \sphinxstylestrong{linee di corrente},
\$\(\begin{cases}
  \dfrac{dX}{dp} =  \lambda(p) 3Y \\
  \dfrac{dY}{dp} = -\lambda(p) 3X \\
  \dfrac{dZ}{dp} =  \lambda(p) t \ ,
 \end{cases}
\quad \rightarrow \quad
 \begin{cases}
  X(p) \dfrac{dX}{dp} + Y(p) \dfrac{dY}{dp} = 0 \\
  \dfrac{dZ}{dP} = \lambda(p) t \ .
 \end{cases}\)\( Integrando la prima, si ottiene l'equazione di una
criconferenza \)X(p)\textasciicircum{}2 + Y(p)\textasciicircum{}2 = R\textasciicircum{}2\( (con \)R\textasciicircum{}2 = X(p\_0)\textasciicircum{}2 + Y(p\_0)\textasciicircum{}2\(,
descrivibile in forma paramterica come \)\(\begin{cases}
 X(p) = R \cos(p) \\
 Y(p) = R \sin(p) \ .
 \end{cases}\)\( Con la parametrizzazione scelta, è possibile ricavare la
relazione \)\textbackslash{}lambda(p) = \sphinxhyphen{}1/3\( e integrare l'equazione per la componente
\)Z\$.

\sphinxAtStartPar
Per il calcolo dell’equazione che descrive le \sphinxstylestrong{triettorie} delle
particelle materiali e le \sphinxstylestrong{linee di fumo}, la soluzione del problema
di Cauchy \$\(\begin{cases}
  \dfrac{dx}{dt} =  3y(t) & x(t_0) = x_0 \\
  \dfrac{dy}{dt} = -3x(t) & y(t_0) = y_0 \\
  \dfrac{dz}{dt} = t  &     z(t_0) = z_0 \ ,
 \end{cases}\)\( ha la forma \)\(\begin{cases}
  x(t,\bm{r_0},t_0) = A \sin(3t) - B \cos(3t) \\
  y(t,\bm{r_0},t_0) = A \cos(3t) + B \sin(3t) \\
  z(t,\bm{r_0},t_0) = z_0 + \dfrac{t^2 - t_0^2}{ 2 } \ .  \\
 \end{cases}\)\( Le costanti di integrazione mancanti \)A\(, \)B\( vengono
calcolate imponendo le condizioni iniziali,
\)\(A = y_0 \cos(3t_0) + x_0 \sin(3t_0) \quad , \quad
  B = y_0 \sin(3t_0) - x_0 \cos(3t_0) \ ,\)\( e la soluzione del problema
in forma parametrica può essere riscritta come \)\(\begin{cases}
  x(t,\bm{r_0},t_0) = x_0 \cos(3(t-t_0)) + y_0 \sin(3(t-t_0)) \\
  y(t,\bm{r_0},t_0) =-x_0 \sin(3(t-t_0)) + y_0 \cos(3(t-t_0)) \\
  z(t,\bm{r_0},t_0) = z_0 + \dfrac{t^2 - t_0^2}{ 2 } \ .  \\
 \end{cases}\)\$

\sphinxstepscope


\subsection{Exercise 3.4}
\label{\detokenize{polimi/fluidmechanics-ita/template/capitoli/03_cinematica/1203in_hints:exercise-3-4}}\label{\detokenize{polimi/fluidmechanics-ita/template/capitoli/03_cinematica/1203in_hints:fluid-mechanics-kinematics-ex-04}}\label{\detokenize{polimi/fluidmechanics-ita/template/capitoli/03_cinematica/1203in_hints::doc}}
\sphinxAtStartPar
Sia dato il campo di moto
\$\(\bm{u}(x,y,z) = \frac{x}{x^2 + y^2 + z^2} \bm{\hat{x}} +
                 \frac{y}{x^2 + y^2 + z^2} \bm{\hat{y}} +
                 \frac{z}{x^2 + y^2 + z^2} \bm{\hat{z}}\)\$ Calcolare
l’equazione delle linee di corrente, delle traiettorie e delle linee di
fumo (curve di emissione) e disegnarle.

\sphinxAtStartPar
\sphinxstylestrong{Suggerimento.} Per risolvere l’esercizio in maniera semplice, si
osservi che il campo di moto è stazionario e ha simmetria sferica: è
quindi conveniente usare un sistema di coordiante sferiche.

\sphinxstepscope


\chapter{Balance equations}
\label{\detokenize{polimi/fluidmechanics-ita/template/capitoli/04_bilanci/04teoria:balance-equations}}\label{\detokenize{polimi/fluidmechanics-ita/template/capitoli/04_bilanci/04teoria:fluid-mechanics-balances}}\label{\detokenize{polimi/fluidmechanics-ita/template/capitoli/04_bilanci/04teoria::doc}}
\sphinxAtStartPar
In questo capitolo vengono introdotti i bilanci di alcune quantità
meccaniche per un mezzo continuo. I bilanci in forma integrale
permettono di descrivere l’evoluzione complessiva (integrale) di un
sistema e vengono ricavati partendo da alcuni principi fondamentali
della meccanica classica: la conservazione della massa, l’equazioni
cardinali della dinamica, il primo principio della termodinamica o
bilancio dell’energia. Vengono scritti prima per un volume materiale e
poi per volumi di controllo o volumi in moto generico, utilizzando il
teorema del trasporto di Reynolds.

\sphinxAtStartPar
Dai bilanci in forma integrale, sotto ipotesi di sufficiente regolarità
dei campi, vengono poi ricavati i bilanci in forma differenziale, che
permettono di descrivere l’evoluzione locale (puntuale) di un sistema.
La forma lagrangiana del bilanci di massa, di quantità di moto e della
vorticità verrà utilizzata per meglio apprezzare il significato del
vincolo di incomprimibilità, il ruolo della pressione (e degli sforzi in
generale) nella dinamica di un fluido e intuire l’influenza del campo di
velocità sul campo di vorticità.

\sphinxAtStartPar
Successivamente, dai bilanci integrali vengono ricavate le relazioni di
salto delle quantità meccaniche. Queste relazioni possono essere
utilizzate per trovare determinare lo stato di un sistema formato da due
sotto\sphinxhyphen{}sistemi, all’interno dei quali i campi sono regolari, ma che sono
separati da una frontiera, attraverso la quale i campi non sono
regolari: alcuni esempi di queste sono le superfici «di scorrimento» in
fluidi non viscosi, attraverso le quali è discontinua la componente
tangenziale della velocità, o le onde d’urto che possono formarsi in
correnti comprimibili di fluidi non viscosi.

\sphinxAtStartPar
Infine, viene fornita una breve introduzione agli esercizi sui bilanci
integrali, che costituisce una prima linea guida al loro svolgimento.


\section{Bilanci in forma integrale}
\label{\detokenize{polimi/fluidmechanics-ita/template/capitoli/04_bilanci/04teoria:bilanci-in-forma-integrale}}\label{\detokenize{polimi/fluidmechanics-ita/template/capitoli/04_bilanci/04teoria:fluid-mechanics-balance-integral}}
\sphinxAtStartPar
Vengono ricavati i bilanci integrali per un volume materiale \(V(t)\)
partendo dai principi fondamentali della meccanica classica.
Successivamente si ricavano i bilanci per un volumi in moto arbitrario
\(v(t)\) e, come caso particolare, volumi di controllo \(V_c\).


\section{Bilancio di massa}
\label{\detokenize{polimi/fluidmechanics-ita/template/capitoli/04_bilanci/04teoria:bilancio-di-massa}}
\sphinxAtStartPar
La massa di un volume materiale è uguale all’integrale sul volume della
densità \(\rho\). Per il \sphinxstylestrong{principio di conservazione della massa}, la
massa di un sistema chiuso (che non ha scambi di materia con l’esterno),
come ad esempio un volume materiale \(V(t)\), rimane costante e quindi la
sua derivata nel tempo deve essere uguale a zero,
\begin{equation*}
\begin{split}\dfrac{d}{dt} \int_{V(t)} \rho = 0 \ .\end{split}
\end{equation*}

\section{Bilancio della quantità di moto}
\label{\detokenize{polimi/fluidmechanics-ita/template/capitoli/04_bilanci/04teoria:bilancio-della-quantita-di-moto}}
\sphinxAtStartPar
La quantità di moto di un volume materiale è uguale all’integrale sul
volume della quantità di moto per unità di volume \(\rho \mathbf{u}\), dove
\(\mathbf{u}\) è la velocità delle particelle materiali. Per la \sphinxstylestrong{prima
equazione cardinale della dinamica}, la derivata nel tempo della
quantità di moto di un sistema è uguale alla risultante delle forze
esterne agenti sul sistema,
\begin{equation*}
\begin{split}\dfrac{d}{dt} \int_{V(t)} \rho \mathbf{u} = \int_{V(t)} \mathbf{f} + \oint_{S(t)} \mathbf{t_n} \ ,\end{split}
\end{equation*}
\sphinxAtStartPar
dove \(\int_{V(t)} \mathbf{f}\) rappresenta la risultante delle forze esterne
di volume e \(\oint_{S(t)} \mathbf{t_n}\) la risultante delle forze esterne di
superficie, avendo indicato con \(\mathbf{f}\) il campo di forze per unità di
volume e \(\mathbf{t_n}\) il vettore sforzo agente sulla supreficie esterna
\(S(t)\) del volume \(V(t)\). Il teorema di Cauchy nella meccanica del
continuo, permette di esprimere il vettore sforzo \(\mathbf{t_n}\) in funzione
del tensore degli sforzi \(\mathbb{T}\) e la normale alla superficie
\(\mathbf{\hat{n}}\), come \(\mathbf{t_n} = \mathbf{\hat{n}} \cdot \mathbb{T}\).


\section{Bilancio del momento quantità di moto}
\label{\detokenize{polimi/fluidmechanics-ita/template/capitoli/04_bilanci/04teoria:bilancio-del-momento-quantita-di-moto}}
\sphinxAtStartPar
Il momento della quantità di moto di un volume materiale è uguale
all’integrale sul volume del momento della quantità di moto per unità di
volume \(\rho \mathbf{r} \times \mathbf{u}\), dove \(\mathbf{r}\) è il vettore che
congiunge il polo con i punti del volume materiale. Per la \sphinxstylestrong{seconda
equazione cardinale della dinamica}, la derivata nel tempo del momento
della quantità di moto di un sistema, rispetto a un polo fisso, è uguale
alla risultante momenti esterni sul sistema,
\begin{equation*}
\begin{split}\dfrac{d}{dt} \int_{V(t)} \rho \mathbf{r} \times \mathbf{u} = \int_{V(t)} \mathbf{r} \times \mathbf{f} + \oint_{S(t)} \mathbf{r} \times \mathbf{t_n} \ ,\end{split}
\end{equation*}
\sphinxAtStartPar
nell’ipotesi che non ci siano momenti esterni per unità di volume e che
il materiale non sia polare (due elementi di materiale adiacenti non si
scambiano momenti ma solo forze).


\section{Bilancio dell’energia totale}
\label{\detokenize{polimi/fluidmechanics-ita/template/capitoli/04_bilanci/04teoria:bilancio-dell-energia-totale}}
\sphinxAtStartPar
L’energia totale di un volume materiale è uguale all’integrale sul
volume della sua energia interna per unità di volume \(\rho e\) e della
sua energia cinetica per unità di volume \(\rho |\mathbf{u}|^2/2\). Combinando
il \sphinxstylestrong{primo principio della termodinamica} (che riguarda solo sistemi in
equilibrio) con il \sphinxstylestrong{teorema dell’energia cinetica} (che non include il
contributo di energia interna), la derivata nel tempo dell’energia
totale del sistema di un sistema è uguale alla differenza tra la potenza
delle forze agenti sul sistema e i flussi di calore uscenti da esso,
\begin{equation*}
\begin{split}\dfrac{d}{dt} \int_{V(t)} \rho e^t = \int_{V(t)} \mathbf{f} \cdot \mathbf{u} + \oint_{S(t)} \mathbf{t_n} \cdot \mathbf{u} - \oint_{S(t)} \mathbf{q} \cdot \mathbf{\hat{n}} + \int_{V(t)} \rho r \ ,\end{split}
\end{equation*}
\sphinxAtStartPar
avendo indicato con \(\mathbf{q}\) il flusso di calore uscente dal volume
materiale \(V(t)\), e con \(r\) l’intensità di una sorgente di calore per
unità di massa \(r\), distributia all’interno del volume \(V(t)\), come ad
esempio il calore rilasciato da una reazione chimica come la
combustione.


\section{Bilanci integrali per volumi in moto arbitrario}
\label{\detokenize{polimi/fluidmechanics-ita/template/capitoli/04_bilanci/04teoria:bilanci-integrali-per-volumi-in-moto-arbitrario}}
\sphinxAtStartPar
Utilizzando il teorema del trasporto di Reynolds, è possibile esprimere
la derivata nel tempo dell’integrale di un campo \(f\) su un volume
materiale \(V(t)\) come somma della derivata nel tempo dell’integrale
dello stesso campo \(f\) su un volume arbitrario \(v(t)\) e al flusso della
quantità \(f\) attraverso la frontiera \(s(t)=\partial v(t)\) di \(v(t)\),
dovuto alla velocità relativa \(\mathbf{u} - \mathbf{v}\) tra le particelle
materiali e la superficie \(s(t)\),
\begin{equation*}
\begin{split}\dfrac{d}{d t} \int_{V(t)} f = \dfrac{d}{d t} \int_{v(t)\equiv V(t)} f +
 \oint_{s(t)\equiv S(t)} f (\mathbf{u} - \mathbf{v}) \cdot \mathbf{\hat{n}} \ .\end{split}
\end{equation*}
\sphinxAtStartPar
I bilanci integrali riferiti a un volume arbitrario \(v(t)\), la cui
superficie \(s(t)\) si muove con velocità \(\mathbf{v}\), risultano
\begin{equation*}
\begin{split}\begin{cases}
 \dfrac{d}{dt} \displaystyle\int_{v(t)} \rho + \oint_{s(t)} \rho (\mathbf{u}-\mathbf{v}) \cdot \mathbf{\hat{n}}= 0  \\
 \dfrac{d}{dt} \displaystyle\int_{v(t)} \rho \mathbf{u} + \oint_{s(t)} \rho \mathbf{u} (\mathbf{u} - \mathbf{v}) \cdot \mathbf{\hat{n}} = \int_{v(t)} \mathbf{f} + \oint_{s(t)} \mathbf{t_n}  \\
 \dfrac{d}{dt} \displaystyle\int_{v(t)} \rho \mathbf{r} \times \mathbf{u} + \oint_{s(t)} \rho \mathbf{r} \times \mathbf{u} (\mathbf{u}-\mathbf{v}) \cdot \mathbf{\hat{n}}= \int_{v(t)} \mathbf{r} \times \mathbf{f} + \oint_{s(t)} \mathbf{r} \times \mathbf{t_n} \\
 \dfrac{d}{dt} \displaystyle\int_{v(t)} \rho e^t + \oint_{s(t)} \rho e^t (\mathbf{u}-\mathbf{v}) \cdot \mathbf{\hat{n}}= \int_{v(t)} \mathbf{f} \cdot \mathbf{u} + \oint_{s(t)} \mathbf{t_n} \cdot \mathbf{u} - \oint_{s(t)} \mathbf{q} \cdot \mathbf{\hat{n}} + \int_{V(t)} \rho r \ .
\end{cases}\end{split}
\end{equation*}

\section{Bilanci integrali per volumi di controllo fissi}
\label{\detokenize{polimi/fluidmechanics-ita/template/capitoli/04_bilanci/04teoria:bilanci-integrali-per-volumi-di-controllo-fissi}}
\sphinxAtStartPar
Come caso particolare dei bilanci integrali riferiti a un volume
arbitrario \(v(t)\), i bilanci integrali riferiti a un volume di controllo
fisso \(V_c\) risultano
\begin{equation*}
\begin{split}\begin{cases}
   \dfrac{d}{d t} \displaystyle\int_{V_c} \rho + \oint_{S_c} \rho \mathbf{u} \cdot \mathbf{\hat{n}} = 0 \\
   \dfrac{d}{d t} \displaystyle\int_{V_c} \rho  \mathbf{u}+ \oint_{S_c} \rho \mathbf{u} \mathbf{u} \cdot \mathbf{\hat{n}} = \int_{V_c} \mathbf{f} + \oint_{S_c} \mathbf{t_n} \\
   \dfrac{d}{d t} \displaystyle\int_{V_c} \rho \mathbf{r} \times \mathbf{u}+ \oint_{S_c} \rho \mathbf{r} \times \mathbf{u} \mathbf{u} \cdot \mathbf{\hat{n}} = \int_{V_c} \mathbf{r} \times \mathbf{f} + \oint_{S_c} \mathbf{r} \times \mathbf{t_n} \\
   \dfrac{d}{d t} \displaystyle\int_{V_c} \rho e^t + \oint_{S_c} \rho e^t \mathbf{u} \cdot \mathbf{\hat{n}} = \int_{V_c} \mathbf{f} \cdot \mathbf{u} + \oint_{S_c} \mathbf{t_n} \cdot \mathbf{u} - \oint_{S_c} \mathbf{q} \cdot \mathbf{\hat{n}} + \int_{V(t)} \rho r  \ .
 \end{cases}\end{split}
\end{equation*}

\section{Bilanci in forma differenziale}
\label{\detokenize{polimi/fluidmechanics-ita/template/capitoli/04_bilanci/04teoria:bilanci-in-forma-differenziale}}\label{\detokenize{polimi/fluidmechanics-ita/template/capitoli/04_bilanci/04teoria:fluid-mechanics-balance-differential}}
\sphinxAtStartPar
Sotto le ipotesi di sufficiente regolarità dei campi che compaiono negli
integrali di superficie, è possibile trasformare gli integrali di
superficie in integrali di volume, applicando il teorema della
divergenza o il lemma del teorema di Green
\begin{equation*}
\begin{split}\oint_{S} f n_i = \int_V f_{/i} \ ,\end{split}
\end{equation*}
\sphinxAtStartPar
avendo indicato con \(f_{/i}\) la
derivata parziale rispetto alla coordinata cartesiana \(x_i\) e con \(n_i\)
la proiezione lungo \(x_i\) della normale uscente dalla superficie
\(S=\partial V\). Una volta scritti tutti i termini come integrali di
volume, sullo stesso volume \(V\), è possibile sfruttare l’arbitrarietà
del volume \(V\) per ricavare i bilanci in forma differenziale. In questa
sezione, si partirà dai bilanci in forma integrale scritti per un volume
di controllo fisso \(V=V_c\), per il quale vale
\begin{equation*}
\begin{split}\dfrac{d}{d t} \int_V f = \int_V \dfrac{\partial f}{\partial t} \ ,\end{split}
\end{equation*}
\sphinxAtStartPar
secondo il teorema del trasporto di Reynolds.


\section{Bilancio di massa}
\label{\detokenize{polimi/fluidmechanics-ita/template/capitoli/04_bilanci/04teoria:id1}}
\sphinxAtStartPar
Usando il teorema del trasporto di Reynolds per volumi di controllo
fissi e applicando il teorema della divergenza al termine di flusso, si
può scrivere
\begin{equation*}
\begin{split}\dfrac{d}{d t} \displaystyle\int_{V} \rho + \oint_{S} \rho \mathbf{u} \cdot \mathbf{\hat{n}} = \int_V \left[ \dfrac{\partial \rho}{\partial t} + \mathbf{\nabla} \cdot (\rho \mathbf{u})\right] = 0 \ .\end{split}
\end{equation*}
\sphinxAtStartPar
Sfruttando l’arbitrarietà del bilancio integrale dal volume considerato
e imponendo che l’integranda sia nulla, si ricava la \sphinxstyleemphasis{forma
conservativa} del bilancio differenziale di massa,
\begin{equation*}
\begin{split}\dfrac{\partial \rho}{\partial t} + \mathbf{\nabla} \cdot (\rho \mathbf{u}) = 0\end{split}
\end{equation*}
\sphinxAtStartPar
Sviluppando la divergenza
\(\mathbf{\nabla} \cdot (\rho \mathbf{u}) = \rho \mathbf{\nabla} \cdot \mathbf{u} + \mathbf{u} \cdot \mathbf{\nabla} \rho\),
e riconoscendo l’espressione della derivata materiale, si ottiene la
\sphinxstyleemphasis{forma convettiva} del bilancio differenziale di massa,
\begin{equation*}
\begin{split}\begin{aligned}
 \dfrac{\partial \rho}{\partial t} + \mathbf{u} \cdot \mathbf{\nabla} \rho &+ \rho \mathbf{\nabla} \cdot \mathbf{u} = 0 \\ 
 \dfrac{D \rho}{D t} = &- \rho \mathbf{\nabla} \cdot \mathbf{u} \ .
\end{aligned}\end{split}
\end{equation*}
\sphinxAtStartPar
Il vincolo cinematico di incomprimibilità \(\mathbf{\nabla} \cdot \mathbf{u} = 0\)
equivale al vincolo «fisico» che impone che la densità delle singole
particelle materiali rimanga costante, \(D\rho/Dt = 0\).


\section{Bilancio di quantità di moto}
\label{\detokenize{polimi/fluidmechanics-ita/template/capitoli/04_bilanci/04teoria:bilancio-di-quantita-di-moto}}
\sphinxAtStartPar
É possibile trasformare in un integrale di volume la risultante degli
sforzi di superficie, utilizzando il teorema di Cauchy per i mezzi
continui,
\begin{equation*}
\begin{split}\mathbf{t_n} = \mathbf{\hat{n}} \cdot \mathbb{T} \quad , \quad 
  t_i = n_j T_{ji} \ ,\end{split}
\end{equation*}
\sphinxAtStartPar
dove \(\mathbf{t_n}\) è il vettore sforzo,
\(\mathbf{\hat{n}}\) la normale alla superficie e \(\mathbb{T}\) il tensore
degli sforzi. La risultante degli sforzi di superficie diventa, usando
un po” di libertà nel passare dalla notazione astratta a quella
indiciale,
\begin{equation*}
\begin{split}\oint_S \mathbf{t_n} = \oint_S t_i = \oint_S n_j T_{ji} =
  \int_V T_{ji/j} = \int_V \mathbf{\nabla} \cdot \mathbb{T} \ .\end{split}
\end{equation*}
\sphinxAtStartPar
Usando il
teorema del trasporto di Reynolds per volumi di controllo fissi e
applicando il teorema della divergenza al termine di flusso,
\begin{equation*}
\begin{split}\oint_S \big\{ \rho \mathbf{u} \mathbf{u} \cdot \mathbf{\hat{n}} \big\}_i = \oint_S \rho u_i u_j n_j = \int_V (\rho u_i u_j)_{/j} = \int_{V} \mathbf{\nabla} \cdot ( \rho \mathbf{u} \otimes \mathbf{u} ) \ ,\end{split}
\end{equation*}
\sphinxAtStartPar
si può scrivere il bilancio di quantità di moto
\begin{equation*}
\begin{split}\displaystyle\int_{V} \dfrac{\partial(\rho \mathbf{u})}{\partial t}  + \int_{V} \mathbf{\nabla} \cdot ( \rho \mathbf{u} \otimes \mathbf{u} ) = \int_{V} \left[ \mathbf{f} +  \mathbf{\nabla} \cdot \mathbb{T} \right] \ .\end{split}
\end{equation*}
\sphinxAtStartPar
Sfruttando l’arbitrarietà del bilancio integrale dal volume considerato
e imponendo che l’integranda sia nulla, si ricava la \sphinxstyleemphasis{forma
conservativa} del bilancio differenziale di quantità di moto,
\begin{equation*}
\begin{split}\dfrac{\partial(\rho \mathbf{u})}{\partial t}  + \mathbf{\nabla} \cdot ( \rho \mathbf{u} \otimes \mathbf{u} - \mathbb{T} ) = \mathbf{f} \ .\end{split}
\end{equation*}
\sphinxAtStartPar
Sviluppando i termini
\begin{equation*}
\begin{split}\begin{aligned}
 \dfrac{\partial (\rho \mathbf{u})}{\partial t} = \rho \dfrac{\partial \mathbf{u}}{\partial t} + \mathbf{u} \dfrac{\partial \rho}{\partial t} \quad & , \quad 
 \dfrac{\partial (\rho u_i)}{\partial t} = \rho \dfrac{\partial u_i}{\partial t} + u_i \dfrac{\partial \rho}{\partial t} \\
 \mathbf{\nabla} \cdot ( \rho \mathbf{u} \otimes \mathbf{u} ) = \rho (\mathbf{u} \cdot \mathbf{\nabla}) \mathbf{u} + \mathbf{u} \mathbf{\nabla} \cdot (\rho \mathbf{u}) \quad & , \quad 
 ( \rho u_i u_j )_{/j} = \rho u_j u_{i/j} + u_i (\rho u_j)_{/j} 
  \ ,
\end{aligned}\end{split}
\end{equation*}
\sphinxAtStartPar
riconoscendo che
\(\mathbf{u} \cdot (\partial \rho/\partial t + \mathbf{\nabla} \cdot (\rho \mathbf{u}))=0\)
come conseguenza della conservazione della massa, si ottiene la \sphinxstyleemphasis{forma
convettiva} dell’equazione della quantità di moto
\begin{equation*}
\begin{split}\begin{aligned}
   \rho \dfrac{\partial\mathbf{u}}{\partial t}  +  \rho (\mathbf{u}  \cdot \mathbf{\nabla} ) \mathbf{u}& = \mathbf{f} + \mathbf{\nabla} \cdot \mathbb{T}  \\ 
   \rho \dfrac{D \mathbf{u}}{D t} & = \mathbf{f} + \mathbf{\nabla} \cdot \mathbb{T} \ . \\ 
  \end{aligned}\end{split}
\end{equation*}

\section{Bilancio del momento della quantità di moto}
\label{\detokenize{polimi/fluidmechanics-ita/template/capitoli/04_bilanci/04teoria:bilancio-del-momento-della-quantita-di-moto}}
\sphinxAtStartPar
Il bilancio del momento della quantità di moto per un mezzo continuo non
polare è equivalente alla condizione di simmetria del tensore degli
sforzi \$\(\mathbb{T}^T = \mathbb{T} \quad , \quad T_{ij} = T_{ji} \ .\)\$


\section{Bilancio dell’energia totale}
\label{\detokenize{polimi/fluidmechanics-ita/template/capitoli/04_bilanci/04teoria:id2}}
\sphinxAtStartPar
Usando un po” di libertà nel passare dalla notazione astratta a quella
indiciale, la potenza degli sforzi di superficie diventa
\begin{equation*}
\begin{split}\begin{aligned}
 \oint_S \mathbf{t_n} \cdot \mathbf{u} = \oint_S t_i u_i = \oint_S n_j T_{ji} u_i & =
  \int_V (T_{ji} u_i)_{/j}= \int_V \mathbf{\nabla} \cdot ( \mathbb{T} \cdot \mathbf{u}) \\
  & = \int_V (T_{ij/j} u_i + T_{ij} u_{j/i}) = \int_V \big( (\mathbf{\nabla} \cdot \mathbb{T}) \cdot \mathbf{u} + \mathbb{T} : \mathbf{\nabla} \mathbf{u} \big) \ , 
\end{aligned}\end{split}
\end{equation*}
\sphinxAtStartPar
avendo utilizzato la simmetria del tensore degli sforzi,
\(T_{ij/j} = \{ \mathbf{\nabla} \cdot \mathbb{T}^T \}_i = \{ \mathbf{\nabla} \cdot \mathbb{T} \}_i\).
Applicando il teorema della divergenza, il termine di flusso di calore

\sphinxAtStartPar
viene scritto come
\begin{equation*}
\begin{split}\oint_S \mathbf{q} \cdot \mathbf{\hat{n}} = \int_V \mathbf{\nabla} \cdot \mathbf{q} \ .\end{split}
\end{equation*}
\sphinxAtStartPar
La \sphinxstyleemphasis{forma conservativa} del bilancio differenziale di energia totale
diventa quindi
\begin{equation*}
\begin{split}\dfrac{\partial (\rho e^t)}{\partial t} + \mathbf{\nabla} \cdot (\rho e^t \mathbf{u} - \mathbb{T} \cdot \mathbf{u} + \mathbf{q}) = \mathbf{f} \cdot \mathbf{u} + \rho r \ .\end{split}
\end{equation*}
\sphinxAtStartPar
Sviluppando la derivata temporale e il termine
\(\mathbf{\nabla} \cdot (\rho e^t \mathbf{u}) = \rho \mathbf{u} \cdot \mathbf{\nabla} e^t + e^t \mathbf{\nabla} \cdot (\rho \mathbf{u})\),
riconoscendo che
\(e^t (\partial \rho/\partial t + \mathbf{\nabla} \cdot (\rho \mathbf{u}))=0\)
come conseguenza della conservazione della massa, si ottiene la \sphinxstyleemphasis{forma
convettiva} dell’equazione dell’energia totale,
\begin{equation*}
\begin{split}\begin{aligned}
   \rho \dfrac{\partial e^t}{\partial t}  +  \rho \mathbf{u}  \cdot  \mathbf{\nabla} e^t & = \mathbf{f} \cdot \mathbf{u} + \mathbf{\nabla} \cdot ( \mathbb{T} \cdot \mathbf{u} ) - \mathbf{\nabla} \cdot \mathbf{q} + \rho r \\ 
   \rho \dfrac{D e^t}{D t} & =  \mathbf{f} \cdot \mathbf{u} + \mathbf{\nabla} \cdot ( \mathbb{T} \cdot \mathbf{u} ) - \mathbf{\nabla} \cdot \mathbf{q} + \rho r  \ . \\ 
  \end{aligned}\end{split}
\end{equation*}

\section{Chiusura del problema}
\label{\detokenize{polimi/fluidmechanics-ita/template/capitoli/04_bilanci/04teoria:chiusura-del-problema}}
\sphinxAtStartPar
Affinché il sistema di equazioni differenziali alle derivate parziali
formato dai bilanci di massa, quantità di moto ed energia totale, con le
condizioni iniziali e al contorno adeguate, sono necessarie l’equazione
di stato del materiale che ne descriva le proprietà termodinamiche%
\begin{footnote}[1]\sphinxAtStartFootnote
Si ricorda che lo stato termodinamico di un sistema monofase è
definito da due variabili termodinamiche indipendenti.
%
\end{footnote} e
i legami costitutivi che esprimano il tensore degli sforzi e il flusso
di calore come funzioni dello stato dinamico e termodinamico del
sistema. Per un fluido, il tensore degli sforzi viscosi \(\mathbb{T}\) può
essere scritto come la somma del contributo idrostatico dovuto alla
pressione \(p\) e il tensore degli sforzi viscosi \(\mathbb{S}\), funzione
delle derivate spaziali del campo di velocità. Un fluido che ha un
\sphinxstyleemphasis{legame costitutivo lineare} tra il tensore degli sforzi viscosi e il
gradiente di velocità \(\mathbf{\nabla} \mathbf{u}\), viene definito \sphinxstylestrong{fluido
newtoniano}. Per un fluido newtoniano isotropo, il legame costitutivo
che definisce il tensore degli sforzi è
\begin{equation*}
\begin{split}\mathbb{T} = -p \mathbb{I} + 2 \mu \mathbb{D} + \lambda (\mathbf{\nabla} \cdot \mathbf{u}) \mathbb{I} \ ,\end{split}
\end{equation*}
\sphinxAtStartPar
dove \(\mu\) e \(\lambda\) sono rispettivamente il coefficiente di viscosità
dinamica e il secondo coefficiente di viscosità, \(p\) è la pressione
(«termodinamica»), \(\mathbb{D}\) il tensore velocità di deformazione. In
generale, sia la pressione \(p\) sia i coefficienti di viscosità dipendono
dallo stato termodinamico del sistema. Il flusso di calore \(\mathbf{q}\) per
conduzione può essere descritto dalla \sphinxstylestrong{legge di Fourier}, che lo mette
in relazione con il gradiente della temepratura tramite il coefficiente
di conduzione termica \(k\), in generale funzione dello stato
termodinamico del sistema,
\begin{equation*}
\begin{split}\mathbf{q} = - k \mathbf{\nabla} T \ .\end{split}
\end{equation*}
\sphinxAtStartPar
L’introduzione di queste leggi costitutive nelle equazioni di bilancio,
aggiunge nuove incognite al sistema, per le quali non abbiamo ricavato
un’equazione dinamica. Sono quindi indispensabili la legge di stato che
fornisca le relazioni necessarie,
\begin{equation*}
\begin{split}\begin{aligned}
  p = p(\rho,e) \quad , & \quad \mu = \mu(\rho,e) \\
  T = T(\rho,e) \quad , & \quad \lambda = \lambda(\rho,e) \\
  & \quad k = k(\rho,e) \ ,
 \end{aligned}\end{split}
\end{equation*}
\sphinxAtStartPar
avendo scelto le variabili termodinamiche delle quali è
nota l’equazione dinamica come due variabili termodinamiche
indipendenti: la densità \(\rho\) e l’energia interna \(e\). Ve


\section{Altre equazioni di bilancio}
\label{\detokenize{polimi/fluidmechanics-ita/template/capitoli/04_bilanci/04teoria:altre-equazioni-di-bilancio}}
\sphinxAtStartPar
Combinando i bilanci delle quantità meccaniche ottenuti partendo dai
principi fondamentali della fisica, si possono ottenere le equazioni di
bilanci di altre quantità, come ad esempio l’energia cinetica
\(\rho|\mathbf{u}|^2/2\), l’energia interna \(e\), e la vorticità
\(\mathbf{\omega} = \nabla \times \mathbf{u}\).


\subsection{Equazione dell’energia cinetica}
\label{\detokenize{polimi/fluidmechanics-ita/template/capitoli/04_bilanci/04teoria:equazione-dell-energia-cinetica}}
\sphinxAtStartPar
Moltiplicando scalarmente il bilancio della quantità di moto per il
vettore velocità \(\mathbf{u}\), si può scrivere l’equazione di bilancio
dell’energia cinetica. In forma conservativa,
\begin{equation*}
\begin{split}\dfrac{\partial}{\partial t}\dfrac{\rho|\mathbf{u}|^2}{2}  + \mathbf{\nabla} \cdot \left( \rho \mathbf{u} \dfrac{|\mathbf{u}|^2}{2} \right) = \mathbf{f} \cdot \mathbf{u} + \mathbf{u} \cdot ( \mathbf{\nabla} \cdot \mathbb{T} ) \ ,\end{split}
\end{equation*}
\sphinxAtStartPar
in forma convettiva,
\begin{equation*}
\begin{split}\begin{aligned}
   \rho \dfrac{\partial}{\partial t} \dfrac{|\mathbf{u}|^2}{2} +  \rho \mathbf{u}  \cdot \mathbf{\nabla} \dfrac{|\mathbf{u}|^2}{2} & = \mathbf{f} \cdot \mathbf{u} + \mathbf{u} \cdot (\mathbf{\nabla} \cdot \mathbb{T} ) \\ 
   \rho \dfrac{D }{D t}\dfrac{|\mathbf{u}|^2}{2} & = \mathbf{f}\cdot \mathbf{u} + \mathbf{u} \cdot ( \mathbf{\nabla} \cdot \mathbb{T} ) \ . \\ 
  \end{aligned}\end{split}
\end{equation*}

\subsection{Equazione dell’energia interna}
\label{\detokenize{polimi/fluidmechanics-ita/template/capitoli/04_bilanci/04teoria:equazione-dell-energia-interna}}
\sphinxAtStartPar
Dalla differenza del bilancio dell’energia totale e dell’energia
cinetica, si ottiene il bilancio dell’energia interna. In forma
conservativa,
\begin{equation*}
\begin{split}\dfrac{\partial (\rho e)}{\partial t} + \mathbf{\nabla} \cdot (\rho e \mathbf{u}) = \mathbb{T}:\mathbf{\nabla}\mathbf{u} - \mathbf{\nabla} \cdot \mathbf{q} + \rho r \ ,\end{split}
\end{equation*}
\sphinxAtStartPar
in forma convettiva,
\begin{equation*}
\begin{split}\begin{aligned}
   \rho \dfrac{\partial e}{\partial t} +  \rho \mathbf{u}  \cdot \mathbf{\nabla}e & =  \mathbb{T}:\mathbf{\nabla}\mathbf{u} - \mathbf{\nabla} \cdot \mathbf{q} + \rho r \\
   \rho \dfrac{D e}{D t} & =  \mathbb{T}:\mathbf{\nabla}\mathbf{u} - \mathbf{\nabla} \cdot \mathbf{q} + \rho r \ . \\ 
  \end{aligned}\end{split}
\end{equation*}

\subsection{Equazione della vorticità}
\label{\detokenize{polimi/fluidmechanics-ita/template/capitoli/04_bilanci/04teoria:equazione-della-vorticita}}
\sphinxAtStartPar
Applicando l’operatore di rotore al bilancio della quantità di moto, si
ottiene l’equazione dinamica della vorticità. Per un fluido newtoniano
(con coefficienti di viscosità costanti e uniformi),
\begin{equation*}
\begin{split}\begin{aligned}
 \dfrac{\partial \mathbf{\omega}}{\partial t} + (\mathbf{u} \cdot \mathbf{\nabla} ) \mathbf{\omega} & =
  (\mathbf{\omega} \cdot \mathbf{\nabla}) \mathbf{u} - \mathbf{\omega} (\mathbf{\nabla} \cdot \mathbf{u}) +
  \nu \Delta \mathbf{\omega} + \dfrac{\mathbf{\nabla} \rho \times \mathbf{\nabla} p}{\rho^2} \\
   \dfrac{D \mathbf{\omega}}{D t}  & =
  (\mathbf{\omega} \cdot \mathbf{\nabla}) \mathbf{u} - \mathbf{\omega} (\mathbf{\nabla} \cdot \mathbf{u}) +
  \nu \Delta \mathbf{\omega} + \dfrac{\mathbf{\nabla} \rho \times \mathbf{\nabla} p}{\rho^2} \ ,
 \end{aligned}\end{split}
\end{equation*}
\sphinxAtStartPar
dove è stata introdotta la viscosità cinematica del fluido,
\(\nu = \mu / \rho\).


\subsection{Equazione dell’entropia}
\label{\detokenize{polimi/fluidmechanics-ita/template/capitoli/04_bilanci/04teoria:equazione-dell-entropia}}
\sphinxAtStartPar
Nell’ipotesi di equilibrio termodinamico delle singole particelle
materiali%
\begin{footnote}[2]\sphinxAtStartFootnote
Se i tempi caratteristici della termodinamica sono di gran lunga
inferiori al tempo caratteristico del fenomeno fluidodinamico, si
può immaginare che la singola particella fluida sia in continuo
stato di equilibrio termodinamico locale. Si possono quindi
estendere i risultati della Termodinamica, che in generale studia
sistemi in equilibrio, alla singola particella fluida.
%
\end{footnote} si può ricavare dal primo principio della termodinamica,
\$\(de = T ds - P dv = T ds + \dfrac{p}{\rho^2} d\rho \ ,\)\$ l’equazione di
bilancio dell’entropia in forma convettiva,
\begin{equation*}
\begin{split}T \dfrac{D s}{D t} = \dfrac{D e}{D t} - \dfrac{p}{\rho^2} \dfrac{D \rho}{D t} \qquad \rightarrow \qquad
 \rho \dfrac{D s}{D t} = \dfrac{1}{T} \left( \rho \dfrac{D e}{D t} -  \dfrac{p}{\rho} \dfrac{D \rho}{D t} \right) \ .\end{split}
\end{equation*}
\sphinxAtStartPar
Utilizzando il bilancio dell’energia interna e il bilancio di massa, si
può scrivere
\begin{equation*}
\begin{split}\rho \dfrac{D s}{D t} = \dfrac{1}{T} \left( \mathbb{T} : \mathbf{\nabla} \mathbf{u} - \mathbf{\nabla} \cdot \mathbf{q} + \rho r + p \mathbf{\nabla} \cdot \mathbf{u} \right) \ ,\end{split}
\end{equation*}
\sphinxAtStartPar
e separando i contributi viscosi da quelli di presisone nel tensore
degli sforzi, \(\mathbb{T} = - p \mathbb{I} + \mathbb{S}\),%
\begin{footnote}[3]\sphinxAtStartFootnote
Dovrebbe essere facile dimostrare che
\(\mathbb{I}:\mathbf{\nabla}\mathbf{u} = \mathbf{\nabla} \cdot \mathbf{u}\).
%
\end{footnote}
\begin{equation*}
\begin{split}\rho \dfrac{D s}{D t} = \dfrac{1}{T} \left( \mathbb{S} : \mathbf{\nabla} \mathbf{u} - \mathbf{\nabla} \cdot \mathbf{q} + \rho r \right) \ .\end{split}
\end{equation*}
\sphinxAtStartPar
Nel caso di fluidi newtoniani,
\(\mathbb{S} = 2\mu\mathbb{D} + \lambda\mathbf{\nabla} \cdot \mathbf{u} \mathbb{I}\),
l’equazione dell’entropia in forma differenziale convettiva diventa
\begin{equation*}
\begin{split}\rho \dfrac{D s}{D t} = \dfrac{1}{T} \left( 2 \mu \mathbb{D} : \mathbb{D} + \lambda |\mathbf{\nabla} \cdot \mathbf{u}|^2 - \mathbf{\nabla} \cdot \mathbf{q} + \rho r \right) \ .\end{split}
\end{equation*}
\sphinxAtStartPar
Utilizzando la legge di Fourier, \(\mathbf{q} = -k \mathbf{\nabla} T\), per il
flusso di calore per conduzione, si può riscrivere il termine di
divergenza del flusso di calore,
\begin{equation*}
\begin{split}\dfrac{1}{T} \mathbf{\nabla} \cdot \mathbf{q} =
  \mathbf{\nabla} \cdot \left( \dfrac{\mathbf{q}}{T} \right) + \mathbf{q} \cdot \dfrac{\mathbf{\nabla} T}{T^2} =
  \mathbf{\nabla} \cdot \left( \dfrac{\mathbf{q}}{T} \right) - k \mathbf{\nabla}{T} \cdot \dfrac{\mathbf{\nabla} T}{T^2} =
  \mathbf{\nabla} \cdot \left( \dfrac{\mathbf{q}}{T} \right) - k \dfrac{|\mathbf{\nabla} T|^2}{T^2} \ ,\end{split}
\end{equation*}
\sphinxAtStartPar
e quindi riscrivere l’equazione dell’entropia, in forma conservativa e
convettiva,
\begin{equation*}
\begin{split}\begin{aligned}
 \dfrac{\partial (\rho s)}{\partial t} + \mathbf{\nabla} \cdot (\rho s \mathbf{u} ) & = \dfrac{1}{T} \left( 2 \mu \mathbb{D} : \mathbb{D} + \lambda |\mathbf{\nabla} \cdot \mathbf{u}|^2 \right) + k \dfrac{|\mathbf{\nabla}T|^2}{T^2} - \mathbf{\nabla} \cdot \left( \dfrac{\mathbf{q}}{T} \right) + \dfrac{\rho r}{T} \\
 \rho \dfrac{D s}{D t} & = \dfrac{1}{T} \left( 2 \mu \mathbb{D} : \mathbb{D} + \lambda |\mathbf{\nabla} \cdot \mathbf{u}|^2 \right) + k \dfrac{|\mathbf{\nabla}T|^2}{T^2} - \mathbf{\nabla} \cdot \left( \dfrac{\mathbf{q}}{T} \right) + \dfrac{\rho r}{T}  \ .
\end{aligned}\end{split}
\end{equation*}
\sphinxAtStartPar
In questa equazione si riconoscono tutti i fenomeni fisici che
influenzano l’entropia di una particella materiale. Si possono
distinguere i contributi dovuti alla \sphinxstyleemphasis{non idealità} del fluido
considerato, legati ai fenomeni viscosi e di conduzione del calore, e i
contributi dovuti ai flussi di calore forniti alla sistema. I fenomeni
viscosi e i processi di conduzione del calore fanno aumentare
l’entropia, poiché
\begin{equation*}
\begin{split}T, \, \mu, \, \lambda, \, k \, \geq 0
 \qquad \text{e} \qquad
 \mathbb{D}:\mathbb{D}, \, |\mathbf{\nabla} \cdot \mathbf{u}|, \, |\mathbf{\nabla} T| \, \geq 0 \ .\end{split}
\end{equation*}
\sphinxAtStartPar
Gli ultimi due termini rappresentano i flussi di calore forniti al
sistema e si presentano nella forma \(Q/T\), flusso di calore diviso la
temperatura della particella, coerentemente con la definizione
dell’entropia in Termodinamica,
\begin{equation*}
\begin{split}dS = \dfrac{\delta Q}{T} \ .\end{split}
\end{equation*}
\sphinxAtStartPar
Questi
due termini possono dare un contributo positivo o nevativo, a seconda
del segno della sorgente di calore \(r\) e del flusso di valore \(\mathbf{q}\).
Il bilancio dell’entropia in forma integrale per un volume materiale,
\begin{equation*}
\begin{split}\dfrac{d}{dt}\int_{V(t)} \rho s = \int_{V(t)} \dfrac{1}{T} \left( 2 \mu \mathbb{D} : \mathbb{D} + \lambda |\mathbf{\nabla} \cdot \mathbf{u}|^2 + k \dfrac{|\mathbf{\nabla}T|^2}{T} \right)
 - \oint_{\partial V(t)} \dfrac{\mathbf{q}}{T} \cdot \mathbf{\hat{n}} + \int_{V(t)} \dfrac{\rho r}{T} \ ,\end{split}
\end{equation*}
\sphinxAtStartPar
permette di interpretare il interpretare il ruolo dei fenomeni non
ideali (viscosità e conduzione del calore) come sorgente di volume
sempre positiva dell’entropia, riconoscrere il ruolo della sorgente (o
pozzo) di entropia di intensità per unità di massa \(r/T\) svolto da una
sorgente di calore per unità di massa \(r\), e il ruolo di flusso di
entropia \(\mathbf{q}/T\) attraverso il contorno del volume \(V(t)\) svolto da
un flusso di calore \(\mathbf{q}\).


\section{Relazioni di salto}
\label{\detokenize{polimi/fluidmechanics-ita/template/capitoli/04_bilanci/04teoria:relazioni-di-salto}}\label{\detokenize{polimi/fluidmechanics-ita/template/capitoli/04_bilanci/04teoria:fluid-mechanics-balance-jump}}
\sphinxAtStartPar
…


\bigskip\hrule\bigskip


\sphinxstepscope


\section{Exercises}
\label{\detokenize{polimi/fluidmechanics-ita/template/capitoli/04_bilanci/exercises:exercises}}\label{\detokenize{polimi/fluidmechanics-ita/template/capitoli/04_bilanci/exercises:fluid-mechanics-balances-exercises}}\label{\detokenize{polimi/fluidmechanics-ita/template/capitoli/04_bilanci/exercises::doc}}
\sphinxstepscope


\subsection{Approfondimenti su alcuni bilanci}
\label{\detokenize{polimi/fluidmechanics-ita/template/capitoli/04_bilanci/0410in:approfondimenti-su-alcuni-bilanci}}\label{\detokenize{polimi/fluidmechanics-ita/template/capitoli/04_bilanci/0410in:fluid-mechanics-balances-in-depth}}\label{\detokenize{polimi/fluidmechanics-ita/template/capitoli/04_bilanci/0410in::doc}}
\sphinxAtStartPar
In questa sezione vengono analizzate alcune equazioni di bilancio in
forma differenziale (è quindi necessario che queste equazioni siano
valide!): vengono usate sia la rappresentazione euleriana sia la
rappresentazione lagrangiana, al fine di ottenere la migliore
comprensione dei fenomeni fisici coinvolti.

\sphinxAtStartPar
Si indicano con \(\bm{x}_0\) le coordinate lagrangiane, solidali con il
continuo; si indicano con \(\bm{x}\) le coordinate euleriane. I due
sistemi di coordinate sono legati tra di loro dalle relazioni
\begin{equation*}
\begin{split}\begin{aligned}
 \bm{x} = \bm{x}(\bm{x}_0,t) \\
 \frac{D \bm{x}}{D t} = \left.\frac{\partial \bm{x}}{\partial t}\right|_{\bm{x}_0} = 
 \bm{u}
\end{aligned}\end{split}
\end{equation*}
\sphinxAtStartPar
La derivata \(\partial/\partial t\) indica la derivata
temporale fatta a coordinata euleriana \(\bm{x}\) costante. La derivata
materiale \(D/D t\) indica la derivata fatta «a coordinata lagrangiana»
costante e rappresenta quindi la variazione temporale di una quantità
legata alla particella materiale, che si muove come il continuo, per la
definizione di coordinate materiali.

\sphinxAtStartPar
Il legame tra \(D/Dt\) e \(\partial/\partial t\) si trova utilizzando le
regole di derivazione per funzioni composte. Scrivendo la funzione
generica \(f\) come
\begin{equation*}
\begin{split}f(\bm{x},t) = f(\bm{x}(\bm{x}_0,t),t)
  = f_0(\bm{x}_0,t) = f_0(\bm{x}_0(\bm{x},t),t) ,$$ si ottiene
$$\frac{D}{Dt} f = \left.\frac{\partial}{\partial t}\right|_{\bm{x}_0} f(\bm{x},t) =
   \left.\frac{\partial}{\partial t}\right|_{\bm{x}_0} f(\bm{x}(\bm{x_0},t),t) = 
   \left.\frac{\partial f}{\partial t}\right|_{\bm{x}} +
   \left.\frac{\partial \bm{x}}{\partial t}\right|_{\bm{x}_0} \cdot
   \left.\frac{\partial f}{\partial \bm{x}}\right|_{t}
   = \frac{\partial f}{\partial t} +
    \bm{u} \cdot \bm{\nabla} f .\end{split}
\end{equation*}

\subsubsection{Continuità}
\label{\detokenize{polimi/fluidmechanics-ita/template/capitoli/04_bilanci/0410in:continuita}}

\subsubsection{Quantità di moto}
\label{\detokenize{polimi/fluidmechanics-ita/template/capitoli/04_bilanci/0410in:quantita-di-moto}}

\subsubsection{Vorticità}
\label{\detokenize{polimi/fluidmechanics-ita/template/capitoli/04_bilanci/0410in:vorticita}}
\sphinxstepscope


\subsection{Exercise 4.1}
\label{\detokenize{polimi/fluidmechanics-ita/template/capitoli/04_bilanci/0412in:exercise-4-1}}\label{\detokenize{polimi/fluidmechanics-ita/template/capitoli/04_bilanci/0412in:fluid-mechanics-balances-ex-01}}\label{\detokenize{polimi/fluidmechanics-ita/template/capitoli/04_bilanci/0412in::doc}}
\sphinxAtStartPar
I bilanci integrali consentono di valutare le azioni integrali (forze,
momenti, potenza) scambiati tra un fluido e un corpo a contatto con
esso, senza conoscere nel dettaglio il campo di moto del fluido di
interesse, ma valutando il flusso netto delle quantità meccaniche di
interesse (massa, quantità di moto, momento della quantità di moto,
energia, entalpia e calore) attraverso la superficie di contorno del
volume fluido di interesse. Il contorno del dominio fluido \(v(t)\) viene
suddiviso nella parte a contatto con il corpo di interesse \(s_{f,s}(t)\)
e nella parte rimanente
\(s_{f,free}(t) = \partial v(t) \backslash s_{f,s}(t)\).


\subsubsection{Bilancio della quantità di moto e risultante delle forze.}
\label{\detokenize{polimi/fluidmechanics-ita/template/capitoli/04_bilanci/0412in:bilancio-della-quantita-di-moto-e-risultante-delle-forze}}
\sphinxAtStartPar
La risultante delle forze agenti sul corpo%
\begin{footnote}[1]\sphinxAtStartFootnote
La risultante delle forze delle azioni scambiate con il fluido. A
questa andranno sommate le forze di volume, come ad esempio il peso
del corpo stesso.
%
\end{footnote} sarà uguale all’integrale
del vettore sforzo agente sulla superficie \(s_{s,f}(t)\),
\$\(\bm{R}^s = \oint_{s_{s,f}(t)} \bm{t}_{n,s} \ ,\)\( avendo indicato con
\)s\_\{s,f\}(t)\( la superficie del solido con normale uscente dalla
superficie solida ed entrante nel solido e con \)\textbackslash{}bm\{t\}\sphinxstyleemphasis{\{n,s\}\( il vettore
sforzo agente sul solido, uguale e contrario allo sforzo agente sul
fluido nello stesso punto, \)\textbackslash{}bm\{t\}}\{n,s\} = \sphinxhyphen{} \textbackslash{}bm\{t\}\sphinxstyleemphasis{n\(, per il principio
di azione e reazione (terzo principio della dinamica). Non è stato
aggiunto il pedice \)f\( al vettore sforzo agente sul fluido, poiché siamo
in un corso di fluidodinamica e il soggetto è il fluido, quando è
sottointeso. Si può riconoscere la risultante \)\textbackslash{}bm\{R\}\textasciicircum{}s\( all'interno del
bilancio integrale della quantità di moto per il volume fluido \)v(t)\(,
\)\(\begin{aligned}
 \dfrac{d}{dt} \displaystyle\int_{v(t)} \rho \bm{u} + \oint_{\partial v(t)} \rho \bm{u} (\bm{u}-\bm{v}) \cdot \bm{\hat{n}}= \int_{v(t)} \bm{f} + \oint_{\partial v(t)} \bm{t_n} \ .
\end{aligned}\)\( Si analizzano i termini di superficie, considerando
separatamente i contributi delle superfici \)s}\{f,s\}\( e \)s\_\{f,free\}\(. Se
il solido ha una superficie impermeabile al fluido e non c'è flusso di
massa, la velocità del fluido e del solido sono uguali,
\)\textbackslash{}bm\{u\} = \textbackslash{}bm\{v\}\(, sulla superficie \)s\_\{f,s\}\(. Di conseguenza rimane
solo il contributo del flusso della quantità di moto attraverso la
superficie \)s\_\{f,free\}\(, mentre il termine di flusso attraverso
\)s\_\{f,s\}\( è nullo. L'integrale sul contorno \)\textbackslash{}partial v(t)\( del vettore
sforzo può essere suddiviso nella somma dell'integrale svolto sulla
superficie a contatto con il solido e sulla superficie libera,
\)\(\begin{aligned}
 \oint_{\partial v(t)} \bm{t_n} & = 
 \oint_{s_{f,s}(t)} \bm{t_n} + \oint_{s_{f,free}(t)} \bm{t_n} = \\
 & = - \oint_{s_{s,f}(t)} \bm{t}_{n,s} + \oint_{s_{f,free}(t)} \bm{t_n} =
 -\bm{R}^s + \oint_{s_{f,free}(t)} \bm{t_n} \ .
\end{aligned}\)\( Spesso sulla superficie libera \)s\_\{f,free\}(t)\( possono
essere trascurati gli sforzi viscosi: in questo caso, il vettore sforzo
si riduce al solo effetto della pressione \)\textbackslash{}bm\{t\_n\} = \sphinxhyphen{}p \textbackslash{}bm\{\textbackslash{}hat\{n\}\}\$.

\sphinxAtStartPar
Ritornando al bilancio della quantità di moto, si può scrivere
\$\$\textbackslash{}bm\{R\}\textasciicircum{}s = \sphinxhyphen{} \textbackslash{}int\_\{s\_\{f,free\}\} \textbackslash{}rho \textbackslash{}bm\{u\} (\textbackslash{}bm\{u\}\sphinxhyphen{}\textbackslash{}bm\{v\}) \textbackslash{}cdot \textbackslash{}bm\{\textbackslash{}hat\{n\}\}
\begin{itemize}
\item {} 
\sphinxAtStartPar
\textbackslash{}int\_\{s\_\{f,free\}(t)\} \textbackslash{}bm\{t\_n\}

\item {} 
\sphinxAtStartPar
\textbackslash{}int\_\{v(t)\} \textbackslash{}bm\{f\} \sphinxhyphen{} \textbackslash{}dfrac\{d\}\{dt\} \textbackslash{}int\_\{v(t)\} \textbackslash{}rho \textbackslash{}bm\{u\}\$\( Nel caso
in cui il problema sia stazionario e che le forze di volume nel fluido
siano trascurabili, gli ultimi due termini is annullano. Se poi si
possono trascurare gli sforzi viscosi su \)s\_\{f,free\}\(, la superficie
\)s\_\{s,free\}\( è una superficie chiusa (si pensi alla superficie
"all'infinito" attorno a un corpo, come esempio) e la pressione è
costante su questa superficie chiusa, l'integrale degli sforzi su
\)s\_\{f,free\}\( è anch'esso nullo, poiché
\)\(\oint_{s_{f,free}(t)} \bm{t_n} = - \oint_{s_{f,free}(t)} p \bm{\hat{n}} = - p \oint_{s_{f,free}(t)} \bm{\hat{n}} \equiv 0 \ ,\)\(
e la risultante delle forze agenti sul solido si riduce a
\)\(\bm{R}^s = - \int_{s_{f,free}} \rho \bm{u} (\bm{u}-\bm{v}) \cdot \bm{\hat{n}} \ .\)\$

\end{itemize}


\subsubsection{Bilancio del momento della quantità di moto e risultante dei momenti.}
\label{\detokenize{polimi/fluidmechanics-ita/template/capitoli/04_bilanci/0412in:bilancio-del-momento-della-quantita-di-moto-e-risultante-dei-momenti}}
\sphinxAtStartPar
Riproponendo un ragionamento analogo, dal bilancio del momento della
quantità di moto si può ricavare la risultante dei momenti agenti su un
corpo, \$\(\bm{M} = \oint_{s_{s,f}} \bm{r} \times \bm{t}_{n,s} \ .\)\( Nel
caso semplificato in cui il problema sia stazionario, le forze di volume
sono trascurabili, gli sforzi viscosi sono trascurabili sulla superficie
\)s\_\{f,free\}(t)\( chiusa, sulla quale agisce una pressione costante, la
risultante dei momenti agenti sul solido si riduce a
\)\(\bm{M}^s = - \int_{s_{f,free}} \rho \bm{r} \times \bm{u} (\bm{u}-\bm{v}) \cdot \bm{\hat{n}} \ ,\)\(
dove \)\textbackslash{}bm\{r\}\( è il raggio vettore tra i punti sulla superficie
\)s\_\{f,free\}(t)\$ e il polo rispetto al quale si calcolano i momenti.


\subsubsection{Bilancio dell’energia totale.}
\label{\detokenize{polimi/fluidmechanics-ita/template/capitoli/04_bilanci/0412in:bilancio-dell-energia-totale}}
\sphinxAtStartPar
Tramite il bilancio dell’energia totale si può ricavare la potenza
fornita (o assorbita) da un corpo al fluido, e/o il calore scambiato con
esso. Gli esercizi che utilizzeranno il bilancio di energia totale
ricorderanno alcuni esercizi di Fisica Tecnica. Lo scopo di questi
esercizi è quello di proporre un punto di vista più maturo a tali
problemi, partendo ai bilanci integrali nella loro forma più generale e
opportunamente semplificati considerando grandezze uniformi sulle
sezioni (o equivalenti grandezze medie) e ipotesi sullo scambio di
calore tra il fluido e l’esterno. Verranno analizzati sistemi aperti e
chiusi, nella speranza di fornire un approccio di validità generale a
problemi già trattati durante il corso di Fisica Tecnica, senza alcuna
pretesa di coprire tutti gli argomenti e i dettagli trattati in quel
corso, ma piuttosto consentire una visione del problema generale che
coinvolga scambi di massa, lavoro e calore del sistema con l’esterno,
facilmente specializzabile a casi particolari, che riduca al minimo lo
sforzo mnemonico richiesto da molti casi particolari, apparentemente
scorrelati l’uno dall’altro, a vantaggio di una maggiore «sensibilità»
sul fenomeno fisico.

\sphinxAtStartPar
Sfruttando la suddivisione della superficie del volume fluido
\(\partial v = s_{f,free} \cup s_{f,s}\), si può riscrivere il bilancio
dell’energia totale,
\$\(\dfrac{d}{dt} \displaystyle\int_{v(t)} \rho e^t + \oint_{\partial v(t)} \rho e^t (\bm{u}-\bm{v}) \cdot \bm{\hat{n}}= \int_{v(t)} \bm{f} \cdot \bm{u} + \oint_{\partial v(t)} \bm{t_n} \cdot \bm{u} - \oint_{\partial v(t)} \bm{q} \cdot \bm{\hat{n}} + \int_{v(t)} \rho r \ .\)\(
riconoscendo la potenza
\)\(W = \oint_{s_{f,s}} \bm{t_n} \cdot \bm{u} \ ,\)\( fornita da un corpo
solido al fluido, \)\(\begin{aligned}
 \dfrac{d}{dt} \displaystyle\int_{v(t)} \rho e^t & + \oint_{\partial v(t)} \rho e^t (\bm{u}-\bm{v}) \cdot \bm{\hat{n}} = \\
  & = \int_{v(t)} \bm{f} \cdot \bm{u} + \oint_{s_{f,free}(t)} \bm{t_n} \cdot \bm{u} + W - \oint_{\partial v(t)} \bm{q} \cdot \bm{\hat{n}} + \int_{v(t)} \rho r \ .
\end{aligned}\)\( Se non c'è flusso di massa attraverso la superficie
solida, \)\textbackslash{}bm\{u\} = \textbackslash{}bm\{v\}\( su \)s\_\{f,s\}\(. Se la superficie libera
\)s\_\{f,free\}\( del volume di controllo è fissa, \)\textbackslash{}bm\{v\}= \textbackslash{}bm\{0\}\( su
\)s\_\{f,free\}\(. Separando il contributo degli sforzi di pressione da
quelli viscosi, \)\textbackslash{}bm\{t\_n\} = \sphinxhyphen{}p\textbackslash{}bm\{\textbackslash{}hat\{n\}\} + \textbackslash{}bm\{s\_n\}\( sulla superficie
\)s\_\{s,free\}\(, il bilancio dell'energia totale diventa, \)\(\begin{aligned}
 \dfrac{d}{dt} \displaystyle\int_{v(t)} \rho e^t & + \oint_{s_{f,free}(t)} \rho h^t \bm{u} \cdot \bm{\hat{n}} = \\
  & = \int_{v(t)} \bm{f} \cdot \bm{u} + \oint_{s_{f,free}(t)} \bm{s_n} \cdot \bm{u} + W - \oint_{\partial v(t)} \bm{q} \cdot \bm{\hat{n}} + \int_{v(t)} \rho r \ ,
\end{aligned}\)\( avendo introdotto l'entalpia totale
\)h\textasciicircum{}t = e\textasciicircum{}t + \textbackslash{}frac\{p\}\{\textbackslash{}rho\} = e + \textbackslash{}frac\{p\}\{\textbackslash{}rho\} + \textbackslash{}frac\{|\textbackslash{}bm\{u\}|\textasciicircum{}2\}\{2\}\(.
Se si trascurano la potenza degli sforzi viscosi su \)s\_\{s,free\}\( e la
potenza delle forze di volume \)\textbackslash{}bm\{f\}\(, il bilancio dell'energia totale
del fluido contenuto nel volume \)v(t)\( diventa
\)\(\dfrac{d}{dt} \displaystyle\int_{v(t)} \rho e^t + \oint_{s_{f,free}(t)} \rho h^t \bm{u} \cdot \bm{\hat{n}}
 \, = \,
  W - \oint_{\partial v(t)} \bm{q} \cdot \bm{\hat{n}} + \int_{v(t)} \rho r \ .\)\$


\subsubsection{Sistemi aperti}
\label{\detokenize{polimi/fluidmechanics-ita/template/capitoli/04_bilanci/0412in:sistemi-aperti}}
\sphinxAtStartPar
Per un sistema aperto in cui sono soddisfatte le ipotesi già elencate,
si può scrivere
\$\$\textbackslash{}dfrac\{d\}\{dt\} \textbackslash{}displaystyle\textbackslash{}int\_\{v(t)\} \textbackslash{}rho e\textasciicircum{}t = \sphinxhyphen{} \textbackslash{}oint\_\{s\_\{f,free\}(t)\} \textbackslash{}rho h\textasciicircum{}t \textbackslash{}bm\{u\} \textbackslash{}cdot \textbackslash{}bm\{\textbackslash{}hat\{n\}\}
\begin{itemize}
\item {} 
\sphinxAtStartPar
W \sphinxhyphen{} \textbackslash{}oint\_\{\textbackslash{}partial v(t)\} \textbackslash{}bm\{q\} \textbackslash{}cdot \textbackslash{}bm\{\textbackslash{}hat\{n\}\} + \textbackslash{}int\_\{v(t)\} \textbackslash{}rho r \textbackslash{} ,\$\(
e sinteticamente \)\(\dfrac{d E^t}{d t} = \Phi_{h^t} + W + \dot{Q} \ ,\)\(
avendo definito l'energia totale interna \)E\textasciicircum{}t\( al volume \)v(t)\(
studiato, il flusso netto di entalpia totale \)\textbackslash{}Phi\_\{h\textasciicircum{}t\}\( attraverso la
superficie \)s\_\{s,free\}\(, e il flusso di calore \)\textbackslash{}dot\{Q\}\( fornito al
fluido contenuto all'interno di \)v(t)\(, \)\(\begin{aligned}
E & = \int_{v(t)} \rho e \\
\Phi_{h^t} & = \int_{s_{f,free}} \rho h^t \bm{u} \cdot \bm{\hat{n}}
           = \int_{s_{f,free}} \rho \left( e + \dfrac{p}{\rho} + \dfrac{|\bm{u}|^2}{2} \right) \bm{u} \cdot \bm{\hat{n}} \\
\dot{Q} & = - \oint_{\partial v(t)} \bm{q} \cdot \bm{\hat{n}} + \int_{v(t)} \rho r \ . 
\end{aligned}\)\$

\end{itemize}


\subsubsection{Sistemi chiusi}
\label{\detokenize{polimi/fluidmechanics-ita/template/capitoli/04_bilanci/0412in:sistemi-chiusi}}
\sphinxAtStartPar
Per un sistema chiuso (nessuno scambio di massa con l’esterno) in cui i
termini cinetici sono trascurabili, \(e^t = e\), il bilancio di energia
diventa sintenticamente,
\$\$\%\textbackslash{}dfrac\{d\}\{dt\} \textbackslash{}displaystyle\textbackslash{}int\_\{v(t)\} \textbackslash{}rho e

\sphinxAtStartPar
\textbackslash{}dfrac\{d E\}\{dt\} = W + \textbackslash{}dot\{Q\} \textbackslash{} ,\$\( avendo definito
\)E = \textbackslash{}displaystyle\textbackslash{}int\_\{v(t)\} \textbackslash{}rho e\(, come l'energia interna del fluido
contenuto nel volume \)v(t)\$. Questa formula corrisponde al primo
principio della Termodinamica, formulato in termini di potenza e non di
energia, in cui è stata utilizzata la convenzione di potenza delle forze
positiva e flusso di calore positivo se fornito al fluido.%
\begin{footnote}[2]\sphinxAtStartFootnote
In Termodinamica, che studia sistemi in equilibrio, il primo
principio è formulato in termini di energia come,
\$\(\Delta E = Q - L \ ,\)\( in cui la variazione di energia \)\textbackslash{}Delta E\(
tra due stati termodinamici del sistema corrisponde alla differenza
del calore \)Q\( fornito al sistema e al lavoro \)L\$ svolto \sphinxstylestrong{dal}
sistema.
%
\end{footnote}


\bigskip\hrule\bigskip


\sphinxstepscope


\subsection{Exercise 4.2}
\label{\detokenize{polimi/fluidmechanics-ita/template/capitoli/04_bilanci/0406in:exercise-4-2}}\label{\detokenize{polimi/fluidmechanics-ita/template/capitoli/04_bilanci/0406in:fluid-mechanics-balances-ex-02}}\label{\detokenize{polimi/fluidmechanics-ita/template/capitoli/04_bilanci/0406in::doc}}
\sphinxAtStartPar
+:———————————:+:———————————:+
| Si consideri una rete idraulica   | \sphinxincludegraphics{{polimi/fluidmechanics-ita/template/capitoli/04_bilanci/fig/rete}.eps}\{width=»9 |
| come quella rappresentata in      | 0\%»\}                              |
| figura. All’interno dei tubi      |                                   |
| scorre acqua. Sia nota le         |                                   |
| velocità media dell’acqua         |                                   |
| all’interno di alcuni dei rami    |                                   |
| della rete: \(U_1 = 1\, m/s\),      |                                   |
| \(U_2 = 1.5\, m/s\),                |                                   |
| \(U_3 = 0.5\, m/s\),                |                                   |
| \(U_7 = 2\, m/s\) e                 |                                   |
| \(U_8 = 0.3\, m/s\). Il verso della |                                   |
| velocità è indicato dalle frecce  |                                   |
| sul disegno. Determinare la       |                                   |
| portata volumetrica, la portata   |                                   |
| in massa e la velocità media      |                                   |
| all’interno di ciascun ramo della |                                   |
| rete sapendo che l’acqua ha una   |                                   |
| densità pari a                    |                                   |
| \(\overline{\rho} = 999\ kg/m^3\),  |                                   |
| e che il diametro dei tubi è      |                                   |
| rispettivamente \(D_1=0.4\ m\),     |                                   |
| \(D_2=0.2\ m\), \(D_3=0.2\ m\),       |                                   |
| \(D_4=0.3\ m\), \(D_5=0.5\ m\),       |                                   |
| \(D_6=0.25\ m\), \(D_7=0.3\ m\),      |                                   |
| \(D_8=0.6\ m\).                     |                                   |
|                                   |                                   |
| (\(Q_1 = 0.13\ m^3/s\),             |                                   |
| \(Q_2 = 0.05\ m^3/s\),              |                                   |
| \(Q_3 = 0.02\  m^3/s\),             |                                   |
| \(Q_4 = 0.13\ m^3/s\),              |                                   |
| \(Q_5 = 0.06\ m^3/s\),              |                                   |
| \(Q_6 = 0.13\  m^3/s\),             |                                   |
| \(Q_7 = 0.14\ m^3/s\),              |                                   |
| \(Q_8 = 0.08\ m^3/s\),              |                                   |
| \(U_1 = 1   \ m/s\),                |                                   |
| \(U_2 = 1.5\  m/s\),                |                                   |
| \(U_3 = 0.5\   m/s\),               |                                   |
| \(U_4 = 1.87\ m/s\),                |                                   |
| \(U_5 = 0.29\ m/s\),                |                                   |
| \(U_6 = 2.69\  m/s\),               |                                   |
| \(U_7 = 2   \ m/s\),                |                                   |
| \(U_8 = 0.3\  m/s\),                |                                   |
| \(\overline{Q}_1 = 125.5\  kg/s\),  |                                   |
| \(\overline{Q}_2 = 47.08\  kg/s\),  |                                   |
| \(\overline{Q}_3 = 15.69\  kg/s\),  |                                   |
| \(\overline{Q}_4 = 131.8\  kg/s\),  |                                   |
| \(\overline{Q}_5 = 54.49\  kg/s\),  |                                   |
| \(\overline{Q}_6 = 131.8\  kg/s\),  |                                   |
| \(\overline{Q}_7 = 141.2\  kg/s\),  |                                   |
| \(\overline{Q}_8 = 84.74\  kg/s\))  |                                   |
+———————————–+———————————–+

\sphinxAtStartPar
Bilancio integrale della massa. Teoria delle reti: bilancio ai nodi.

\sphinxAtStartPar
Se il regime di moto è stazionario, la portata massica è costante e
indipendente dalla sezione considerata all’interno di ogni singolo tubo.
Il bilancio di massa nell”\(i\)\sphinxhyphen{}esimo tubo è,
\$\(\underbrace{\dfrac{d}{dt} \int_{V_i} \bm{\rho}}_{=0} = \oint_{S_i} \rho \bm{u} \cdot \bm{\hat{n}} = \oint_{S_{i,{\alpha}}}\rho \bm{u} \cdot \bm{\hat{n}} + \oint_{S_{i,\beta}} \rho \bm{u} \cdot \bm{\hat{n}} = \tilde{Q}_{i,\alpha} + \tilde{Q}_{i,\beta} \quad \rightarrow \quad \tilde{Q}_{i,\alpha} = -  \tilde{Q}_{i,\beta} \ ,\)\(
avendo indicato \)S\_\{i,\{\textbackslash{}alpha\}\}\( e \)S\_\{i,\{\textbackslash{}beta\}\}\( le due sezioni in
"ingresso" e "uscita" del tubo \)V\_i\(, con \)\textbackslash{}bm\{\textbackslash{}hat\{n\}\}\(,
\)\textbackslash{}tilde\{Q\}\sphinxstyleemphasis{\{\textbackslash{}alpha\}\( e \)\textbackslash{}tilde\{Q\}}\{\textbackslash{}beta\}\( la normale uscente e i flussi
di massa uscenti dal volume \)V\_i\(. Se si calcola il flusso di massa
\)\textbackslash{}overline\{Q\}\_i\( attraverso le sezioni del tubo con normale identificata
dal "verso di percorrenza" del tubo, uno dei due termini cambia segno e
si dimostra che la portata è costante sulle sezioni del singolo tubo,
\)\(\overline{Q}_{i,\alpha} = \overline{Q}_{i,\beta} =: \overline{Q}_{i} \ .\)\(
Utilizzando il verso delle frecce indicato in figura per stabilire il
segno dei flussi di massa, il bilancio di massa ai nodi porta al sistema
lineare, \)\(\begin{cases}
   \overline{Q}_1 + \overline{Q}_2 + \overline{Q}_3 - \overline{Q}_4 - \overline{Q}_5 = 0 & \text{(bil. al nodo in alto)} \\
   \overline{Q}_5 + \overline{Q}_8 - \overline{Q}_7 = 0 & \text{(bil. al nodo a sinistra)} \\
   \overline{Q}_4 - \overline{Q}_6 = 0 & \text{(bil. al nodo a destra)} \ , \\
 \end{cases}\)\( nel quale le incognite sono i flussi \)\textbackslash{}overline\{Q\}\_4\(,
\)\textbackslash{}overline\{Q\}\_5\(, \)\textbackslash{}overline\{Q\}\_6\(, una volta calcolati gli altri flussi
con i dati forniti dal testo del problema,
\)\textbackslash{}overline\{Q\}\_k = \textbackslash{}rho \textbackslash{}frac\{\textbackslash{}pi\}\{4\}D\_k\textasciicircum{}2 U\_k\(, \)k=1,2,3,7,8\(.
Successivamente si calcolano le portate volumetriche \)Q\_k\( incognite,
dividendo le portate massiche \)\textbackslash{}overline\{Q\}\_k\( per la densità \)rho\(,
\)\(Q_k = \dfrac{\overline{Q}_k}{\rho} \quad , \quad k = 1:8 \ .\)\$

\sphinxstepscope


\subsection{Exercise 4.3}
\label{\detokenize{polimi/fluidmechanics-ita/template/capitoli/04_bilanci/0407in:exercise-4-3}}\label{\detokenize{polimi/fluidmechanics-ita/template/capitoli/04_bilanci/0407in:fluid-mechanics-balances-ex-03}}\label{\detokenize{polimi/fluidmechanics-ita/template/capitoli/04_bilanci/0407in::doc}}
\sphinxAtStartPar
+:———————————:+:———————————:+
| Si sta riempiendo una bombola per |                                   |
| immersioni subacquee. Sapendo che |                                   |
| la pompa aspira aria a pressione  |                                   |
| ambiente di \(1.01\times10^5\ Pa\)  |                                   |
| e alla temperatura di \(293\ K\) in |                                   |
| un condotto di sezione \(1\ cm^2\)  |                                   |
| in cui la velocità media è di     |                                   |
| \(0.5\ m/s\) e che non ci sono      |                                   |
| perdite nel sistema di pompaggio, |                                   |
| determinare la rapidità di        |                                   |
| variazione della massa d’aria e   |                                   |
| della sua densità all’interno     |                                   |
| della bombola, sapendo che il     |                                   |
| volume della bombola è pari a     |                                   |
| \(0.02 \  m^3\).                    |                                   |
|                                   |                                   |
| (\(\frac{dM}{dt} = 6.01 \times 10^ |                                   |
| {-5}\ kg/s, \frac{d \rho}{d t} =  |                                   |
| 3.00 \times 10^{-3}\ kg/(m^3 s)\)) |                                   |
| .                                 |                                   |
+———————————–+———————————–+

\sphinxAtStartPar
Bilancio integrale della massa. Legge dei gas perfetti.

\sphinxAtStartPar
Sono date la pressione \(p\) e la temperatura \(T\) all’uscita della pompa.
É nota l’area \(S\) della sezione e la velocità media \(U\) su quella
sezione. Si trova la variazione di massa all’interno della bombola
grazie al bilancio integrale di massa nel volume della bombola \(V\)
(volume di controllo, fisso),
\$\(\dfrac{d M}{d t} = \dfrac{d}{d t} \int_V \rho = -\oint_S \rho \bm{u} \cdot \bm{u} =
 \rho_{in} S_{in} U \ ,\)\( dove si è indicato con \)M\( la massa totale,
\)S\_\{in\}\( l'area della sezione del tubo utilizzato per riempire la
bombola e \)\textbackslash{}rho\_\{in\}\(, la densità sulla sezione di ingresso, dove sono
note la pressione \)P\_\{in\}\( e la temperatura \)T\_\{in\}\(. Ipotizzando che
valga la legge di stato dei gas perfetti, la densità sulla sezione di
ingresso vale \)\(\rho_{in} = \dfrac{P_{in}}{R T_{in}} \ ,\)\( dove
\)R=287 J/(kg\textbackslash{} K)\( è la costante dei gas per l'aria. La derivata nel
tempo della massa d'aria nella bombola vale quindi
\)\(\frac{d M}{d t} = 6.0 \cdot 10^{-5} \dfrac{kg}{s} \ .\)\( Supponendo che
la densità dell'aria si uniforme all'interno della bombola, si può
calcolare la sua derivata nel tempo,
\)\(\dfrac{d \rho}{d t} = \dfrac{1}{V} \dfrac{d}{dt} \int_V \rho = 2.0 \cdot 10^{-3} \dfrac{kg}{m^3 s} \ .\)\$

\sphinxstepscope


\subsection{Exercise 4.4}
\label{\detokenize{polimi/fluidmechanics-ita/template/capitoli/04_bilanci/0402in:exercise-4-4}}\label{\detokenize{polimi/fluidmechanics-ita/template/capitoli/04_bilanci/0402in:fluid-mechanics-balances-ex-04}}\label{\detokenize{polimi/fluidmechanics-ita/template/capitoli/04_bilanci/0402in::doc}}
\sphinxAtStartPar
+:———————————:+:———————————:+
| Un getto d’acqua                  | \sphinxincludegraphics{{polimi/fluidmechanics-ita/template/capitoli/04_bilanci/fig/coanda}.eps}\{width= |
| (\(\rho=999\ kg/m^3\)) stazionario, | «90\%»\}                            |
| piano e orizzontale viene         |                                   |
| indirizzato su un cilindro,       |                                   |
| lambendone la superficie e        |                                   |
| venendo deviato di un angolo      |                                   |
| \(\alpha =15^\circ\). Determinare   |                                   |
| la forza agente su una porzione   |                                   |
| del cilindro di lunghezza pari a  |                                   |
| \(H = 2\ m\), dovuta sia al getto   |                                   |
| d’acqua, sia all’aria             |                                   |
| circostante, sapendo che:         |                                   |
|                                   |                                   |
| \sphinxhyphen{}   il fluido che circonda il     |                                   |
|     getto e il cilindro è aria in |                                   |
|     quiete a pressione            |                                   |
|     atmosferica di \(101325\ Pa\);  |                                   |
|                                   |                                   |
| \sphinxhyphen{}   la larghezza del getto è      |                                   |
|     \(h=2\ cm\);                    |                                   |
|                                   |                                   |
| \sphinxhyphen{}   la portata d’acqua per unità  |                                   |
|     di lunghezza nel getto è      |                                   |
|     \(Q = 199\ kg\ m^{-1}\ s^{-1}\) |                                   |
| .                                 |                                   |
|                                   |                                   |
| Sufficientemente lontano dal      |                                   |
| cilindro, il profilo di velocità  |                                   |
| sulle sezioni del getto è         |                                   |
| uniforme. Illustrare tutte le     |                                   |
| ipotesi semplificative adottate   |                                   |
| nella risoluzione dell’esercizio. |                                   |
|                                   |                                   |
| (\(\bm{F} = 1026\ \hat{\bm{x}} - 1 |                                   |
| 35\ \hat{\bm{y}} \ N\))            |                                   |
+———————————–+———————————–+

\sphinxAtStartPar
Bilanci integrali di massa e quantità di moto. Equazioni di equilibrio
(equazioni fondamentali della dinamica classica). Principio di azione e
reazione. Integrale della normale su una superficie chiusa è
identicamente nullo. Effetto Coanda (esempio della bustina da té sotto
il rubinetto).

\sphinxAtStartPar
Vengono fatte alcune ipotesi: il problema stazionario; attorno al getto
e al solido, l’aria è in quiete con pressione uniforme \(p_a\); il profilo
di velocità è uniforme sulle sezioni del getto considerate nelle
equazioni di bilancio.

\sphinxAtStartPar
Partendo dalle equazioni di bilancio per il volume di controllo \(V_{f}\)
occupato dal fluido, rielaborando il termine degli sforzi di superficie
sforzi di superficie, si ricava la risultante \(\bm{R}\) agente sul solido
in funzione del flusso di quantità di moto del fluido attraverso la
superficie \(S_{f} = \partial V_f\).

\sphinxAtStartPar
Innanzitutto viene ricavata l’espressione della risultante \(\bm{R}\)
agente sul solido.
\begin{itemize}
\item {} 
\sphinxAtStartPar
Vengono scritte le equazioni di bilancio per il fluido, considerando
il volume \(V_f\) \$\$\textbackslash{}begin\{cases\}
\textbackslash{}dfrac\{d\}\{d t\} \textbackslash{}displaystyle\textbackslash{}int\_\{V\_f\} \textbackslash{}rho + \textbackslash{}oint\_\{S\_f\} \textbackslash{}rho \textbackslash{}bm\{u\} \textbackslash{}cdot \textbackslash{}hat\{\textbackslash{}bm\{n\}\} = 0 \& \textbackslash{}qquad \textbackslash{}text\{(massa)\} \textbackslash{}
\textbackslash{}dfrac\{d\}\{d t\} \textbackslash{}displaystyle\textbackslash{}int\_\{V\_f\} \textbackslash{}rho \textbackslash{}bm\{u\} + \textbackslash{}oint\_\{S\_f\} \textbackslash{}rho \textbackslash{}bm\{u\} \textbackslash{}bm\{u\} \textbackslash{}cdot \textbackslash{}hat\{\textbackslash{}bm\{n\}\} =
\textbackslash{}oint\_\{S\_f\} \textbackslash{}bm\{t\_n\} = 0

\begin{sphinxVerbatim}[commandchars=\\\{\}]
    \PYGZam{} \PYGZbs{}qquad \PYGZbs{}text\PYGZob{}(quantità di moto)\PYGZcb{}  \PYGZpc{}\PYGZbs{}Rb\PYGZca{}\PYGZob{}ext\PYGZcb{}
  \PYGZbs{}end\PYGZob{}cases\PYGZcb{}\PYGZdl{}\PYGZdl{}
\end{sphinxVerbatim}

\item {} 
\sphinxAtStartPar
Viene introdotta l’ipotesi di stazionarietà del fenomeno,
\(\frac{d}{dt}\equiv 0\). La risultante degli sforzi viene scritta
come somma degli sforzi di pressione e degli sforzi viscosi,
\$\$\textbackslash{}begin\{split\}

\sphinxAtStartPar
\& \textbackslash{}oint\_\{S\_f\} \textbackslash{}rho \textbackslash{}bm\{u\} \textbackslash{}bm\{u\} \textbackslash{}cdot \textbackslash{}hat\{\textbackslash{}bm\{n\}\}
= \textbackslash{}oint\_\{S\_\{f\}\}  \{\textbackslash{}bm\{t\}\}\sphinxstyleemphasis{\{\textbackslash{}bm\{n\}\} =
\textbackslash{}oint}\{S\_\{f\}\}  \{\textbackslash{}bm\{s\}\}\sphinxstyleemphasis{\{\textbackslash{}bm\{n\}\} \sphinxhyphen{} \textbackslash{}oint}\{S\_f\} p \{\textbackslash{}hat\{\textbackslash{}bm\{n\}\}\}\_\{f\} \textbackslash{} .
\textbackslash{}end\{split\}\$\$

\item {} 
\sphinxAtStartPar
Viene manipolato il termine degli sforzi di superficie. Il contorno
\(S_f\) del volume fluido viene scomposto come unione della superficie
a contatto con il solido \(S_{fs}\), delle superfici «laterali»
\(S_{f\ell}\) (attraverso le quali non c’è flusso di quantità
meccaniche, poichè \(\bm{u}\cdot\bm{\hat{n}} = 0\)) a contatto con
l’aria in quiete e le sezioni «di ingresso» \(S_{f,1}\) e «di uscita»
\(S_{f,2}\) sulle quali la velocità è uniforme, utilizzate per i
bilanci integrali per il volume fluido. Viene indicata con
\(\bm{\hat{n}_f}\) la normale uscente dal volume \(V_f\). Il contorno
\(S_s\) del solido viene scomposto come unione della superficie a
contatto con il fluido \(S_{sf}\) e della superficie \(S_{s\ell}\) a
contatto con l’aria in quiete. Viene indicata con \(\bm{\hat{n}_s}\)
la normale uscente dal volume \(V_s\). In questo modo, la superficie
\(S_{fs}\) coincide con la superficie \(S_{sf}\), a meno della normale
invertita, \(\bm{\hat{n}_f} = \bm{\hat{n}_s}\). Su queste superfici,
per il terzo principio della dinamica, lo sforzo \({\bm{t_n}}_{sf}\)
agente sul solido dovuto al fluido è uguale e contrario allo sforzo
\({\bm{t_n}}_{fs}\) agente sul fluido dovuto al fluido,
\({\bm{t_n}}_{sf}=-{\bm{t_n}}_{fs}\). La superficie formata
dall’unione
\(S_{f\ell} \cup S_{f,1} \cup S_{f,2} \cup S_{s\ell} =:S_{ext}\) è una
superficie chiusa con normale uscente \(\bm{\hat{n}}\) uguale a
\(\bm{\hat{n}_f}\) sulle prime tre superfici e uguale a \(\bm{\hat{n}}\)
su \(S_{s\ell}\). Lo sforzo agente su \(S_{ext}\) è uguale a
\(-p_a\bm{\hat{n}}\), poiché le superfici libere sono a contatto con
aria in quiete con pressione \(p_a\) e le traiettorie delle particelle
rettilinee (senza curvatura%
\begin{footnote}[1]\sphinxAtStartFootnote
Vedi commento sull’equazione della quantità di moto e sulle
traiettorie delle particelle
%
\end{footnote}) sulle sezioni \(S_{f,1}\) e
\(S_{f,2}\).
\begin{align*}\!\begin{aligned}
\begin{aligned}
      \oint_{S_f} \bm{t_n} & = 
      \int_{S_{f\ell}} \bm{t_n} + \int_{S_{f,1+2}} \bm{t_n} + \int_{S_{fs}} \bm{t_n} = & \text{($\bm{t_n} |_{S_{f\ell},S_{f,1+2}} = -p_a \bm{\hat{n}_f}$ )}\\
      & = - \int_{S_{f\ell}\cup S_{f,1+2}} p_a \bm{\hat{n}_f} + \int_{S_{fs}} \bm{t_n} = & \text{(somma e sottrazione di $\int_{S_{fs}} p_a \bm{\hat{n}_f}$)}\\
      & = \underbrace{- \int_{S_{f\ell}\cup S_{f,1+2}} p_a \bm{\hat{n}_f} - \int_{S_{fs}} p_a \bm{\hat{n}_f}}_{-\oint_{S_f} p_a \bm{\hat{n}_f}=0}
      + \int_{S_{fs}} p_a \bm{\hat{n}_f} + \int_{S_{fs}} \bm{t_n} = & \text{($\bm{\hat{n}_f} = -\bm{\hat{n}_s}$, ${\bm{t_n}}_{fs} = - {\bm{t_n}}_{sf}$ su $S_{fs}$)} \\
      & = - \int_{S_{sf}} p_a \bm{\hat{n}}_{s} - \int_{S_{sf}} {\bm{t_n}}_{sf} = &
       \text{($\oint_{S_s=S_{sf}\cup S_{s\ell}} p_a \bm{\hat{n}_s} = 0)$} \\
      & = + \int_{S_{s\ell}} p_a \bm{\hat{n}}_{s} - \int_{S_{sf}} {\bm{t_n}}_{sf} = &
       \text{(${\bm{t_n}}_s = -p_a\bm{\hat{n}_s}$ su $S_{s\ell}$} \\
      & = - \int_{S_{s\ell}} {\bm{t_n}}_{s} - \int_{S_{sf}} {\bm{t_n}}_{sf} = - \oint_{S_{s}} {\bm{t_n}}_{s} = \\
    %  \text{($S_{cyl} = S_c \cup S_{c_l}$ e $\int_{S_{cyl}} p_a \bm{n} = 0$)}\\
    %  & = \int_{{S_c}_l} p_a \bm{n}_{cyl} + \int_{S_c} \bm{t_n} = &
    %  \text{($\bm{t}_{\bm{n}_{s}}|_{S_{c_l}} = -p_a \bm{n}_{cyl}$, $\bm{t}_{\bm{n}_{cyl}}|_{S_c} = - \bm{t_n}$)} \\
    %  & = - \int_{{S_c}_l} \bm{t}_{\bm{n}_{cyl}} - \int_{S_c} \bm{t}_{\bm{n}_{cyl}} = \\
    %  & = - \int_{S_{cyl}} \bm{t}_{\bm{n}_{cyl}} \\
      & = - \bm{R} \ ,
    \end{aligned}$$ dove $\bm{R}$ è la risultante degli sforzi di
    superficie agente sul solido. In questo esercizio è il contributo
    delle forze di volume (ad esempio il peso) agenti sul solido.\\
-   Sostituendo nell'equazione del bilancio della quantità di moto si
    ottiene:
    $$\bm{R} = - \oint_{S_f} \rho \bm{u} \bm{u} \cdot \hat{\bm{n}}\\
\end{aligned}\end{align*}
\item {} 
\sphinxAtStartPar
Considerando solo le superfici di \(V_f\) attraverso le quali c’è un
flusso non nullo di quantità di moto, la risultante delle forze
diventa
\$\(\bm{R} = - \int_{S_{f,1}} \rho \bm{u} \bm{u} \cdot \hat{\bm{n}} 
          - \int_{S_{f,2}} \rho \bm{u} \bm{u} \cdot \hat{\bm{n}}\)\(
dove le quantità all'interno degli integrali sono riferite alle
superfici di integrazione. Sulle sezioni \)S\_\{f,1\}\(, \)S\_\{f,2\}\( la
velocità è uniforme con modulo \)U\( (dalla continuità, la velocità
sulle due sezioni è uguale poichè l'area delle due sezioni è uguale)
diretta lungo la linea media del getto. Le componenti cartesiane
della risultante \)\textbackslash{}bm\{R\}\( sono \)\(\begin{split}
  & R_x = \frac{Q^2 H}{\rho h} \sin \alpha \\
  & R_y = - \frac{Q^2 H}{\rho h} (1-\cos \alpha) \ ,
\end{split}\)\$ riferite agli assi rappresentati in figura.

\end{itemize}


\bigskip\hrule\bigskip


\sphinxstepscope


\subsection{Exercise 4.5}
\label{\detokenize{polimi/fluidmechanics-ita/template/capitoli/04_bilanci/0409in:exercise-4-5}}\label{\detokenize{polimi/fluidmechanics-ita/template/capitoli/04_bilanci/0409in:fluid-mechanics-balances-ex-05}}\label{\detokenize{polimi/fluidmechanics-ita/template/capitoli/04_bilanci/0409in::doc}}
\sphinxAtStartPar
+:———————————:+:———————————:+
| Un getto d’acqua                  | !{[}image{]}(./fig/gettoPiattello.eps |
| (\(\rho=999\ kg/m^3\)) stazionario, | )\{width=»90\%»\}                    |
| piano e verticale viene           |                                   |
| indirizzato su un oggetto di      |                                   |
| massa \(M\), tenuto da esso in      |                                   |
| equilibrio. Il getto ha           |                                   |
| distribuzione di velocità         |                                   |
| uniforme \(U\) lungo lo spessore    |                                   |
| \(H\), mentre la distribuzione sul  |                                   |
| bordo dell’oggetto è triangolare  |                                   |
| di spessore \(h\) con velocità      |                                   |
| massima \(V\). Si calcoli la        |                                   |
| velocità \(V\) e la massa \(M\)       |                                   |
| dell’oggetto supponendo che:      |                                   |
|                                   |                                   |
| \sphinxhyphen{}   il fluido che circonda il     |                                   |
|     getto e il solido è aria in   |                                   |
|     quiete a pressione            |                                   |
|     atmosferica di                |                                   |
|     \(P_a = 101325\  Pa\);          |                                   |
|                                   |                                   |
| \sphinxhyphen{}   si possa trascurare la        |                                   |
|     gravità nel bilancio di       |                                   |
|     quantità di moto, ma non      |                                   |
|     nell’equilibrio del corpo.    |                                   |
|                                   |                                   |
| (\(V = U H / h ; M = \rho U^2 H /  |                                   |
| g\))                               |                                   |
+———————————–+———————————–+

\sphinxAtStartPar
Bilanci integrali di massa e quantità di moto. Equazioni di equilibrio
(equazioni fondamentali della dinamica classica). Principio di azione e
reazione. Integrale della normale su una superficie chiusa è
identicamente nullo.

\sphinxAtStartPar
Ipotesi: problema stazionario; sulla superficie libera del corpo e del
fluido agisce solo la pressione ambiente \(p_a\); nessun effetto della
gravità nei bilanci del fluido.

\sphinxAtStartPar
Si sceglie un asse \(y\) diretto verso l’alto.
\begin{itemize}
\item {} 
\sphinxAtStartPar
Scrittura delle equazioni di bilancio per il fluido.
\begin{equation*}
\begin{split}\begin{cases}
           \frac{d}{d t} \int_{\Omega} \rho + \oint_{\partial \Omega} \rho \bm{u} \cdot \hat{\bm{n}} = 0 & \qquad \text{(massa)} \\
           \frac{d}{d t} \int_{\Omega} \rho \bm{u} + \oint_{\partial \Omega} \rho \bm{u} \bm{u} \cdot \hat{\bm{n}} +
            \oint_{\partial \Omega} p \hat{\bm{n}} - \oint_{\partial \Omega} \bm{s_n} 
            -\int_V \rho \bm{g} = 0  
            & \qquad \text{(quantità di moto)}  %\Rb^{ext}
          \end{cases}\end{split}
\end{equation*}
\sphinxAtStartPar
A queste, va aggiunta l’equazione di equilibrio del corpo sottoposto
alla forza di gravità: \(\bm{F} + M \bm {g} = 0\).

\item {} 
\sphinxAtStartPar
Dopo aver semplificato il bilancio di massa, da esso si ricava la
velocità \(V\). La velocità sui due bordi “di uscita” è
\(v(s) = V s/h\), avendo chiamato \(s\) la coordinata che descrive tale
superficie per valori compresi tra \(0\) e \(h\).
\$\$0 = \textbackslash{}int\_\{S\_in\} \textbackslash{}rho \textbackslash{}bm\{u\} \textbackslash{}cdot \textbackslash{}bm\{\textbackslash{}hat\{n\}\} +   \textbackslash{}int\_\{S\_\{out1\}\} \textbackslash{}rho \textbackslash{}bm\{u\} \textbackslash{}cdot \textbackslash{}bm\{\textbackslash{}hat\{n\}\} +
\textbackslash{}int\_\{S\_\{out2\}\} \textbackslash{}rho \textbackslash{}bm\{u\} \textbackslash{}cdot \textbackslash{}bm\{\textbackslash{}hat\{n\}\} =
\begin{itemize}
\item {} 
\sphinxAtStartPar
\textbackslash{}rho U H + 2 \textbackslash{}int\_0\textasciicircum{}h \textbackslash{}rho V \textbackslash{}frac\{s\}\{h\} ds = \textbackslash{}rho \textbackslash{}displaystyle\textbackslash{}left{[}

\item {} 
\sphinxAtStartPar
U H + 2 \textbackslash{}frac\{1\}\{2\} V h\textbackslash{}right{]}\$\$

\end{itemize}

\sphinxAtStartPar
E quindi \(V = U \frac{H}{h}\).

\item {} 
\sphinxAtStartPar
Le equazioni vengono opportunamente semplificate secondo le ipotesi
fatte (vengono eliminati i termini non stazionari e il termine
contenente le forze di volume \sphinxhyphen{} gravità). Il bordo del dominio
fluido \(\partial \Omega\) viene indicato con \(S_f\). I contributi di
pressione e viscosi vengono raccolti nel «vettore di sforzo»
complessivo.
\begin{equation*}
\begin{split}\begin{split}
    % & \Rb = \oint_{S_{s}}  {\bm{t}}_{\bm{n}} = 
    % \oint_{S_{s}}  {\bm{s}}_{\bm{n}} - \oint_{S_{s}} p {\hat{\bm{n}}}_{s} \\
     & \oint_{S_f} \rho \bm{u} \bm{u} \cdot \hat{\bm{n}}= 
     \oint_{S_{f}}  {\bm{s}}_{\bm{n}} - \oint_{S_f} p {\hat{\bm{n}}}
      = \oint_{S_{f}}  {\bm{t}}_{\bm{n}} 
    \end{split}\end{split}
\end{equation*}
\item {} 
\sphinxAtStartPar
Riscrittura del termine di contorno. Si indica con \(S_f\) il contorno
fluido: questo è costituito dall’unione del controno a contatto con
il corpo \(S_c\) e quella «libera» \(S_l\). Il contorno del corpo
\(S_{s}\) è suddiviso nel contorno \(S_c\) a contatto con il fluido e
nel contorno libero \(S_{c_l}\).

\sphinxAtStartPar
Nei passaggi successivi si ricava il legame tra sforzi sul contorno
del dominio fluido e la forza agente sul corpo. Si usano le ipotesi
che sulle superfici libere agisca solo la pressione ambiente. Si usa
il fatto che l’integrale di una quantità costante per la normale su
una superficie chiusa è nullo. Vengono definite le normali \(\bm{n}\)
e \(\bm{n_s}\) come la normale uscente dal volume del fluido e quella
uscente dal solido. Si definiscono \(\bm{t}_{\bm{n}}\) e
\(\bm{t}_{\bm{n}_{s}}\) come lo sforzo agente sul fluido e quello
agente sul solido. Si usa infine il fatto che \(\bm{n}=-\bm{n}_{s}\)
(normali uscenti dai due domini, uguali e contrarie) e
\(\bm{t_n}=-\bm{t}_{\bm{n}_s}\) sulla superficie in comune (sforzi
agenti sulla superficie comune, uguali e contrari; principio di
azione e reazione).
\begin{equation*}
\begin{split}\begin{aligned}
      \oint_{S_f} \bm{t_n} & = 
      \int_{S_l} \bm{t_n} + \int_{S_c} \bm{t_n} = & \text{($\bm{t_n} |_{S_l} = -p_a \bm{n}$ )}\\
      & = - \int_{S_l} p_a \bm{n} + \int_{S_c} \bm{t_n} = & \text{(somma e sottrazione di $\int_{S_c} p_a \bm{n}$)}\\
      & = \underbrace{- \int_{S_l} p_a \bm{n} - \int_{S_c} p_a \bm{n}}_{=0}
      + \int_{S_c} p_a \bm{n} + \int_{S_c} \bm{t_n} = & \text{($\bm{n} = -\bm{n}_{s}$)} \\
      & = - \int_{S_c} p_a \bm{n}_{s} + \int_{S_c} \bm{t_n} = &
      \text{($S_{s} = S_c \cup S_{c_l}$ e $\int_{S_{s}} p_a \bm{n} = 0$)}\\
      & = \int_{{S_c}_l} p_a \bm{n}_{s} + \int_{S_c} \bm{t_n} = &
      \text{($\bm{t}_{\bm{n}_{s}}|_{S_{c_l}} = -p_a \bm{n}_{s}$, $\bm{t}_{\bm{n}_{s}}|_{S_c} = - \bm{t_n}$)} \\
      & = - \int_{{S_c}_l} \bm{t}_{\bm{n}_{s}} - \int_{S_c} \bm{t}_{\bm{n}_{s}} = \\
      & = - \int_{S_{s}} \bm{t}_{\bm{n}_{s}} \\
      & = - \bm{R}
    \end{aligned}\end{split}
\end{equation*}
\item {} 
\sphinxAtStartPar
Sostituendo nell’equazione del bilancio della quantità di moto si
ottiene:
\$\(\bm{R} = - \oint_{S_f} \rho \bm{u} \bm{u} \cdot \hat{\bm{n}}\)\$

\item {} 
\sphinxAtStartPar
Data la simmetria del problema si riconosce che non ci può essere
una componente orizzontale. I contributi nel bilancio della quantità
di moto sulla superficie di contatto tra corpo e fluido e sulla
superficie laterale del getto sono nulli poichè è nullo il flusso su
tali superfici. I contributi sulle sezioni “di uscita” sono uguali e
contrari. Rimane quindi solo il contributo dalla sezione “in
ingresso”.
\begin{equation*}
\begin{split}\bm{F} = - \oint_{S_f} \rho \bm{u} \bm{u} \cdot \bm{\hat{n}} = 
               - \oint_{S_in} \rho \bm{u} \bm{u} \cdot \bm{\hat{n}} = 
               \rho U^2 H \bm{\hat{y}}\end{split}
\end{equation*}
\item {} 
\sphinxAtStartPar
Si scrive l’equilibrio del corpo \(\bm{F} + M \bm{g} = 0\), con
\(\bm{g} = - g \bm{\hat{y}}\). Da questo segue che
\(M = F/g = \frac{\rho U^2 H}{g}\).

\end{itemize}

\sphinxAtStartPar
\sphinxstyleemphasis{Osservazioni.} Nell’elaborazione dei termini della quantità di moto è
contenuta la forma della risultante delle forze sull’oggetto vista in
classe.

\sphinxAtStartPar
Come giustamente osservato da qualcuno in classe, la massa è per unità
di lunghezza, poichè stiamo considerando un caso bidimensionale.

\sphinxstepscope


\subsection{Exercise 4.6}
\label{\detokenize{polimi/fluidmechanics-ita/template/capitoli/04_bilanci/0403in:exercise-4-6}}\label{\detokenize{polimi/fluidmechanics-ita/template/capitoli/04_bilanci/0403in:fluid-mechanics-balances-ex-06}}\label{\detokenize{polimi/fluidmechanics-ita/template/capitoli/04_bilanci/0403in::doc}}
\sphinxAtStartPar
+:———————————:+:———————————:+
| Il motore a getto in figura è     | !{[}image{]}(./fig/motore\_a\_getto.eps |
| alimentato con una portata        | )\{width=»90\%»\}                    |
| \(\dot{m}_c =                      |                                   |
| 1.1\ kg/s\) di carburante liquido  |                                   |
| iniettato in direzione ortogonale |                                   |
| all’asse del motore. Calcolare la |                                   |
| spinta \(T\) del motore ipotizzando |                                   |
| che:                              |                                   |
|                                   |                                   |
| \sphinxhyphen{}   il carburante vaporizzi e     |                                   |
|     diffonda completamente;       |                                   |
|                                   |                                   |
| \sphinxhyphen{}   le sezioni di ingresso e      |                                   |
|     uscita abbiano area uguale e  |                                   |
|     pari ad \(A = 0.5\ m^2\);       |                                   |
|                                   |                                   |
| \sphinxhyphen{}   sia l’aria in ingresso che i  |                                   |
|     gas di scarico siano a        |                                   |
|     pressione atmosferica         |                                   |
|     \(P_{atm}=26400\ Pa\);          |                                   |
|                                   |                                   |
| \sphinxhyphen{}   la velocità di ingresso e di  |                                   |
|     uscita siano uniformi sulle   |                                   |
|     rispettive sezioni;           |                                   |
|                                   |                                   |
| \sphinxhyphen{}   siano note la densità         |                                   |
|     dell’aria in ingresso         |                                   |
|     \(\rho_1 = 0.42\,              |                                   |
|           kg/m^3\), la velocità di |                                   |
|     ingresso \(V_1 = 240\ m/s\) e   |                                   |
|     la velocità di efflusso       |                                   |
|     \(V_2 = 980\ m/s\).             |                                   |
|                                   |                                   |
| (\(T = -38374\hat{\bm{x}}\ N\))     |                                   |
+———————————–+———————————–+
\begin{equation*}
\begin{split}T = \rho V_1 A (V_2-V_1) + V_2 \dot{m}_c \ .\end{split}
\end{equation*}
\sphinxAtStartPar
Bilanci integrali di massa e quantità di moto. \$\(\begin{cases}
  \frac{d}{dt} \int_V \rho = -\oint_{\partial V} \rho \bm{u} \cdot \hat{\bm{n}}  & \text{(massa)} \\
  \frac{d}{dt} \int_V \rho \bm{u} = -\oint_{\partial V} \rho \bm{u} \bm{u} \cdot \hat{\bm{n}}
  +\int_V \bm{f} - \oint_{\partial V} p \hat{\bm{n}} + \oint_{\partial V} \bm{s_n} & \text{(quantità di moto)}
\end{cases}\)\$

\sphinxAtStartPar
Ipotesi: Regime stazionario. Fluido non viscoso (?). Profilo costante di
velocità. No gravità.
\begin{itemize}
\item {} 
\sphinxAtStartPar
Scrittura dei bilanci integrali con le semplificazioni opportune,
derivanti dalle ipotesi. \$\(\begin{cases}
      \oint_{\partial V} \rho \bm{u} \cdot \hat{\bm{n}} = 0  & \text{(massa)} \\
      \oint_{\partial V} \rho \bm{u} \bm{u} \cdot \hat{\bm{n}} = \oint_{\partial V} \bm{t_n} & \text{(quantità di moto)}
     \end{cases}\)\$

\item {} 
\sphinxAtStartPar
Ulteriore semplificazione usando l’ipotesi di profili di velocità
uniformi \$\(\begin{cases}
      - \rho_1 V_1 A_1 -\dot{m}_c + \rho_2 V_2 A_2 = 0  \\
      - \rho_1 \vec{V_1} V_1 A_1 + \rho_2 \vec{V_2} V_2 A_2 - \dot{m}_c \vec{v}_c = \oint_{S1\cup S2\cup S3} \bm{t_n}
     \end{cases}\)\$

\item {} 
\sphinxAtStartPar
Relazione tra l’integrale della pressione e la risultante delle
forze agenti sul gomito, sfruttando il fatto che l’integrale della
normale su tutta la superficie è identicamente nullo. Si
identificano con \(S_1\) la superficie di ingresso, \(S_2\) la
superficie di uscita, \(S_3\) la superficie laterale interna del
motore, \(S_{3_o}\) la superficie laterale esterna del motore.
\$\(\begin{aligned}
      \displaystyle\oint_{S_1\cup S_2\cup S_3} \bm{t_n} & = \displaystyle\oint_{S_1\cup S_2\cup S_3} \bm{t_n} + \underbrace{\displaystyle\oint_{S_1\cup S_2\cup S{3_o}} p_a \hat{\bm{n}}}_{=0} = \\
      & = -\int_{S_1} (p-p_a) \hat{\bm{n}} - \int_{S_2} (p-p_a) \hat{\bm{n}} + \int_{S_{3_o}} p_a \hat{\bm{n}} + \int_{S_3} \bm{t_n}  = \qquad(p|_{S_1} = p|_{S_2} = p_a) \\
      & = \int_{S_{3_o}} p_a \hat{\bm{n}} + \int_{S_3} \bm{t_n} = \\
      & = \oint_{S_{eng}} \bm{t_n} = - \vec{F}
     \end{aligned}\)\$

\item {} 
\sphinxAtStartPar
L’equazione della quantità di moto diventa quindi:
\$\(- \rho_1 \vec{V_1} V_1 A_1 + \rho_2 \vec{V_2} V_2 A_2 - \dot{m}_c \vec{v}_c = - \vec{F}\)\$

\item {} 
\sphinxAtStartPar
Mettendo a sistema l’equazione del bilancio di massa e la proiezione
in direzione orizzontale dell’equazione della quantità di moto (si
assume che l’iniezione del combustibile, e quindi \(\bm{v}_c\), sia
perpendicolare all’asse x e quindi non compare nel bilancio della
quantità di moto in direzione x): \$\(\begin{cases}
    \rho_2 V_2 A = \rho_1 V_1 A + \dot{m}_c \\
    -\rho_1 V_1^2 A + \rho_2 V_2^2 A = -F_x 
  \end{cases}\)\$

\sphinxAtStartPar
Si ottiene
\begin{equation*}
\begin{split}\begin{aligned}
        F_x & = \rho_1 V_1^2 A - \rho_2 V_2^2 A = \\
            & = \rho_1 V_1^2 A - (\rho_2 V_2 A) V_2 = \\
            & = \rho_1 V_1^2 A - V_2 (\rho_1 V_1 A + \dot{m}_c) = \\
            & = \rho_1 V_1 A (V_1 - V_2) - V_2 \dot{m}_c
      \end{aligned}\end{split}
\end{equation*}
\sphinxAtStartPar
E la spinta coincide con la componente lungo x appena calcolata:
\$\(T = \rho_1 V_1 A (V_2 - V_1) + V_2 \dot{m}_c\)\$

\sphinxAtStartPar
La spinta risulta quindi: \(T = -F_x = 38374N\).

\sphinxAtStartPar
\sphinxstyleemphasis{Interpretazione dei risultati e osservazioni.}

\sphinxAtStartPar
In prima approssimazione, la spinta in un motore a getto è una
funzione della portata d’aria e della differenza di velocità tra
ingresso e uscita. Spesso in molte applicazioni il termine
\(\dot{m}_c\) è trascurabile.

\sphinxAtStartPar
Ragionare in questo caso sulla validità dell’approssimazione
\(\bm{t_n} = -p\bm{\hat{n}}\) nella definizione della risultante delle
forze sul motore.

\end{itemize}

\sphinxstepscope


\subsection{Exercise 4.7}
\label{\detokenize{polimi/fluidmechanics-ita/template/capitoli/04_bilanci/0404in:exercise-4-7}}\label{\detokenize{polimi/fluidmechanics-ita/template/capitoli/04_bilanci/0404in:fluid-mechanics-balances-ex-07}}\label{\detokenize{polimi/fluidmechanics-ita/template/capitoli/04_bilanci/0404in::doc}}

\bigskip\hrule\bigskip


\sphinxAtStartPar
(\(\bm{F} = -1765.03\hat{\bm{x}} + 1765.03\hat{\bm{y}}\  N\))   \sphinxincludegraphics{{polimi/fluidmechanics-ita/template/capitoli/04_bilanci/fig/gomito_01}.eps}\{width=»70\%»\}


\bigskip\hrule\bigskip


\sphinxAtStartPar
Bilanci integrali di massa e quantità di moto. … \$\(\begin{cases}
  \frac{d}{dt} \int_V \rho = -\oint_{\partial V} \rho \bm{u} \cdot \hat{\bm{n}}  & \text{(massa)} \\
  \frac{d}{dt} \int_V \rho \bm{u} = -\oint_{\partial V} \rho \bm{u} \bm{u} \cdot \hat{\bm{n}}
  +\int_V \bm{F} - \oint_{\partial V} p \hat{\bm{n}} + \oint_{\partial V} {\bm{s_n}} & \text{(quantità di moto)}
\end{cases}\)\$

\sphinxAtStartPar
Vengono fatte alcune ipotesi: regime stazionario, fluido incomprimibile,
fluido non viscoso, profili costanti di velocità, no gravità. Si
scrivono i bilanci integrali semplificati, si riconoscono in essi e si
calcolano le azioni scambiate con il corpo.
\begin{itemize}
\item {} 
\sphinxAtStartPar
Scrittura dei bilanci integrali opportunamente semplificati
(ipotesi). \$\(\begin{cases}
      \oint_{\partial V} \rho \bm{u} \cdot \hat{\bm{n}} = 0  & \text{(massa)} \\
      \oint_{\partial V} \rho \bm{u} \bm{u} \cdot \hat{\bm{n}} = \oint_{\partial V} \bm{t_n} & \text{(quantità di moto)}
     \end{cases}\)\$

\item {} 
\sphinxAtStartPar
Ulteriore semplificazione usando l’ipotesi di densità costante e
profili di velocità uniformi \$\(\begin{cases}
      -V_1 A_1 + V_2 A_2 = 0 \qquad \qquad \qquad \Rightarrow  V_1 = V_2 = V \\
      - \rho \vec{V_1} V_1 A_1 + \rho \vec{V_2} V_2 A_2 = \oint_{\partial V} \bm{t_n}
     \end{cases}\)\$

\item {} 
\sphinxAtStartPar
Relazione tra l’integrale degli sforzi sulla superficie e la
risultante delle forze agenti sul gomito, sfruttando il fatto che
l’integrale della normale su tutta la superficie è identicamente
nullo. Si identificano con \(S_1\) la superficie di ingresso, \(S_2\) la
superficie di uscita, \(S_3\) la superficie laterale.
\$\(\begin{aligned}
       \displaystyle\oint_{S_1\cup S_2\cup S_3} \bm{t_n} & =  \displaystyle\oint_{S_1\cup S_2\cup S_3} \bm{t_n} + \underbrace{\displaystyle\oint_{S_1\cup S_2\cup S_3} p_a \hat{n}}_{=0} = \\
      & = -\oint_{S_1} (p-p_a) \hat{n} - \oint_{S_2} (p-p_a) \hat{n} + \underbrace{\oint_{S_3} (\bm{t_n}+p_a \hat{n})}_{=-\bm{F}} =  \\
      & = -\oint_{S_1} (p-p_a) \hat{n} - \oint_{S_2} (p-p_a) \hat{n} - \bm{F} 
     \end{aligned}\)\$

\item {} 
\sphinxAtStartPar
Proiezione lungo i due assi del sistema di riferimento della
risultante delle forze agenti sul gomito (dopo averla inserita
nell’equazione di bilancio della quantità di moto) \$\(\begin{cases}
    F_x = - \rho V^2 A - (p_2 - p_a)A   \\
    F_y =  \rho V^2 A + (p_1 - p_a)A  
  \end{cases}\)\$

\end{itemize}

\sphinxstepscope


\subsection{Exercise 4.7}
\label{\detokenize{polimi/fluidmechanics-ita/template/capitoli/04_bilanci/0405in:exercise-4-7}}\label{\detokenize{polimi/fluidmechanics-ita/template/capitoli/04_bilanci/0405in:fluid-mechanics-balances-ex-07}}\label{\detokenize{polimi/fluidmechanics-ita/template/capitoli/04_bilanci/0405in::doc}}

\bigskip\hrule\bigskip


\sphinxAtStartPar
(\(F_x = 1.876\,10^6\ N\), \(F_y = -6.251\,10^6\ N\))   \sphinxincludegraphics{{polimi/fluidmechanics-ita/template/capitoli/04_bilanci/fig/gomito_galleria}.eps}\{width=»60\%»\}


\bigskip\hrule\bigskip


\sphinxAtStartPar
Bilanci integrali di massa e quantità di moto. \$\(\begin{cases}
  \frac{d}{dt} \int_V \rho = -\oint_{\partial V} \rho \bm{u} \cdot \hat{\bm{n}}  & \text{(massa)} \\
  \frac{d}{dt} \int_V \rho \bm{u} = -\oint_{\partial V} \rho \bm{u} \bm{u} \cdot \hat{\bm{n}}
  +\int_V \bm{f} - \oint_{\partial V} p \hat{\bm{n}} + \oint_{\partial V} {\bm{t}_s} & \text{(quantità di moto)}
\end{cases}\)\$

\sphinxAtStartPar
Vengono fatte alcune ipotesi: regime stazionario, fluido incomprimibile,
fluido non viscoso, profili costanti di velocità, no gravità. Si
scrivono i bilanci integrali semplificati, si riconoscono in essi e si
calcolano le azioni scambiate con il corpo.
\begin{itemize}
\item {} 
\sphinxAtStartPar
Scrittura dei bilanci integrali con le semplificazioni opportune,
derivanti dalle ipotesi. \$\(\begin{cases}
      \oint_{\partial V} \rho \bm{u} \cdot \hat{\bm{n}} = 0  & \text{(massa)} \\
      \oint_{\partial V} \rho \bm{u} \bm{u} \cdot \hat{\bm{n}} = \oint_{\partial V} \bm{t_n} & \text{(quantità di moto)}
     \end{cases}\)\$

\item {} 
\sphinxAtStartPar
Ulteriore semplificazione usando l’ipotesi di densità costante e
profili di velocità uniformi \$\(\begin{cases}
      -V_1 A_1 + V_2 A_2 = 0  & \quad \Rightarrow \quad V_1 A_1 = V_2 A_2 = Q \\
      - \rho \vec{V_1} V_1 A_1 + \rho \vec{V_2} V_2 A_2 = \oint_{\partial V} \bm{t_n}
     \end{cases}\)\$

\item {} 
\sphinxAtStartPar
Relazione tra l’integrale della pressione e la risultante delle
forze agenti sul gomito, sfruttando il fatto che l’integrale della
normale su tutta la superficie è identicamente nullo. Si
identificano con \(S_1\) la superficie di ingresso, \(S_2\) la
superficie di uscita, \(S_3\) la superficie laterale.
\$\(\begin{aligned}
      \displaystyle\oint_{S_1\cup S_2\cup S_3} p \hat{n} & =  \displaystyle\oint_{S_1\cup S_2\cup S_3} \bm{t_n} + \displaystyle\oint_{S_1\cup S_2\cup S_3} p_a \hat{n} = \\
      & = -\oint_{S_1} (p-p_a) \hat{n} - \oint_{S_2} (p-p_a) \hat{n} + \underbrace{\oint_{S_3} (\bm{t_n}+p_a\hat{n})}_{=-\bm{f}}  =  \\
      & = -\oint_{S_1} (p-p_a) \hat{n} - \oint_{S_2} (p-p_a) \hat{n} - \bm{f}
     \end{aligned}\)\$

\item {} 
\sphinxAtStartPar
L’equazione della quantità di moto diventa quindi:
\$\(- \rho \bm{V_1} V_1 A_1 + \rho \bm{V_2} V_2 A_2 = - (p_1 - p_a) A_1 \hat{n}_1 - (p_2 - p_a) A_2 \hat{n}_2 - \bm{F}\)\$

\item {} 
\sphinxAtStartPar
Proiezione lungo i due assi del sistema di riferimento della
risultante delle forze agenti sul gomito. Se si considera \(p_a = 0\),
i risultati numerici sono i seguenti: \$\(\begin{cases}
    F_x = \rho \frac{Q^2}{A_1} + (p_1 - p_a)A_1  & \quad \Rightarrow \quad   F_x = 1.876 \cdot 10^6 N  \\
    F_y = -  \rho \frac{Q^2}{A_2} - (p_2 - p_a)A_2  & \quad \Rightarrow \quad   F_y =-6.250 \cdot 10^6 N
  \end{cases}\)\$

\end{itemize}

\sphinxstepscope


\subsection{Exercise 4.8}
\label{\detokenize{polimi/fluidmechanics-ita/template/capitoli/04_bilanci/0408in:exercise-4-8}}\label{\detokenize{polimi/fluidmechanics-ita/template/capitoli/04_bilanci/0408in:fluid-mechanics-balances-ex-08}}\label{\detokenize{polimi/fluidmechanics-ita/template/capitoli/04_bilanci/0408in::doc}}
\sphinxAtStartPar
+:———————————:+:———————————:+
| Un numero elevato di profili è    | \sphinxincludegraphics{{polimi/fluidmechanics-ita/template/capitoli/04_bilanci/fig/wings}.eps}\{width=» |
| disposto come in figura. Il       | 95\%»\}                             |
| profilo di ingresso è uniforme    |                                   |
| \(\bm{u} = U_\infty \bm{\hat{x}}\), |                                   |
| mentre il profilo di uscita ha    |                                   |
| andamento                         |                                   |
| \(\bm{u} = \beta U_\infty (\cos \t |                                   |
| heta \bm{\hat{x}} - \sin \theta \ |                                   |
| bm{\hat{y}})                      |                                   |
| \sin{\frac{\pi \eta}{H}}\) in ogni |                                   |
| canale (sia \(\eta\) la coordinata  |                                   |
| che descrive la sezione di        |                                   |
| uscita). Sulla sezione di         |                                   |
| ingresso la pressione media vale  |                                   |
| \(P_1\), sulla sezione di uscita    |                                   |
| \(P_2\).                            |                                   |
|                                   |                                   |
| Calcolare il fattore \(\beta\) del  |                                   |
| profilo di velocità in uscita e   |                                   |
| la risultante delle forze (per    |                                   |
| unità di apertura) agente sul     |                                   |
| singolo profilo.                  |                                   |
|                                   |                                   |
| (Risultati:                       |                                   |
| \(\beta = \frac{\pi}{2 \cos \theta |                                   |
| }, \bm{F} = [(P_1 - P_2) H + \rho |                                   |
|  U^2 H ((1-\pi^2/8) ]\bm{\hat{x}} |                                   |
|  + \pi^2/8 \tan \theta \bm{\hat{y |                                   |
| }})\))                             |                                   |
+———————————–+———————————–+

\sphinxAtStartPar
Bilanci integrali di massa e quantità di moto. Equazioni di equilibrio
(equazioni fondamentali della dinamica classica). Principio di azione e
reazione. Integrale della normale su una superficie chiusa è
identicamente nullo. Simmetria.
\begin{itemize}
\item {} 
\sphinxAtStartPar
Ricavare il coefficiente \(\beta\) dal bilancio di massa

\item {} 
\sphinxAtStartPar
Usare le ipotesi di simmetria nel bilancio di quantità di moto per
annullare alcuni termini

\end{itemize}

\sphinxAtStartPar
Si ricava il coefficiente \(\beta\) dal bilancio di massa in forma
integrale. Si utilizza la simmetria del problema nel bilancio di
quantità moto per ricavare le azioni sui profili.

\sphinxstepscope


\subsection{Exercise 4.9}
\label{\detokenize{polimi/fluidmechanics-ita/template/capitoli/04_bilanci/0401in:exercise-4-9}}\label{\detokenize{polimi/fluidmechanics-ita/template/capitoli/04_bilanci/0401in:fluid-mechanics-balances-ex-09}}\label{\detokenize{polimi/fluidmechanics-ita/template/capitoli/04_bilanci/0401in::doc}}
\sphinxAtStartPar
+:———————————:+:———————————:+
| Calcolare la resistenza di un     | \sphinxincludegraphics{{polimi/fluidmechanics-ita/template/capitoli/04_bilanci/fig/airfoil}.eps}\{width |
| profilo immerso in una corrente   | =»90\%»\}                           |
| stazionaria con velocità          |                                   |
| asintotica \({\bm{V}}_\infty\),     |                                   |
| sapendo la distribuzione della    |                                   |
| componente di velocità \(u(y)\)     |                                   |
| parallela a \({\bm{V}}_\infty\) a   |                                   |
| valle del profilo e assumendo     |                                   |
| che:                              |                                   |
|                                   |                                   |
| \sphinxhyphen{}   la pressione statica sul      |                                   |
|     contorno del volume di        |                                   |
|     controllo sia costante e pari |                                   |
|     a quella della corrente       |                                   |
|     indisturbata a monte del      |                                   |
|     profilo;                      |                                   |
|                                   |                                   |
| \sphinxhyphen{}   sul lato superiore e          |                                   |
|     inferiore del volume di       |                                   |
|     controllo sia possibile       |                                   |
|     trascurare la componente      |                                   |
|     lungo l’asse \(x\) della        |                                   |
|     perturbazione della velocità  |                                   |
|     dovuta alla presenza del      |                                   |
|     profilo:                      |                                   |
|     \(\bm{V} = (V_{\infty}+u,v) \s |                                   |
| imeq (V_{\infty},v).\)             |                                   |
|                                   |                                   |
| (\(R = \int_0^{ H}\rho \, u(y) [V_ |                                   |
| {\infty}-u(y)] dy.\))              |                                   |
+———————————–+———————————–+

\sphinxAtStartPar
Bilanci integrali di massa e quantità di moto. Equazioni di equilibrio
(equazioni fondamentali della dinamica classica). Principio di azione e
reazione. Integrale della normale su una superficie chiusa è
identicamente nullo. Esperienza in laboratorio sul \sphinxstyleemphasis{difetto di scia}.

\sphinxAtStartPar
Vengono scritti i bilanci integrali di massa e quantità di moto,
opportunamente semplificati (ipotesi di stazionarietà
\(\frac{d}{dt} \equiv 0\), densità costante \(\rho = \bar{\rho}\), ipotesi
sulle condizioni sul bordo esterno del dominio); all’interno dei bilanci
si possono riconoscere i termini legati alle azioni scambiate dal fluido
con il profilo (l’incognita del problema); si sfrutta infine la
geometria rettangolare del contorno esterno e le ipotesi su di esso per
ottenere una forma ulteriormente semplificata dei bilanci e trovare la
soluzione del problema.
\begin{itemize}
\item {} 
\sphinxAtStartPar
Scrittura e semplificazione dei bilanci di massa e quantità di moto.
\$\(\begin{cases}
       \frac{d}{d t} \int_{\Omega} \rho + \oint_{\partial \Omega} \rho \bm{u} \cdot \hat{\bm{n}} = 0 & \qquad \text{(massa)} \\
       \frac{d}{d t} \int_{\Omega} \rho \bm{u} + \oint_{\partial \Omega} \rho \bm{u} \bm{u} \cdot \hat{\bm{n}} +
        \oint_{\partial \Omega} p \hat{\bm{n}} - \oint_{\partial \Omega} \bm{s_n} = 0  
        & \qquad \text{(quantità di moto)}  %\Rb^{ext}
      \end{cases}\)\$

\sphinxAtStartPar
Nel problema, il controno del dominio fluido \(\partial \Omega\) è
costituito dal bordo rettangolare \(\gamma_\infty\) lontano dal
profilo e dal bordo \(\gamma_p\) coincidente con il profilo stesso. La
forza \(\bm{F}\) agente sul profilo è l’integrale degli sforzi
generati dal fluido (uguali e contrari agli sforzi agenti sul
fluido) sul contorno del profilo. Inoltre si può fare l’ipotesi di
sforzi viscosi nulli e pressione costante sul bordo esterno:
l’integrale sul dominio esterno si riduce all’integrale della
normale su una superficie chiusa ed è quindi nullo. Si può dunque
scrivere:
\$\(\oint_{\partial \Omega} (-p \hat{\bm{n}} + \bm{s_n}) = \oint_{\partial \Omega} \bm{t_n} = \underbrace{\oint_{\gamma_p} \bm{t_n}}_{=-\bm{F}} + \underbrace{\oint_{\gamma_\infty} \bm{t_n}}_{=0} = -\bm{F}\)\$

\sphinxAtStartPar
\sphinxstyleemphasis{Osservazione. A differenza di quanto fatto in classe, non è stata
fatta l’ipotesi di fluido non viscoso; il contributo all’infinito si
annulla con l’ipotesi di pressione costante all’infinito e sforzi
viscosi trascurabili. Per ritrovarsi con gli appunti, sostituire
\(\bm{t_n}\) con \(-p\bm{\hat{n}}\)}.

\sphinxAtStartPar
Dopo aver fatto l’ipotesi di stazionarietà e aver inserito la
definizione di \(\bm{F}\) appena data, le equazioni di bilancio
possono essere scritte come: \$\(\begin{cases}
      & \oint_{\partial \Omega} \rho \bm{u} \cdot \hat{\bm{n}} = 0  \\
      & \bm{F} = - \oint_{\partial \Omega} \rho \bm{u} \bm{u} \cdot \hat{\bm{n}} 
      \end{cases}
    \label{eqn:airfoil_bil_int}\)\$

\sphinxAtStartPar
Il bilancio di quantità di moto può essere scritto esplicitando e
separando le componenti vettoriali. \$\(\begin{aligned}
       F_x\bm{\hat{x}} + F_y\bm{\hat{y}} 
& = - \oint_{\partial \Omega} \rho (u \bm{x} + v \bm{y}) \bm{u} \cdot \hat{\bm{n}} \\
& =       - \bm{\hat{x}} \oint_{\partial \Omega} \rho u \bm{u} \cdot \hat{\bm{n}} -  \bm{\hat{y}} \oint_{\partial \Omega} \rho v  \bm{u} \cdot \hat{\bm{n}}
    \end{aligned}\)\$

\item {} 
\sphinxAtStartPar
Scrittura delle equazioni di bilancio in componenti (sfruttando la
geometria rettangolare del bordo esterno: \(\gamma_1\) indica il bordo
di sinistra, \(\gamma_2\) il bordo inferiore, \(\gamma_3\) quello di
destra, \(\gamma_4\) quello superiore).

\sphinxAtStartPar
\sphinxstyleemphasis{Attenzione: la normale è quella uscente dal dominio fluido. Sul
contorno del profilo, la normale è entrante nel profilo. In più: non
fare confusione tra azioni del profilo agenti sul fluido e azioni
del fluido agenti sul profilo!} \$\(\begin{cases}
      & 0 = \oint_{\partial \Omega} \rho \bm{u} \cdot \hat{\bm{n}} = -\int_{\gamma_1} \rho u
      -\int_{\gamma_2} \rho v +\int_{\gamma_3} \rho u +\int_{\gamma_4} \rho v \\
      & F_x = +\int_{\gamma_1} \rho u^2 +\int_{\gamma_2} \rho u v -\int_{\gamma_3} \rho u^2 -\int_{\gamma_4} \rho u v \\
      & F_y = +\int_{\gamma_1} \rho u v +\int_{\gamma_2} \rho v^2 -\int_{\gamma_3} \rho u v -\int_{\gamma_4} \rho v^2
     \end{cases}\)\$

\item {} 
\sphinxAtStartPar
Ipotesi sulla velocità sui lati orizzontali
(\(u|_{\gamma_2} = u|_{\gamma_4} = V_\infty\) costante), per poter
ulteriormente semplificare il risultato. \$\(\begin{cases}
        \int_{\gamma_2} \rho v  -\int_{\gamma_4} \rho v = -\int_{\gamma_1} \rho u+\int_{\gamma_3} \rho u\\
       F_x = +\int_{\gamma_1} \rho u^2 -\int_{\gamma_3} \rho u^2 + V_\infty \left[ \int_{\gamma_2} \rho v  -\int_{\gamma_4} \rho v \right]
     \end{cases}\)\( E inserendo la prima nella seconda:
\)\(\begin{aligned}
       F_x  & = \int_{\gamma_1} \rho u^2 -\int_{\gamma_3} \rho u^2 + V_\infty \left[-\int_{\gamma_1} \rho u+\int_{\gamma_3} \rho u \right] = \\
       & = \int_{\gamma_1} \rho u (u-V_\infty) + \int_{\gamma_3} \rho u (V_\infty-u) = \quad \text{(\)u|\sphinxstyleemphasis{\{\textbackslash{}gamma\_1\} = V}\textbackslash{}infty  \textbackslash{}Rightarrow \( il primo integrale è nullo)} \\
       & = \int_{\gamma_3} \rho u (V_\infty-u) = \\
       & = \int_{0}^{H} \rho u(y) (V_\infty - u(y)) dy
  \end{aligned}
\label{eqn:difetto_scia}\)\$

\end{itemize}

\sphinxAtStartPar
\sphinxstylestrong{Osservazioni.} Tramite la misura del campo di velocità in galleria è
possibile stimare la resistenza del corpo. Le condizioni di «aria
libera» e in galleria sono diverse. In generale, in galleria il fluido è
confinato dalle pareti di galleria, maggiormente «vincolato». Inoltre
sulle pareti della galleria esiste una condizione di adesione,
\(\bm{u}=\bm{0}\): per la conservazione della massa, il rallentamento del
fluido in corrispondenza delle pareti della galleria viene compensato da
un incremento della velocità nella regione «più lontana» dalla parete,
rispetto a un corpo in aria libera. Per tenere conto di effetti di
\sphinxstylestrong{bloccaggio} dovuti al confinamento in galleria, è necessario compiere
delle correzioni delle misure sperimentali. Agli effetti di bloccaggio,
vanno aggiunti gli effetti di \sphinxstylestrong{galleggiamento} dovuti al gradiente di
pressione lungo la galleria, che danno un effetto di resistenza
aggiuntiva. Inoltre è importante che la dimensione del corpo rispetto
alla dimensione della galleria non sia né «troppo grosso» (per problemi
di “bloccaggio”), né, di solito, «troppo piccolo» (per motivi di
similitudine; ma sarà argomento di puntate successive del corso…). É
importante avere in mente la necessità di prestare attenzione a questi
aspetti, quando vengono svolte attività sperimentali. Ma questo sarà
argomento di altri capitoli o di altri corsi…

\sphinxstepscope


\subsection{Exercise 4.9 \sphinxhyphen{} Experiment: wake defect}
\label{\detokenize{polimi/fluidmechanics-ita/template/capitoli/04_bilanci/0401SciaExp:exercise-4-9-experiment-wake-defect}}\label{\detokenize{polimi/fluidmechanics-ita/template/capitoli/04_bilanci/0401SciaExp:fluid-mechanics-balances-ex-09}}\label{\detokenize{polimi/fluidmechanics-ita/template/capitoli/04_bilanci/0401SciaExp::doc}}


\sphinxAtStartPar
L’esercizio svolto in precedenza risulta propedeutico per l’analisi dei
dati ottenuti tramite alcune attività sperimentali, per ottenere delle
risultanti di forze e momenti da misure del campo di velocità (e
pressione, a volte) tramite i bilanci integrali. Le attività svolte nel
mondo reale sono affette da imprecisioni e incertezze. La
quantificazione (o almeno la stima) dell’incertezza del risultato di
un’attività sperimentale è parte integrante del risultato stesso. I
valori \(x_i, \ i=1:N\) di grandezze misurate possono essere combinati per
calcolare delle grandezze derivate \(f(x_i)\). I \sphinxstyleemphasis{datasheet} che
accompagnano uno strumento raccolgono anche le informazioni sulla sua
incertezza di misura, spesso in forma di intervallo di confidenza o di
scarto quadratico medio. L’incertezza sulle misure sperimentali \(x_i\) si
propaga sul valore della funzione \(f(x_i)\). Nell’ipotesi che le
incertezze di misura sulle variabili \(d_i\) siano tra di loro
indipendenti e non correlate, è possibile utilizzare la \sphinxstylestrong{formula RSS}
(\sphinxstylestrong{root\sphinxhyphen{}sum\sphinxhyphen{}squares}) per la propagazione dell’incertezza. Se la misura
\(x_i\) ha incertezza \(\sigma_{x_i}\), una stima dell’incertezza su \(f\)
vale
\$\(\sigma_f^2 = \sum_{i=1}^{N} \left( \dfrac{\partial f}{\partial x_i} \right)^2 \sigma_{x_i}^2 \ .\)\(
L'incertezza \)\textbackslash{}sigma\textasciicircum{}2\_f\( sulla quantità \)f\$, obiettivo dell’attività
sperimentale, è un indicatore della bontà del metodo sperimentale
utilizzato ed del sistema di miusra disponibile per tale attività. In
generale, l’incertezza sulla grandezza desiderata deve essere «molto
minore» della grandezza stessa: in caso contrario, l’apparato
sperimentale risulterebbe indeguato all’esperimento. Essendo parte
integrante del risultato, è buona regola indicare l’incertezza sui
risultati delle attività sperimentali, ad esempio fornendone il valore
numerico, il valore relativo alla misura o gli intervalli di confidenza
sui grafici.


\subsubsection{Risultante delle forze: bilancio di quantità di moto di un volume di controllo .}
\label{\detokenize{polimi/fluidmechanics-ita/template/capitoli/04_bilanci/0401SciaExp:risultante-delle-forze-bilancio-di-quantita-di-moto-di-un-volume-di-controllo}}
\sphinxAtStartPar
Esistono metodi sperimentali, come ad esempio la \sphinxstylestrong{PIV} (Particle Image
Velocimetry o, in italiano, velocimetria a immagini di particelle), che
permettono di ottenere il campo di velocità in un determinato istante
all’interno di un dominio di misura, un piano bidimensionale o un volume
tridimensionale. Il bilancio di quantità di moto del volume di controllo
contenente un corpo solido permette poi di calcolare la risultante delle
forze scambiate tra corpo e fluido.

\sphinxAtStartPar
Per semplicità, viene considerato un campo di moto bidimensionale,
\(\bm{u}(x,y)=u(x,y)\bm{\hat{x}}+v(x,y)\bm{\hat{y}}\). Ad esempio, il
campo di moto attorno alla mezzeria di un’ala allungata senza freccia
investita da una corrente con un angolo di incidenza ridotto è in buona
approssimazione bidimensionale. In questo caso, misure PIV (PIV\sphinxhyphen{}2D\sphinxhyphen{}2C)
forniscono le due componenti (2C) del campo di velocità nel piano (2D)
di misura. Tramite il bilancio della quantità di moto del dominio
bidimensionale, è possibile ottenere una stima della risultante delle
forze (per unità di apertura) che esercita il fluido sul profilo di ala
tagliato dal piano di misura. Considerando gli effetti viscosi
trascurabili, al di fuori di regioni di dimensione ridotta nell’ambito
di applicazioni aeronautiche (alto numero di Reynolds, strato limite e
scie sottili), il bilancio integrale della quantità di moto del fluido
nel volume di misura fornisce, in un caso stazionario, la risultante
delle forze \({\bm{R}}\) agenti sul corpo,
\$\(\bm{R} = -\oint_S \rho \bm{u} \bm{u} \cdot \bm{\hat{n}} - \oint_S p \bm{\hat{n}} \ ,\)\(
avendo trascurato il contributo delle forze di volume. Nell'ipotesi, più
che sensata per molte applicazioni aeronautiche, che sia valido il
teorema di Bernoulli sulla frontiera \)S\( del volume di controllo, la
pressione viene espressa in funzione della velocità locale e dello stato
della corrente asintotica,
\)\(p = p_\infty + \rho \dfrac{|\bm{U_\infty}|^2}{2} - \rho \dfrac{|\bm{u}|^2}{2} \ .\)\(
Inserendo questa espressione della pressione nell'espressione della
risultante delle forze ed eliminando gli integrali (nulli) su una
superficie chiusa delle quantità costanti moltiplicate per la normale
alla superficie, come ad esempio \)\textbackslash{}oint\_S p\_\textbackslash{}infty \textbackslash{}bm\{\textbackslash{}hat\{n\}\}\(, si può
esprimere la risultante \)\textbackslash{}bm\{R\}\( della forza aerodinamica agente sul
corpo in funzione della sola velocità del fluido sulla frontiera \)S\(,
\)\(\bm{R} = -\oint_S \rho \bm{u} \bm{u} \cdot \bm{\hat{n}} + \oint_S \rho \dfrac{|\bm{u}|^2}{2} \bm{\hat{n}} \ .\)\(
Sotto queste ipotesi, la forza aerodinamica agente sul corpo, in questo
esempio l'obiettivo della misura, è stata scritta unicamente come
funzione del campo di velocità sulla superficie \)S\$, fornito come
«risultato diretto» dell’attività seprimentale. Per semplicità, la
densità del fluido viene considerata costante e nota senza incertezza:
nel caso che anche il campo di densità fosse affetto da incertezza, la
formula RSS permette di aggiungere abbastanza facilmente il suo effetto
a quello dovuto all’incertezza sulla misura del campo di velocità.


\subsubsection{Risultante delle forze: discretizzazione.}
\label{\detokenize{polimi/fluidmechanics-ita/template/capitoli/04_bilanci/0401SciaExp:risultante-delle-forze-discretizzazione}}
\sphinxAtStartPar
Per la sua natura, la PIV fornisce dei risultati discreti (non
continui): di solito, il campo di velocità viene misurato sui nodi di
una griglia cartesiana. Per il calcolo della risultante \(\bm{R}\) sono
necessari solamente gli \(N_n\) nodi esterni \(\bm{x_i}, \ i=1:N_n\), posti
sul contorno della griglia. Il campo di velocità viene approssimato
(linearmente, per semplicità) utilizzando un approccio simile a quello
impiegato nella modellazione numerica a elementi finiti. Viene
introdotto un insieme completo di funzioni di base
\(\phi_i(\bm{x}), \ i=1:N_{n}\), lineari a tratti sul contorno \(S\), grazie
alle quali è possibile scrivere l’approssimazione \(\bm{u}^h\) del campo
di velocità \$\(\label{exp:u:fem-exp}
 \bm{u}(\bm{x}) \approx \bm{u}^h(\bm{x}) = \displaystyle\sum_{i=1}^{N_n} \phi_i(\bm{x}) \bm{U_i} \ .\)\(
Utilizzando funzioni di base lagrangiane, per le quali il valore della
funzione \)i\(-esima \)\textbackslash{}phi\_i(\textbackslash{}bm\{x\})\( è uguale a uno sul nodo \)i\(-esimo
\)\textbackslash{}bm\{x\}\sphinxstyleemphasis{i\( e zero sugli altri nodi,
\)\(\phi_i(\bm{x_j}) = \delta_{ij} \qquad , \qquad \displaystyle\sum_{i=1}^{N_n} \phi_i(\bm{x}) = 1  \ , \forall i=1:N_n \ ,\)\(
i coefficienti \)\textbackslash{}bm\{U\_i\}\( della
([\[exp:u:fem-exp\]](#exp:u:fem-exp){reference-type="ref"
reference="exp:u:fem-exp"}) concidono con i valori nodali,
\)\textbackslash{}bm\{U\_i\}:=\textbackslash{}bm\{u\}(\textbackslash{}bm\{x\_i\})\( ricavati nei punti \)\textbackslash{}bm\{x\_i\}\( tramite la
misura sperimentale. Introducendo il campo di velocità approssimato
\)\textbackslash{}bm\{u\}\textasciicircum{}h(\textbackslash{}bm\{x\})\( nell'espressione della risultante delle forze, si
ottiene una formula nella quale compaiono gli integrali di superficie
del prodotto delle funzioni di base e del versore normale,
\)\(\label{eqn:RU}
\begin{aligned}
 \bm{R} \approx \bm{R}^h & = -\oint_S \rho \bm{u}^h \bm{u}^h \cdot \bm{\hat{n}} + \oint_S \rho \dfrac{\bm{u}^h \cdot \bm{u}^h}{2} \bm{\hat{n}} = \\
 & = - \rho \sum_{i=1}^{N_n} \sum_{j=1}^{N_n} \bm{U}_i \bm{U}_j \cdot \oint_S \phi_i(\bm{x}) \phi_j(\bm{x}) \bm{\hat{n}}(\bm{x})  + \dfrac{1}{2} \rho \sum_{i=1}^{N_n} \sum_{j=1}^{N_n}\bm{U}_i \cdot \bm{U}_j \oint_S  \phi_i(\bm{x}) \phi_j(\bm{x}) \bm{\hat{n}}(\bm{x}) = \\ 
 & = - \rho \sum_{i=1}^{N_n} \sum_{j=1}^{N_n} \bm{U}_i \bm{U}_j \cdot \bm{I}_{ij}  + \dfrac{1}{2} \rho \sum_{i=1}^{N_n} \sum_{j=1}^{N_n}\bm{U}_i \cdot \bm{U}_j \bm{I}_{ij} \ , 
\end{aligned}\)\( dove sono stati introdotti i vettori
\)\textbackslash{}bm\{I\}}\{ij\} = \textbackslash{}oint\_S  \textbackslash{}phi\_i(\textbackslash{}bm\{x\}) \textbackslash{}phi\_j(\textbackslash{}bm\{x\}) \textbackslash{}bm\{\textbackslash{}hat\{n\}\}(\textbackslash{}bm\{x\})\$,
facilmente calcolabili in maniera analitica, come spiegato nella sezione
§\DUrole{xref,myst}{0.0.0.9}\{reference\sphinxhyphen{}type=»ref» reference=»sec:fem»\}.


\subsubsection{Sensitività della risultante al campo di velocità.}
\label{\detokenize{polimi/fluidmechanics-ita/template/capitoli/04_bilanci/0401SciaExp:sensitivita-della-risultante-al-campo-di-velocita}}
\sphinxAtStartPar
Per ricavare tramite la formula RSS l’incertezza sulla misura della
risultante delle forze \(\bm{R}\) dall’incertezza sulle misure del campo
di velocità \(\bm{u}(\bm{x})\), è necessario calcolare la \sphinxstyleemphasis{variazione} di
\(\bm{R}\) rispetto al campo \(\bm{u}(\bm{x})\). Perturbando il campo di
velocità \(\bm{u}(\bm{x})\) con la variazione \(\delta \bm{u}(\bm{x})\), e
trascurando i termini di ordine superiore al primo, dopo aver sottratto
l’equazione «non perturbata», si ottiene la perturbazione della
risultante delle forze \(\delta \bm{R}\), \$\(\label{eqn:sens:cont}
\begin{aligned}
  \bm{R} + \delta \bm{R} & = -\oint_S \rho (\bm{u}+\delta\bm{u}) (\bm{u}+\delta\bm{u}) \cdot \bm{\hat{n}} + \oint_S \dfrac{1}{2}\rho (\bm{u}+\delta\bm{u}) \cdot (\bm{u}+\delta\bm{u}) \bm{\hat{n} } \\
  \rightarrow \delta \bm{R} & = -\oint_S \rho \left[ \bm{u} \bm{\hat{n}} \cdot \delta\bm{u} + \bm{u} \cdot \bm{\hat{n}} \delta\bm{u}\right]  + \oint_S \rho \bm{\hat{n}} \bm{u} \cdot \delta\bm{u} \\
 & = \oint_S \rho \left[ - \bm{u} \otimes \bm{\hat{n}} - (\bm{u} \cdot \bm{\hat{n}})\mathbb{I} + \bm{\hat{n}} \otimes \bm{u} \right] \cdot \delta\bm{u} = \\
 & = \oint_S \bm{\nabla}_u \bm{R} \cdot \delta\bm{u} = \\
 & =\oint_S \begin{bmatrix} \nabla_u R_x &  \nabla_v R_x \\ \nabla_u R_y &  \nabla_v R_y  \end{bmatrix} \cdot \begin{bmatrix} \delta u \\ \delta v \end{bmatrix} =
 \oint_S \begin{bmatrix} \bm{\nabla}_{\bm{u}} R_x \cdot \delta \bm{u} \\ \bm{\nabla}_{\bm{u}} R_y \cdot \delta \bm{u} \end{bmatrix} \ ,
\end{aligned}\)\( avendo introdotto il campo tensoriale della sensitività
\)\textbackslash{}bm\{\textbackslash{}nabla\}\_\{u\} \textbackslash{}bm\{R\}(\textbackslash{}bm\{x\})\( della risultante delle forze rispetto
al campo di velocità \)\textbackslash{}bm\{u\}(\textbackslash{}bm\{x\})\( ed evidenziato l'influenza delle
due componenti del campo di velocità sulle due componenti di forza.
L'equazione precedente può essere scritta con notazione indiciale
\)\(\qquad \delta R_i = \oint_S \nabla_{u_j} R_i \delta u_j = -\rho \oint_S \left[ u_i n_j + u_k n_k \delta_{ij} - n_i u_j \right] \delta u_j \ ,\)\(
o esplicitamente in coordinate cartesiane, per ricavare l'espressione
della sensitività della componenti della forza dalle singole componenti
del campo di velocità, \)\(\label{eqn:sens:cart:simple}
\begin{aligned}
  \qquad & \begin{cases}
  \delta R_x & = \rho \displaystyle\oint_S \left[ -u n_x - u n_x - v n_y + u n_x \right] \delta u + \rho \oint_S \left[ -u n_y + v n_x   \right] \delta v \\
  \delta R_y & = \rho \displaystyle\oint_S \left[ -v n_x + u n_y \right] \delta u + \rho \oint_S \left[ -v n_y - u n_x - v n_y + v n_y \right] \delta v \\
\end{cases}  \vspace{5mm} \\
 \rightarrow & \begin{cases}
 \delta R_x & =
 \rho \displaystyle\oint_S \left[ -u n_x - v n_y \right] \delta u + \rho \oint_S \left[ -u n_y + v n_x   \right] \delta v =
 \oint_S \nabla_{u} R_x \ \delta u + \oint_S \nabla_v R_x \delta v \\
 \delta R_y & = \rho \displaystyle\oint_S \left[ -v n_x + u n_y \right] \delta u + \rho \oint_S \left[ -v n_y - u n_x \right] \delta v =
 \oint_S \nabla_{u} R_y \ \delta u + \oint_S \nabla_v R_y \delta v \ .
\end{cases}
\end{aligned}\)\$


\subsubsection{Sensitività della risultante alle misure di velocità.}
\label{\detokenize{polimi/fluidmechanics-ita/template/capitoli/04_bilanci/0401SciaExp:sensitivita-della-risultante-alle-misure-di-velocita}}
\sphinxAtStartPar
Partendo dall’espansione
(\DUrole{xref,myst}{{[}exp:u:fem\sphinxhyphen{}exp{]}}\{reference\sphinxhyphen{}type=»ref»
reference=»exp:u:fem\sphinxhyphen{}exp»\}) del campo di velocità, la variazione del
campo \(\bm{u}^h(\bm{x})\) diventa \$\(\label{exp:du:fem-exp}
 \delta \bm{u}^h(\bm{x}) = \displaystyle\sum_{i=1}^{N_n} \phi_i(\bm{x}) \delta \bm{U}_i \ ,\)\(
avendo indicato con \)\textbackslash{}delta \textbackslash{}bm\{U\}\sphinxstyleemphasis{i\( la variazione dei valori nodali
del campo di velocità. Le funzioni di base sono note, e quindi la loro
variazione è nulla.[^1] Introducendo l'espressione
([\[exp:du:fem-exp\]](#exp:du:fem-exp){reference-type="ref"
reference="exp:du:fem-exp"}) di \)\textbackslash{}delta \textbackslash{}bm\{u\}\textasciicircum{}h(\textbackslash{}bm\{x\})\( all'interno
della formula ([\[eqn:sens:cont\]](#eqn:sens:cont){reference-type="ref"
reference="eqn:sens:cont"}) che lega la variazione \)\textbackslash{}delta \textbackslash{}bm\{R\}\( alla
variazione \)\textbackslash{}delta \textbackslash{}bm\{u\}(\textbackslash{}bm\{x\})\(,
\)\(\delta \bm{R} = \oint_S \bm{\nabla}_{\bm{u}} \bm{R} \cdot \delta \bm{u} 
  = \sum_{i=1}^{N_n} \oint_S \phi_i(\bm{x}) \bm{\nabla}_{\bm{u}} \bm{R} \cdot \delta \bm{U}_i  
 = \sum_{i=1}^{N_n} \bm{\nabla}_{\bm{U}_i} \bm{R} \cdot \delta \bm{U}_i \ ,\)\(
si ricava l'espressione della sensitività
\)\textbackslash{}bm\{\textbackslash{}nabla\}}\{\textbackslash{}bm\{U\}\_i\} \textbackslash{}bm\{R\}\( della risultante delle forze rispetto
alla misura di velocità \)\textbackslash{}bm\{U\}\sphinxstyleemphasis{i\(, in funzione della sensitività
\)\textbackslash{}bm\{\textbackslash{}nabla\}}\{\textbackslash{}bm\{U\}\sphinxstyleemphasis{i\} \textbackslash{}bm\{R\}(\textbackslash{}bm\{x\})\( della risultante rispetto al
campo di velocità \)\textbackslash{}bm\{u\}(\textbackslash{}bm\{x\})\( e alle funzioni di base
\)\textbackslash{}phi\_i(\textbackslash{}bm\{x\})\(,
\)\(\bm{\nabla}_{\bm{U}_i} \bm{R} = \oint_S \phi_i(\bm{x}) \bm{\nabla}_{\bm{u}} \bm{R} \ .\)\(
La sensitività \)\textbackslash{}bm\{\textbackslash{}nabla\}}\{\textbackslash{}bm\{U\}\_i\} R\_k\( della componente \)R\_k\( della
risultante delle forze rispetto alla misura \)\textbackslash{}bm\{U\}\_i\( è quindi
\)\(\bm{\nabla}_{\bm{U}_i} R_k 
  = \oint_S \phi_i(\bm{x}) \bm{\nabla}_{\bm{u}} R_k \ .\)\$


\subsubsection{Sensitività della risultante alle misure di velocità: discretizzazione.}
\label{\detokenize{polimi/fluidmechanics-ita/template/capitoli/04_bilanci/0401SciaExp:sensitivita-della-risultante-alle-misure-di-velocita-discretizzazione}}
\sphinxAtStartPar
Inserendo l’approssimazione \(\bm{u}^h\) nella formula della sensitività
\(\bm{\nabla}_{\bm{u}} \bm{R}\), è possibile calcolare la sensitività
della risultante alle misure di velocità \(\bm{U}_i\),
\$\(\label{eqn:sens:RU}
\begin{aligned}
 \bm{\nabla}_{\bm{U}_i} \bm{R} & = \oint_S \phi_i(\bm{x}) \bm{\nabla}_{\bm{u}} \bm{R} =\\
 & = \oint_S \phi_i(\bm{x}) \rho \left[ - \bm{u} \otimes \bm{\hat{n}} - (\bm{u} \cdot \bm{\hat{n}})\mathbb{I} + \bm{\hat{n}} \otimes \bm{u} \right]  = \\
 & = \rho \displaystyle\sum_{j=1}^{N_n} \oint_S \phi_i(\bm{x}) \phi_j(\bm{x}) \left[ - \bm{U}_j \otimes \bm{\hat{n}} - (\bm{U}_j \cdot \bm{\hat{n}})\mathbb{I} + \bm{\hat{n}} \otimes \bm{U}_j \right]  = \\
 & = \rho \displaystyle\sum_{j=1}^{N_n} \left[ - \bm{U}_j \otimes \bm{I}_{ij} - (\bm{U}_j \cdot \bm{I}_{ij})\mathbb{I} + \bm{I}_{ij} \otimes \bm{U}_j \right] \ ,
\end{aligned}\)\( avendo riconosciuto i vettori \)\textbackslash{}bm\{I\}\_\{ij\}\( definiti in
precedenza. La sensitività della componente \)R\_k\( alla misura \)\textbackslash{}bm\{U\}\sphinxstyleemphasis{i\(
vale \)\(\bm{\nabla}_{\bm{U}_i} R_k =  
  \rho \displaystyle\sum_{j=1}^{N_n} \left[ - U_{j,k} \bm{I}_{ij} - (\bm{U}_j \cdot \bm{I}_{ij}) \bm{\hat{e}}_k + I_{ij,k} \bm{U}_j \right] \ ,\)\(
dove \)\textbackslash{}bm\{\textbackslash{}hat\{e\}\}\sphinxstyleemphasis{k\( è il versore in direzione \)k\( e \)U}\{j,k\}\(,
\)I}\{ij,k\}\( le componenti in quella direzione della misura \)\textbackslash{}bm\{U\}\sphinxstyleemphasis{i\( e
del vettore \)\textbackslash{}bm\{I\}}\{ij\}\$.


\subsubsection{Osservazione 1.}
\label{\detokenize{polimi/fluidmechanics-ita/template/capitoli/04_bilanci/0401SciaExp:osservazione-1}}
\sphinxAtStartPar
Si può dimostrare che le sensitività \(\bm{\nabla}_{\bm{U_i}} \bm{R}\)
sono le componenti del gradiente della formula
(\DUrole{xref,myst}{{[}eqn:RU{]}}\{reference\sphinxhyphen{}type=»ref» reference=»eqn:RU»\}) che
esprime \(\bm{R}\) come una funzione quadratica delle variabili
\(\bm{U}_i\).


\subsubsection{Osservazione 2.}
\label{\detokenize{polimi/fluidmechanics-ita/template/capitoli/04_bilanci/0401SciaExp:osservazione-2}}
\sphinxAtStartPar
Utilizzando la formula generale
(\DUrole{xref,myst}{{[}eqn:sens:RU{]}}\{reference\sphinxhyphen{}type=»ref»
reference=»eqn:sens:RU»\}) o utilizzando la forma discretizzata delle
espressioni
(\DUrole{xref,myst}{{[}eqn:sens:cart:simple{]}}\{reference\sphinxhyphen{}type=»ref»
reference=»eqn:sens:cart:simple»\}), si può dimostrare che
\$\(\begin{aligned}
 \nabla_{U_{i,x}} R_x & = \nabla_{U_{i,y}} R_y = - \rho \displaystyle\sum_{j=1}^{N_n} \bm{U}_{j} \cdot \bm{I}_{ij} \\
 -\nabla_{U_{i,y}} R_x & = \nabla_{U_{i,x}} R_y = - \rho \displaystyle\sum_{j=1}^{N_n} \bm{U}_j  \times \bm{I}_{ij}\cdot \bm{\hat{z}} \\
\end{aligned}\)\$


\subsubsection{Incertezza sulla risultante dall’incertezza sulla misura di velocità.}
\label{\detokenize{polimi/fluidmechanics-ita/template/capitoli/04_bilanci/0401SciaExp:incertezza-sulla-risultante-dall-incertezza-sulla-misura-di-velocita}}
\sphinxAtStartPar
Utilizzando la formula del campo \(\bm{u}^h\), viene calcolata la varianza
\(\sigma^2_{R_k}\) della componente \(R_k\), \$\(\begin{aligned}
 \sigma^2_{R_k} & = E[\delta R_k \delta R_k] = \rho^2 E\left[ \oint_{S(\bm{x})} \bm{\nabla}_{\bm{u}} R_k (\bm{x}) \cdot \delta \bm{u}(\bm{x}) \oint_{S(\bm{y})} \bm{\nabla}_{\bm{u}} R_k (\bm{y}) \cdot \delta \bm{u}(\bm{y}) \right] = \\
 & = \oint_{S(\bm{x})} \oint_{S(\bm{y})}  \bm{\nabla}_{\bm{u}} R_k (\bm{x}) \cdot E\left[ \delta \bm{u}(\bm{x}) \otimes \delta \bm{u}(\bm{y}) \right] \cdot \bm{\nabla}_{\bm{u}} R_k (\bm{y}) \approx \\
 & = \oint_{S(\bm{x})} \oint_{S(\bm{y})}  \bm{\nabla}_{\bm{u}} R_k (\bm{x}) \cdot \sum_{i=1}^{N_n} \sum_{j=1}^{N_n} \phi_i(\bm{x}) \phi_j(\bm{y}) E\left[ \delta \bm{U}_i \otimes \delta \bm{U}_j \right] \cdot \bm{\nabla}_{\bm{u}} R_k (\bm{y}) \ ,
\end{aligned}\)\( dove sono state indicate esplicitamente le variabili
indipendenti \)\textbackslash{}bm\{x\}\(, \)\textbackslash{}bm\{y\}\( sulle quali devono essere svolte le
integrazioni. Si fa l'ipotesi che l'incertezza della misura della
componente in un punto sia indipendente dalla misura delle altre
componenti della velocità nello stesso punto e dalla velocità negli
altri punti del dominio. Si ipotizza inoltre che l'incertezza sulla
singola misura in tutto il dominio sia uguale a \)\textbackslash{}sigma\textasciicircum{}2\_U\( su tutte le
componenti della velocità. L'espressione dei valori attesi
\)E{[}\textbackslash{}delta \textbackslash{}bm\{U\}\_i \textbackslash{}otimes \textbackslash{}delta \textbackslash{}bm\{U\}\sphinxstyleemphasis{j{]}\( diventa quindi
\)\(E[\delta \bm{U}_i \otimes \delta \bm{U}_j] = \sigma_U^2 \delta_{ij} \mathbb{I}\)\(
e di conseguenza l'incertezza della componente di forza \)R\_k\(,
\)\(\begin{aligned}
 \sigma^2_{R_k} & = \oint_{S(\bm{x})} \oint_{S(\bm{y})}  \bm{\nabla}_{\bm{u}} R_k (\bm{x}) \cdot \sum_{i=1}^{N_n} \phi_i(\bm{x}) \phi_i(\bm{y}) \cdot \bm{\nabla}_{\bm{u}} R_k (\bm{y}) \sigma^2_U  = \\  
  & = \sum_{i=1}^{N_n}\left\{ \oint_{S(\bm{x})} \bm{\nabla}_{\bm{u}} R_k (\bm{x})   \phi_i(\bm{x}) \right\} \cdot \left\{ \oint_{S(\bm{y})} \bm{\nabla}_{\bm{u}} R_k (\bm{y}) \phi_i(\bm{y}) \right\} \sigma^2_U  = \\   
  & = \sum_{i=1}^{N_n} \bm{\nabla}_{\bm{U}_i} R_k \cdot \bm{\nabla}_{\bm{U}_i} R_k \ \sigma^2_U  = \\ 
  & = \sum_{i=1}^{N_n} | \bm{\nabla}_{\bm{U}_i} R_k |^2 \ \sigma^2_U = \\
  & = \sum_{i=1}^{N_n} \big( (\nabla_{U_{i,x}} R_k) ^2 + (\nabla_{U_{i,y}} R_k) ^2 \big) \ \sigma^2_U \ , 
\end{aligned}\)\( avendo riconosciuto la sensitività
\)\textbackslash{}bm\{\textbackslash{}nabla\}}\{\textbackslash{}bm\{U\}\_i\} R\_k\( della componente di forza \)R\_k\( rispetto
alla misura della velocità \)\textbackslash{}bm\{U\}\_i = \textbackslash{}bm\{u\}(\textbackslash{}bm\{x\}\_i)\$.


\subsubsection{Cenni sugli elementi finiti. \{\#sec:fem\}}
\label{\detokenize{polimi/fluidmechanics-ita/template/capitoli/04_bilanci/0401SciaExp:cenni-sugli-elementi-finiti-sec-fem}}
\sphinxAtStartPar
In questo paragrafo si fornisce qualche dettaglio sulla discretizzazione
«a elementi finiti» usata nel calcolo della risultante aerodinamica e
della sua incertezza. Un dominio \(S\), come ad esempio la superficie di
controno del volume di controllo considerato, viene suddiviso negli
elementi \(S_k\), l’unione dei quali costituisce il dominio \(S\)
\$\(S = \bigcup_{k=1}^{N_e} S_k\)\( e che non hanno punti in comune tra di
loro se non i bordi. Vengono poi definite delle funzioni di base
\)\textbackslash{}phi\_i(\textbackslash{}bm\{x\})\(, grazie alle quali è possibile approssimare (sulle
quali viene proiettata) una funzione generica
\)\(f(\bm{x}) = \displaystyle\sum_{i=1}^{N_n} \phi_i(\bm{x}) f_i \ .\)\( La
dipendenza dalla variabile spaziale \)\textbackslash{}bm\{x\}\( è contenuta nelle funzioni
di base \)\textbackslash{}phi(\textbackslash{}bm\{x\})\(, le quali vengono moltiplicate per i coefficienti
\)f\_i\(. In generale, le funzioni \)\textbackslash{}phi\_i(\textbackslash{}bm\{x\})\( sono regolari a tratti,
essendo regolari all'interno dei singoli elementi \)S\_k\( e continue sui
loro bordi. Nel metodo degli *elementi finiti*, le funzioni di base sono
*a supporto compatto*, cioè sono diverse da zero solo su un dominio
chiuso e limitato: il carattere "locale" delle singole funzioni di base
viene sfruttato nel metodo degli elementi finiti per operare con matrici
sparse, all'interno delle quali solo pochissimi elementi sono diversi da
zero in ogni riga o colonna. Il supporto della funzione \)\textbackslash{}phi\_i(\textbackslash{}bm\{x\})\(
è la parte di dominio al di di fuori della quale la funzione è nulla.
Nel metodo degli elementi finiti, il supporto di \)\textbackslash{}phi\_i(\textbackslash{}bm\{x\})\( è
costitutito dagli elementi \)S\_k\( ai quali appartiene il nodo \)\textbackslash{}bm\{x\}\sphinxstyleemphasis{i\(.
Indichiamo il supporto di \)\textbackslash{}phi\_i(\textbackslash{}bm\{x\})\( con \)B\_i\(. Le funzioni di
base vengono definite lagrangiane, se la funzione \)i\(-esima
\)\textbackslash{}phi\_i(\textbackslash{}bm\{x\})\( è uguale a uno sul nodo \)i\(-esimo \)\textbackslash{}bm\{x\}\sphinxstyleemphasis{i\( e zero
sugli altri nodi,
\)\(\phi_i(\bm{x}_j) = \delta_{ij} \qquad , \qquad \displaystyle\sum_{i=1}^{N_n} \phi_i(\bm{x}) = 1  \ , \forall i=1:N_n \ .\)\(
In questo caso, i coefficienti \)f\_i\( concidono con i valori nodali della
funzione \)f(\textbackslash{}bm\{x\})\(, \)f\_i:=f(\textbackslash{}bm\{x\_i\})\(. Viene definita una
*connettività* della griglia degli elementi finiti, che consiste in un
elenco ordinato dell'indice dei nodi di ogni elemento: in questa maniera
viene definita una numerazione locale dei nodi di ogni singolo elemento,
che risulta utile nel calcolo degli integrali. Viene indicato con
\)I\_k = \{ i}\{k1\} , i}\{k2\} , \textbackslash{}dots , i\_\{kn\} \}\(, l'elenco degli \)n\( nodi
dell'elemento \)S\_k\$.

\sphinxAtStartPar
In figura \DUrole{xref,myst}{2}\{reference\sphinxhyphen{}type=»ref»
reference=»fig:base\sphinxhyphen{}fcn»\} è rappresentata una parte di una suddivisione
in elementi finiti \(S_{k}\) di un dominio monodimensionale, sul quale
sono definite delle funzioni di base lagrangiane, lineari a tratti, a
supporto compatto: ad esempio, la funzione di base \(\phi_{i2}(\bm{x})\) è
diversa da zero solo sugli elementi \(S_{e1}\) e \(S_{e2}\). Ogni elemento
ha due nodi. Se viene definita la connettività nodi\sphinxhyphen{}elemento,
\$\(\label{eqn:conn:ex}
  \begin{aligned}
    I_{e1} = \{ i_1 , i_2 \} \ , \\
    I_{e2} = \{ i_2 , i_3 \} \ , \\
    I_{e3} = \{ i_4 , i_3 \} \ , \\
  \end{aligned}\)\( il nodo \)i\_1\( è il primo nodo (quello che ha l'indice
\)=1\( nella numerazione locale) dell'elemento \)S\_\{e1\}\(, il nodo \)i\_2\( è
il secondo nodo di \)S\_\{e1\}\( e il primo di \)S\_\{e2\}\(, il nodo \)i\_3\( è il
secondo nodo sia di \)S\_\{e2\}\( sia di \)S\_\{e3\}\(, il nodo \)i\_4\( è il primo
nodo di \)S\_\{e3\}\$.

\sphinxAtStartPar
\sphinxincludegraphics{{polimi/fluidmechanics-ita/template/capitoli/04_bilanci/fig/base-functions}}\{\#fig:base\sphinxhyphen{}fcn
width=»55\%»\} \sphinxincludegraphics{{polimi/fluidmechanics-ita/template/capitoli/04_bilanci/fig/base-functions-ref}}\{\#fig:base\sphinxhyphen{}fcn
width=»30\%»\}

\sphinxAtStartPar
Si utilizzano ora le proprietà della base di funzioni lineari a tratti
\(\phi_i(\bm{x})\) per calcolare i vettori \(\bm{I}_{ij}\) che compaiono nel
calcolo della risultante delle forze e nella sua varianza,
\$\(\bm{I}_{ij} := \oint_S \phi_i(\bm{x}) \phi_j(\bm{x}) \bm{\hat{n}}(\bm{x}) \ .\)\(
Gli unici termini \)\textbackslash{}bm\{I\}\sphinxstyleemphasis{\{ij\}\( che non sono nulli sono quelli in cui
compaiono due funzioni, che hanno supporti a intersezione non nulla,
\)B\_i \textbackslash{}cap B\_j \textbackslash{}neq 0\(. In questi termini, il dominio di integrazione può
essere limitato alla sola intersezione dei supporti delle due funzioni,
essendo il prodotto di queste nullo al di fuori di esso. Ad esempio,
facendo riferimento alla figura [2](#fig:base-fcn){reference-type="ref"
reference="fig:base-fcn"}, il termine \)\textbackslash{}bm\{I\}}\{i2,i1\}\( può essere
riscritto come
\)\(\bm{I}_{i2,i1} = \oint_S \phi_{i2}(\bm{x})\phi_{i1}(\bm{x})\bm{\hat{n}}   
                =  \int_{B_{i2}\cap B_{i1}} \phi_{i2}(\bm{x})\phi_{i1}(\bm{x})\bm{\hat{n}}   
                =  \int_{S_{e1}} \phi_{i2}(\bm{x})\phi_{i1}(\bm{x})\bm{\hat{n}} \ ,\)\(
il termine \)\textbackslash{}bm\{I\}\sphinxstyleemphasis{\{i2,i2\}\( può essere riscritto come
\)\(\bm{I}_{i2,i2} = \oint_S \phi_{i2}(\bm{x})\phi_{i2}(\bm{x})\bm{\hat{n}}   
                =  \int_{B_{i2}} \phi_{i2}(\bm{x})\phi_{i2}(\bm{x})\bm{\hat{n}}   
                =  \int_{S_{e1}\cup S_{e2}} \phi_{i2}(\bm{x})\phi_{i2}(\bm{x})\bm{\hat{n}} \ ,\)\(
mentre il termine \)\textbackslash{}bm\{I\}}\{i2,i4\}\( è nullo. Gli integrali sugli elementi
\)S\_i\( nello spazio "fisico" possono essere calcolati sull'elemento di
riferimento \)\textbackslash{}hat\{S\}\(, definito in \)\textbackslash{}xi \textbackslash{}in {[}\sphinxhyphen{}1,1{]}\(. La trasformazione
di coordinate che porta l'elemento di riferimento \)\textbackslash{}hat\{S\}\( nell'
elemento fisico \)S\_\{k\}\( delimitato dai punti di coordinata \)x\_\{k1\}\( e
\)x\_\{k2\}\( è
\)\(x = \dfrac{x_{k2}+x_{k1}}{2} + \dfrac{x_{k2}-x_{k1}}{2} \xi \ \)\( e il
suo "determinante" è
\)\(\dfrac{\partial x}{\partial \xi} = \dfrac{x_{k2}-x_{k1}}{2} = \dfrac{\ell_k}{2} \ .\)\(
Se si considera costante il versore normale
\)\textbackslash{}bm\{\textbackslash{}hat\{n\}\} = \textbackslash{}bm\{\textbackslash{}hat\{n\}\}\sphinxstyleemphasis{\{S}\{e1\}\}\( sull'elemento finito \)S\_\{e1\}\( e
si utilizza la connettività nodi-griglia dell'esempio definita in
([\[eqn:conn:ex\]](#eqn:conn:ex){reference-type="ref"
reference="eqn:conn:ex"}), l'integrale \)\textbackslash{}bm\{I\}\sphinxstyleemphasis{\{i2,i1\}\( può essere
trasformato nell'integrale sull'elemento di riferimento
\)\(\begin{aligned}
 \bm{I}_{i2,i1} = \int_{S_{e1}} \phi_{i2}(x)\phi_{i1}(x)\bm{\hat{n}} dx & = \int_{\tilde{S}} \varphi_2(\xi) \varphi_1(\xi)  \dfrac{\partial x}{\partial \xi}  d\xi \ \bm{\hat{n}}_{S_{e1}} = \\
 & = \int_{\xi=-1}^{1}\varphi_2(\xi) \varphi_1(\xi)  \dfrac{\partial x}{\partial \xi}  d\xi \ \bm{\hat{n}}_{S_{e1}} \ , 
\end{aligned}\)\( avendo riconosciuto il legame tra l'elemento \)S}\{e1\}\(
nel dominio fisico e quello di riferimento \)\textbackslash{}hat\{S\}\(,
\)\textbackslash{}phi\_\{i\}(x) = \textbackslash{}phi\_i(x(\textbackslash{}xi)) = \textbackslash{}varphi\_\{i\textasciicircum{}\{\textbackslash{}ell\}\}(\textbackslash{}xi)\(, dove è stato
indicato con \)i\textasciicircum{}\{\textbackslash{}ell\}\( l'indice locale del nodo globale con indice \)i\(:
dalla connettività dell'elemento \)S\_\{e1\}\( risulta \)i\_1\textasciicircum{}\textbackslash{}ell = 1\(
\)i\_2\textasciicircum{}\textbackslash{}ell = 2\(. Il "determinante" della trasformazione è noto e
costante, \)\textbackslash{}partial x/\textbackslash{}partial \textbackslash{}xi|\sphinxstyleemphasis{\{S}\{e1\}\} = \textbackslash{}ell\_\{S\_\{e1\}\}/2\(.
L'espressione delle funzioni sull'elemento locale è facilmente
ricavabile. Le funzioni di base lagrangiane devono essere uguali a \)1\(
in un nodo e zero in tutti gli altri. Considerando i punti \)\textbackslash{}xi=\sphinxhyphen{}1\( e
\)x=1\( come primo e secondo nodo dell'elemento di riferimento \)\textbackslash{}hat\{S\}\(,
le funzioni definite sull'elemento di riferimento valgono
\)\(\varphi_1(\xi) = \dfrac{1}{2}(1-\xi) \quad , \quad
 \varphi_2(\xi) = \dfrac{1}{2}(1+\xi) \ .\)\( É immediato calcolare il
valore degli integrali sull'elemento di riferimento, \)\(\begin{aligned}
  \int_{-1}^{1} \varphi_1(\xi) \varphi_1(\xi) d\xi = \dfrac{2}{3} \quad & , \quad   
  \int_{-1}^{1} \varphi_1(\xi) \varphi_2(\xi) d\xi = \dfrac{1}{3} \\ 
  \int_{-1}^{1} \varphi_2(\xi) \varphi_1(\xi) d\xi = \dfrac{1}{3} \quad & , \quad   
  \int_{-1}^{1} \varphi_2(\xi) \varphi_2(\xi) d\xi = \dfrac{2}{3} \ .  
\end{aligned}\)\( Questi valori vengono infine utilizzati nel calcolo dei
vettori \)\textbackslash{}bm\{I\}\_\{ij\}\(. I vettori dell'esempio valgono \)\(\begin{aligned}
 \bm{I}_{i2,i1} & 
 = \int_{S_{e1}} \phi_{i2}(x)\phi_{i1}(x)\bm{\hat{n}} dx = \\ 
 & = \int_{\xi=-1}^{1}\varphi_2(\xi) \varphi_1(\xi)  \dfrac{\partial x}{\partial \xi}\bigg|_{S_{e1}}  d\xi \ \bm{\hat{n}}_{S_{e1}} = \dfrac{1}{3}\dfrac{\ell_{e1}}{2} \bm{\hat{n}}_{S_{e1}} = \dfrac{\ell_{e1}}{6} \bm{\hat{n}}_{S_{e1}}  \ , \\
 \bm{I}_{i2,i2} & = \int_{S_{e1}\cup S_{e2}} \phi_{i2}(x)\phi_{i2}(x)\bm{\hat{n}} \ dx = \\ 
 & = \int_{S_{e1}} \phi_{i2}(x)\phi_{i2}(x)\bm{\hat{n}} \ dx +    
     \int_{S_{e2}} \phi_{i2}(x)\phi_{i2}(x)\bm{\hat{n}} \ dx = \\ 
 & = \int_{\xi=-1}^{1}\varphi_2(\xi) \varphi_2(\xi)  \dfrac{\partial x}{\partial \xi}\bigg|_{S_{e1}}  d\xi \ \bm{\hat{n}}_{S_{e1}} + 
     \int_{\xi=-1}^{1}\varphi_1(\xi) \varphi_1(\xi)  \dfrac{\partial x}{\partial \xi}\bigg|_{S_{e2}}  d\xi \ \bm{\hat{n}}_{S_{e2}} = \\ 
 & = \dfrac{\ell_{e1}}{3} \bm{\hat{n}}_{S_{e1}} + \dfrac{\ell_{e2}}{3} \bm{\hat{n}}_{S_{e2}} \ . \\
\end{aligned}\)\$

\sphinxstepscope


\subsection{Exercise 4.10}
\label{\detokenize{polimi/fluidmechanics-ita/template/capitoli/04_bilanci/04ma01in:exercise-4-10}}\label{\detokenize{polimi/fluidmechanics-ita/template/capitoli/04_bilanci/04ma01in:fluid-mechanics-balances-ex-10}}\label{\detokenize{polimi/fluidmechanics-ita/template/capitoli/04_bilanci/04ma01in::doc}}
\sphinxAtStartPar
+:———————————:+:———————————:+
| Viene dato l’irrigatore           |                                   |
| rappresentato in figura, del      |                                   |
| quale sono note le sue dimensioni |                                   |
| geometriche, \(R_0\), \(R_1\),        |                                   |
| \(\ell\), \(h\). L’irrigatore è       |                                   |
| libero di ruotrare attorno        |                                   |
| all’asse \(z\). Si conosce la       |                                   |
| densità del fluido \(\rho\) e la    |                                   |
| velocità «di ingresso» \(U_0\)      |                                   |
| uniforme sulla sezione \(S_0\).     |                                   |
| Supponendo                        |                                   |
|                                   |                                   |
| \sphinxhyphen{}   la pressione uniforme sulle   |                                   |
|     sezioni \(S_0\), \(S_1\), \(S_2\) e |                                   |
|     uguale alla pressione         |                                   |
|     atmosferica dell’aria attorno |                                   |
|     all’irrigatore                |                                   |
|                                   |                                   |
| \sphinxhyphen{}   la \sphinxstylestrong{velocità relativa}      |                                   |
|     rispetto al moto              |                                   |
|     dell’irrigatore uniforme      |                                   |
|     sulle sezioni \(S_1\), \(S_2\),   |                                   |
|                                   |                                   |
| \sphinxhyphen{}   gli effetti gravitazionali    |                                   |
|     trascurabili                  |                                   |
|                                   |                                   |
| viene chiesto di calcolare la     |                                   |
| velocità \(V\) e la velocità di     |                                   |
| rotazione dell’irrigatore         |                                   |
| \(\Omega\), a regime.               |                                   |
+———————————–+———————————–+

\sphinxAtStartPar
Si scrivono i bilanci integrali di massa e momento della quantità di
moto per il volume fluido \(V_t\) contenuto all’interno dell’irrigatore,
delimitato dalla parete interna dell’irrigatore \(S_{f,s}\), dalla sezione
di ingresso \(S_0\) e dalle due di uscita \(S_1\), \(S_2\). Si introducono due
sistemi di riferimento cartesiani, uno inerziale,
\(\left\{ \bm{\hat{x}}, \bm{\hat{y}}, \bm{\hat{z}} \right\}\), l’altro
solidale con l’irrigatore,
\(\left\{ \bm{\hat{X}}, \bm{\hat{Y}}, \bm{\hat{Z}} \right\}\), con l’asse
\(Z\) coincidente con l’asse \(z\). Il bilancio di massa per un volume
\(V_t\),
\$\(\dfrac{d}{dt} \int_{V_t} \rho + \oint_{\partial V_t} \rho \left(\bm{u} - \bm{b} \right) \cdot \bm{\hat{n}} = 0 \ ,\)\$
viene semplificato
\begin{itemize}
\item {} 
\sphinxAtStartPar
utilizzando l’ipotesi di stazionarietà%
\begin{footnote}[1]\sphinxAtStartFootnote
Dovrebbe essere chiaro che il concetto di «stazionarietà» dipende
dal tipo di descrizione adottata per rappresentare il problema,
euleriana, lagrangiana o arbitraria. Come esempio, in questo
esercizio utilizziamo una descrizione arbitraria, utilizzando un
volume di controllo che ruota insieme all’irrigatore. Per un
osservatore inerziale il problema a regime non è stazionario, ma
periodico. Per un’osservatore non inerziale solidale con
l’irrigatore, le quantità del problema non variano con il tempo e
quindi a lui il problema a regime risulta stazionario;
%
\end{footnote} del problema

\item {} 
\sphinxAtStartPar
riconoscendo che la superficie \(S_{s,f}\) non dà contributo al
bilancio, poiché la velocità del fluido sul contorno solido deve
essere uguale alla velocità del corpo, per la condizione al contorno
di \sphinxstylestrong{adesione}: \(\bm{u} = \bm{b}\) in ogni punto di \(S_{s,f}\). Anche
se fosse stato utilizzato un modello non viscoso per rappresentare
il problema, sarebbe valida la condizione al contorno di \sphinxstylestrong{non
penetrazione}:
\(\bm{u} \cdot \bm{\hat{n}} = \bm{b} \cdot \bm{\hat{n}}\) su tutti i
punti di \(S_{s,f}\);

\item {} 
\sphinxAtStartPar
la velocità della superficie \(S_0\) è nulla, \(\bm{b} = \bm{0}\) su
\(S_0\) e quindi la velocità relativa coincide con la velocità
assoluta \(\bm{U_0} = U_0 \bm{\hat{z}}\), dato del problema;

\item {} 
\sphinxAtStartPar
per i dati del problema, la velocità relativa del fluido sulle
sezioni \(S_1\), \(S_2\) è uniforme: \(\bm{u}-\bm{b} = V \bm{\hat{X}}\) su
\(S_1\), \(\bm{u}-\bm{b} = -V \bm{\hat{X}}\) su \(S_2\)

\end{itemize}

\sphinxAtStartPar
Dal bilancio di massa si trova quindi il modulo della velocità relativa
della corrente sulle sezioni \(S_1\), \(S_2\),
\$\(0 = - \rho U_0 \pi R_0^2 + 2 \rho V \pi R_1^2 \qquad \rightarrow \qquad
  V = \dfrac{1}{2} \left(\dfrac{R_0}{R_1}\right)^2 U_0 \ .\)\$

\sphinxAtStartPar
Il bilancio del momento della quantità di moto per il volume fluido
\(V_t\), \$\(\dfrac{d}{dt} \int_{V_t} \rho \bm{r} \times \bm{u} +
 \int_{\partial V_t} \rho \bm{r} \times \bm{u} \left( \bm{u} - \bm{b} \right) \cdot \bm{\hat{n}} =
 \int_{V_t} \rho \bm{r} \times \bm{g} + \int_{\partial V_t} \bm{r} \times \bm{t_n} \ .\)\(
Senza riportare i dettagli (TODO: riportare i dettagli) usando l'ipotesi
che gli effetti gravitazionali trascurabili e che la pressione sia
uniforme sulle superfici \)S\_0\(, \)S\_1\(, \)S\_2\( e uguale alla pressione
atmosferica attorno al'irrigatore, il bilancio del momento di quantità
di moto diventa
\)\(\bm{M}^s = - \dfrac{d}{dt} \int_{V_t} \rho \bm{r} \times \bm{u} - \oint_{\partial V_t} \rho \bm{r} \times \bm{u} \left( \bm{u} - \bm{b} \right) \cdot \bm{\hat{n}} \ ,\)\(
avendo messo in evidenza la risultante dei momenti \)\textbackslash{}bm\{M\}\textasciicircum{}s\( agenti sul
solido, rispetto all'origine dei sistemi di riferimento. Il contributo
della superficie laterale \)S\_\{s,f\}\( è nullo, poiché è nullo il termine
\)(\textbackslash{}bm\{u\}\sphinxhyphen{}\textbackslash{}bm\{b\}) \textbackslash{}cdot \textbackslash{}bm\{\textbackslash{}hat\{n\}\}\(; il contributo della superficie
\)S\_0\( è nullo per simmetria. Si può dimostrare (TODO: riportare i
dettagli) che il termine con la derivata temporale non genera un momento
attorno all'asse \)z\( di rotazione dell'irrigatore, e scrivere
\)\(\begin{aligned}
  M^s_z & = - \bm{\hat{z}} \cdot \oint_{\partial V_t} \rho \bm{r} \times \bm{u} \left( \bm{u} - \bm{b} \right) \cdot \bm{\hat{n}} = {\color{red} (...)} \\
   & = - 2 \, \bm{\hat{z}} \cdot \oint_{S_1} \rho \bm{r} \times \bm{u} \left( \bm{u} - \bm{b} \right) \cdot \bm{\hat{n}} \ ,
\end{aligned}\)\( avendo riconosciuto che i contributi al momento delle
superfici \)S\_1\( e \)S\_2\( sono uguali. Calcoliamo ora il termine di
superficie, esprimendo tutti i termini nel sistema di coordinate
solidale con l'irrigatore, \)\(\begin{aligned}
 \bm{b} = \bm{\Omega} \times \bm{r} & = \Omega \bm{\hat{Z}} \times \left( X \bm{\hat{X}} + Y \bm{\hat{Y}} + Z \bm{\hat{Z}}  \right) = -\Omega Y \bm{\hat{X}} + \Omega X \bm{\hat{Y}} \\
 \left( \bm{u} - \bm{b} \right) \cdot \bm{\hat{n}} = \bm{u}_{rel} \cdot \bm{\hat{n}} & =  V \bm{\hat{X}} \cdot \bm{\hat{X}} = V \\
 \bm{r} \times \bm{u} = \bm{r} \times (\bm{u}_{rel} + \bm{b} ) & = \left( X \bm{\hat{X}} + Y \bm{\hat{Y}} + Z \bm{\hat{Z}} \right) \times \left[ \left(V-\Omega Y\right)\bm{\hat{X}} + \Omega X \bm{\hat{Y}} \right] = \\
  & = - \Omega X Z \bm{\hat{X}} + \left( V - \Omega Y \right) Z \bm{\hat{Y}} + 
  \left[ \Omega \left( X^2 + Y^2 \right) - V \, Y \right] \bm{\hat{Z}} \ , 
\end{aligned}\)\( e usando un sistema di coordinate polari con origine nel
centro della sezione circolare \)S\_1\(, \)\(\begin{cases}
 X = h \\
 Y = \ell + r \cos\theta \\
 Z = r \sin\theta \\
\end{cases}\)\( \)\$\textbackslash{}begin\{aligned\}
\begin{itemize}
\item {} 
\sphinxAtStartPar
\textbackslash{}dfrac\{M\_z\textasciicircum{}s\}\{2\} \& = \textbackslash{}bm\{\textbackslash{}hat\{z\}\} \textbackslash{}cdot \textbackslash{}oint\_\{S\_1\} \textbackslash{}rho \textbackslash{}bm\{r\} \textbackslash{}times \textbackslash{}bm\{u\} \textbackslash{}left( \textbackslash{}bm\{u\} \sphinxhyphen{} \textbackslash{}bm\{b\} \textbackslash{}right) \textbackslash{}cdot \textbackslash{}bm\{\textbackslash{}hat\{n\}\} = \textbackslash{}
\& = \textbackslash{}rho \textbackslash{}int\_\{\textbackslash{}theta=0\}\textasciicircum{}\{2\textbackslash{}pi\}\textbackslash{}int\_\{r=0\}\textasciicircum{}\{R\_1\}
\textbackslash{}left{[} \textbackslash{}Omega\textbackslash{}left( h\textasciicircum{}2 + (\textbackslash{}ell + r \textbackslash{}cos\textbackslash{}theta)\textasciicircum{}2 \textbackslash{}right) \sphinxhyphen{} V \textbackslash{}left( \textbackslash{}ell + r \textbackslash{}cos \textbackslash{}theta \textbackslash{}right)  \textbackslash{}right{]} , V ,  r , dr , d\textbackslash{}theta = \textbackslash{}
\& = \textbackslash{}rho \textbackslash{}left{[} \textbackslash{}Omega \textbackslash{}left( h\textasciicircum{}2 + \textbackslash{}ell\textasciicircum{}2 \textbackslash{}right) \sphinxhyphen{} V \textbackslash{}ell \textbackslash{}right{]} , V , 2 \textbackslash{}pi \textbackslash{}dfrac\{R\_1\textasciicircum{}2\}\{2\} +
\textbackslash{}Omega , V , \textbackslash{}dfrac\{R\_1\textasciicircum{}4\}\{4\}, \textbackslash{}pi = \textbackslash{}
\& = \textbackslash{}pi R\_1\textasciicircum{}2 \textbackslash{}rho , V \textbackslash{}left{[} \textbackslash{}Omega \textbackslash{}left( h\textasciicircum{}2 + \textbackslash{}ell\textasciicircum{}2 + \textbackslash{}dfrac\{R\_1\textasciicircum{}2\}\{4\} \textbackslash{}right) \sphinxhyphen{} V \textbackslash{}ell \textbackslash{}right{]} \textbackslash{} .
\textbackslash{}end\{aligned\}\$\( A regime, la componente \)z\( della risultante dei momenti
deve essere nulla e la velocità angolare deve essere uguale a
\)\(\Omega = \dfrac{\ell}{h^2 + \ell^2 + \dfrac{R_1^2}{4}} V
\qquad \rightarrow \qquad
\Omega = \dfrac{1}{2}\dfrac{\ell}{h^2 + \ell^2 + \dfrac{R_1^2}{4}} \left( \dfrac{R_0}{R_1} \right)^2 \, U_0 \ .\)\$

\end{itemize}


\bigskip\hrule\bigskip


\sphinxstepscope


\subsection{Exercise 4.11}
\label{\detokenize{polimi/fluidmechanics-ita/template/capitoli/04_bilanci/04e01in:exercise-4-11}}\label{\detokenize{polimi/fluidmechanics-ita/template/capitoli/04_bilanci/04e01in:fluid-mechanics-balances-ex-11}}\label{\detokenize{polimi/fluidmechanics-ita/template/capitoli/04_bilanci/04e01in::doc}}

\bigskip\hrule\bigskip


\sphinxAtStartPar
Viene chiesto di determinare la potenza dei motori della galleria a circuito aperto rappresentata in figura, sapendo che la velocità massima desiderata nella sezione di prova è \(V_{test} = 30 \, m/s\), l’area della sezione di prova è \(A_{test} = 1.0 \, m^2\) e l’area della sezione in cui è alloggiato il ventilatore che mette in moto l’aria è \(A_{fan} = 2.0 \, m^2\). Si supponga che la corrente sia incomprimibile e che la densità dell’aria sia \(\rho = 1.1 \, kg/m^3\). In una prima fase, si trascuri la caduta di pressione attraverso il nido d’ape e gli schermi presenti tra la sezione 1 e la sezione 2 del condotto. Successivamente si ripeta il calcolo con una caduta di pressione \(P_1 - P_2 = k \rho U^2\), con \(k = \dots\).


\bigskip\hrule\bigskip


\sphinxAtStartPar
\sphinxincludegraphics{{polimi/fluidmechanics-ita/template/capitoli/04_bilanci/fig/wt}}\{width=»90\%»\}

\sphinxAtStartPar
Si studia la galleria a circuito aperto rappresentata in figura
utilizzando i bilanci integrali scritti per alcuni volumi di controllo
fissi, per ricavare l’andamento della velocità e della pressione
all’interno della galleria e infine ricavare la potenza dei motori,
necessaria per garantire le condizioni di progetto nella sezione di
prova. Si ipotizza un funzionamento stazionario, si trascurano gli
effetti viscosi nel volume e sulle pareti della galleria e le forze di
volume. In particolare, grazie alle ipotesi fatte, si possono
semplificare il bilancio di massa, \$\(\begin{aligned}
 & \dfrac{d}{dt} \int_V \rho = - \oint_{S} \rho \bm{u} \cdot \bm{\hat{n}} 
 \hspace{1.0cm} \rightarrow \hspace{1.0cm} \oint_{S} \rho \bm{u} \cdot \bm{\hat{n}} = 0 \ ,
\end{aligned}\)\( e il bilancio dell'energia cinetica, \)\$\textbackslash{}begin\{aligned\}
\textbackslash{}dfrac\{d\}\{dt\} \textbackslash{}int\_V \textbackslash{}rho \textbackslash{}dfrac\{|\textbackslash{}bm\{u\}|\textasciicircum{}2\}\{2\} \& = \sphinxhyphen{} \textbackslash{}oint\_\{S\} \textbackslash{}rho \textbackslash{}dfrac\{|\textbackslash{}bm\{u\}|\textasciicircum{}2\}\{2\} \textbackslash{}bm\{u\} \textbackslash{}cdot \textbackslash{}bm\{\textbackslash{}hat\{n\}\} + \textbackslash{}oint\_S \textbackslash{}bm\{t\_n\} \textbackslash{}cdot \textbackslash{}bm\{u\} \sphinxhyphen{} \textbackslash{}int\_V \textbackslash{}bm\{\textbackslash{}nabla\} \textbackslash{}bm\{u\} : \textbackslash{}mathbb\{T\} + \textbackslash{}int\_V \textbackslash{}rho \textbackslash{}bm\{g\} \textbackslash{}
\& = \sphinxhyphen{} \textbackslash{}oint\_\{S\} \textbackslash{}rho \textbackslash{}dfrac\{|\textbackslash{}bm\{u\}|\textasciicircum{}2\}\{2\} \textbackslash{}bm\{u\} \textbackslash{}cdot \textbackslash{}bm\{\textbackslash{}hat\{n\}\} + \textbackslash{}oint\_S \textbackslash{}bm\{t\_n\} \textbackslash{}cdot \textbackslash{}bm\{u\} +
\textbackslash{}underbrace\{\textbackslash{}int\_V  p , \textbackslash{}bm\{\textbackslash{}nabla\} \textbackslash{}cdot \textbackslash{}bm\{u\}\}\_\{=0 \textbackslash{}text\{, se\} \textbackslash{}bm\{\textbackslash{}nabla\} \textbackslash{}cdot \textbackslash{}bm\{u\} = 0\}
\begin{itemize}
\item {} 
\sphinxAtStartPar
\textbackslash{}underbrace\{ \textbackslash{}int\_V 2 \textbackslash{}mu \textbackslash{}mathbb\{D\} : \textbackslash{}mathbb\{D\}\}\_\{=0 \textbackslash{}text\{, se\} \textbackslash{}mu = 0\} + \textbackslash{}int\_V \textbackslash{}rho \textbackslash{}bm\{g\} \textbackslash{}
\textbackslash{}hspace\{1.0cm\} \textbackslash{}rightarrow \textbackslash{}hspace\{1.0cm\}  \&

\item {} 
\sphinxAtStartPar
\textbackslash{}oint\_\{S\} \textbackslash{}rho \textbackslash{}dfrac\{|\textbackslash{}bm\{u\}|\textasciicircum{}2\}\{2\} \textbackslash{}bm\{u\} \textbackslash{}cdot \textbackslash{}bm\{\textbackslash{}hat\{n\}\} + \textbackslash{}oint\_S \textbackslash{}bm\{t\_n\} \textbackslash{}cdot \textbackslash{}bm\{u\} = 0 \textbackslash{} .
\textbackslash{}end\{aligned\}\$\$

\end{itemize}

\sphinxAtStartPar
Viene svolta la prima parte dell’esercizio, trascurando le perdite di
pressione che avvengono tra la sezione 1 e la sezione 2, a causa della
presenza dei nidi d’ape e delle reti. Si scrive il bilancio di massa per
un volume di fluido che ha come superficie di contorno la superficie
\(S_0\), la superficie laterale del tubo di flusso e una superficie \(S_i\)
all’interno della galleria. Assumendo grandezze uniformi sulla sezione,
si può scrivere \$\(\rho A_0 U_0 = \rho A_i U_i \ ,\)\( cioè che il flusso
di massa \)\textbackslash{}dot\{m\}\( che attraversa le sezioni della galleria è costante.
Se sono note le condizioni di progetto in camera di prova, da esser si
può calcolare il flusso di massa,
\)\(\dot{m} = \rho A_3 U_3 = \rho A_{test} V_{test} = \dots \ .\)\( Poiché
la velocità all'infinito è nulla, \)U\_0 \textbackslash{}rightarrow 0\(, l'area della
sezione all'infinito a monte deve tendere all'infinito
\)A\_0 \textbackslash{}rightarrow \textbackslash{}infty\(. Si scrive poi il bilancio di energia cinetica
per un volume di controllo che ha come contorno la superficie \)S\_0\(
all'infinito a monte, dove viene aspirata l'aria in uno stato di quiete,
la superficie laterale del tubo di flusso, la superficie interna della
galleria e la sezione \)S\_4\( alla fine del divergente, poco prima
dell'imbocco dei ventilatori. Poiché non ci sono organi meccanici in
movimento, il termine \)\textbackslash{}oint\_S \textbackslash{}bm\{t\_n\} \textbackslash{}cdot \textbackslash{}bm\{u\}\( è nullo, e
assumendo grandezze fisiche costanti sulle sezioni si può scrivere,
\)\(\rho A_0 U_0 \left( \dfrac{U_0^2}{2} + \dfrac{P_0}{\rho} \right) =
 \rho A_4 U_4 \left( \dfrac{U_4^2}{2} + \dfrac{P_4}{\rho} \right) \ .\)\(
Poiché il flusso di massa che attraversa le sezioni considerate è
costante, il bilancio di energia cinetica si riduce a un'espressione che
ricorda quella del teorema di Bernoulli, così come viene enunciato alle
scuole superiori,
\)\(P_0 + \dfrac{1}{2} \rho \, U_0^2  = P_4 + \dfrac{1}{2} \rho \, U_4^2
 \qquad \rightarrow \qquad B_4 = B_0 = P_{atm} \ ,\)\( avendo introdotto
la definizione del "binomio di Bernoulli", \)B\_i = P\_i + \textbackslash{}rho U\_i\textasciicircum{}2 / 2\(.
Si scrive poi il bilancio di energia cinetica per il volume fluido
\)V(t)\( che contiene il ventilatore, delimintato dalle superifci \)S\_4\(,
\)S\_5\( e la superficie interna della galleria e dalla superficie
(mobile!) del ventilatore. Il bilancio diventa
\)\(\int_{S_4} \rho \dfrac{|\bm{u}|^2}{2} \bm{u} \cdot \bm{\hat{n}} - \bm{t_n} \cdot \bm{u} + 
\int_{S_5} \rho \dfrac{|\bm{u}|^2}{2} \bm{u} \cdot \bm{\hat{n}} - \bm{t_n} \cdot \bm{u} = \int_{S_{fan}} \bm{t_n} \cdot \bm{u} \ ,\)\(
essendo il termine a destra dell'uguale la potenza delle forze
essercitata dal ventilatore sul fluido, contraria a quella esercitata
dal fluido sul ventilatore, ma uguale a quella che deve fornire il
motore elettrico per poter garantire la rotazione del ventialore stesso.
Se si trascurano gli sforzi viscosi sulle superfici \)S\_4\( ed \)S\_5\(,
\)\textbackslash{}bm\{t\_n\} = \textbackslash{}bm\{s\_n\} \sphinxhyphen{} P \textbackslash{}bm\{\textbackslash{}hat\{n\}\}\(, e se si esplicita la potenza che
deve essere fornita dai motori, il bilancio diventa, \)\(W_{mot} = 
\int_{S_4} \left( \rho \dfrac{|\bm{u}|^2}{2} + P \right) \left( \bm{u} \cdot \bm{\hat{n}} \right)  + 
\int_{S_5} \left( \rho \dfrac{|\bm{u}|^2}{2} + P \right) \left( \bm{u} \cdot \bm{\hat{n}} \right) \ ,\)\(
e facendo l'ipotesi di grandezze fisiche costanti sulle sezioni,
\)\(W_{mot} = \rho A_5 U_5 \left( \dfrac{U_5^2}{2} + \dfrac{P_5}{\rho} \right) 
         - \rho A_4 U_4 \left( \dfrac{U_4^2}{2} + \dfrac{P_4}{\rho} \right)
         = \dot{m} \left( B_5 - B_4 \right) \ .\)\( Ricordando che il
"binomio di Bernoulli" nelle sezioni 1:4 è uguale al "binomio di
Bernoulli" nella sezione \)S\_0\(, e quindi uguale alla pressione ambiente
\)P\_\{atm\}\(, nell'ipotesi che la pressione nella sezione \)S\_5\( sia uguale
alla pressione atmosferica \)P\_\{atm\}\( all'esterno del tubo di flusso, la
potenza del motore diventa, \)\(W_{mot} = \dot{m} \dfrac{U_5^2}{2} \ ,\)\(
e, riferendosi alle grandezze fisiche in camera di prova, può essere
scritta come \)\(\begin{aligned}
 & W_{mot} = \dot{m} \left( \dfrac{A_{test}}{A_{fan}} \right)^2 \dfrac{V_{test}^2}{2} \\
 & \hspace{3.0cm} \rightarrow \qquad 
 W_{mot} = \dfrac{1}{2} \rho A_{test} \left( \dfrac{A_{test}}{A_{fan}} \right)^2 V_{test}^3 = \dots kW \ .
\end{aligned}\)\( La formula della potenza dei motori necessaria al
funzionamento della galleria mette in evidenza la dipendenza dal cubo
della velocità di prova e dal quadrato del rapporto tra l'area della
sezione di prova e l'area della sezione all'imbocco delle ventole.
Questo ultimo termine dovrebbe chiarire uno degli obiettivi del
divergente della galleria: rallentare la corrente dopo la sezione di
prova, per poter ridurre la potenza dei motori da installare per
garantire il funzionamento dell'impianto. **Osservazione.** Potrebbe
suscitare qualche perplessità il fatto che la corrente in uscita
dall'impianto con velocità \)U\_5 \textbackslash{}simeq V\_\{fan\}\( abbia una pressione
uguale alla pressione ambiente, \)P\_\{atm\}\(, come il fluido in quiete
all'esterno del tubo di flusso. Provando ad applicare il teorema di
Bernoulli tra un punto sulla sezione del tubo di flusso \)S\_5\( e un punto
all'esterno del tubo di flusso,
\)\(P_5 + \dfrac{1}{2}\rho V_{fan}^2 = P_5^{out}
 \quad \rightarrow \quad 
 P_{atm} + \dfrac{1}{2}\rho V_{fan}^2 = P_{atm} \ ,\)\( si giungerebbe
alla conclusione che \)V\_\{fan\} = 0\(. L'errore risiede nell'applicazione
del teorema di Bernoulli nella formula vista alla scuola superiore (o in
altri corsi universitari), nonostante alcune ipotesi (che verranno
presentate nel prosieguo del corso) non siano rispettate. In
particolare, per collegare un punto sulla sezione \)S\_5\( e un punto
all'esterno del tubo di flusso viene attraversato uno strato di
mescolamento tra la corrente in moto che esce dalla galleria e il fluido
in quiete all'esterno: la presenza di questo strato di mescolamento, nel
quale la corrente non è irrotazionale \)\textbackslash{}bm\{\textbackslash{}omega\} \textbackslash{}neq 0\$, fa cadere le
ipotesi del teorema di Bernoulli e lo rende quindi inapplicabile. Tutte
le parti evidenziate in rosso devono quindi essere considerate errate.

\sphinxstepscope


\subsection{Exercise 4.12}
\label{\detokenize{polimi/fluidmechanics-ita/template/capitoli/04_bilanci/04e02in:exercise-4-12}}\label{\detokenize{polimi/fluidmechanics-ita/template/capitoli/04_bilanci/04e02in:fluid-mechanics-balances-ex-12}}\label{\detokenize{polimi/fluidmechanics-ita/template/capitoli/04_bilanci/04e02in::doc}}
\sphinxAtStartPar
+:———————————:+:———————————:+
| Il funzionamento di un motore     |                                   |
| alternativo a benzina (a quattro  |                                   |
| tempi) può essere rappresentato   |                                   |
| in prima approssimazione con un   |                                   |
| ciclo termodinamico Otto ideale,  |                                   |
| rappresentato da una compressione |                                   |
| adiabatica, una fase veloce di    |                                   |
| combustione a volume costante     |                                   |
| (nel punto morto superiore del    |                                   |
| moto del pistone, PMS) e          |                                   |
| un’espansione adiabatica. Le fasi |                                   |
| di aspirazione e scarico dei gas  |                                   |
| combusti sono anch’essi ideali.   |                                   |
| L’aspirazione avviene a pressione |                                   |
| costante durante il movimento del |                                   |
| pistone dal PMS al punto morto    |                                   |
| inferiore (PMI). La fase di       |                                   |
| scarico avviene in due fasi:      |                                   |
| durante la prima fase la          |                                   |
| pressione diminuisce molto        |                                   |
| velocemente (approssimata da una  |                                   |
| trasformazione a volume costante) |                                   |
| a causa dell’apertura della       |                                   |
| valvola di scarico quando il      |                                   |
| pistone si trova al PMI; durante  |                                   |
| la seconda fase i gas combusti    |                                   |
| sono spinti fuori dalla camera di |                                   |
| combustione dal movimento         |                                   |
| ascendente del pistone che si     |                                   |
| riporta al PMS, per l’inizio del  |                                   |
| ciclo termodinamico successivo.   |                                   |
| Del motore sono noti:             |                                   |
|                                   |                                   |
| \sphinxhyphen{}   il rapporto di compressione,  |                                   |
|     definito come il rapporto tra |                                   |
|     il volume massimo (pistone al |                                   |
|     PMI) e minimo (pistone al     |                                   |
|     PMS) della camera di          |                                   |
|     combustione,                  |                                   |
|     \(r = V_1 / V_2 = 10\);         |                                   |
|                                   |                                   |
| \sphinxhyphen{}   la cilindrata, definita come  |                                   |
|     la corsa del pistone per      |                                   |
|     l’area della sezione del      |                                   |
|     cilindro, e uguale alla       |                                   |
|     differenza                    |                                   |
|     \(C = N (V_2 - V_1) = 1000 \ c |                                   |
| c\),                               |                                   |
|     essendo \(N\) il numero di      |                                   |
|     cilindri del motore;          |                                   |
|                                   |                                   |
| \sphinxhyphen{}   le condizioni termodinamiche  |                                   |
|     dell’aria all’aspirazione     |                                   |
|     \(P_0 = 85570 \, Pa\),          |                                   |
|     \(T_0 = 25°C\);                 |                                   |
|                                   |                                   |
| \sphinxhyphen{}   il rapporto in massa tra      |                                   |
|     benzina e aria,               |                                   |
|     \(f = m_f / m_a = 0.06\);       |                                   |
|                                   |                                   |
| \sphinxhyphen{}   il potere calorifico della    |                                   |
|     benzina usata                 |                                   |
|     \(\Delta h = 43 \, MJ\);        |                                   |
|                                   |                                   |
| \sphinxhyphen{}   la pressione nel basamento    |                                   |
|     del motore,                   |                                   |
|     \(p_{b} = 150000 \, Pa\)        |                                   |
|     uniforme e costante. Si       |                                   |
|     calcoli la potenza media      |                                   |
|     erogata dal motore a un       |                                   |
|     regime di rotazione di        |                                   |
|     \(\Omega = 3000 RPM\),          |                                   |
|     assumendo un rendimento       |                                   |
|     meccanico \(\eta = 0.8\). Si    |                                   |
|     rappresenti l’aria come un    |                                   |
|     gas bi\sphinxhyphen{}atomico perfetto       |                                   |
|     (\(\gamma = c_P/ c_v = 1.4\))   |                                   |
|     con costante dei gas          |                                   |
|     \(R = 287 J / (kg \, K)\), e si |                                   |
|     trascuri l’effetto del        |                                   |
|     carburante sul valore dei     |                                   |
|     calori specifici e sulla      |                                   |
|     massa presente all’interno    |                                   |
|     della camera di combustione.  |                                   |
|     Si trascurino inoltre gli     |                                   |
|     scambi di calore per          |                                   |
|     conduzione con l’esterno del  |                                   |
|     cilindro durante la           |                                   |
|     compressione e l’espansione   |                                   |
|     (trasformazioni adiabatiche). |                                   |
|     Si trascurino i termini       |                                   |
|     cinetici nell’energia totale  |                                   |
|     in camera di combustione,     |                                   |
|     facendo coincidere l’energia  |                                   |
|     totale con l’energia interna  |                                   |
|     \(e^t = e = c_v T\), e si       |                                   |
|     assuma che le variabili       |                                   |
|     termodinamiche siano uniformi |                                   |
|     (costanti in spazio, non in   |                                   |
|     tempo) in camera di           |                                   |
|     combustione.                  |                                   |
+———————————–+———————————–+

\sphinxAtStartPar
\sphinxincludegraphics{{polimi/fluidmechanics-ita/template/capitoli/04_bilanci/fig/otto_cycle}}\{width=»45\%»\}

\sphinxAtStartPar
Ogni fase del ciclo termodinamico viene analizzata con i bilanci
integrali, per il volume corrispondente alla camera di combustione di un
cilindro. Questo volume è un sistema aperto durante la fase di
aspirazione e scarico (scambia massa con l’esterno), mentre è un sistema
chiuso durante la compressione, la combustione e l’espansione (valvole
chiuse, nessuno scambio di massa con l’esterno). Si calcola il lavoro
svolto dal sistema durante un ciclo e si divide per il periodo per
ricavare la potenza media.

\sphinxAtStartPar
Conoscendo il numero dei cilindri \(N=3\), il rapporto di compressione \(r\)
e la cilindrata \(C\) è possibile ricavare il valore del volume massimo
\(V_1\) e minimo \(V_2\) della camera di combustione. \$\(\begin{cases}
 N( V_2 - V_1 ) = C \\
 V_1 / V_2 = r
\end{cases} \qquad \rightarrow \qquad
\begin{cases}
 V_1 = \dfrac{r}{r-1} \dfrac{C}{N} \\ \\ 
 V_2 = \dfrac{1}{r-1} \dfrac{C}{N}
\end{cases}\)\$ Si analizzano ora le fasi del ciclo termodinamico,
fornendo una breve descrizione e ponendo attenzione allo scambio di
massa (sistema chiuso/aperto), lavoro e calore con l’esterno.
\begin{itemize}
\item {} 
\sphinxAtStartPar
\sphinxstylestrong{Aspirazione}, \(0 \rightarrow 1\): la prima fase del ciclo Otto è
l’aspirazione. Durante la fase di aspirazione (ideale), la valvola
di aspirazione è aperta e il sistema scambia massa con l’esterno: il
pistone si sposta dal PMS al PMI e la camera di combustione si
riempie d’aria a pressione e temperatura costante,
\$\(p_1 = p_0 \qquad , \qquad
 T_1 = T_0 \qquad , \qquad
 \rho_1 = \rho_0 = \dfrac{p_0}{R T_0} =\)\( La massa contenuta nella
camera di combustione alla chiusura della valvola, in coincidenza
del PMI, è \)\(m = \rho_1 V_1 = \dots \ .\)\( Durante la fase di
aspirazione, il pistone deve vincere la sovrapressione del basamento
(di solito la pressione nel basamento è superiore a quella aspirata
in camera di combustione). Dal PMS al PMI un pistone assorbe parte
della potenza fornita dagli altri pistoni. Il lavoro che assorbe è
\)L\_\{01\} = \sphinxhyphen{}(p\_b\sphinxhyphen{}p\_0) , c\( (negativo poichè assorbito), essendo \)c\$
la corsa del pistone e la differenza di pressione costante durante
l’aspirazione. Questo lavoro assorbito durante l’aspirazione sarà
uguale e contrario a quello fornito durante lo scarico ideale dei
gas, che avviene alla stessa differenza di pressione con un moto
opposto.

\item {} 
\sphinxAtStartPar
\sphinxstylestrong{Compressione}, \(1 \rightarrow 2\): la seconda fase del ciclo
termodinamico è la compressione del fluido che avviene a causa del
movimento verso l’alto del pistone. Il sistema è chiuso: le valvole
sono chiuse e si ipotizza che non ci sia trafilamento (\sphinxstyleemphasis{blow\sphinxhyphen{}by})
tra il pistone e la superficie laterale del cilindro. Il bilancio di
energia totale per il fluido contenuto all’interno del volume \(V(t)\)
(variabile nel tempo, a causa del moto del pistone) della camera di
combustione,
\$\(\dfrac{d}{dt} \displaystyle\int_{V(t)} \rho e^t + \oint_{S(t)} \rho e^t (\bm{u}-\bm{v}) \cdot \bm{\hat{n}}= \int_{V(t)} \bm{f} \cdot \bm{u} + \oint_{S(t)} \bm{t_n} \cdot \bm{u} - \oint_{S(t)} \bm{q} \cdot \bm{\hat{n}} + \int_{V(t)} \rho r \ .\)\(
può essere semplificato, trascurando l'effetto delle forze di
volume, \)\textbackslash{}bm\{f\} = \textbackslash{}bm\{0\}\(, trascurando la trasmissione del calore
con l'esterno (trasformazione adiabatica),
\)\textbackslash{}bm\{q\} \textbackslash{}cdot \textbackslash{}bm\{\textbackslash{}hat\{n\}\} = 0\(, e non essendoci sorgenti di calore,
\)r = 0\(. Inoltre non c'è flusso di massa attraverso il contorno
\)S(t)\( del volume, \)(\textbackslash{}bm\{u\}\sphinxhyphen{}\textbackslash{}bm\{v\}) \textbackslash{}cdot \textbackslash{}bm\{\textbackslash{}hat\{n\}\} = 0\(, e
l'unica superficie in movimento della camera di combustione
corrisponde al cielo (la faccia superiore) del pistone, \)S\_\{c\}\(.
Trascurando il contributo cinetico e approssimando l'energia totale
\)e\textasciicircum{}t = e + |\textbackslash{}bm\{u\}|\textasciicircum{}2/2\( con l'energia interna \)e\(, il bilancio di
energia diventa,
\)\(\dfrac{d}{dt} \displaystyle\int_{V(t)} \rho e = \int_{S_c(t)} \bm{t_n} \cdot \bm{u} \ ,\)\(
legando la derivata temporale dell'energia del fluido nella camera
di combustione alla potenza delle forze esercitate dal pistone sul
fluido. La potenza delle forze agenti sul pistone è uguale
all'integrale superficiale del prodotto scalare vettore sforzo
\)\textbackslash{}bm\{t\_\{n,s\}\}\( agente sul solido per la velocità \)\textbackslash{}bm\{v\}\( della
superficie del solido, \)\(\begin{aligned}
 W_{12} = \oint_{S_{s}} \bm{t_{n,s}} \cdot \bm{v}
 & = \int_{S_{s,c}} \bm{t_{n,s}} \cdot \bm{v} + \int_{S_{s,b}} \bm{t_{n,s}} \cdot \bm{v}  + \int_{S_{s,lat}} \bm{t_{n,s}} \cdot \bm{v} = \\
 & = - \int_{S_{c}} \bm{t_{n}} \cdot \bm{u} - \int_{S_{s,b}} p_b\bm{\hat{n}_{s}} \cdot \bm{v} \ ,
\end{aligned}\)\( avendo suddiviso la superficie del cilindro \)S\_s\(
come l'unione della superficie superiore \)S\_\{s,c\}\( (cielo),
superficie laterale \)S\_\{s,lat\}\( (dal contributo nullo, per
simmetria), e superifice inferiore \)S\_\{s,b\}\( esposta verso il
basamento del motore, sulla quale agisce uno sforzo dovuto alla
pressione \)p\_b\( dell'ambiente all'interno del basamento. É stato
indicata con \)\textbackslash{}bm\{\textbackslash{}hat\{n\}\sphinxstyleemphasis{s\}\( la normale uscente dalla superficie
del solido e con \)\textbackslash{}bm\{t}\{n,s\}\}\( il vettore sforzo agente su un punto
della superficie del solido, uguale e contrario a quello agente sul
fluido \)\textbackslash{}bm\{t\_n\} = \sphinxhyphen{}\textbackslash{}bm\{t\_\{n,s\}\}\( per il principio di azione e
reazione. Inoltre, le condizioni al contorno impongono che il fluido
e il solido abbiano la stessa velocità \)\textbackslash{}bm\{u\} = \textbackslash{}bm\{v\}\( sulle
superfici di contatto. Si può quindi riscrivere il bilancio di
energia del fluido in funzione della potenza \)W\_\{12\}\( trasmessa al
pistone,
\)\(\dfrac{d}{dt} \displaystyle\int_{V(t)} \rho e = - W_{12} - p_b S_{c} \ v(t) = - W_{12} - p_b \dfrac{d V}{d t} \ ,\)\(
essendo \)v(t)\( la velocità del pistone, per ottenere la potenza
trasmessa al pistone dal fluido (sarà una potenza richiesta, \)<0\(),
\)\(\begin{aligned}
  W_{12}(t) & = - \dfrac{d}{dt} \displaystyle\int_{V(t)} \rho e - p_b \dfrac{d V}{d t} = \\
  & = - \dfrac{d}{dt} \left( \rho V e \right) - p_b \dfrac{d V}{d t} = \\
  & = - m \dfrac{d e}{d t} - p_b \dfrac{d V}{d t} \ ,
\end{aligned}\)\( nell'ipotesi di variabili termodinamiche uniformi
nel volume, ricordando che la massa contenuta nella camera di
combustione \)m = \textbackslash{}rho V\( rimane costante, essendo un sistema chiuso,
se si trascura l'effetto di trafilamento tra le pareti di cilindro e
pistone (ridotte al minimo da fasce elastiche e anelli raschiaolio
sul pistone e sovra-pressione nel basamento). Integrando in tempo la
potenza istantantea \)W\_\{12\}(t)\(, tra il punto 1 e il punto 2 del
ciclo, si ottiene il lavoro di compressione
\)\(L_{12} = - m (e_2 - e_1) - p_b ( V_2 - V_1 ) \ .\)\( Utilizzando la
legge di stato dei gas perfetti \)p = \textbackslash{}rho R T\( e il legame tra le
variabili termodinamiche durante una trasformazione adiabatica
\)p/\textbackslash{}rho\textasciicircum{}\textbackslash{}gamma = \textbackslash{}text\{cost\}\(, si ottiene
\)\(e_2 - e_1 = c_v ( T_2 - T_1 ) = c_v T_1 \left[ \left( \dfrac{\rho_2}{\rho_1} \right)^{\gamma-1} - 1 \right] = c_v T_1 \left( r^{\gamma-1} - 1\right) \ .\)\$

\item {} 
\sphinxAtStartPar
\sphinxstylestrong{Combustione}, \(2 \rightarrow 3\): la terza fase del ciclo
termodinamico è la combustione. Viene iniettato il combustibile
all’interno della camera di combustione, innescata dall’accensione
di una candela in un motore a benzina classico. Durante l’iniezione
del combustibile il sistema è aperto. In prima approssimazione si
può trascurare la variazione di massa,
\(m + m_f = m ( 1 + f ) \simeq m\). In prima approssimazione, si può
rappresentare questa fase con una trasformazione isocora (volume
costante) associata a un aumento di pressione e temperatura, a causa
di una combustione (completa) veloce in corrispondenza del PMS. Il
bilancio di energia che descrive questa fase diventa
\$\(\begin{aligned}
 \dfrac{d}{dt} \displaystyle\int_{V(t)} \rho e & = \int_{V(t)} \rho r \\
 m \dfrac{d e}{d t} & = \dot{m}_f \Delta h \qquad \rightarrow \qquad
 e_3 - e_2 = \dfrac{ m_f }{ m } \Delta h  = f \Delta h \ .
\end{aligned}\)\( Utilizzando l'espressione dell'energia interna
\)e = c\_v T\(, \)\(c_v T_3 = c_v T_2 + f \Delta h \ .\)\$

\item {} 
\sphinxAtStartPar
\sphinxstylestrong{Espansione}, \(3 \rightarrow 4\): la quarta fase del ciclo è
l’espansione. Trascurando gli scambi di calore con l’esterno, la
trasformazione è adiabatica. Facendo le stesse ipotesi fatte per la
fase di compressione, si ottiene un lavoro di espansione (fornito al
pistone, \(>0\)) \$\(L_{34} = -m (e_4-e_3) - p_b ( V_4 - V_3 ) \ .\)\(
Utilizzando la legge di stato dei gas perfetti \)p = \textbackslash{}rho R T\( e il
legame tra le variabili termodinamiche durante una trasformazione
adiabatica \)p/\textbackslash{}rho\textasciicircum{}\textbackslash{}gamma = \textbackslash{}text\{cost\}\(, si ottiene
\)\(\begin{aligned}
 e_4 - e_3 & = c_v ( T_4 - T_3 ) = \\
  & = c_v T_3 \left[ \left( \dfrac{\rho_4}{\rho_3} \right)^{\gamma-1} - 1 \right] = \\
  & = c_v T_3 \left( r^{-\gamma+1} - 1\right) = \\
  & = c_v T_2 \left( r^{-\gamma+1} - 1\right) +
   f \Delta h \left( r^{-\gamma+1} - 1\right)  = \\ 
  & = c_v T_1 \left( 1 - r^{ \gamma-1}\right) +
   f \Delta h \left( r^{-\gamma+1} - 1\right)  = \ .
\end{aligned}\)\$

\item {} 
\sphinxAtStartPar
\sphinxstylestrong{Scarico}, \(4 \rightarrow 5, 5 \rightarrow 6\): la fase di scarico
(libera) è considerata istantanea e quindi non viene compiuto lavoro
da parte del fluido sul sistema meccanico. Durante la fase di
scarico forzata, mentre si muove dal PMI al PMS, il pistone compie
un lavoro \(L_{46} = (p_b - p_0) c\), uguale e contrario a quello
compiuto durante la fase di aspirazione se la pressione di
aspirazione e di scarico sono uguali (\(p_0 = p_1 = p_5\)).

\end{itemize}

\sphinxAtStartPar
Il lavoro complessivo fornito dal fluido al sistema meccanico durante un
ciclo è quindi uguale a \$\(\begin{aligned}
 L = L_{12} + L_{34} & = \dots = \\
 & = f \, m \, \Delta h \left( 1 - r^{-\gamma+1}\right)  \ .
\end{aligned}\)\( Il risultato ottenuto può essere facilmente interpretato
in termini termodinamici, essendo \)Q\_\{in\} = f , m , \textbackslash{}Delta h\( il
calore fornito alla macchina termica e \)\textbackslash{}eta = 1 \sphinxhyphen{} r\textasciicircum{}\{\sphinxhyphen{}\textbackslash{}gamma+1\}\( il
rendimento del ciclo Otto espresso in funzione del rapporto di
compressione \)r\(, \)\(L = \eta \, Q_{in} \ .\)\( Nonostante il risultato
ottenuto non sia nuovo, lo svolgimento dovrebbe fornire uno svolgimento
più dettagliato che parta dai principi fisici, rappresentati dai bilanci
integrali, ed evidenziare il ruolo delle ipotesi fatte per ricavare il
risultato, come ad esempio l'assenza di flussi di calore durante la fase
di compressione e espansione adiabatica. Per ottenere la potenza media
fornita dal motore, bisogna moltiplicare il lavoro \)L\( fornito da un
pistone per il numero \)N\( dei cilindri del motore e dividere per il
periodo del ciclo \)T = \textbackslash{}dfrac\{2 \textbackslash{}pi\}\{\textbackslash{}Omega\} \textbackslash{}dfrac\{n\}\{2\}\(, essendo
\)\textbackslash{}Omega\( la velocità di rotazione dell'albero motore ed \)n = 4\( il
numero dei tempi del motore,
\)\(W = \dfrac{N L}{T} = \dfrac{\Omega }{n \pi} \, f \, \Delta h \, \rho_1 N V_1 \, \left( 1 - r^{1-\gamma} \right) \ ,\)\(
e introducendo la definizione di cilindrata,
\)\(W = \dfrac{N L}{T} = \dfrac{\Omega }{n \pi} \, f \, \Delta h \, \rho_1 C \dfrac{r}{r-1} \, \left( 1 - r^{1-\gamma} \right) = 43.14 \, kW = 58.6 \, CV \ .\)\$

\sphinxstepscope


\subsection{Exercise 4.13}
\label{\detokenize{polimi/fluidmechanics-ita/template/capitoli/04_bilanci/04e03in:exercise-4-13}}\label{\detokenize{polimi/fluidmechanics-ita/template/capitoli/04_bilanci/04e03in:fluid-mechanics-balances-ex-13}}\label{\detokenize{polimi/fluidmechanics-ita/template/capitoli/04_bilanci/04e03in::doc}}
\sphinxAtStartPar
+:———————————:+:———————————:+
| Un aereo vola alla velocità       |                                   |
| \(V=250 \, m/s\) alla quota         |                                   |
| \(z=10000 \, m\), dove la pressione |                                   |
| e la temperatura atmosferica sono |                                   |
| \(P_0 = 26500 \, Pa\) e             |                                   |
| \(T_0 = 223.25 \, K\), spinto dal   |                                   |
| motore a getto rappresentato in   |                                   |
| figura. Sapendo che:              |                                   |
|                                   |                                   |
| \sphinxhyphen{}   \(0 \rightarrow 1\): la presa   |                                   |
|     d’aria è progettata per       |                                   |
|     ottenere una compressione     |                                   |
|     adiabatica ideale             |                                   |
|     (isentropica), con            |                                   |
|     \(P_1/P_0 = 1.5\);              |                                   |
|                                   |                                   |
| \sphinxhyphen{}   \(1 \rightarrow 2\): il         |                                   |
|     compressore ideale ha una     |                                   |
|     sezione di ingresso           |                                   |
|     \(A_1 = \dots\) e produce un    |                                   |
|     rapporto di pressione totale  |                                   |
|     \(P_2^t/P_1^t = 40.0\), tramite |                                   |
|     una trasformazione adiabatica |                                   |
|     ideale;                       |                                   |
|                                   |                                   |
| \sphinxhyphen{}   \(2 \rightarrow 3\): il         |                                   |
|     combustore garantisce una     |                                   |
|     perfetta combustione          |                                   |
|     mantenendo costante la        |                                   |
|     pressione totale al suo       |                                   |
|     interno \(P_2^t = P_3^t\); il   |                                   |
|     flusso di calore prodotto     |                                   |
|     dalla combustione è uguale a  |                                   |
|     \(\dot{Q}_c = \dot{m}_f \Delta |                                   |
|  h_c\),                            |                                   |
|     dove \(\dot{m}_f\) è il flusso  |                                   |
|     di massa di combustibile e    |                                   |
|     \(\Delta h_c = 46 \, MJ/kg\) il |                                   |
|     suo potere calorifico; la     |                                   |
|     temperatura totale            |                                   |
|     all’ingresso della turbina è  |                                   |
|     \(T_4^t = 1600 \, K\);          |                                   |
|                                   |                                   |
| \sphinxhyphen{}   \(3 \rightarrow 4\): nella      |                                   |
|     turbina avviene un’espansione |                                   |
|     adiabatica ideale, in modo    |                                   |
|     tale da garantire la potenza  |                                   |
|     necessaria a mantenere in     |                                   |
|     moto il compressore;          |                                   |
|                                   |                                   |
| \sphinxhyphen{}   \(4 \rightarrow 5\):            |                                   |
|     nell’ugello avviene           |                                   |
|     un’espansione adiabatica      |                                   |
|     ideale, che porta il gas a    |                                   |
|     espandersi fino alla          |                                   |
|     pressione ambiente            |                                   |
|     \(P_5 = P_0\).                  |                                   |
|                                   |                                   |
| Si considerino tutti i componenti |                                   |
| meccanici ideali, si trascurino   |                                   |
| gli effetti viscosi dove          |                                   |
| possibile e si consideri l’aria e |                                   |
| la miscela di gas combusti come   |                                   |
| un gas biatomico ideale, con      |                                   |
| costante dei gas                  |                                   |
| \(R = 287 \, J/(kg \, K)\) e calori |                                   |
| specifici costanti. Viene chiesto |                                   |
| di calcolare:                     |                                   |
|                                   |                                   |
| \sphinxhyphen{}   il rapporto in massa tra      |                                   |
|     flusso di combustibile e      |                                   |
|     flusso di aria,               |                                   |
|     \(f = \dot{m}_f / \dot{m}_a\);  |                                   |
|                                   |                                   |
| \sphinxhyphen{}   la spinta \(T\) fornita dal     |                                   |
|     motore.                       |                                   |
+———————————–+———————————–+

\sphinxAtStartPar
\sphinxincludegraphics{{polimi/fluidmechanics-ita/template/capitoli/04_bilanci/fig/jet_engine}}\{width=»95\%»\}

\sphinxAtStartPar
Durante lo svolgimento dell’esercizio vengono utilizzati i bilanci
integrali di massa,
\$\(\dfrac{d}{dt} \displaystyle\int_{V(t)} \rho + \oint_{S(t)} \rho (\bm{u}-\bm{v}) \cdot \bm{\hat{n}} = 0 \ ,\)\(
quantità di moto,
\)\(\dfrac{d}{dt} \displaystyle\int_{V(t)} \rho \bm{u} + \oint_{S(t)} \rho \bm{u} (\bm{u}-\bm{v}) \cdot \bm{\hat{n}}= \int_{V(t)} \bm{f} + \oint_{S(t)} \bm{t_n} \ ,\)\(
ed energia totale,
\)\(\dfrac{d}{dt} \displaystyle\int_{V(t)} \rho e^t + \oint_{S(t)} \rho e^t (\bm{u}-\bm{v}) \cdot \bm{\hat{n}}= \int_{V(t)} \bm{f} \cdot \bm{u} + \oint_{S(t)} \bm{t_n} \cdot \bm{u} - \oint_{S(t)} \bm{q} \cdot \bm{\hat{n}} + \int_{V(t)} \rho r \ .\)\$
In particolare, il bilancio di quantità di moto permette di ricavare la
formula della spinta del motore in funzione del flusso di quantità di
moto attraverso un volume di controllo opportunamente scelto. Il
bilancio di energia totale permette di analizzare i singoli componenti
del motore.

\sphinxAtStartPar
Per risolvere il problema, è necessario ricavare la spinta del motore in
funzione della portata massica trattata e della differenza di velocità
del fluido in ingresso e in uscita dal motore. Successivamente viene
analizzato il sistema motore per calcolare la velocità di efflusso dei
gas. Si considera il problema stazionario, con forze di volume \(\bm{f}\)
trascurabili. Si svolge uno studio «quasi\sphinxhyphen{}1D» considerando variabili
uniformi sulle varie sezioni del motore.


\subsubsection{Formula della spinta.}
\label{\detokenize{polimi/fluidmechanics-ita/template/capitoli/04_bilanci/04e03in:formula-della-spinta}}
\sphinxAtStartPar
Nell’ipotesi che la pressione dei gas in uscita dall’ugello sia uguale
alla pressione ambiente, il bilancio di quantità di moto del fluido
trattato dal motore permette di ottenere la stima della trazione
generata dal motore,
\$\(T = \dot{m}_5 V_5 - \dot{m}_0 V_0 = \dot{m}_0 ( V_5 - V_0 ) + \dot{m}_f V_5 \ .\)\(
Per ricavare la trazione \)T\$ è necessario ricavare i valori del flusso
di massa d’aria ingerito dal motore, il flusso di combustibile e la
velocità di efflusso dei gas combusti, studiando in dettaglio il fluido
all’interno del motore


\subsubsection{Analisi del motore.}
\label{\detokenize{polimi/fluidmechanics-ita/template/capitoli/04_bilanci/04e03in:analisi-del-motore}}
\sphinxAtStartPar
Si studia l’evoluzione della corrente che attraversa il motore.
\begin{itemize}
\item {} 
\sphinxAtStartPar
\(0 \rightarrow 1\), presa d’aria: l’aria che approccia l’ingresso del
compressore \(S_1\) subisce una compressione libera adiabatica ideale.
Dato lo stato termodinamico TD(0), con
\(\rho_0 = P_0/ (R T_0) = 0.414 \, kg/m^3\), e il rapporto di
pressione \(P_1 / P_0\), è possibile calcolare lo stato termodinamico
TD(1):
\$\(P_1 = \left( \dfrac{P_1}{P_0} \right) P_0 = 39750 \, Pa \qquad , \qquad
\rho_1 = \left( \dfrac{P_1}{P_0} \right)^{\frac{1}{\gamma}} \rho_0 = 0.553 \, kg/m^3 \ .\)\(
Una volta note la pressione e la densità, è possibile calcolare la
temperatura e l'entalpia del fluido,
\)\(T_1 = \dfrac{P_1}{R T_1} = 250.67 \, K \qquad , \qquad h_1 = c_P T_1 = 2.52 \cdot 10^5 \, J/kg \ .\)\(
Si calcola ora il flusso di massa che entra nel volume,
\)\(\dot{m}_0 = \dot{m}_1 \qquad , \qquad \rho_0 V_0 A_0 = \rho_1 V_1 A_1 \ .\)\(
Si calcola il flusso di massa utilizzando la sezione 1. Poiché non
ci sono organi meccanici che assorbono o forniscono potenza, non ci
sono sorgenti di calore e possono essere trascurati gli effetti
viscosi, tra le sezioni 0 e 1 si conserva il flusso di entalpia
totale, \)\(\dot{m}_0 h_0^t = \dot{m}_1 h_1^t 
  \quad \rightarrow \quad h_0^t = h_1^t = h_1 + \dfrac{V_1^2}{2} = 2.56 \cdot 10^5 \, J/kg \ .\)\(
\)\(\begin{aligned}
 \rightarrow \qquad V_1 & = \sqrt{2(h_0^t - h_1)} = 86.09 \, m/s \\
 \dot{m}_1 & = \rho_1 A_1 V_1 = 47.57 \, kg/s \\
\end{aligned}\)\$

\item {} 
\sphinxAtStartPar
\(1 \rightarrow 2\), compressore: lo stato termodinamico totale in
uscita del compressore è legato allo stato totale in ingresso da una
trasformazione isentropica,
\$\(P_2^t = \left( \dfrac{P_2^t}{P_1^t} \right) P_1^t = 1.67 \cdot 10^6 \, Pa \qquad , \qquad
   T_2^t = \left( \dfrac{P_2^t}{P_1^t} \right)^{\frac{\gamma-1}{\gamma}} T_1^t = 729.76 \, K \ .\)\(
Trascurando gli effetti viscosi sulla superficie di ingresso e di
uscita del compressore, in assenza di scambi di calore, la potenza
fornita dal compressore al fluido vale
\)\(W_{12} = \dot{m}_1 ( h_2^t - h_1^t ) = 22.72 \, MW \ .\)\$

\item {} 
\sphinxAtStartPar
\(2 \rightarrow 3\), combustore: la temperatura totale \(T_3^t = \dots\)
in ingresso alla turbina è un dato del problema determinato dai
limiti tecnologici legati alla realizzazione delle palette del
rotore della turbina e al fenomeno di creeping. Nel combustore non
ci sono organi meccanici in movimento che forniscano o assorbano
potenza dal fluido. Si trascurano gli effetti viscosi e le forze di
volume. Se si ipotizza la combustione completa del comustibile
iniettato come origine del calore generato e si trascura il flusso
di entalpia totale attraverso l’iniettore, il bilancio di energia
totale in regime stazionario diventa
\$\(\dot{m}_3 h_3^t - \dot{m}_2 h_2^t - \underbrace{\dot{m}_f h_f^t}_{\approx 0} = \dot{Q}_c = \dot{m}_f \Delta h_c
 \ .\)\( Poiché il flusso di massa dei gas combusti uscenti dal
combustore \)\textbackslash{}dot\{m\}\_3\( è uguale alla somma del flusso d'aria
\)\textbackslash{}dot\{m\}\_2\( e il flusso di combustibile \)\textbackslash{}dot\{m\}\_f\( entranti,
\)\(\dot{m}_3 = \dot{m}_2 + \dot{m}_f \ ,\)\( il rapporto tra il flusso
di massa del combustibile e dell'aria diventa,
\)\(f := \dfrac{\dot{m}_f}{\dot{m}_2}
    = \dfrac{h_3^t - h_2^t}{\Delta h_c - h_3^t}
    = \dfrac{T_3^t - T_2^t}{\Delta h_c / c_P - T_3^t} = 0.0197  \ .\)\(
Se si ipotizza che la pressione totale rimanga costante all'interno
del combustore, lo stato termodinamico totale in uscita dal
combustore è determinato dal valore della pressione e della
temperatura totale, \)P\_3\textasciicircum{}t\( e \)T\_3\textasciicircum{}t\(,
\)\(\rho_3^t = \dfrac{P_3^t}{R T_3^t} = 3.64 \, kg/m^3 \qquad , \qquad h_3^t = c_P T_3^T = 1.61 \cdot 10^6 \, J/kg \ .\)\$

\item {} 
\sphinxAtStartPar
\(3 \rightarrow 4\), turbina: la turbina deve generare la potenza
\(W_{34}\) necessaria a muovere il compressore,
\$\(W_{12} + W_{34} = 0 \ .\)\( Se si trascurano gli effetti viscosi e
si ipotizza un processo adiabatico, la potenza della turbina è
uguale alla differenza del flusso di entalpia totale tra l'uscita e
l'ingresso della turbina,
\)\(W_{34} = \dot{m}_3 ( h_4^t - h_3^t ) \ .\)\( L'entalpia totale
all'uscita della turbina vale
\)\(h_4^t = h_3^t - \dfrac{1}{1+f} ( h_2^t - h_1^t ) = 1.13 \cdot 10^6 \, J/kg 
\qquad \rightarrow \qquad T_4^t = 1124.6 \, K \ .\)\( La
trasformazione isentropica lega lo stato termodinamico totale TD(3)
in ingresso alla turbina allo stato termodinamico totale TD(4) in
uscita, \)\(\begin{aligned}
 P_4^t & = \left(\dfrac{T_4^t}{T_3^t} \right)^{\frac{\gamma}{\gamma-1}} P_3^t = 4.87 \cdot 10^5 \, Pa \\
 \rho_4^t & = 1.51 \, kg/m^3 \ .
\end{aligned}\)\$

\item {} 
\sphinxAtStartPar
\(4 \rightarrow 5\), ugello: se si considera un’espansione libera
nell’ugello ideale, trascurando gli effetti viscosi e gli scambi di
calore, il bilancio dell’energia totale in assenza di organi
meccanici che generino o assorbano potenza dal fluido equivale alla
conservazione del flusso dell’entropia totale,
\$\(\dot{m}_4 h_4^t = \dot{m}_5 h_5^t \qquad \rightarrow \qquad h_4^t = h_5 + \dfrac{V_5^2}{2} 
 \qquad \rightarrow \qquad V_5 = \sqrt{2(h_4^t - h_5)} \\\)\( Se
l'ugello non è bloccato la pressione dei gas in uscita è uguale alla
pressione atmosferica, \)P\_5 = P\_0\(. La trasformazione isentropica
tra 4 e 5, permette di ricavare lo stato termodinamico TD(5),
\)\(\begin{aligned}
  \rho_5 & = \left( \dfrac{P_5}{P_4^t} \right)^{\frac{1}{\gamma}} \rho_4^t = 0.189 \, kg/m^3 \\
     T_5 & = 489.5 \, K \qquad \rightarrow \qquad h_5 = 4.92 \cdot 10^5 \, J/kg .
\end{aligned}\)\( La velocità di efflusso dei gas combusti vale quindi
\)\(V_5 = \sqrt{2(h_4^t - h_5)} = 1129.6 \, m/s .\)\$

\end{itemize}

\sphinxAtStartPar
La spinta fornita dal motore in questa condizione di volo vale
\$\(T = \dot{m}_5 V_5 - \dot{m}_0 V_0 = 42.90 \, kN \ .\)\$

\sphinxstepscope


\chapter{Bernoulli theorems and vorticity dynamics}
\label{\detokenize{polimi/fluidmechanics-ita/template/capitoli/05_bernoulli/05teoria:bernoulli-theorems-and-vorticity-dynamics}}\label{\detokenize{polimi/fluidmechanics-ita/template/capitoli/05_bernoulli/05teoria:fluid-mechanics-bernoulli}}\label{\detokenize{polimi/fluidmechanics-ita/template/capitoli/05_bernoulli/05teoria::doc}}
\sphinxAtStartPar
Per fluidi incomprimibili o barotropici (per i quali la pressione è
funzione solo della densità), il teorema di Bernoulli si ottiene dal
bilancio della quantità di moto. Si elencano qui tre forme del teorema
di Bernoulli, ognuna caratterizzata da diverse ipotesi. Tramite
l’identità vettoriale
\begin{equation*}
\begin{split}\mathbf{\nabla} (\mathbf{a} \cdot \mathbf{b}) = (\mathbf{a} \cdot \mathbf{\nabla}) \mathbf{b} +  (\mathbf{b} \cdot \mathbf{\nabla}) \mathbf{a} + \mathbf{a} \times (\mathbf{\nabla} \times \mathbf{b}) + \mathbf{b} \times (\mathbf{\nabla} \times \mathbf{a}),\end{split}
\end{equation*}
\sphinxAtStartPar
applicata al termine convettivo \((\mathbf{u} \cdot \mathbf{\nabla}) \mathbf{u}\), è
possible ottenere la forma del Crocco dell’equazione della quantità di
moto
\begin{equation*}
\begin{split}\label{eqn:bilanci:crocco}
\begin{aligned}
 & \dfrac{\partial \mathbf{u}}{\partial t} + (\mathbf{u} \cdot \mathbf{\nabla}) \mathbf{u} - \nu \Delta \mathbf{u} + \mathbf{\nabla} P = \mathbf{g}  & \\ &  &  \bigg( (\mathbf{u} \cdot \mathbf{\nabla})\mathbf{u} = \mathbf{\nabla} \frac{\mathbf{u} \cdot \mathbf{u}}{2} + (\mathbf{\nabla} \times \mathbf{u}) \times \mathbf{u} \bigg) \\
 & \rightarrow \dfrac{\partial \mathbf{u}}{\partial t} + \mathbf{\nabla} \frac{|\mathbf{u}|^2}{2} + \mathbf{\omega} \times \mathbf{u} - \nu \Delta \mathbf{u} + \mathbf{\nabla} P = \mathbf{g} , & \\
\end{aligned}\end{split}
\end{equation*}
\sphinxAtStartPar
avendo indicato con \(P\) il potenziale termodinamico,
\(P =\) che si riduce al rapporto \(p/\rho\) nel caso di densità costante e
con \(\mathbf{g}\) le forze per unità di massa.


\section{Prima forma del teorema di Bernoulli}
\label{\detokenize{polimi/fluidmechanics-ita/template/capitoli/05_bernoulli/05teoria:prima-forma-del-teorema-di-bernoulli}}
\sphinxAtStartPar
Nel caso di fluido non viscoso, incomprimibile o barotropico, in regime
stazionario (\(\partial / \partial t \equiv 0\)), con forze di massa
conservative \(\mathbf{g} = -\mathbf{\nabla} \chi\), il trinomio di Bernoulli
\(|\mathbf{u}|^2/2 + P + \chi\) è costante lungo le linee di corrente e le
linee vorticose, cioè
\begin{equation*}
\begin{split}\mathbf{\hat{t}} \cdot \mathbf{\nabla} \left( \frac{|\mathbf{u}|^2}{2} + P + \chi \right) = 0 ,\end{split}
\end{equation*}
\sphinxAtStartPar
con \(\mathbf{\hat{t}}\) versore tangente alle linee di corrente o alle linee
vorticose. Infatti, il termine \(\mathbf{\omega} \times \mathbf{u}\)
nell’equazione della quantità di moto nella forma di Crocco
(\DUrole{xref,myst}{{[}eqn:bilanci:crocco{]}}\{reference\sphinxhyphen{}type=»ref»
reference=»eqn:bilanci:crocco»\}) è perpendicolare in ogni punto del
dominio alle linee di corrente (\(\mathbf{\hat{t}}\) parallelo al campo di
velocità \(\mathbf{u}\)) e alle linee vorticose (\(\mathbf{\hat{t}}\) parallelo al
campo di vorticità \(\mathbf{\omega}\)): moltiplicando scalarmente l’equazione
(\DUrole{xref,myst}{{[}eqn:bilanci:crocco{]}}\{reference\sphinxhyphen{}type=»ref»
reference=»eqn:bilanci:crocco»\}) scritta per un fluido non viscoso
(\(\nu = 0\)) per il versore \(\mathbf{\hat{t}}\) , il prodotto scalare
\(\mathbf{\hat{t}} \cdot (\mathbf{\omega} \times \mathbf{u})\) è identicamente nullo.


\section{Seconda forma del teorema di Bernoulli}
\label{\detokenize{polimi/fluidmechanics-ita/template/capitoli/05_bernoulli/05teoria:seconda-forma-del-teorema-di-bernoulli}}
\sphinxAtStartPar
Nella corrente irrotazionale (\(\mathbf{\omega} = \mathbf{0}\)) di un fluido non
viscoso, incomprimibile o barotropico, in regime stazionario, con forze
di massa conservative \(\mathbf{g} = -\mathbf{\nabla} \chi\), il trinomio di
Bernoulli \(|\mathbf{u}|^2/2 + P + \chi\) è costante in tutto il dominio, cioè
\begin{equation*}
\begin{split}\mathbf{\nabla} \left( \frac{|\mathbf{u}|^2}{2} + P + \chi \right) = 0  \quad \rightarrow \quad 
  \frac{|\mathbf{\nabla} \phi|^2}{2} + P + \chi = C.\end{split}
\end{equation*}

\section{Terza forma del teorema di Bernoulli}
\label{\detokenize{polimi/fluidmechanics-ita/template/capitoli/05_bernoulli/05teoria:terza-forma-del-teorema-di-bernoulli}}
\sphinxAtStartPar
Nella corrente irrotazionale (\(\mathbf{\omega} = \mathbf{0}\)) di un fluido non
viscoso, incomprimibile o barotropico, in un dominio semplicemente
connesso (nel quale è quindi possibile definire il potenziale cinetico
\(\phi\), t.c. \(\mathbf{u} = \nabla \phi\), con forze di massa conservative
\(\mathbf{g} = -\mathbf{\nabla} \chi\), il quadrinomio di Bernoulli
\(\partial \phi / \partial t + |\mathbf{u}|^2/2 + P + \chi\) è uniforme
(costante in spazio, in generale \sphinxstylestrong{non} in tempo) in tutto il dominio,
cioè
\begin{equation*}
\begin{split} \left(\mathbf{\nabla}\dfrac{\partial \mathbf{u}}{\partial t}  + \frac{|\mathbf{\nabla} \phi|^2}{2} + P + \chi \right) = 0  \quad \rightarrow \quad 
 \dfrac{\partial \mathbf{u}}{\partial t}  + \frac{|\mathbf{\nabla} \phi|^2}{2} + P + \chi = C(t).\end{split}
\end{equation*}

\section{Teoremi di Bernoulli per fluidi viscosi incomprimibili}
\label{\detokenize{polimi/fluidmechanics-ita/template/capitoli/05_bernoulli/05teoria:teoremi-di-bernoulli-per-fluidi-viscosi-incomprimibili}}
\sphinxAtStartPar
Mentre la prima forma del teorema di Bernoulli non è valida se non viene
fatta l’ipotesi di fluido non viscoso%
\begin{footnote}[1]\sphinxAtStartFootnote
Moltiplicando scalarmente l’equazione
(\DUrole{xref,myst}{{[}eqn:bilanci:crocco{]}}\{reference\sphinxhyphen{}type=»ref»
reference=»eqn:bilanci:crocco»\}) per il versore \(\mathbf{\hat{t}}\), il
termine \(\mathbf{\hat{t}}\cdot \nu \Delta \mathbf{u}\) non si annulla. Rimane
quindi
\begin{equation*}
\begin{split}\mathbf{\hat{t}} \cdot \mathbf{\nabla} \left( \frac{|\mathbf{u}|^2}{2} + P + \chi \right) - \nu \mathbf{\hat{t}} \cdot \Delta \mathbf{u} = 0\end{split}
\end{equation*}%
\end{footnote}, la seconda e la terza forma
sono ancora valide per fluidi viscosi incomprimibili. Infatti, usando
l’identità vettoriale
\begin{equation*}
\begin{split}\Delta \mathbf{u} = \mathbf{\nabla} (\mathbf{\nabla}\cdot \mathbf{u})
  - \mathbf{\nabla} \times (\mathbf{\nabla} \times \mathbf{u}) ,\end{split}
\end{equation*}
\sphinxAtStartPar
si scopre che il
laplaciano del campo di velocità per correnti irrotazionali
(\(\mathbf{\nabla} \times \mathbf{u} = \mathbf{0}\)) di fluidi incomprimibili
(\(\mathbf{\nabla} \cdot \mathbf{u} = 0\)) è nullo.

\sphinxAtStartPar
L’ipotesi di fluido non viscoso non è direttamente necessaria per la
seconda e la terza forma del teorema di Bernoulli, ma lo diventa tramite
l’ipotesi di corrente irrotazionale. Sotto opportune ipotesi sulla
corrente asintotica, verificate in molti casi di interesse aeronautico,
si dimostra che (quasi) tutto il campo di moto è irrotazionale solo se
viene fatta l’ipotesi di fluido non viscoso. Questo modello viene
utilizzato per studiare correnti di interesse aeronautico, nelle quali
gli effetti della viscosità sono (quasi ovunque) trascurabili: un
esempio è la corrente, uniforme a monte, che investe un corpo
aerodinamico a bassi angoli di incidenza (corpo affusolato, attorno al
quale non si verifichino separazioni) per alti numeri di Reynolds: in
queste correnti, le zone vorticose sono confinate in regioni di spessore
sottile (strato limite sulla superficie dei corpi solidi e scie libere).


\bigskip\hrule\bigskip


\sphinxstepscope


\section{Exercises}
\label{\detokenize{polimi/fluidmechanics-ita/template/capitoli/05_bernoulli/exercises:exercises}}\label{\detokenize{polimi/fluidmechanics-ita/template/capitoli/05_bernoulli/exercises:fluid-mechanics-bernoulli-exercises}}\label{\detokenize{polimi/fluidmechanics-ita/template/capitoli/05_bernoulli/exercises::doc}}
\sphinxstepscope


\subsection{Exercise 5.1}
\label{\detokenize{polimi/fluidmechanics-ita/template/capitoli/05_bernoulli/0503in:exercise-5-1}}\label{\detokenize{polimi/fluidmechanics-ita/template/capitoli/05_bernoulli/0503in:fluid-mechanics-bernoulli-ex-01}}\label{\detokenize{polimi/fluidmechanics-ita/template/capitoli/05_bernoulli/0503in::doc}}
\sphinxAtStartPar
+:———————————:+:———————————:+
| Determinare la portata d’acqua    | \sphinxincludegraphics{{polimi/fluidmechanics-ita/template/capitoli/05_bernoulli/fig/venturi}.eps}\{width |
| che scorre all’interno del tubo   | =»90\%»\}                           |
| di Venturi rappresentato in       |                                   |
| figura, quando sia trascurabile   |                                   |
| ogni effetto dissipativo          |                                   |
| all’interno della corrente e la   |                                   |
| velocità uniforme nelle sezioni   |                                   |
| considerate e a monte del         |                                   |
| Venturi. Dati: densità dell’acqua |                                   |
| \(\overline{\rho}= 999\ kg/m^3\),   |                                   |
| densità dell’aria                 |                                   |
| \(\overline{\rho}= 1.225\ kg/m^3\), |                                   |
| diametro del tubo \(D=2\  cm\),     |                                   |
| diametro della sezione di gola    |                                   |
| \(d=1\ cm\), altezze:               |                                   |
| \(z_1 = 10\ cm\), \(z_2 = 1.2\  m\),  |                                   |
| \(z_3 = 5\ cm\), \(z_4 = 0.5\ m\).    |                                   |
|                                   |                                   |
| (\(Q=3.01\, 10^{-4}\ m/s\),         |                                   |
| \(\overline{Q}=3.005\, 10^{-1}\ kg |                                   |
| /s\))                              |                                   |
+———————————–+———————————–+

\sphinxAtStartPar
Teorema di Bernoulli. Equazione della vorticità. Conseguenze delle
ipotesi di stazionarietà, fluido incomprimibile, non viscoso,
irrotazionale. Dominio di applicabilità del teorema di Bernoulli.
Condizioni all’interfaccia. Legge di Stevino.

\sphinxAtStartPar
Il problema viene risolto in diversi passi successivi: in principio
vengono fatte alcune ipotesi semplificative (\(\rho = \bar{\rho}\),
\(\mu=0\), \(\frac{\partial}{\partial t}=0\)); poi si utilizza l’equazione
della vorticità per semplificare ulteriormente il problema; si determina
il dominio in cui è applicabile il teorema di Bernoulli con le ipotesi
fatte; si osserva che la parte restante del problema è un problema di
statica; si determinano le condizioni di interfaccia tra i due domini;
solo a questo punto è possibile scrivere il sistema di equazioni dal
quale ricavare le quantità richieste dal problema.
\begin{itemize}
\item {} 
\sphinxAtStartPar
Il testo del problema consente di fare le seguenti ipotesi: fluido
incomprimibile, non viscoso, stazionario.

\item {} 
\sphinxAtStartPar
L’ipotesi di flusso non viscoso e quella di velocità uniforme a
monte permettono di definire il domino all’interno del quale è
possibile applicare il teorema di Bernoulli, aggiungendo l’ipotesi
di irrotazionalità alle tre ipotesi precedenti. Infatti, l’equazione
della vorticità può essere scritta come:
\$\(\frac{D \bm{\omega}}{Dt} = (\bm{\omega} \cdot \bm{\nabla}) \bm{u}\)\(
La derivata materiale rappresenta la variazione di una quantità
associata a una particella materiale che segue il moto del fluido.
Poiché la vorticità nella sezione a monte è nulla (il profilo di
velocità è uniforme quindi le derivate spaziali sono nulle), la
vorticità rimane nulla (\)\textbackslash{}frac\{d f\}\{d t\} = a f\(, se \)f=0\(
all'istante iniziale la sua derivata in quell'istante è nulla,
quindi \)f\$ non varia, quindi rimane uguale a zero).

\item {} 
\sphinxAtStartPar
Il dominio in cui è possibile applicare il teorema di Bernoulli con
le ipotesi di incomprimibilità, assenza di viscosità ed effetti
dissipativi, stazionarietà, \sphinxstylestrong{irrotazionalità} e connessione
semplice del dominio, coincide con il tubo di Venturi stesso.
Infatti in corrispondenza delle prese a parete cade l’ipotesi di
irrotazionalità.

\sphinxAtStartPar
Secondo le ipotesi fatte il fluido è non viscoso. Questo assicura
che la vorticità sia nulla lungo le linee di corrente. Nel tubo del
manometro però il fluido è fermo. Per un fluido non viscoso in
corrispondenza dell’interfaccia non ci deve essere discontinuità
nella componente normale all’interfaccia stessa e nella pressione.
La componente normale è nulla da entrambe le parti della
discontinuità; la componente tangenziale è però discontinua: mentre
nel tubo di Venturi è diversa da zero, nel tubo del manometro è
nulla. Questo comporta che la vorticità non sia nulla (bensì
infinita: «differenza finita in uno spessore infinitesimo») e di
conseguenza la non validità in questa regione delle ipotesi fatte in
precedenza.

\sphinxAtStartPar
Si possono quindi distinguere due regioni (il tubo di Venturi e il
manometro) che non possono «parlare» tra di loro con il teorema di
Bernoulli, ma solo tramite la condizione di \sphinxstylestrong{interfaccia}
(continuità degli sforzi: per fluidi non viscosi questa condizione
coincide con la continuità della pressione).

\item {} 
\sphinxAtStartPar
É possibile ora scrivere il sistema risolvente:
\begin{align*}\!\begin{aligned}
\begin{cases}
     P_A + \frac{1}{2}U_A^2 + \rho g z_A = P_{B_1} + \frac{1}{2}U_{B_1}^2 + \rho g z_{B_1} & \text{(Bernoulli A-$B_1$)} \\
     P_{B_1} = P_{B_2} & \text{(interfaccia $B_1$-$B_2$)} \\
     P_{B_2} + \rho g z_{B_2} = P_C + \rho g z_C & \text{(Stevino $B_2$-C)} \\
     P_C + \rho_a g z_C = P_D + \rho_a g z_D & \text{(Stevino C-D)} \\
     P_D + \rho g z_D = P_{E_2} + \rho g z_{E_2} & \text{(Stevino D-$E_2$)} \\
     P_{E_2} = P_{E_1} & \text{(interfaccia $E_2$-$E_1$)} \\
     P_{E_1} + \frac{1}{2} \rho U_{E_1}^2 + \rho g z_{E_1} = P_F + \frac{1}{2} \rho U_F^2 + \rho g z_F & \text{(Bernoulli $E_1$-F)} \\
     P_A + \frac{1}{2}\rho U_A^2 + \rho g z_A = P_F + \frac{1}{2}\rho U_F^2 + \rho g z_F & \text{(Bernoulli A-F)} \\
     \rho \frac{\pi D^2}{4} U_A = \rho \frac{\pi d^2}{4} U_F & \text{(continuità A-F)}
    \end{cases}$$ che, osservando che $z_{B_1} = z_{B_2} = z_B$,
    $z_{E_1} = z_{E_2} = z_E$ e applicando le ipotesi fatte in
    precedenza ($U_A = u_{B_1}$, $U_F = u_{E_1}$,
    $P_{B_1} = P_{B_2} = P_B$, $P_{E_1} = P_{E_2} = P_E$), diventa:\\
$$\begin{cases}
     P_A + \rho g z_A = P_{B} + \rho g z_{B} & \text{(Bernoulli A-B)} \\
     P_{B_2} + \rho g z_{B} = P_C + \rho g z_C & \text{(Stevino B-C)} \\
     P_C + \rho_a g z_C = P_D + \rho_a g z_D & \text{(Stevino C-D)} \\
     P_D + \rho g z_D = P_{E} + \rho g z_{E} & \text{(Stevino D-E)} \\
     P_{E_1} + \rho g z_{E} = P_F +  + \rho g z_F & \text{(Bernoulli E-F)} \\
     P_A + \frac{1}{2} \rho U_A^2 + \rho g z_A = P_F + \frac{1}{2}\rho U_F^2 + \rho g z_F & \text{(Bernoulli A-F)} \\
     D^2 U_A = d^2 U_F & \text{(continuità A-F)}
    \end{cases}\\
\end{aligned}\end{align*}
\sphinxAtStartPar
Anche se il numero di equazioni è minori del numero di incognite,
prova che il sistema è indeterminato, si dimostra che \(U_A\) e \(U_F\)
sono determinate (nelle equazioni intervengono sempre differenze di
pressioni, ed è questo il motivo dell’indeterminazione).

\item {} 
\sphinxAtStartPar
Soluzione del sistema: il sistema può essere risolto come più si
preferisce. Per esempio, partendo da quella che può essere una
«lettura dello strumento» \(\Delta z = z_C - z_D\) e «chiudendo il
ciclo ABCDEF»:
\begin{equation*}
\begin{split}\rho_a g \Delta z = P_D - P_C\end{split}
\end{equation*}\begin{equation*}
\begin{split}\begin{cases}
        P_D = P_E + \rho g (h_E - h_D) = P_F + \rho g (h_F - h_D)\\
        P_C = P_B + \rho g (h_B - h_C) = P_A + \rho g (h_A - h_C)\\
      \end{cases} \\\end{split}
\end{equation*}\begin{equation*}
\begin{split}\begin{aligned}
      \Rightarrow \quad P_D - P_C & = (P_F + \rho g h_F) - (P_A + \rho g h_A) + \rho g \Delta z = 
      & \text{(Bernoulli A-F)}\\
       & = \frac{1}{2}\rho U_A^2 - \frac{1}{2}\rho U_F^2 + \rho g \Delta z = 
      & \text{(continuità)}\\
       & = -\frac{1}{2}\rho U_A^2 \displaystyle\left( \frac{D^4}{d^4} - 1 \right) + \rho g \Delta z \\
    \end{aligned}\end{split}
\end{equation*}
\sphinxAtStartPar
E quindi:
\begin{equation*}
\begin{split}(\rho - \rho_a ) g \Delta z = \frac{1}{2}\rho U_A^2 \displaystyle\left( \frac{D^4}{d^4} - 1 \right)\end{split}
\end{equation*}\begin{equation*}
\begin{split}U_A = \displaystyle\sqrt{\frac{2 (1 - \rho_a / \rho) g \Delta z}{\frac{D^4}{d^4} - 1}}\end{split}
\end{equation*}
\sphinxAtStartPar
Inserendo i valori numerici, si trova: \(U = 0.956 m/s\),
\(Q = 3.0 \cdot 10^{-4} m^3/s\), \(\bar{Q} = 3.0 \cdot 10^{-1} kg/s\).

\end{itemize}

\sphinxAtStartPar
\sphinxstyleemphasis{Osservazioni.} É importante saper riconoscere i limiti di applicabilità
di formule e teoremi nel rispetto delle ipotesi con le quali essi
vengono formulati.

\sphinxAtStartPar
Considerazioni analoghe dovranno essere svolte anche in esercizi simili
a questo, riguardanti le soluzioni esatte delle equazioni di
Navier\sphinxhyphen{}Stokes.

\sphinxstepscope


\subsection{Exercise 5.2}
\label{\detokenize{polimi/fluidmechanics-ita/template/capitoli/05_bernoulli/0501in:exercise-5-2}}\label{\detokenize{polimi/fluidmechanics-ita/template/capitoli/05_bernoulli/0501in:fluid-mechanics-bernoulli-ex-02}}\label{\detokenize{polimi/fluidmechanics-ita/template/capitoli/05_bernoulli/0501in::doc}}
\sphinxAtStartPar
+:———————————:+:———————————:+
| Si consideri il serbatoio         | \sphinxincludegraphics{{polimi/fluidmechanics-ita/template/capitoli/05_bernoulli/fig/serbatoio}.eps}\{wid |
| rappresentato in figura,          | th=»90\%»\}                         |
| \(D=2\ m\), \(H=4.4\ m\) al cui       |                                   |
| interno è contenuta acqua,        |                                   |
| \(\overline{\rho}=999\,{kg/m^3}\).  |                                   |
| Supponendo il fluido non viscoso, |                                   |
| determinare la velocità di        |                                   |
| efflusso del fluido dall’ugello   |                                   |
| del serbatoio, \(h=0.4\ m\) e       |                                   |
| \(d = 1\ cm\), e la sua portata,    |                                   |
| sia in massa sia in volume.       |                                   |
|                                   |                                   |
| (\(U = 8.86\ m/s\),                 |                                   |
| \(Q=6.96\, 10^{-4}\ m^3/s\),        |                                   |
| \(\overline{Q}=0.695\ kg/s\))       |                                   |
+———————————–+———————————–+

\sphinxAtStartPar
Teorema di Bernoulli, nel caso incomprimibile, non viscoso,
«stazionario» (da come è fatto il disegno, il livello del serbatorio
sembra diminuire…assumiamo che così non sia), con forze che ammettono
potenziale e dominio semplicemente connesso. Se si fa l’ipotesi che il
flusso sia irrotazionale sulla sezione di ingresso, nel caso non
viscoso, si mantiene irrotazionale ovunque (equazione della vorticità).
\$\(\frac{D \bm{\omega}}{Dt} = (\bm{\omega} \cdot \bm{\nabla}) \bm{u}\)\$

\sphinxAtStartPar
Si può quindi scrivere il teorema di Bernoulli nella forma:
\$\(\frac{P}{\rho} + \frac{|\bm{u}|^2}{2} + gh = \text{cost}\)\$

\sphinxAtStartPar
Il problema si risolve mettendo a sistema il teorema di Bernoulli
(opportunamente semplificato; vedi sopra) con il bilancio integrale di
massa. Si ipotizza che sulle due sezioni agisca la stessa pressione
esterna.
\begin{equation*}
\begin{split}\begin{cases}
  A_1 u_1 = A_2 u_2 & (massa) \\
  \frac{u_1^2}{2} + g h_1 = \frac{u_2^2}{2} + g h_2 & (Bernoulli)
\end{cases}\end{split}
\end{equation*}
\sphinxAtStartPar
Svolgendo i passaggi, ricordando che le superfici sono circolari,
risulta:
\$\(u_2 = \sqrt{\frac{2 g (h_1-h_2)}{1-\displaystyle\left(\frac{d_2}{d_1}\right)^4}}\)\$

\sphinxAtStartPar
Si calcolano poi le portate volumetriche e di massa. \$\(\begin{aligned}
  & Q = A_2 u_2 \\
  & \dot{m} = \rho Q
\end{aligned}\)\$

\sphinxstepscope


\subsection{Exercise 5.3}
\label{\detokenize{polimi/fluidmechanics-ita/template/capitoli/05_bernoulli/0502in:exercise-5-3}}\label{\detokenize{polimi/fluidmechanics-ita/template/capitoli/05_bernoulli/0502in:fluid-mechanics-bernoulli-ex-03}}\label{\detokenize{polimi/fluidmechanics-ita/template/capitoli/05_bernoulli/0502in::doc}}
\sphinxAtStartPar
+:———————————:+:———————————:+
| Si consideri il flusso d’acqua,   | \sphinxincludegraphics{{polimi/fluidmechanics-ita/template/capitoli/05_bernoulli/fig/canale}.eps}\{width= |
| \(\overline{\rho}=999\ kg/m^3\),    | «90\%»\}                            |
| nel canale rappresentato in       |                                   |
| figura. Nel primo tratto l’acqua  |                                   |
| scorre con una velocità uniforme  |                                   |
| \(U_1 = 1\ m/s\) e l’altezza del    |                                   |
| pelo libero rispetto al fondo del |                                   |
| canale è \(h_1 = 1.5\ m\).          |                                   |
| Determinare la velocità           |                                   |
| dell’acqua \(U_2\) e l’altezza del  |                                   |
| pelo libero \(h_2\) nel secondo     |                                   |
| tratto del canale, sapendo che    |                                   |
| l’altezza del fondo del primo     |                                   |
| tratto rispetto al fondo del      |                                   |
| secondo tratto è \(H=0.5\ m\). Si   |                                   |
| trascuri qualunque effetto        |                                   |
| dissipativo.                      |                                   |
|                                   |                                   |
| (Soluzione 1: \(U_2 = 0.741\ m/s\), |                                   |
| \(h_2 = 2.022\ m\). Soluzione 2:    |                                   |
| \(U_2 = 5.940\ m/s\),               |                                   |
| \(h_2 = 0.252\ m\))                 |                                   |
+———————————–+———————————–+

\sphinxAtStartPar
Teorema di Bernoulli nel caso incomprimibile, non viscoso, stazionario,
irrotazionale. Soluzione di equazioni di terzo grado: metodo grafico e
numerico. Correnti in canali aperti: soluzioni «fisiche», numero di
Froude \(Fr\), correnti subrcritiche e supercritiche.

\sphinxAtStartPar
L’esercizio viene risolto in due passi, che richiedono diversi livelli
di conoscenza della dinamica dei fluidi in canali aperti: in un primo
tempo, vengono ricavate le soluzioni ammissibili (\(h_2 > 0\), \(U_2 > 0\))
del problema; in un secondo tempo, viene scelta la soluzione fisica del
problema, tra le due soluzioni ammissibili trovate in precedenza.


\subsubsection{Parte 1.}
\label{\detokenize{polimi/fluidmechanics-ita/template/capitoli/05_bernoulli/0502in:parte-1}}
\sphinxAtStartPar
L’esercizio viene risolto mettendo a sistema il teorema di Bernoulli
riferito a una linea di corrente sul pelo libero (sul quale agisce la
pressione ambiente \(P_a\), costante) e l’equazione di continuità. Grazie
alle ipotesi elencate in precedenza, si può scrivere il sistema
risolvente come:
\begin{equation*}
\begin{split}\label{eqn:bern_cont}
  \begin{cases}
    \frac{1}{2}\rho U_1^2 + \rho g(h_1+H) = \frac{1}{2}\rho U_2^2 +
    \rho g h_2 \\
    h_1 U_1 = h_2 U_2
  \end{cases}\end{split}
\end{equation*}
\sphinxAtStartPar
Il sistema è di due equazioni (non lineari) nelle incognite \(U_2\) e
\(h_2\). Se si ricava una delle due incognite da un’equazione e la si
inserisce nell’altra, si ottiene un’equazione di terzo grado. Per
esempio, ricavando \(h_2\) dalla prima e inserendola nella seconda, per
l’incognita \(U_2\) si ottiene l’equazione di terzo grado:
\begin{equation*}
\begin{split}h_1 U_1 = U_2 \displaystyle\left( \frac{U_1^2 - U_2^2}{2 g} + (h_1 + H)    \right)\end{split}
\end{equation*}
\sphinxAtStartPar
I metodi numerici convergono (quando convergono) a una soluzione, senza
informazioni su quante soluzioni esistono effettivamente: prima di
risolvere l’equazione di terzo grado con un metodo numerico è utile un
primo approccio analitico al problema.

\sphinxAtStartPar
Per questo cerchiamo le soluzioni del sistema di due equazioni per via
grafica. Le equazioni del sistema
\DUrole{xref,myst}{{[}eqn:bern\_cont{]}}\{reference\sphinxhyphen{}type=»ref»
reference=»eqn:bern\_cont»\} definiscono curve nel piano \((U_2,h_2)\). Se
scegliamo di usare come asse orizzontale quello delle \(U_2\), la prima
equazione definisce una parabola con la concavità diretta verso il basso
(\(h_2 = - 0.5  U_2^2 /g +...\)), mentre la seconda un’iperbole.

\sphinxAtStartPar
Esistono due soluzioni con senso fisico (\(h_2 \ge 0, U_2 \ge 0\)). Ora
che sappiamo quante soluzioni cercare e dove cercarle, possiamo
procedere con un metodo numerico, dando guess iniziali in un intorno
della soluzione. Le due soluzioni sono: \$\(\begin{aligned}
  A :
  \begin{cases}
   U_2 = 0.741 \ m/s \\
   h_2 = 2.022 \ m
  \end{cases}
   \quad
  B :
  \begin{cases}
   U_2 = 5.940 \ m/s \\
   h_2 = 0.252 \ m
  \end{cases}
\end{aligned}\)\$


\subsubsection{Parte 2.}
\label{\detokenize{polimi/fluidmechanics-ita/template/capitoli/05_bernoulli/0502in:parte-2}}
\sphinxAtStartPar
É plausibile farsi una domanda: al netto delle ipotesi fatte sul regime
di moto (fluido incomprimibile, non viscoso), il modello è in grado di
descrivere il fenomeno fisico e stabilire quale delle due soluzioni
amissibili trovate è la soluzione «fisica»? Seguendo la trattazione del
problema svolta in \sphinxhref{http://heidarpour.iut.ac.ir/sites/heidarpour.iut.ac.ir/files//u32/open-chaudhry.pdf}{Chaudhry, \sphinxstyleemphasis{Open\sphinxhyphen{}Channel Flow}, paragrafo 2\sphinxhyphen{}7:
Channel transition e paragrafi
vicini},
è possibile trovare l’unica soluzione fisica del problema. Viene
introdotta la notazione usata da Chaudhry, che contrasta in parte con
quella usata finora. Si tornerà alla notazione usata nella prima parte
dell’esercizio, solo alla fine per scrivere i risultati.

\sphinxAtStartPar
La variabile \(z(x)\) descrive la quota del fondo del canale, la variabile
\(y(x)\) descrive la profondità della corrente, riferita al fondo del
canale. Si indica con \(Q = V y\) la portata in volume, costante. Il
trinomio di Bernoulli \(H\), diviso per \(\rho\) e \(g\), è costante lungo il
canale. Si ricorda che sulla linea di corrente in corrispondenza del
pelo libero agisce una pressione costante uguale alla pressione ambiente
\(P_a\). Se si introduce la coordinata orizzontale \(x\), \$\(\begin{aligned}
 0 = \dfrac{d H}{d x} = 
 & \dfrac{d (y+z)}{d x} + \dfrac{d}{d x} \dfrac{V^2}{2 g}     = \\
 & = \dfrac{d (y+z)}{d x} + \dfrac{d}{d x} \dfrac{Q^2}{2 g y^2} = \\
 & = \dfrac{d (y+z)}{d x} - \dfrac{Q^2}{g y^3} \dfrac{d y}{d x} = \\
 & = \dfrac{d (y+z)}{d x} - \dfrac{V^2}{g y} \dfrac{d y}{d x} = \\ 
 & = \dfrac{d (y+z)}{d x} - \text{Fr}^2 \dfrac{d y}{d x} = \\ 
 & = \dfrac{d z}{d x} - (\text{Fr}^2 - 1) \dfrac{d y}{d x} \\ 
\end{aligned}\)\( dove è stato introdotto il numero di Froude
\)\textbackslash{}textit\{Fr\} = V(y)\textasciicircum{}2 / g y\(, e qui è stata esplicitata la dipendenza
dalla profondità \)y\(, funzione a sua volta funzione della coordinata
\)x\(. Si trova così il legame tra la profondità della corrente \)y(x)\(, la
quota del fondo \)z(x)\( e lo stato della corrente, descritto dal numero
di Froude. \)\(\label{eqn:flow_depth}
 \dfrac{d z}{d x} = (\text{Fr}^2(y(x)) - 1) \dfrac{d y}{d x}\)\( Vengono
definiti due regimi di moto: subcritico \)\textbackslash{}textit\{Fr\} < 1\(, supercritico
\)\textbackslash{}textit\{Fr\} > 1\(. Il profilo del fondo \)z(x)\(, e quindi la sua
derivata, è noto dal progetto del canale. La profondità della corrente
\)y(x)\$ può essere ottenuta integrando l’eq.
\DUrole{xref,myst}{{[}eqn:flow\_depth{]}}\{reference\sphinxhyphen{}type=»ref»
reference=»eqn:flow\_depth»\} con le condizioni iniziali note sulla
sezione di ingresso.

\sphinxAtStartPar
Per risolvere il nostro esercizio è sufficiente ragionare sui segni dei
tre termini dell’eq.
\DUrole{xref,myst}{{[}eqn:flow\_depth{]}}\{reference\sphinxhyphen{}type=»ref»
reference=»eqn:flow\_depth»\}: \(dz/dx \le 0\), quindi i due fattori alla
destra dell’uguale devono essere discordi. Il numero di Froude sulla
sezione di ingresso del problema vale
\(\text{Fr}_1 = U^2_1 / (g h_1) = 0.068\), quindi il contenuto della
parentesi tonda è negativo (e negativo rimane, al variare di \(x\); di
questo dovete fidarvi…). Deve quindi essere \(dy/dx \ge 0\). Tornando
alla notazione usata nella prima parte dell’esercizio, dove la
profondità della corrente è indicata con \(h(x)\), \(dh(x)/dx \ge 0\).
Poichè la profondità della corrente aumenta sempre, la soluzione
«fisica» tra le due ammissibili è la soluzione \(A\), per la quale
\(h_2 > h_1\).
\begin{equation*}
\begin{split}\begin{cases}
   U_2 = 0.741 \ m/s \\
   h_2 = 2.022 \ m
  \end{cases}\end{split}
\end{equation*}

\subsubsection{Cosa non è stato detto.}
\label{\detokenize{polimi/fluidmechanics-ita/template/capitoli/05_bernoulli/0502in:cosa-non-e-stato-detto}}
\sphinxAtStartPar
É stato fatto solo un accenno al ragionamento che consente di
determinare l’unica soluzione «fisica» del problema delle correnti in
canali aperti che variano con continuità. Non si dirà nulla sui salti
idraulici (che portano la corrente da uno stato supercritico a uno
subcritico), dei quali si possono trovare esempi nei fiumi o sul fondo
di un lavandino. Si accenna solo alla uguaglianza formale del problema
del moto di un fluido incomprimibile in un canale aperto, con il moto
monodimensionale di un fluido comprimibile, dove il ruolo del numero di
Froude \(\textit{Fr}\) sarà svolto dal numero di Mach \(M\), la definizione
di stato sub\sphinxhyphen{} e supercritico, sarà sostituita da quella di condizione
sub\sphinxhyphen{} e supersonica, il salto idraulico troverà il suo fenomeno
corrispondente nelle onde d’urto.

\sphinxstepscope


\subsection{Exercise 5.4}
\label{\detokenize{polimi/fluidmechanics-ita/template/capitoli/05_bernoulli/0504in:exercise-5-4}}\label{\detokenize{polimi/fluidmechanics-ita/template/capitoli/05_bernoulli/0504in:fluid-mechanics-bernoulli-ex-04}}\label{\detokenize{polimi/fluidmechanics-ita/template/capitoli/05_bernoulli/0504in::doc}}
\sphinxAtStartPar
+:———————————:+:———————————:+
| Dato il condotto a sezione        | !{[}image{]}(./fig/condottocircolare. |
| circolare rappresentato in        | eps)\{width=»90\%»\}                 |
| figura, determinare la portata in |                                   |
| massa d’olio,                     |                                   |
| \(\overline{\rho} = 850\ kg/m^3\),  |                                   |
| attraverso il condotto stesso     |                                   |
| sapendo che il diametro del       |                                   |
| condotto è \(d=0.5\ m\), che la     |                                   |
| differenza di altezza fra i peli  |                                   |
| liberi è \(H=40\ cm\), che il       |                                   |
| diametro del tubo «a U» è di \(2\)  |                                   |
| mm. Si trascuri qualunque effetto |                                   |
| dissipativo, si assuma uniforme   |                                   |
| la velocità in una sezione        |                                   |
| sufficientemente lontana a monte  |                                   |
| e si consideri che nel tubo «a U» |                                   |
| sia presente aria in condizioni   |                                   |
| normali.                          |                                   |
|                                   |                                   |
| (\(\overline{Q}= 467.2\  kg/s\))    |                                   |
+———————————–+———————————–+

\sphinxAtStartPar
Teorema di Bernoulli nell’ipotesi di stazionarietà, fluido
incomprimibile, non viscoso, irrotazionale. Equazione della vorticità
nel caso non viscoso. Legge di Stevino.

\sphinxAtStartPar
Vengono fatte alcune ipotesi semplificative (\(\rho = \bar{\rho}\),
\(\mu=0\), \(\frac{\partial}{\partial t}=0\)); si utilizza poi l’equazione
della vorticità per semplificare ulteriormente il problema: se si assume
che il profilo di velocità all’ingresso sia uniforme, e quindi a
vorticità nulla, il fluido nel canale rimane irrotazionale
(dall’equazione della vorticità per fluidi non viscosi).

\sphinxAtStartPar
Gli unici due punti che possono creare problemi sono i collegamenti del
tubo con il canale. Sulla linea di corrente che incontra l’imbocco del
tubicino, il fluido subisce un rallentamento dalla velocità di ingresso
fino ad arrestarsi: su questa linea di corrente è possibile applicare il
teorema di Bernoulli. In corrispondenza del’altro collegamento, si
incontra una superficie di discontinuità a vorticità infinita: non è
quindi possibile attraversare questa superficie applicando direttamente
il teorema di Bernoulli, ma bisogna ricorrere alle condizioni di
interfaccia tra i due domini, quello interno al canale e quello interno
al tubo, nel quale possono essere applicate le equazioni della statica.

\sphinxAtStartPar
Vengono definiti i punti \(A\) all’ingresso sulla linea di corrente che
arriva alla presa del tubo all’interno del canale; il punto \(B\)
coincidente con la presa del tubo all’interno del canale; \(C\) il pelo
libero di destra all’interno del tubo «a U», \(D\) il pelo libero di
sinistra. Si definiscono anche \(h_C\) e \(h_D\) come quote dei peli liberi
(oss. \(H = h_C - h_D\)).

\sphinxAtStartPar
\sphinxincludegraphics{{polimi/fluidmechanics-ita/template/capitoli/05_bernoulli/fig/Canale01}.eps}\{width=»35\%»\}

\sphinxAtStartPar
Il sistema risolvente è: \$\(\begin{cases}
  P_A + \frac{1}{2} \rho v_A^2 + \rho g h_A =
   P_B + \frac{1}{2} \rho v_B^2 + \rho g h_B\\
  P_B + \rho g h_B = P_C + \rho g h_C   \\
  P_C + \rho_a g h_C = P_D + \rho_a g h_D\\
  P_D + \rho g h_D = P_{E_2} + \rho g h_{E_2} \\
  P_{E_2} = P_{E_1} \\
  P_{E_1} + \frac{1}{2} \rho u_{E_1}^2 + \rho g h_{E_1} = P_F + \frac{1}{2} \rho u_F^2 + \rho g h_F\\
 P_F + \frac{1}{2} \rho v_F^2 + \rho g h_F = P_A + \frac{1}{2} \rho v_A^2 + \rho g h_A \\
  \bar{Q} = \rho \frac{\pi}{4}d^2 U
\end{cases}\)\$

\sphinxAtStartPar
Osservando che \(h_A = h_B\), \(h_E = h_F\), \(v_A = v_F = U\), \(v_B = 0\),
supponendo \(u_E = U\) (ipotizzando dimensioni e intrusività trascurabile
della sonda), il sistema semplificato diventa: \$\(\begin{cases}
  P_A + \frac{1}{2} \rho U^2  = P_B \\
  P_B + \rho g h_A = P_C + \rho g h_C   \\
  P_C + \rho_a g h_C = P_D + \rho_a g h_D\\
  P_D + \rho g h_D = P_E + \rho g h_E \\
  P_E + \frac{1}{2} \rho u_E^2 = P_F + \frac{1}{2} \rho U^2 \\
 P_F + \rho g h_E = P_A + \rho g h_A \\
  \bar{Q} = \rho \frac{\pi}{4}d^2 U
\end{cases}\)\$

\sphinxAtStartPar
Risolvendo per U, avendo definito \(H = h_C - h_D\):
\$\(\frac{1}{2} \rho U^2 = P_B - P_A = ... = (\rho - \rho_a) g H \quad \Rightarrow \quad 
  U = \sqrt{2\displaystyle\left(1-\frac{\rho_a}{\rho}\right) g H}\)\$

\sphinxAtStartPar
Inserendo i valori numerici: \(U = 2.799 m/s\), \(\bar{Q} = 467.15 kg/s\).

\sphinxstepscope


\subsection{Exercise 5.5}
\label{\detokenize{polimi/fluidmechanics-ita/template/capitoli/05_bernoulli/0506in:exercise-5-5}}\label{\detokenize{polimi/fluidmechanics-ita/template/capitoli/05_bernoulli/0506in:fluid-mechanics-bernoulli-ex-05}}\label{\detokenize{polimi/fluidmechanics-ita/template/capitoli/05_bernoulli/0506in::doc}}
\sphinxAtStartPar
+:———————————:+:———————————:+
| Si consideri un getto             | \sphinxincludegraphics{{polimi/fluidmechanics-ita/template/capitoli/05_bernoulli/fig/getto_vert}.eps}\{wi |
| stazionario, assialsimmetrico,    | dth=»60\%»\}                        |
| d’acqua in condizioni standard,   |                                   |
| diretto verso l’alto, in          |                                   |
| atmosfera uniforme, secondo la    |                                   |
| verticale \(z\), e uscente con      |                                   |
| velocità uniforme e costante      |                                   |
| \(V = 20\ m/s\) da un ugello        |                                   |
| circolare di diametro             |                                   |
| \(d = 5\  cm\). Si assuma che:      |                                   |
|                                   |                                   |
| \sphinxhyphen{}   la curvatura delle linee di   |                                   |
|     flusso sia trascurabile;      |                                   |
|                                   |                                   |
| \sphinxhyphen{}   sia trascurabile ogni perdita |                                   |
|     di energia.                   |                                   |
|                                   |                                   |
| Si determinino:                   |                                   |
|                                   |                                   |
| \sphinxhyphen{}   il diametro \(D\) del getto     |                                   |
|     alla quota \(Z = 15\ m\)        |                                   |
|     (misurata dal piano d’uscita  |                                   |
|     dall’ugello);                 |                                   |
|                                   |                                   |
| \sphinxhyphen{}   la massima quota ideale \(H\)   |                                   |
|     cui può giungere il getto.    |                                   |
|                                   |                                   |
| (\(D = 6.97\ cm\), \(H = 20.39\  m\)) |                                   |
+———————————–+———————————–+

\sphinxAtStartPar
Teorema di Bernoulli nell’ipotesi di stazionarietà, fluido
incomprimibile, non viscoso, irrotazionale. Equazione della vorticità
nel caso non viscoso.
\begin{itemize}
\item {} 
\sphinxAtStartPar
Il primo quesito del problema viene risolto mettendo a sistema
l’equazione di Bernoulli (ipotesi…) e l’equazione della
continuità. \$\(\begin{cases}
  \frac{1}{2} \rho V^2  = \frac{1}{2}\rho u^2(z) + \rho g z\\
  V d^2 = u(z) D^2
\end{cases} \qquad \Rightarrow \qquad D = \frac{d}
{\displaystyle\left[1 - \frac{2 g z}{V^2}\right]^{\frac{1}{4}}}\)\(
Inserendo i valori numerici \)D = 6.97 \textbackslash{}text\{cm\}\$.

\item {} 
\sphinxAtStartPar
Il secondo quesito si ottiene ricavando dal teorema di Bernoulli la
quota alla quale la velocità è nulla.

\sphinxAtStartPar
\$\(\frac{1}{2} \rho V^2 = \rho g H \qquad \Rightarrow \qquad 
  H = \frac{1}{2} \frac{V^2}{g}\)\( Inserendo i valori numerici
\)H = 20.39 \textbackslash{}text\{m\}\$.

\end{itemize}

\sphinxstepscope


\subsection{Exercise 5.6}
\label{\detokenize{polimi/fluidmechanics-ita/template/capitoli/05_bernoulli/0507in:exercise-5-6}}\label{\detokenize{polimi/fluidmechanics-ita/template/capitoli/05_bernoulli/0507in:fluid-mechanics-bernoulli-ex-06}}\label{\detokenize{polimi/fluidmechanics-ita/template/capitoli/05_bernoulli/0507in::doc}}
\sphinxAtStartPar
In un gioco d’acqua (\(\rho=999\ kg/m^3\)), un disco di diametro
\(D=35\ cm\) viene sollevato da un getto che fuoriesce con velocità
\(V_0=10\ m/s\) da un foro di diametro \(d_0=8\ cm\) concentrico all’asse
del disco, così come illustrato schematicamente in figura. Noto che in
condizioni stazionarie la quota raggiunta dal disco è di poco superiore
alla quota \(H=2\ m\), si richiede di determinare:
\begin{itemize}
\item {} 
\sphinxAtStartPar
la velocità \(V_1\) e il diametro \(d_1\) del getto alla quota \(H\)
supponendo trascurabili tra le sezioni \(0\) e \(1\) sia la curvatura
delle linee di flusso che ogni forma di dissipazione;

\item {} 
\sphinxAtStartPar
lo spessore \(h\) del film d’acqua all’estremità del disco assumendo
che il profilo di velocità radiale sia lineare con velocità massima
\(V_2=V_1\).

\item {} 
\sphinxAtStartPar
la massa \(m\) del disco considerando trascurabili sia gli sforzi
viscosi all’interfaccia tra l’atmosfera circostante
(\(P_{atm}=101325\ Pa\)) e il getto d’acqua che la forza
gravitazionale agente sul fluido tra la quota \(H\) e la quota del
disco.

\end{itemize}

\sphinxAtStartPar
\sphinxstyleemphasis{Per la risoluzione del problema si assumano condizioni di
assialsimmetria.}

\sphinxAtStartPar
\sphinxincludegraphics{{polimi/fluidmechanics-ita/template/capitoli/05_bernoulli/fig/jet}}\{width=»100\%»\}

\sphinxAtStartPar
Teorema di Bernoulli nell’ipotesi di stazionarietà, fluido
incomprimibile, non viscoso, irrotazionale. Bilanci integrali.
\begin{itemize}
\item {} 
\sphinxAtStartPar
Il primo punto viene risolto mettendo a sistema il teorema di
Bernoulli e la continuità. \$\(\begin{cases}
  \frac{1}{2} \rho V_0^2  = \frac{1}{2}\rho V_1^2(z) + \rho g H\\
  V_0 d_0^2 = V_1 d_1^2
\end{cases} \qquad \Rightarrow \qquad 
\begin{cases}
  V_1 = V_0\sqrt{1 - 2 g H / V_0^2} \\
  d_1 = \displaystyle\left[1 - \frac{2 g H}{V_0^2}\right]^{-\frac{1}{4}} d_0
\end{cases}\)\$

\item {} 
\sphinxAtStartPar
Il secondo punto viene risolto utilizzando solamente il bilancio di
massa.
\$\(Q = \frac{\pi}{4} \rho V_0 d_0^2 = \frac{\pi}{4} \rho V_1 d_1^2 = \frac{\pi}{2} D h V_2 \qquad \Rightarrow \qquad h = \frac{d_1^2}{2 D}\)\$

\item {} 
\sphinxAtStartPar
Il terzo punto viene risolto applicando il bilancio della quantità
di moto in direzione verticale per trovare la forza applicata dal
disco al fluido. Infine si scrive l’equilibrio del disco soggetto
alla stessa forza con verso opposto (principio di azione e reazione)
e al proprio peso.

\sphinxAtStartPar
Dal bilancio si ottiene che la componente verticale della forza che
si scambiano fluido e disco è uguale a
\(\rho V_1^2 \frac{\pi}{4} d_1^2\). La massa del disco è quindi
\(m = \frac{\pi}{4} d_1^2 \frac{\rho V_1^2}{g}\)

\end{itemize}

\sphinxstepscope


\subsection{Exercise 5.7}
\label{\detokenize{polimi/fluidmechanics-ita/template/capitoli/05_bernoulli/0508in:exercise-5-7}}\label{\detokenize{polimi/fluidmechanics-ita/template/capitoli/05_bernoulli/0508in:fluid-mechanics-bernoulli-ex-07}}\label{\detokenize{polimi/fluidmechanics-ita/template/capitoli/05_bernoulli/0508in::doc}}
\sphinxAtStartPar
Un getto di acqua (\(\rho = 1000 \ kg/m^3\)) colpisce una lamina di massa
per unità di apertura \(m = 1 \ kg/m\) inclinata di un angolo
\(\alpha = 30\degree\) rispetto all’orizzontale, connessa a terra con una
molla di costante elastica \(k = 10^5 \ N/m^2\). Il getto esce con profilo
uniforme \(U=10 \ m/s\) da una fessura larga \(d_1 = 5 \ cm\) Determinare:
\begin{itemize}
\item {} 
\sphinxAtStartPar
la velocità \(U_2\) (uniforme) e lo spessore \(d_2\) del getto alla
quota \(H=1 \ m\) sopra la fessura di uscita, supponendo trascurabili
ogni forma di dissipazione e la curvatura delle linee di corrente;

\item {} 
\sphinxAtStartPar
la velocità massima \(V\) del profilo triangolare di spessore
\(h = 2 \ cm\), identico su entrambe le estremità della lamina;

\item {} 
\sphinxAtStartPar
la deformazione della molla, considerando trascurabili gli sforzi
viscosi all’interfaccia tra il getto e l’atmosfera circostante
(\(P_a = 101325 \ Pa\)) e la gravità agente sul fluido al di sopra
della quota \(H\).

\end{itemize}

\sphinxAtStartPar
\sphinxincludegraphics{{polimi/fluidmechanics-ita/template/capitoli/05_bernoulli/fig/jet_angle}}\{width=»100\%»\}

\sphinxAtStartPar
Teorema di Bernoulli nell’ipotesi di stazionarietà, fluido
incomprimibile, non viscoso, irrotazionale. Bilanci integrali.
\begin{itemize}
\item {} 
\sphinxAtStartPar
continuità + Bernoulli \$\(\begin{cases}
        \rho d U = \rho d_2 U_2 \\
        \frac{1}{2} \rho U^2 = \frac{1}{2} \rho U_2^2 + \rho g H
      \end{cases}
      \qquad \Rightarrow \qquad
      \begin{cases}
        d_2 = d_1 \left( 1 - \dfrac{2 g H}{U^2} \right)^{-1/2} = 0.0558 \ m \\
        U_2 = U \left( 1 - \dfrac{2 g H}{U^2} \right)^{1/2} = 8.96 \ m/s \\
      \end{cases}\)\$

\item {} 
\sphinxAtStartPar
continuità: in ingresso profilo uniforme, in uscita due profili
triangolari.
\$\(U d_1 = 2 \dfrac{1}{2} V h \Rightarrow V = U \dfrac{d_1}{h} = 25 \ m/s\)\$

\item {} 
\sphinxAtStartPar
bilancio di massa + equilibrio corpo: pressione \(P_a\) ovunque; i due
flussi di quantità di moto sulla lamina si bilanciano: rimane solo
il termine in ingresso
\$\(\bm{R}_{fl} = - \oint_{\partial \Omega} \rho \bm{u} \bm{u}
    \cdot \bm{\hat{n}}  =  \dots  = \rho U^2 \dfrac{d_1^2}{d_2} \bm{\hat{y}} = 4482.7 \ N \bm{\hat{y}}\)\(
\)\(k \Delta x = m g - R \Rightarrow \Delta x = - 0.0447 \ m \ \text{(compressione)}\)\$

\end{itemize}

\sphinxstepscope


\chapter{Exact solutions of Navier\sphinxhyphen{}Stokes equations}
\label{\detokenize{polimi/fluidmechanics-ita/template/capitoli/06_slnEsatte/0600in:exact-solutions-of-navier-stokes-equations}}\label{\detokenize{polimi/fluidmechanics-ita/template/capitoli/06_slnEsatte/0600in:fluid-mechanics-exact-solutions}}\label{\detokenize{polimi/fluidmechanics-ita/template/capitoli/06_slnEsatte/0600in::doc}}
\sphinxAtStartPar
\sphinxstylestrong{todo} \sphinxstyleemphasis{…bla bla…} Low\sphinxhyphen{}\(Re\) stable flows…


\section{Introduzione e linee guida per la soluzione dei problemi}
\label{\detokenize{polimi/fluidmechanics-ita/template/capitoli/06_slnEsatte/0600in:introduzione-e-linee-guida-per-la-soluzione-dei-problemi}}
\sphinxAtStartPar
========================================================

\sphinxAtStartPar
É possibile ricavare alcune soluzioni esatte stazionarie delle equazioni
di Navier\sphinxhyphen{}Stokes, che descrivono il moto di un fluido viscoso, quando il
dominio ha una geometria «semplice». In alcuni casi, come la corrente in
un canale piano (Newton\sphinxhyphen{}Couette), la corrente in un tubo a sezione
circolare (Poiseuille), o la corrente nel setto tra due cilindri rotanti
(Taylor\sphinxhyphen{}Couette), per semplificare le equazioni è possibile sfruttare
l’omogeneità del dominio (in qualche direzione) e, per ipotesi, della
corrente. Nella maggioranza delle soluzioni esatte, i termini non
lineari nelle equazioni si annullano, permettendo di ricavare abbastanza
facilmente la soluzione delle equazioni.

\sphinxAtStartPar
In generale, le soluzioni stazionarie esatte presentate in questo
capitolo sono significative quando il regime di moto è laminare. Senza
entrare molto nel dettaglio, una soluzione stazionaria è una soluzione
di equilibrio delle equazioni di Navier\sphinxhyphen{}Stokes, per la quale
\(\partial \mathbf{u}/\partial t = \mathbf{0}\). Un regime di moto instazionario
può manifestarsi a causa di una «instabilità intrinseca» della corrente
o a causa di una enorme amplificazione (\sphinxstyleemphasis{ricettività}) di disturbi,
anche di intensità minima, sempre presenti in natura%
\begin{footnote}[1]\sphinxAtStartFootnote
Il regime di moto periodico (e ordinato) che si manifesta nella
corrente attorno a un cilindro quando il numero di Reynolds supera
un valore critico (\(Re_c \approx 46\)) è il risultato di una
«instabilità intrinseca» (\sphinxstyleemphasis{globale}) parametrica del sistema. La
soluzione stazionaria stabile esistente per \(Re < Re_c\) diventa
instabile quando il parametro \(Re\) eccede il valore critico e nasce
un ciclo limite (stabile) nel piano delle fasi del sistema. Il moto
periodico e ordinato del sistema osservato nello sviluppo della
\sphinxstyleemphasis{scia di Von Karman} a valle del cilindro, corrisponde alla dinamica
del sistema sul ciclo limite. Mentre la corrente attorno a un corpo
tozzo risulta abbastanza insensibile ai disturbi e perturbazioni
esterni, altre correnti possono amplificare perturbazioni di
intensità ridotta di diversi ordini di grandezza. Alcuni esempi sono
uno strato limite, uno strato di mescolamento o un getto. In queste
correnti dominate dalla convezione, l’enorme amplificazione può
avvenire tramite meccanismi \sphinxstyleemphasis{non\sphinxhyphen{}modali}, che caratterizzano di
sistemi dinamici lineari stabili non simmetrici.
%
\end{footnote}. Entrambi i
processi che allontanano la corrente dalla condizione di equilibrio
vengono innescati o amplificati all’aumentare del numero di Reynolds
caratteristico della corrente. Qualitativamente, si può quindi affermare
che le soluzioni stazionarie esatte sono rappresentative del fenomeno
fisico quando il numero di Reynolds caratteristico assume valori
«sufficientemente bassi», per i quali non si verificano instabilità
intrinseche nella corrente e per i quali le perturbazioni e gli effetti
di estremità (ad esempio, all’imbocco di un tubo) vengono smorzati dalla
viscosità, rendendo la corrente stazionaria e omogenea.

\sphinxAtStartPar
In questa introduzione non c’è nessuna velleità di una descrizione
precisa e completa di quelli che sono gli argomenti di studio della
\sphinxstyleemphasis{stabilità fluidodinamica}, ma solamente la necessità di precisare i
«limiti di validità» delle soluzioni esatte ricavate in questo capitolo.


\section{Equazioni di Navier\sphinxhyphen{}Stokes in coordinate cartesiane e cilindriche}
\label{\detokenize{polimi/fluidmechanics-ita/template/capitoli/06_slnEsatte/0600in:equazioni-di-navier-stokes-in-coordinate-cartesiane-e-cilindriche}}
\sphinxAtStartPar
Le equazioni di Navier\sphinxhyphen{}Stokes vengono scritte nel sistema di coordinate
più adeguato alla descrizione del problema, come ad esempio possono
essere le coordinate cartesiane o quelle cilindriche. Le equazioni di
Navier\sphinxhyphen{}Stokes per un fluido incomprimibile
\begin{equation*}
\begin{split}\begin{cases}
 \rho \dfrac{\partial \mathbf{u}}{\partial t}
 + \rho (\mathbf{u} \cdot \mathbf{\nabla}) \mathbf{u}
 - \mu \nabla^2 \mathbf{u} + \mathbf{\nabla} p = \rho \mathbf{g} \\
 \mathbf{\nabla} \cdot \mathbf{u} = 0
\end{cases}\end{split}
\end{equation*}
\sphinxAtStartPar
accompagnate dalle condizioni iniziali e dalle condizioni
al contorno opportune (e, qualora servissero, dalle condizioni di
compatibilità dei dati), possono essere scritte ad esempio in un sistema
di coordinate cartesiane
\begin{equation*}
\begin{split}\begin{cases}
  \rho \dfrac{\partial u}{\partial t}
  + \rho \left( u \dfrac{\partial u}{\partial x}
  + v  \dfrac{\partial u}{\partial y}
  + w  \dfrac{\partial u}{\partial z} \right)- \mu \left( 
  \dfrac{\partial^2 u}{\partial x^2} +
  \dfrac{\partial^2 u}{\partial y^2} +
  \dfrac{\partial^2 u}{\partial z^2} \right)
  + \dfrac{\partial p}{\partial x} = \rho g_x \\
  \rho \dfrac{\partial v}{\partial t}
  + \rho \left( u \dfrac{\partial v}{\partial x}
  + v  \dfrac{\partial v}{\partial y}
  + w  \dfrac{\partial v}{\partial z} \right)- \mu \left( 
  \dfrac{\partial^2 v}{\partial x^2} +
  \dfrac{\partial^2 v}{\partial y^2} +
  \dfrac{\partial^2 v}{\partial z^2} \right)
  + \dfrac{\partial p}{\partial y} = \rho g_y \\
  \rho \dfrac{\partial w}{\partial t} + 
  \rho \left( u \dfrac{\partial w}{\partial x}
  + v \dfrac{\partial w}{\partial y} 
  + w \dfrac{\partial w}{\partial z} \right)- \mu \left( 
  \dfrac{\partial^2 w}{\partial x^2} +
  \dfrac{\partial^2 w}{\partial y^2} +
  \dfrac{\partial^2 w}{\partial z^2} \right)
  + \dfrac{\partial p}{\partial z} = \rho g_z \\
  \dfrac{\partial u}{\partial x}
+ \dfrac{\partial v}{\partial y}
+ \dfrac{\partial w}{\partial z} = 0
\end{cases}\end{split}
\end{equation*}
\sphinxAtStartPar
o in un sistema di coordinate cilindriche
\begin{equation*}
\begin{split}\begin{cases}
\rho \dfrac{\partial u_r}{\partial t}
+ \rho \left( \mathbf{u} \cdot \mathbf{\nabla}u_r - \dfrac{u_\theta^2}{r} \right)
- \mu \left(\nabla^2 u_r 
   - \dfrac{u_r}{r^2} 
   - \dfrac{2}{r^2}\dfrac{\partial u_\theta}{\partial \theta} \right)  
   + \dfrac{\partial p}{\partial r} = \rho g_r \\
\rho \dfrac{\partial u_\theta}{\partial t}
+ \rho \left( \mathbf{u} \cdot \mathbf{\nabla} u_\theta + \dfrac{u_\theta u_r}{r} \right)
- \mu \left(\nabla^2 u_\theta 
   - \dfrac{u_\theta}{r^2} 
   + \dfrac{2}{r^2}\dfrac{\partial u_r}{\partial \theta}  \right) 
+ \dfrac{1}{r} \dfrac{\partial p}{\partial \theta} = \rho g_\theta\\
\rho \dfrac{\partial u_z}{\partial t}
+ \rho \mathbf{u} \cdot \mathbf{\nabla} u_z
- \mu \nabla^2 u_z
+ \dfrac{\partial p}{\partial z} = \rho g_z \\
\dfrac{1}{r}\dfrac{\partial}{\partial r}\left( r u_r \right) 
+ \dfrac{1}{r}\dfrac{\partial u_\theta}{\partial \theta} 
+ \dfrac{\partial u_z}{\partial z} = 0
\end{cases}\end{split}
\end{equation*}
\sphinxAtStartPar
dove
\begin{equation*}
\begin{split}\begin{aligned}
  & \mathbf{a} \cdot \mathbf{\nabla} b = a_r \dfrac{\partial b}{\partial r} 
     + \dfrac{a_\theta}{r} \dfrac{\partial b}{\partial \theta}  
     + a_z \dfrac{\partial b}{\partial z} \\
  & \nabla^2 f = \dfrac{1}{r}\dfrac{\partial}{\partial r}
                      \left(r \dfrac{\partial f}{\partial r} \right) +
               \dfrac{1}{r^2} \dfrac{\partial^2 f}{\partial \theta^2} + 
               \dfrac{\partial^2 f}{\partial z^2} 
  \end{aligned}\end{split}
\end{equation*}

\section{Esempio in coordinate cartesiane: corrente di Poseuille}
\label{\detokenize{polimi/fluidmechanics-ita/template/capitoli/06_slnEsatte/0600in:esempio-in-coordinate-cartesiane-corrente-di-poseuille}}
\sphinxAtStartPar
Nel caso di corrente bidimensionale di Poiseuille in un canale piano, si
usano le equazioni scritte nel sistema di coordinate cartesiane. Si
sceglie l’asse \(x\) orientato lungo il canale e l’asse \(y\) perpendicolare
alle pareti. Si fanno alcune ipotesi:
\begin{itemize}
\item {} 
\sphinxAtStartPar
stazionarietà: \(\dfrac{\partial}{\partial t} = 0\);

\item {} 
\sphinxAtStartPar
omogeneità della coordinata \(x\): il campo di velocità è indipendente
dalla coordinata \(x\). La derivata di tutte le componenti della
velocità rispetto ad \(x\) è nulla:
\(\dfrac{\partial u_i}{\partial x} = 0\). É invece ammissibile che la
pressione vari lungo \(x\): da un punto di vista fisico, è necessario
un gradiente di pressione che equilibri gli sforzi a parete dovuti
alla viscosità e che «spinga» il fluido nel canale; dal punto di
vista matematico, è già stato accennato al ruolo particolare che
svolge quel campo indicato con \(p\), diverso dalla pressione
termodinamica nel caso di fluido incomprimibile; si osservi poi che
il campo \(p\) non compare mai nelle equazioni, se non sotto
l’operatore di gradiente (o all’interno delle condizioni al
contorno, che «fissano» un valore di \(p\): è già stato sottolineato
più volte che spesso il moto di un fluido incomprimibile è
indipendente dal valore assoluto del campo \(p\), mentre dipende da
differenze, o dalle derivate, di \(p\)!).

\item {} 
\sphinxAtStartPar
sfruttando la bidimensionalità della corrente, l’omogeneità della
coordinata \(x\) e il vincolo di incomprimibilità si ottiene:
\begin{equation*}
\begin{split}\underbrace{\dfrac{\partial u}{\partial x}}_{=0} + \dfrac{\partial v}{\partial y} = 0.\end{split}
\end{equation*}
\sphinxAtStartPar
Questo implica che la componente \(v\) della velocità è costante in
tutto il canale; sfruttando le condizioni al contorno di adesione a
parete \(\mathbf{u} = (u,v) = \mathbf{0}\) è evidente che la costante deve
essere nulla: \(v = 0\).

\item {} 
\sphinxAtStartPar
supponiamo qui che, se vengono considerate le forze di volume, esse
sono costanti e dirette lungo \(-\mathbf{\hat{y}}\).

\end{itemize}

\sphinxAtStartPar
Le equazioni diventano quindi
\begin{equation*}
\begin{split}\begin{cases}\label{eqn:poiseuille}
- \mu \left( \dfrac{d^2 u}{d y^2} \right)
  + \dfrac{\partial p}{\partial x} = 0 \\
 \dfrac{\partial p}{\partial y} = - \rho g
\end{cases}\end{split}
\end{equation*}
\sphinxAtStartPar
dove la derivata parziale in \(y\) della componente \(u\) è
stata sostituita dalla derivata ordinaria, poiché la velocità
\(\mathbf{u}(y)\) (e quindi tutte le sue componenti) dipende solo dalla
coordinata \(y\). La seconda equazione integrata dà come risultato (\(p\)
dipende sia da \(x\) sia da \(y\)):
\begin{equation*}
\begin{split}p(x,y) = -\rho g y + f(x)\end{split}
\end{equation*}
\sphinxAtStartPar
Inserita
nella prima:
\begin{equation*}
\begin{split}\mu \left( \dfrac{d^2 u}{d y^2} \right) =
 \dfrac{\partial p}{\partial x} = \dfrac{\partial f}{\partial x}\end{split}
\end{equation*}
\sphinxAtStartPar
Nell’ultima equazione, i termini a sinistra dell’uguale sono funzione
solo della variabile indipendente \(y\), quelli a destra dell’uguale solo
di \(x\): affinché l’uguaglianza possa essere sempre valida, i due termini
devono essere costanti; si sceglie di definire questa costante \(-G_P\)
(con questa \(G_P\) assumerà valore positivo). Si possono quindi risolvere
le due equazioni
\begin{equation*}
\begin{split}\label{eqn:poiseuille2}
\begin{cases}
  \mu \left( \dfrac{d^2 u}{d y^2} \right) = - G_P \\
  \dfrac{\partial p}{\partial x} = -G_P
\end{cases}\end{split}
\end{equation*}
\sphinxAtStartPar
accompagnate dalle opportune condizioni al contorno.
Osservando il sistema
(\DUrole{xref,myst}{{[}eqn:poiseuille{]}}\{reference\sphinxhyphen{}type=»ref»
reference=»eqn:poiseuille»\}), nelle equazioni compare la derivata
seconda in \(y\) della componente \(u\) della velocità, la derivata prima
sia in \(x\) sia in \(y\) di \(p\): è ragionevole pensare che servano due
condizioni al contorno in \(y\) per \(u\), una condizione al contorno per
\(p\) in \(x\) e una in \(y\). In particolare, sulle pareti del canale (alto
\(H\)) la velocità del fluido deve essere nulla, per le condizioni al
contorno di adesione. Per quando riguarda la pressione, si può fissare
il valore in un punto del dominio, ad esempio l’origine degli assi
\(p(0,0) = p_0\).
\begin{equation*}
\begin{split}\begin{cases}
  u(x,0) = 0 \\
  u(x,H) = 0 \\
  p(0,0) = p_0
 \end{cases}\end{split}
\end{equation*}
\sphinxAtStartPar
Le equazioni
(\DUrole{xref,myst}{{[}eqn:poiseuille2{]}}\{reference\sphinxhyphen{}type=»ref»
reference=»eqn:poiseuille2»\}) con le condizioni al contorno appena
elencate danno come risultato:
\begin{equation*}
\begin{split}\begin{cases}
    u(y) = -\dfrac{G_P}{2 \mu} y (y - H) \\
    p(x,y) = p_0 - \rho g y - G_P x
  \end{cases}\end{split}
\end{equation*}

\section{Calcolo del vettore sforzo}
\label{\detokenize{polimi/fluidmechanics-ita/template/capitoli/06_slnEsatte/0600in:calcolo-del-vettore-sforzo}}
\sphinxAtStartPar
Se vengono chieste azioni (risultanti di forze o momenti) esercitate dal
fluido sul solido, è necessario calcolare lo sforzo a parete
\(\mathbf{t}_{n,s}\) esercitato sul solido, uguale e contrario allo sforzo
esercitato dal solido sul fluido \(\mathbf{t}_{n}\). Il vettore sforzo
\(\mathbf{t}_n\) su una superficie con giacitura definita dal versore normale
\(\mathbf{\hat{n}}\) si può esprimere come il prodotto del versore
\(\mathbf{\hat{n}}\) e il tensore degli sforzi \(\mathbb{T}\),
\begin{equation*}
\begin{split}\label{eqn:stress_tensor}
 \mathbf{t}_n = \mathbf{\hat{n}} \cdot \mathbb{T} =
 \mathbf{\hat{n}} \cdot \big[-p\mathbb{I} + 2\mu\mathbb{D} \big] = 
  - p\mathbf{\hat{n}} + \mathbf{s}_n \ ,\end{split}
\end{equation*}
\sphinxAtStartPar
avendo utilizzato la relazione costitutiva
\(\mathbb{S} = 2 \mu \mathbb{D}\) per un fluido incomprimibile newtoniano,
che lega il tensore degli sforzi viscosi \(\mathbb{S}\) al tensore
velocità di deformazione \(\mathbb{D}\) tramite il coefficiente di
viscosità dinamica \(\mu\). Il vettore
\(\mathbf{s_n} = \mathbf{\hat{n}} \cdot \mathbb{S}\) è il vettore degli sforzi
viscosi. É possibile trasformare la relazione
(\DUrole{xref,myst}{{[}eqn:stress\_tensor{]}}\{reference\sphinxhyphen{}type=»ref»
reference=»eqn:stress\_tensor»\}) in una relazione che contenga solamente
operazioni tra vettori,
\begin{equation*}
\begin{split}\label{eqn:stress_vector}
 \mathbf{t}_n = -p \mathbf{\hat{n}} +
 \mu \big[2 (\mathbf{\hat{n}} \cdot \mathbf{\nabla}) \mathbf{u} +
  \mathbf{\hat{n}} \times (\mathbf{\nabla} \times \mathbf{u}) \big]  = 
  - p\mathbf{\hat{n}} + \mathbf{s}_n \ .\end{split}
\end{equation*}
\sphinxAtStartPar
Questa espressione può risultare vantaggiosa quando è richiesto il
calcolo del vettore degli sforzi in sistemi di coordinate non
cartesiani. Mentre esistono molte tabelle che raccolgono l’espressione
delle operazioni vettoriali in sistemi di coordinate non cartesiane,
sono più rare tabelle che raccolgano la forma in componenti di
operazioni tensoriali.

\sphinxAtStartPar
In sistemi di coordinate cartesiane, è facile calcolare il vettore
sforzo come prodotto tensoriale tra il versore normale \(\mathbf{\hat{n}}\) e
il tensore degli sforzi, le cui componenti sono facili da calcolare
\begin{equation*}
\begin{split}\begin{aligned}
 \mathbb{T} & = -p\mathbb{I} + 2\mu\mathbb{D} = -p\mathbb{I} + 2\mu \left[ \dfrac{1}{2} \left( \mathbf{\nabla}\mathbf{u} +\mathbf{\nabla}^T \mathbf{u} \right) \right] \\
 T_{ij} & = -p \delta_{ij} + 2 \mu D_{ij} = -p \delta_{ij} + 2 \mu \left[ \dfrac{1}{2} \left( \dfrac{\partial u_i}{\partial x_j} + \dfrac{\partial u_j}{\partial x_i} \right) \right] \ .
 \end{aligned}\end{split}
\end{equation*}
\sphinxAtStartPar
Ad esempio, per una corrente in uno spazio
bidimensionale descritto dalle coordinate cartesiane \((x,y)\) le
componenti del tensore degli sforzi possono essere raccolte in forma
matriciale,
\begin{equation*}
\begin{split}\mathbb{T} =
 -p \begin{bmatrix}
   1 & 0 \\ 0 & 1
 \end{bmatrix} +
 2 \mu \begin{bmatrix}
  \dfrac{\partial u}{\partial x} & 
  \dfrac{1}{2}\bigg(\dfrac{\partial u}{\partial y} + \dfrac{\partial v}{\partial x}\bigg) \\
  \dfrac{1}{2}\bigg(\dfrac{\partial u}{\partial y} + \dfrac{\partial u}{\partial y}\bigg) &
  \dfrac{\partial v}{\partial y} & 
 \end{bmatrix}\end{split}
\end{equation*}
\sphinxAtStartPar
Sfruttando la simmetria del tensore degli sforzi,
\(T_{ij} = T_{ji}\), il vettore sforzo \(t_i = n_j T_{ji} = T_{ij} n_j\) può
essere calcolato come prodotto matrice vettore. Come esempio, viene
calcolato lo sforzo a parete in un canale piano, nel quale scorre un
fluido con un campo di velocità che ha solamente la componente parallela
alle pareti che dipende dalla coordinata perpendicolare ad esse,
\(\mathbf{u}(\mathbf{r}) = u(y) \mathbf{\hat{x}}\). Facendo riferimento alla corrente
di Poiseuille della sezione precedente, il vettore sforzo agente sul
fluido in corrispondenza della parete interiore a \(y=0\), si ottiene
moltiplicando il versore normale uscente dal fluido
\(\mathbf{\hat{n}} = - \mathbf{\hat{y}}\) per il tensore degli sforzi,
\begin{equation*}
\begin{split}\begin{aligned}
 \begin{bmatrix} t_x \\ t_y \end{bmatrix} & = 
 \begin{bmatrix}
   -p & \mu \dfrac{\partial u}{\partial y} \\ \mu \dfrac{\partial u}{\partial y} & -p
 \end{bmatrix} 
 \begin{bmatrix} n_x \\ n_y \end{bmatrix} = 
% \begin{bmatrix}
%   -p & \mu \dfrac{G_P H}{2\mu} \\ \mu \dfrac{G_P H}{2\mu} & -p
% \end{bmatrix}
 \begin{bmatrix}
   -p & \mu \dfrac{\partial u}{\partial y} \\ \mu \dfrac{\partial u}{\partial y} & -p
 \end{bmatrix} 
 \begin{bmatrix} 0 \\ -1 \end{bmatrix} = 
 \begin{bmatrix} - \mu \dfrac{\partial u}{\partial y} \\ p \end{bmatrix} \\
 & \qquad \rightarrow \qquad \mathbf{t_n} =  - \mu \dfrac{\partial u}{\partial y} \mathbf{\hat{x}} + p \mathbf{\hat{y}} = - \dfrac{G_P H}{2} \mathbf{\hat{x}} + p \mathbf{\hat{y}} \ .
\end{aligned}\end{split}
\end{equation*}
\sphinxAtStartPar
Lo sforzo sulla parete inferiore è l’opposto
\(\mathbf{t}_{n,s} = \frac{G_P H}{2} \mathbf{\hat{x}} - p \mathbf{\hat{y}}\). Sulla
parete superiore, a \(y=H\), la normale uscente dal fluido è
\(\mathbf{\hat{n}} = \mathbf{\hat{y}}\), la derivata
\(\partial u/\partial y (x,H) = -G_P/(2\mu)\). Svolgendo i conti, come
fatto per la parete inferiore, si ottiene che lo sforzo agente sulla
parete superiore è
\(\mathbf{t}_{n,s} = \frac{G_P H}{2} \mathbf{\hat{x}} + p \mathbf{\hat{y}}\).


\subsection{Equivalenza tra l’espressione tensoriale e vettoriale del vettore sforzo.}
\label{\detokenize{polimi/fluidmechanics-ita/template/capitoli/06_slnEsatte/0600in:equivalenza-tra-l-espressione-tensoriale-e-vettoriale-del-vettore-sforzo}}
\sphinxAtStartPar
Per i più curiosi e i più «matematici», si dimostra infine l’equivalenza
tra (\DUrole{xref,myst}{{[}eqn:stress\_tensor{]}}\{reference\sphinxhyphen{}type=»ref»
reference=»eqn:stress\_tensor»\}) e
(\DUrole{xref,myst}{{[}eqn:stress\_vector{]}}\{reference\sphinxhyphen{}type=»ref»
reference=»eqn:stress\_vector»\}). Questa dimostrazione viene fatta
ricorrendo alla notazione indiciale, sfruttando le proprietà di
permutazione degli indici del simbolo \(\epsilon_{ijk}\) e la proprietà
dei simboli \(\epsilon_{ijk}\) e \(\delta_{a,b}\),
\begin{equation*}
\begin{split}\epsilon_{kij}\epsilon_{klm} = \delta_{il}\delta_{jm} - \delta_{im}\delta_{jl} \ .\end{split}
\end{equation*}
\sphinxAtStartPar
La componente \(i\)\sphinxhyphen{}esima di
\(\mathbf{\hat{n}} \times \mathbf{\nabla} \times \mathbf{u}\) è
\begin{equation*}
\begin{split}\begin{aligned}
 \left\{ \mathbf{\hat{n}} \times \mathbf{\nabla} \times \mathbf{u} \right\}_i & =
 \epsilon_{ijk} n_j \left\{ \mathbf{\nabla} \times \mathbf{u} \right\}_k = \\
  & = \epsilon_{ijk} n_j \epsilon_{klm} \partial_l u_m = \\
  & = \epsilon_{kij} \epsilon_{klm} n_j \partial_l u_m = \\
  & = (\delta_{il}\delta_{jm} - \delta_{im}\delta_{jl} ) n_j \partial_l u_m = \\
  & = n_j \partial_i u_j - n_j \partial_j u_i = \\
  & = n_j \partial_i u_j + n_j \partial_j u_i - 2 n_j \partial_j u_i = \\
  & = 2 n_j \dfrac{1}{2}\left(\partial_i u_j + n_j \partial_j u_i \right) - 2 n_j \partial_j u_i = \\
  & = \left\{2 \mathbf{\hat{n}} \cdot \mathbb{D} - 2 ( \mathbf{\hat{n}} \cdot \mathbf{\nabla} ) \mathbf{u} \right\}_i
\end{aligned}\end{split}
\end{equation*}
\sphinxAtStartPar
Il contributo viscoso al vettore sforzo è uguale al
primo termine a destra dell’uguale moltiplicato per la viscosità
dinamica \(\mu\), \(\mathbf{s}_n = 
 2 \mu \mathbf{\hat{n}} \cdot \mathbb{D}\); il vettore sforzo \(\mathbf{t}_n\) è la
somma del vettore degli sforzi viscosi \(\mathbf{s}_n\) e del vettore degli
sforzi (normali) dovuti alla «pressione», \$\textbackslash{}mathbf\{t\}\_n = \textbackslash{}mathbf\{s\}\_n
\begin{itemize}
\item {} 
\sphinxAtStartPar
p \textbackslash{}mathbf\{\textbackslash{}hat\{n\}\}\$. Si ottiene così l’identità desiderata

\end{itemize}
\begin{equation*}
\begin{split}\label{eqn:stress_tensor-3}
 \mathbf{t}_n = \mathbf{\hat{n}} \cdot \mathbb{T} =
  \mathbf{\hat{n}} \cdot \big[-p\mathbb{I} + 2\mu\mathbb{D} \big] = 
  -p \mathbf{\hat{n}} +
 \mu \big[2 (\mathbf{\hat{n}} \cdot \mathbf{\nabla}) \mathbf{u} +
  \mathbf{\hat{n}} \times (\mathbf{\nabla} \times \mathbf{u}) \big] = 
  - p\mathbf{\hat{n}} + \mathbf{s}_n.\end{split}
\end{equation*}
\sphinxAtStartPar
Si ricorda che le identità vettoriali e tensoriali sono indipendenti dal
sistema di riferimento in cui vengono scritte le componenti: per la loro
dimostrazione si può utilizzare un sistema di coordinate qualsiasi
(spesso le coordinate cartesiani sono un sistema di coordinate
conveniente, poiché le espressioni delle operazioni e degli operatori
differenziali sono semplici da ricordare e utilizzare).


\subsection{Osservazione: vettore sforzo in coordinate cilindriche.}
\label{\detokenize{polimi/fluidmechanics-ita/template/capitoli/06_slnEsatte/0600in:osservazione-vettore-sforzo-in-coordinate-cilindriche}}
\sphinxAtStartPar
É possibile calcolare le componenti del prodotto
\(\mathbf{\hat{n}} \cdot \mathbb{T}\) svolgendo un prodotto matrice\sphinxhyphen{}vettore
anche per sistemi di coordinate non cartesiani. In questo caso, però, la
forma delle operazioni vettoriali e tensoriali e le componenti del
tensore sono «non banali». Per esempio le coordinate cartesiane del
gradiente \(\mathbf{\nabla} \mathbf{v}\) di un campo vettoriale \(\mathbf{v}\) sono
uguali a \(\partial v_i / \partial x_j\), mentre le componenti in
coordinate cilindriche sono raccolte nella seguente matrice \(3\times 3\),
\begin{equation*}
\begin{split}\begin{bmatrix}
\frac{\partial v_r}{\partial r} & 
 \frac{1}{r}\left( \frac{\partial v_r}{\partial \theta}-v_\theta \right) &
 \frac{\partial v_r}{\partial z}   \\
\frac{\partial v_\theta}{\partial r} & 
 \frac{1}{r}\left( \frac{\partial v_\theta}{\partial \theta}+v_r \right) & 
 \frac{\partial v_\theta}{\partial z} \\
\frac{\partial v_z}{\partial r} &
 \frac{1}{r}\frac{\partial v_z}{\partial \theta} &
 \frac{\partial v_z}{\partial z}
\end{bmatrix} \ ,\end{split}
\end{equation*}
\sphinxAtStartPar
se riferite alla base fisica
\((\mathbf{\hat{r}},\mathbf{\hat{\theta}},\mathbf{\hat{z}})\). Bisogna quindi prestare
attenzione nella scrittura delle componenti di tensori e operatori
quando si usano sistemi di coordinate non cartesiane. Per il calcolo del
vettore sforzo si consiglia quindi di usare, la formula
(\(\ref{eqn:stress_vector}\)) che contiene solo operazioni vettoriali, per
le quali è più facile trovare tavole che ne raccolgano le espressioni in
componenti in diversi sistemi di coordinate.

\sphinxAtStartPar
Per concludere questa sezione, viene ricavata l’espressione del vettore
degli sforzi viscosi in coordinate cartesiane come prodotto
\(2 \mu \mathbf{\hat{n}} \cdot \mathbb{D}\). Poichè il sistema di coordinate
c ilindriche (fisiche, riferite alla base
\((\mathbf{\hat{r}},\mathbf{\hat{\theta}},\mathbf{\hat{z}})\)) è un sistema
ortogonale, le componenti del vettore degli sforzi viscosi possono
essere calcolate con il prodotto matrice vettore,
\begin{equation*}
\begin{split}\begin{bmatrix} t_{r} \\ t_{\theta} \\ t_z \end{bmatrix} = \mu
 \begin{bmatrix}
 2 \frac{\partial v_r}{\partial r} & 
 \frac{\partial v_\theta}{\partial r} + \frac{1}{r}\left( \frac{\partial v_r}{\partial \theta}-v_\theta \right) &
 \frac{\partial v_z}{\partial r} + \frac{\partial v_r}{\partial z}   \\
 sym & 
 \frac{2}{r}\left( \frac{\partial v_\theta}{\partial \theta}+v_r \right) & 
 \frac{\partial v_\theta}{\partial z} +  \frac{1}{r}\frac{\partial v_z}{\partial \theta} \\
 sym &
 sym &
 2 \frac{\partial v_z}{\partial z}
\end{bmatrix}
 \begin{bmatrix} n_{r} \\ n_{\theta} \\ n_z \end{bmatrix} \ .\end{split}
\end{equation*}
\sphinxAtStartPar
Come
esercizio, è possibile utilizzare l’espressione vettoriale
(\DUrole{xref,myst}{{[}eqn:stress\_vector{]}}\{reference\sphinxhyphen{}type=»ref»
reference=»eqn:stress\_vector»\}) per verificare la validità
dell’espressione appena trovata del vettore sforzo in coordinate
cilindriche.


\bigskip\hrule\bigskip


\sphinxstepscope


\section{Exercises}
\label{\detokenize{polimi/fluidmechanics-ita/template/capitoli/06_slnEsatte/exercises:exercises}}\label{\detokenize{polimi/fluidmechanics-ita/template/capitoli/06_slnEsatte/exercises:fluid-mechanics-exact-solutions-exercises}}\label{\detokenize{polimi/fluidmechanics-ita/template/capitoli/06_slnEsatte/exercises::doc}}
\sphinxstepscope


\subsection{Exercise 6.1}
\label{\detokenize{polimi/fluidmechanics-ita/template/capitoli/06_slnEsatte/0604in:exercise-6-1}}\label{\detokenize{polimi/fluidmechanics-ita/template/capitoli/06_slnEsatte/0604in:fluid-mechanics-exact-solutions-ex01}}\label{\detokenize{polimi/fluidmechanics-ita/template/capitoli/06_slnEsatte/0604in::doc}}
\sphinxAtStartPar
+:———————————:+:———————————:+
| In un canale piano, di lunghezza  | !{[}image{]}(./fig/slnEsatte\sphinxhyphen{}newton\sphinxhyphen{}c |
| e apertura infinita, orizzontale, | ouette)\{width=»90\%»\}              |
| di altezza \(H=1.51\ mm\),          |                                   |
| delimitato da una parete          |                                   |
| inferiore fissa e da una parete   |                                   |
| superiore mobile con velocità     |                                   |
| orizzontale, costante e positiva  |                                   |
| \(U=0.31\ m/s\). scorre acqua in    |                                   |
| condizioni standard. Per quale    |                                   |
| valore del gradiente di pressione |                                   |
| \(G_P = -\partial P/\partial x\) la |                                   |
| portata nel canale risulta nulla? |                                   |
| Si trascurino le forze di volume. |                                   |
|                                   |                                   |
| (\(Re = 441\),                      |                                   |
| \(G_p = - 930\  Pa/m\))             |                                   |
+———————————–+———————————–+

\sphinxAtStartPar
Semplificazione delle equazioni di NS in coordinate cartesiane per
descrivere la corrente in un canale piano infinito messo in moto da un
gradiente di pressione (corrente di Poiseuille) e dal trascinamento
dovuto al movimento di una parete del canale (corrente di Newton).

\sphinxAtStartPar
In questo problema, la corrente nel canale ha due «forzanti»: il moto
a (velocità costante) della parete superiore e il gradiente di pressione
\(G_P\) lungo il canale. Il problema chiede di trovare il valore di \(G_P\)
tale che la portata nel canale sia nulla quando i due effetti si
combinano. Il problema viene risolto ricavando il profilo di velocità in
funzione del gradiente di velocità dalle equazioni di NS opportunamente
semplificate e successivamente il valore del gradiente di pressione
necessario ad avere portata nulla. La geometria del problema suggerisce
di utilizzare un sistema di coordinate cartesiane.
\begin{itemize}
\item {} 
\sphinxAtStartPar
Scrittura delle equazioni di NS in coordinate cartesiane in 2
dimensioni. \$\$\textbackslash{}begin\{cases\}
\textbackslash{}dfrac\{\textbackslash{}partial u\}\{\textbackslash{}partial t\} + u \textbackslash{}dfrac\{\textbackslash{}partial u\}\{\textbackslash{}partial x\}
\begin{itemize}
\item {} 
\sphinxAtStartPar
v \textbackslash{}dfrac\{\textbackslash{}partial u\}\{\textbackslash{}partial y\} \sphinxhyphen{} \textbackslash{}nu \textbackslash{}left(
\textbackslash{}dfrac\{\textbackslash{}partial\textasciicircum{}2 u\}\{\textbackslash{}partial x\textasciicircum{}2\} +
\textbackslash{}dfrac\{\textbackslash{}partial\textasciicircum{}2 u\}\{\textbackslash{}partial y\textasciicircum{}2\} \textbackslash{}right)

\item {} 
\sphinxAtStartPar
\textbackslash{}dfrac\{1\}\{\textbackslash{}rho\} \textbackslash{}dfrac\{\textbackslash{}partial p\}\{\textbackslash{}partial x\} = f\_x \textbackslash{}
\textbackslash{}dfrac\{\textbackslash{}partial v\}\{\textbackslash{}partial t\} + u \textbackslash{}dfrac\{\textbackslash{}partial v\}\{\textbackslash{}partial x\}

\item {} 
\sphinxAtStartPar
v \textbackslash{}dfrac\{\textbackslash{}partial v\}\{\textbackslash{}partial y\} \sphinxhyphen{} \textbackslash{}nu \textbackslash{}left(
\textbackslash{}dfrac\{\textbackslash{}partial\textasciicircum{}2 v\}\{\textbackslash{}partial x\textasciicircum{}2\} +
\textbackslash{}dfrac\{\textbackslash{}partial\textasciicircum{}2 v\}\{\textbackslash{}partial y\textasciicircum{}2\} \textbackslash{}right)

\item {} 
\sphinxAtStartPar
\textbackslash{}dfrac\{1\}\{\textbackslash{}rho\} \textbackslash{}dfrac\{\textbackslash{}partial p\}\{\textbackslash{}partial y\} = f\_y \textbackslash{}
\textbackslash{}dfrac\{\textbackslash{}partial u\}\{\textbackslash{}partial x\} + \textbackslash{}dfrac\{\textbackslash{}partial v\}\{\textbackslash{}partial y\} = 0
\textbackslash{}end\{cases\}\$\$

\end{itemize}

\item {} 
\sphinxAtStartPar
Semplificazione delle equazioni di NS per il problema considerato.
Vengono fatte le seguenti ipotesi:
\begin{itemize}
\item {} 
\sphinxAtStartPar
problema stazionario: \(\dfrac{\partial}{\partial t} = 0\);

\item {} 
\sphinxAtStartPar
direzione \(x\) omogenea (canale infinito in direzione \(x\)):
\(\dfrac{\partial u}{\partial x} = \dfrac{\partial v}{\partial x} = 0\);

\sphinxAtStartPar
non si può dire altrettanto della pressione, a causa del ruolo
che questa ha nelle equazioni di NS incomprimibili. Il campo di
pressione può essere interpretato come un moltiplicatore di
Lagrange necessario a imporre il vincolo di incomprimibilità.
Inoltre, ad eccezione di alcune condizioni al contorno, non
appare mai direttamente come pressione \(p\) ma solamente con le
sue derivate spaziali. Da un punto di vista più fisico, la
differenza di pressione lungo il canale è la forzante che mette
in moto il fluido in una corrente di Poiseuille.

\item {} 
\sphinxAtStartPar
la condizione \(\dfrac{\partial u}{\partial x} = 0\) inserita nel
vincolo di incomprimibilità, implica
\(\dfrac{\partial v}{\partial y}=0\); poichè
\(\dfrac{\partial v}{\partial x}=\dfrac{\partial v}{\partial y}=0\)
segue che \(v = \text{cost} = 0\), poiché è nulla a parete per la
condizione al contorno di adesione, \(\bm{u} = \bm{0}\).

\item {} 
\sphinxAtStartPar
no forze di volume: \(\bm{f} = 0\).

\end{itemize}

\sphinxAtStartPar
Le equazioni di NS possono essere semplificate \$\(\begin{cases}
  \nu \dfrac{\partial^2 u}{\partial y^2} = \dfrac{\partial p}{\partial x} \\
  \dfrac{\partial p}{\partial y} = 0  
\end{cases}\)\$

\sphinxAtStartPar
Dalla seconda segue che la pressione può essere funzione solo di
\(x\). Nella prima, il termine a sinistra dell’uguale è funzione solo
di \(y\); quello di destra può essere funzione solo di x:
l’uguaglianza implica che entrambi i membri sono costanti. Definiamo
questa costante come \(G_P = - \dfrac{\partial p}{\partial x}\): si
noti che questo è il «gradiente di pressione» lungo il canale,
cambiato di segno. \$\(\begin{cases}
    - \mu u''(y) = G_P & y \in[0,H] \\
    u(0) = 0  \\ u(H) = U
  \end{cases}\)\$

\item {} 
\sphinxAtStartPar
Soluzione dell’equazione differenziale con dati al contorno: si
integra due volte e si impongono le condizioni al contorno per
ottenere la componente \(u\) del campo di velocità.
\$\(\Rightarrow u(y) = -\dfrac{G_P}{2 \mu} y^2 + \left( \dfrac{G_P}{2 \mu}H
    + \dfrac{U}{H} \right) y \ .\)\$

\item {} 
\sphinxAtStartPar
Calcolo della portata come integrale della velocità.
\$\(Q = \int_{0}^{H} u(y) dy = \dfrac{G_P}{12 \mu} H^3 + \dfrac{1}{2} U H\)\(
Infine, imponendo la condizione di portata nulla \)Q=0\(, si ottiene
il valore di \)G\_P\(:
\)\(G_P = -6\dfrac{\mu U}{H^2} \qquad \Rightarrow \qquad G_P = - 930 Pa/m\)\$

\end{itemize}

\sphinxstepscope


\subsection{Exercise 6.2}
\label{\detokenize{polimi/fluidmechanics-ita/template/capitoli/06_slnEsatte/0607in:exercise-6-2}}\label{\detokenize{polimi/fluidmechanics-ita/template/capitoli/06_slnEsatte/0607in:fluid-mechanics-exact-solutions-ex02}}\label{\detokenize{polimi/fluidmechanics-ita/template/capitoli/06_slnEsatte/0607in::doc}}
\sphinxAtStartPar
+:———————————:+:———————————:+
| Una corrente di Poiseuille di     |                                   |
| acqua (\(\rho = 1000 / kg/m^3\),    |                                   |
| \(\mu =                            |                                   |
|  10^{-3} \ kg/(m s)\)) scorre in   |                                   |
| un canale di altezza              |                                   |
| \(H = 1 \ cm\). Un manometro misura |                                   |
| la differenza di pressione tra le |                                   |
| sezioni in \(x_A = 1.0 \ m\) e      |                                   |
| \(x_B= 2.0 \ m\). Determinare:      |                                   |
|                                   |                                   |
| \sphinxhyphen{}   il gradiente di pressione     |                                   |
|     all’interno del condotto,     |                                   |
|     conoscendo la densità del     |                                   |
|     liquido barometrico           |                                   |
|     \(\bar{\rho} = 1200 \ kg/m^3\)  |                                   |
|     e la differenza di quote      |                                   |
|     \(h = 5 \ mm\);                 |                                   |
|                                   |                                   |
| \sphinxhyphen{}   la velocità massima           |                                   |
|     all’interno del canale;       |                                   |
|                                   |                                   |
| \sphinxhyphen{}   la risultante \(\bm{R}\) delle  |                                   |
|     forze esercitata dal fluido   |                                   |
|     sul tratto di parete          |                                   |
|     superiore compreso tra A e B, |                                   |
|     sapendo che sulla sezione     |                                   |
|     \(x = 0 \ m\) la pressione vale |                                   |
|     \(p_0 = 10^5 \ Pa\). Qual è la  |                                   |
|     relazione tra \(R_x\) e         |                                   |
|     \(p_A - p_B\)? Commento.        |                                   |
|                                   |                                   |
| !{[}image{]}(./fig/manometro\_Poiseuil |                                   |
| le.eps)\{width=»90\%»\}              |                                   |
+———————————–+———————————–+

\sphinxAtStartPar
Soluzione esatte delle equazioni di Navier\sphinxhyphen{}Stokes. Corrente di
Poiseuille nel canale piano 2D. Manometro: leggi della statica
(Stevino).
\begin{itemize}
\item {} 
\sphinxAtStartPar
Per trovare la derivata in direzione \(x\) della pressione all’interno
del canale (\(\partial P/\partial x = - G_P = cost.\) per la corrente
di Poiseuille) risolve il problema di statica all’interno del
manometro. Facendo riferimento al disegno, si utilizza Stevino tra i
punti \(A'-Q_1\), \(Q_1-Q_2\), \(Q_2-B'\) e l’informazione di derivata
della pressione costante in direzione \(x\) all’interno del canale,
tra \(A'\) e \(B'\). \$\(\begin{cases}
  p_{A'} = p_{Q_1} - \rho g z_{Q_1} \\
  p_{Q_1} - \bar{\rho} g z_{Q_1} =   p_{Q_2} - \bar{\rho} g z_{Q_2} \\
  p_{B'} = p_{Q_2} - \rho g z_{Q_2} \\
  p_{A'} - p_{B'} = G_P \Delta x
 \end{cases} \Rightarrow
  G_P = \dfrac{1}{\Delta x}(\bar{\rho}-\rho) g \Delta h\)\( avendo
svolto correttamente i conti e riconosciuto \)z\_\{Q\_2\} \sphinxhyphen{} z\_\{Q\_1\} =
\textbackslash{}Delta h\$.

\item {} 
\sphinxAtStartPar
Ricordando che il profilo di velocità di Poiseuille risulta
\(\bm{u} = \bm{\hat{x}} u(y)\), con
\$\(u(y) = -\dfrac{G_P}{2 \mu} y (y-H),\)\( la velocità massima
all'interno del canale è \)V = u(H/2) =
\textbackslash{}dfrac\{G\_P\}\{8 \textbackslash{}mu\} H\textasciicircum{}2\$

\item {} 
\sphinxAtStartPar
Per calcolare la risultante degli sforzi sul tratto \(A-B\) della
parete superiore, è necessario calcolare il vettore sforzo agente su
di essa e svolgere un semplice integrale. Il vettore sforzo agente
sulla parete superiore risulta
\$\(\bm{t} = - \mu \dfrac{\partial u}{\partial y}\bigg|_{y=H} \bm{\hat{x}} +
           p(x,H) \bm{\hat{y}} \ .\)\( La pressione \)p(x,H)\( sulla
parete superiore, per \)x \textbackslash{}in {[}x\_A,x\_B{]}\( si calcola come segue: si
parte dall'origine del sistema di riferimento \)O\(, in corrispondenza
della quale è noto il valore della pressione \)p\_0\( e ci si muove in
orizzontale ricordando che \)\textbackslash{}partial P/\textbackslash{}partial x = \sphinxhyphen{}G\_P\( e in
verticale ricordando che \)\textbackslash{}partial P/\textbackslash{}partial y = \sphinxhyphen{}\textbackslash{}rho g\(.
\)\$\textbackslash{}begin\{aligned\}
p\_\{A”\} \& = p\_0 \sphinxhyphen{} G\_P x\_A \textbackslash{}
p\_\{A \} \& = p\_\{A”\} \sphinxhyphen{} \textbackslash{}rho g H  \textbackslash{}

\sphinxAtStartPar
\textbackslash{}end\{aligned\}
\textbackslash{}qquad \textbackslash{}rightarrow  \textbackslash{}qquad p(x,H)  = p\_\{A\} \sphinxhyphen{} G\_P ( x  \sphinxhyphen{}x\_A )\$\( Lo
sforzo tangenziale sulla parete è costante e vale
\)\(- \mu \dfrac{\partial u}{\partial y}\bigg|_{y=H} =
    \dfrac{G_P}{2} H\)\( La risultante delle forze (per unità di
lunghezza, poichè il problema è bidimensionale) si ottiene
integrando lo sforzo tra \)A\( e \)B\(. Facendo comparire il valore
\)p\_B\( della pressione in \)B\(, l'espressione della risultante delle
forze diventa \)\(\bm{R} = \dfrac{G_P}{2} H \Delta x \bm{\hat{x}} + 
           \dfrac{1}{2}(p_A + p_B) \Delta x \bm{\hat{y}} \ .\)\$

\end{itemize}

\sphinxstepscope


\subsection{Exercise 6.3}
\label{\detokenize{polimi/fluidmechanics-ita/template/capitoli/06_slnEsatte/0609in:exercise-6-3}}\label{\detokenize{polimi/fluidmechanics-ita/template/capitoli/06_slnEsatte/0609in:fluid-mechanics-exact-solutions-ex03}}\label{\detokenize{polimi/fluidmechanics-ita/template/capitoli/06_slnEsatte/0609in::doc}}
\sphinxAtStartPar
+:———————————:+:———————————:+
| Si consideri una corrente d’acqua |                                   |
| a pelo libero, laminare e         |                                   |
| stazionaria, che scorre su una    |                                   |
| parete piana di lunghezza e       |                                   |
| apertura infinita inclinata di un |                                   |
| angolo \(\alpha\) rispetto          |                                   |
| all’orizzontale. Sul pelo libero  |                                   |
| la pressione è uniforme e uguale  |                                   |
| a \(P_a\). Lo sforzo tangenziale    |                                   |
| fra acqua e aria viene            |                                   |
| considerato nullo.                |                                   |
|                                   |                                   |
| Si calcoli il profilo di velocità |                                   |
| nello strato di acqua e il campo  |                                   |
| di pressione.                     |                                   |
|                                   |                                   |
| \sphinxincludegraphics{{polimi/fluidmechanics-ita/template/capitoli/06_slnEsatte/fig/slnEsatte-scivolo}} |                                   |
| \{width=»90\%»\}                     |                                   |
+———————————–+———————————–+

\sphinxAtStartPar
Semplificazione delle equazioni di NS in casi particolari. Soluzioni
esatte in coordinate cartesiane.

\sphinxAtStartPar
Si scelga un sistema di riferimento cartesiano con l’asse \(x\) orientato
lungo la parete verso il basso e l’asse y perpendicolare ed uscente ad
essa. Sulla corrente di questo problema agiscono le forze di volume
dovute alla gravità. L’ipotesi che la pressione sia uniforme sulla
superficie di interfaccia tra acqua e aria implica che la pressione è
indipendente dalla coordinata \(x\) in tutto il fluido.
\begin{itemize}
\item {} 
\sphinxAtStartPar
Scrittura delle equazioni di NS in coordinate cartesiane in 2
dimensioni. \$\$\textbackslash{}begin\{cases\}
\textbackslash{}dfrac\{\textbackslash{}partial u\}\{\textbackslash{}partial t\} + u \textbackslash{}dfrac\{\textbackslash{}partial u\}\{\textbackslash{}partial x\}
\begin{itemize}
\item {} 
\sphinxAtStartPar
v \textbackslash{}dfrac\{\textbackslash{}partial u\}\{\textbackslash{}partial y\} \sphinxhyphen{} \textbackslash{}nu \textbackslash{}left(
\textbackslash{}dfrac\{\textbackslash{}partial\textasciicircum{}2 u\}\{\textbackslash{}partial x\textasciicircum{}2\} +
\textbackslash{}dfrac\{\textbackslash{}partial\textasciicircum{}2 u\}\{\textbackslash{}partial y\textasciicircum{}2\} \textbackslash{}right)

\item {} 
\sphinxAtStartPar
\textbackslash{}dfrac\{1\}\{\textbackslash{}rho\} \textbackslash{}dfrac\{\textbackslash{}partial p\}\{\textbackslash{}partial x\} = f\_x \textbackslash{}
\textbackslash{}dfrac\{\textbackslash{}partial v\}\{\textbackslash{}partial t\} + u \textbackslash{}dfrac\{\textbackslash{}partial v\}\{\textbackslash{}partial x\}

\item {} 
\sphinxAtStartPar
v \textbackslash{}dfrac\{\textbackslash{}partial v\}\{\textbackslash{}partial y\} \sphinxhyphen{} \textbackslash{}nu \textbackslash{}left(
\textbackslash{}dfrac\{\textbackslash{}partial\textasciicircum{}2 v\}\{\textbackslash{}partial x\textasciicircum{}2\} +
\textbackslash{}dfrac\{\textbackslash{}partial\textasciicircum{}2 v\}\{\textbackslash{}partial y\textasciicircum{}2\} \textbackslash{}right)

\item {} 
\sphinxAtStartPar
\textbackslash{}dfrac\{1\}\{\textbackslash{}rho\} \textbackslash{}dfrac\{\textbackslash{}partial p\}\{\textbackslash{}partial y\} = f\_y \textbackslash{}
\textbackslash{}dfrac\{\textbackslash{}partial u\}\{\textbackslash{}partial x\} + \textbackslash{}dfrac\{\textbackslash{}partial v\}\{\textbackslash{}partial y\} = 0
\textbackslash{}end\{cases\}\$\$

\end{itemize}

\item {} 
\sphinxAtStartPar
Semplificazione delle equazioni di NS per il problema considerato.
Vengono fatte le seguenti ipotesi:
\begin{itemize}
\item {} 
\sphinxAtStartPar
problema stazionario: \(\dfrac{\partial}{\partial t} = 0\);

\item {} 
\sphinxAtStartPar
direzione \(x\) omogenea (canale infinito in direzione \(x\)):
\(\dfrac{\partial u}{\partial x} = \dfrac{\partial v}{\partial x} = 0\);

\sphinxAtStartPar
non si può dire altrettanto della pressione, a causa del ruolo
che questa ha nelle equazioni di NS incomprimibili. Il campo di
pressione può essere interpretato come un moltiplicatore di
Lagrange necessario a imporre il vincolo di incomprimibilità.
Inoltre, ad eccezione di alcune condizioni al contorno, non
appare mai direttamente come pressione \(p\) ma solamente con le
sue derivate spaziali. Da un punto di vista più fisico, la
differenza di pressione lungo il canale è la forzante che mette
in moto il fluido in una corrente di Poiseuille.

\item {} 
\sphinxAtStartPar
la condizione \(\dfrac{\partial u}{\partial x} = 0\) inserita nel
vincolo di incomprimibilità, implica
\(\dfrac{\partial v}{\partial y}=0\); poichè
\(\dfrac{\partial v}{\partial x}=\dfrac{\partial v}{\partial y}=0\)
segue che \(v = \text{cost} = 0\), poiché è nulla a parete per la
condizione al contorno di adesione, \(\bm{u} = \bm{0}\).

\item {} 
\sphinxAtStartPar
no forze di volume:
\(\bm{f} = \rho \bm{g} = \rho g \sin \alpha \bm{\hat{x}} - \rho g \cos \alpha \bm{\hat{y}}\).

\end{itemize}

\sphinxAtStartPar
Le equazioni di NS possono essere semplificate \$\$\textbackslash{}begin\{cases\}
\begin{itemize}
\item {} 
\sphinxAtStartPar
\textbackslash{}mu \textbackslash{}dfrac\{\textbackslash{}partial\textasciicircum{}2 u\}\{\textbackslash{}partial y\textasciicircum{}2\} = \sphinxhyphen{} \textbackslash{}dfrac\{\textbackslash{}partial p\}\{\textbackslash{}partial x\} + \textbackslash{}rho g \textbackslash{}sin \textbackslash{}alpha \textbackslash{}
\textbackslash{}dfrac\{\textbackslash{}partial p\}\{\textbackslash{}partial y\} = \sphinxhyphen{} \textbackslash{}rho g \textbackslash{}cos \textbackslash{}alpha \textbackslash{} .
\textbackslash{}end\{cases\}\$\$

\end{itemize}

\sphinxAtStartPar
Dalla seconda segue che l’espressione del campo di pressione è
\$\(p(x,y) = -\rho g y \cos \alpha + f(x) \ .\)\( L'espressione di
\)f(x)\( può essere calcolata imponendo la condizione al contorno sul
pelo libero, \)p(x,H) = P\_a\(,
\)\(P_a = -\rho g H \cos \alpha + f(x) \qquad \rightarrow \qquad f(x) = P_a + \rho g H \cos \alpha \ .\)\(
La funzione \)f(x)\( è costante, senza dipendere dalla coordinata \)x\(.
Di conseguenza, il campo di pressione dipende solo dalla coordinata
\)y\( \)\(p(x,y) = P_a + \rho g ( H - y ) \cos \alpha \ ,\)\( e la
derivata di \)\textbackslash{}partial p / \textbackslash{}partial x\( è nulla. La componente \)x\(
dell'equazione della quantità di moto diventa quindi un'equazione
ordinaria del secondo ordine \)\(\begin{cases}
    - \mu u''(y) = \rho g  \sin \alpha  \ , \ y \in[0,H] \\
    u(0) = 0  \\ u'(H) = 0 \ ,
  \end{cases}\)\( con le condizioni al contorno di adesione a parete e
di sforzo di taglio nullo all'interfaccia tra aria ed acqua,
\)0=\textbackslash{}tau(H)=\textbackslash{}mu \textbackslash{}dfrac\{\textbackslash{}partial u\}\{\textbackslash{}partial y\}(H)=\textbackslash{}mu u”(H)\(. La
derivata parziale in \)y\( è stata sostituita da quella ordinaria,
poichè la velocità è solo funzione di \)y\$.

\item {} 
\sphinxAtStartPar
Soluzione dell’equazione differenziale con dati al contorno: si
integra due volte e si impongono le condizioni al contorno per
ottenere la componente \(u\) del campo di velocità.
\$\(u(y) = - \dfrac{\rho g}{2 \mu} y( y - H ) \sin \alpha \ .\)\$

\end{itemize}

\sphinxstepscope


\subsection{Exercise 6.4}
\label{\detokenize{polimi/fluidmechanics-ita/template/capitoli/06_slnEsatte/0603in:exercise-6-4}}\label{\detokenize{polimi/fluidmechanics-ita/template/capitoli/06_slnEsatte/0603in:fluid-mechanics-exact-solutions-ex04}}\label{\detokenize{polimi/fluidmechanics-ita/template/capitoli/06_slnEsatte/0603in::doc}}
\sphinxAtStartPar
+:———————————:+:———————————:+
| Si consideri una corrente d’acqua |                                   |
| a pelo libero, laminare e         |                                   |
| stazionaria, che scorre su una    |                                   |
| parete verticale piana di         |                                   |
| lunghezza e apertura infinita. Si |                                   |
| ipotizzi che la pressione         |                                   |
| atmosferica che agisce sul pelo   |                                   |
| libero sia uniforme. Si ipotizzi  |                                   |
| inoltre che lo sforzo tangenziale |                                   |
| fra acqua e aria in               |                                   |
| corrispondenza del pelo libero    |                                   |
| sia nullo.                        |                                   |
|                                   |                                   |
| Assegnata la portata in massa per |                                   |
| unità di apertura                 |                                   |
| \(\overline{Q}=0.5\ kg/(ms)\),      |                                   |
| determinare                       |                                   |
|                                   |                                   |
| 1.  lo spessore \(h\) della         |                                   |
|     corrente d’acqua;             |                                   |
|                                   |                                   |
| 2.  lo sforzo tangenziale a       |                                   |
|     parete;                       |                                   |
|                                   |                                   |
| 3.  la velocità in corrispondenza |                                   |
|     del pelo libero;              |                                   |
|                                   |                                   |
| 4.  la velocità media e il numero |                                   |
|     di Reynolds basato su tale    |                                   |
|     velocità media e sullo        |                                   |
|     spessore della corrente.      |                                   |
|                                   |                                   |
| Si sostituisca poi al pelo libero |                                   |
| una parete solida. Si determini   |                                   |
| quale dovrebbe essere la velocità |                                   |
| di tale parete per ottenere una   |                                   |
| portata nulla.                    |                                   |
|                                   |                                   |
| Dati:                             |                                   |
| \(\overline{\rho}= 999\ kg/m^3\),   |                                   |
| \(\overline{\mu}= 1.15\ 10^{-3} kg |                                   |
| /(ms)\).                           |                                   |
|                                   |                                   |
| (\(h=5.61\, 10^{-4}\  m\),          |                                   |
| \(\tau = 5.494\ Pa\),               |                                   |
| \(u(h)=1.339\ m/s\),                |                                   |
| \(\overline{U}=0.893\  m/s\),       |                                   |
| \(Re=434.8\), \(U=-0.4464\ m/s\).)    |                                   |
+———————————–+———————————–+

\sphinxAtStartPar
Semplificazione delle equazioni di NS in casi particolari. Soluzioni
esatte in coordinate cartesiane.

\sphinxAtStartPar
Si scelga un sistema di riferimento cartesiano con l’asse x orientato
lungo la parete verso il basso e l’asse y perpendicolare ed uscente ad
essa.

\sphinxAtStartPar
Sulla corrente di questo problema agisce la forza di volume dovuta alla
gravità.

\sphinxAtStartPar
L’ipotesi che la pressione sia uniforme sulla superficie di interfaccia
tra acqua e aria implica che la pressione è costante in tutto il fluido:
si vedrà che \(\frac{\partial p}{\partial y}=0\); se sulla superficie
libera la pressione è costante e non varia nello spessore, allora la
pressione è costante in tutto il fluido.
\begin{itemize}
\item {} 
\sphinxAtStartPar
Scrittura delle equazioni di NS in 2 dimensioni.
\begin{equation*}
\begin{split}\begin{cases}
      \frac{\partial u}{\partial t} + u \frac{\partial u}{\partial x}
      + v \frac{\partial u}{\partial y} - \nu \left( 
      \frac{\partial^2 u}{\partial x^2} +
      \frac{\partial^2 u}{\partial y^2} \right)
       + \frac{1}{\rho} \frac{\partial p}{\partial x} = f_x \\
      \frac{\partial v}{\partial t} + u \frac{\partial v}{\partial x}
      + v \frac{\partial v}{\partial y} - \nu \left( 
      \frac{\partial^2 v}{\partial x^2} +
      \frac{\partial^2 v}{\partial y^2} \right)
      + \frac{1}{\rho}  \frac{\partial p}{\partial y} = f_y \\
      \frac{\partial u}{\partial x} + \frac{\partial v}{\partial y} = 0
    \end{cases}\end{split}
\end{equation*}
\item {} 
\sphinxAtStartPar
Semplificazione delle equazioni di NS per il problema da affrontare.

\sphinxAtStartPar
Ipotesi:
\begin{itemize}
\item {} 
\sphinxAtStartPar
problema stazionario: \(\frac{\partial}{\partial t} = 0\);

\item {} 
\sphinxAtStartPar
direzione x omogenea (canale infinito in direzione x):
\(\frac{\partial u}{\partial x} = \frac{\partial v}{\partial x} = 0\);
la pressione nelle equazioni di NS incomprimibili è un
moltiplicatore di Lagrange per imporre il vincolo di
incomprimibilità; inoltre non appare mai, se non nelle
condizioni al contorno, come \(p\) ma solo con le sue derivate
spaziali: quindi non è corretto imporre
\(= \frac{\partial v}{\partial x} = 0\), nonostante la direzione
\(x\) sia omogenea;

\item {} 
\sphinxAtStartPar
\(\frac{\partial u}{\partial x} = 0\) inserito nel vincolo di
incomprimibilità
(\(\frac{\partial u}{\partial x}+\frac{\partial v}{\partial y}=0\))
implica \(\frac{\partial v}{\partial y}=0\); poichè
\(\frac{\partial v}{\partial x}=\frac{\partial v}{\partial y}=0\)
e \(v = 0\) a parete per la condizione al contorno di adesione,
segue che \(v = \text{cost} = 0\);

\item {} 
\sphinxAtStartPar
forze di volume solo in direzione verticale: per come sono stati
orientati gli assi, \(\bm{f} = g \hat{\bm{x}}\).

\end{itemize}
\begin{equation*}
\begin{split}\begin{cases}
      - \mu \frac{\partial^2 u}{\partial y^2} = - \frac{\partial p}{\partial x} + \rho g\\
      \frac{\partial p}{\partial y} = 0  
    \end{cases}\end{split}
\end{equation*}
\sphinxAtStartPar
Dalla seconda segue che la pressione può essere funzione solo di
\(x\). Come già detto in precedenza, la pressione sulla superficie
libera è costante e uguale alla pressione ambiente \(P_a\): se la
pressione non può variare nello spessore, allora è costante ovunque.
La derivata parziale \(\frac{\partial p}
 {\partial x}=0\), il suo gradiente è nullo e quindi la pressione è
costante in tutta la corrente di acqua.

\sphinxAtStartPar
Nella prima, il termine a sinistra dell’uguale è funzione solo di
\(y\); quello di destra è costante e uguale a \(\rho g\). Le condizioni
al contorno sono di adesione a parete e di sforzo di taglio nullo
all’interfaccia tra aria ed acqua:
\(0=\tau(H)=\mu \frac{\partial u}{\partial y}(H)=\mu u'(H)\), dove la
derivata parziale in \(y\) è stata sostituita da quella ordinaria,
poichè la velocità è solo funzione di \(y\).
\begin{equation*}
\begin{split}\begin{cases}
        - \mu u''(y) = \rho g \ , \ y \in[0,H] \\
        u(0) = 0  \\ u'(H) = 0
      \end{cases}\end{split}
\end{equation*}
\item {} 
\sphinxAtStartPar
Soluzione dell’equazione differenziale (semplice) con dati al
contorno.

\sphinxAtStartPar
Risulta:
\$\(\Rightarrow u(y) = - \frac{\rho g}{2 \mu} y^2 + \frac{\rho g}{\mu} H y\)\$

\item {} 
\sphinxAtStartPar
Calcolo della portata come integrale della velocità; si trova così
la relazione tra Q ed H.
\begin{equation*}
\begin{split}Q = \int_{0}^{H} \rho u(y) dy = \frac{1}{3}\frac{\rho^2 g}{\mu} H^3\end{split}
\end{equation*}
\sphinxAtStartPar
E quindi
\$\(H = \left( \frac{3 Q \mu}{\rho^2 g} \right) ^ {\frac{1}{3}}
     \quad \Rightarrow \quad H = 5.61 \cdot 10^{-4} m\)\$

\item {} 
\sphinxAtStartPar
Calcolo dello sforzo a parete
\$\(\tau = \mu u'|_{y=0} = \rho g H \quad \Rightarrow \quad \tau = 5.494 Pa\)\$
\sphinxstyleemphasis{Osservazione.} Equilibrio con la forza di gravità (problema
stazionario).

\item {} 
\sphinxAtStartPar
Calcolo di \(u(H)\). \$\(u(H) = \frac{1}{2}\frac{\rho g}{\mu} H^2 
    \quad \Rightarrow \quad u(H) = 1.342 m/s\)\$

\item {} 
\sphinxAtStartPar
Calcolo velocità media e numero di Reynolds.
\$\(\bar{U} = \frac{1}{H}\int_{0}^{H} u(y) dy = \frac{Q}{\rho H}
    \quad \Rightarrow \quad \bar{U} = \frac{Q}{\rho H}
                                    = \frac{2}{3}u(H) = 0.895 m/s\)\$
\begin{equation*}
\begin{split}Re = \frac{\rho \bar{U} H}{\mu}
        \quad \Rightarrow \quad Re = 434.8\end{split}
\end{equation*}
\end{itemize}

\sphinxAtStartPar
L’ultima parte del problema chiede di sostiutire alla superficie libera,
una parete infinita. L’equazione trovata in precedenza è ancora valida;
è necessario però sostituire la condizione di sforzo tangenziale nullo
con adesione su una parete mobile con velocità costante \(U\).
\begin{equation*}
\begin{split}\begin{cases}
    - \mu u''(y) = \rho g \ , \ y \in[0,H] \\
    u(0) = 0  \\ u(H) = U
  \end{cases}\end{split}
\end{equation*}
\sphinxAtStartPar
Il profilo di velocità è:
\$\(u(y) = \dfrac{\rho g}{2 \mu}(-y^2 + yH) +\dfrac{U}{H}y\)\( dove la
velocità \)U\( è ancora incognita. Per trovarne il valore, si calcola la
portata e la si pone uguale a zero. La portata è uguale a
\)\$Q = \textbackslash{}int\_0\textasciicircum{}H u(y) dy = \textbackslash{}dots = \textbackslash{}dfrac\{1\}\{12\}\textbackslash{}dfrac\{\textbackslash{}rho g H\textasciicircum{}3\}\{\textbackslash{}mu\}
\begin{itemize}
\item {} 
\sphinxAtStartPar
\textbackslash{}dfrac\{1\}\{2\}UH\$\( Imponendo \)Q=0\(,
\)\(U = - \dfrac{\rho g H^2}{6 \mu} \quad \Rightarrow \quad
U = - 0.4474\ m/s\)\$

\end{itemize}

\sphinxstepscope


\subsection{Exercise 6.5}
\label{\detokenize{polimi/fluidmechanics-ita/template/capitoli/06_slnEsatte/0605in:exercise-6-5}}\label{\detokenize{polimi/fluidmechanics-ita/template/capitoli/06_slnEsatte/0605in:fluid-mechanics-exact-solutions-ex05}}\label{\detokenize{polimi/fluidmechanics-ita/template/capitoli/06_slnEsatte/0605in::doc}}
\sphinxAtStartPar
+:———————————:+:———————————:+
| Un manometro a mercurio           | !{[}image{]}(./fig/slnEsatte\sphinxhyphen{}poiseuil |
| (\(\rho_{hg} = 13610 \  kg/m^3\))   | le)\{width=»90\%»\}                  |
| collega due prese di pressione    |                                   |
| posizionate a una distanza di     |                                   |
| \(l = 2 \                          |                                   |
| m\) l’una dall’altra lungo un tubo |                                   |
| orizzontale di diametro           |                                   |
| \(2R = 5 \ cm\) in cui scorre un    |                                   |
| fluido con densità                |                                   |
| \(\rho_{f} = 950 \                 |                                   |
| kg/m^3\). Se la differenza fra le  |                                   |
| altezze dei peli liberi del       |                                   |
| liquido manometrico nelle due     |                                   |
| colonne vale \(\Delta h = 4 \ cm\)  |                                   |
| e la portata volumetrica che      |                                   |
| scorre nel tubo è \(Q= 6\ m^3/s\),  |                                   |
| quanto valgono la viscosità \(\mu\) |                                   |
| del fluido e lo sforzo a parete   |                                   |
| \(\tau_w\)?                         |                                   |
|                                   |                                   |
| (\(\mu = 6.36\,10^{-5} \ kg/(m\, s |                                   |
| )\),                               |                                   |
| \(\tau_w = 31.05\, \bm{z}\ N/m^2\)) |                                   |
+———————————–+———————————–+

\sphinxAtStartPar
Semplificazione delle equazioni di NS in casi particolari. Soluzioni
esatte in coordinate cilindriche. Legge di Stevino.

\sphinxAtStartPar
Scrittura del contributo viscoso del vettore sforzo come:
\$\(\begin{aligned}
  \bm{s_n} & = \mathbb{S} \cdot \bm{\hat{n}} = \\
           & = \mu [\bm{\nabla} \bm{u} + \bm{\nabla}^T \bm{u}] \cdot \bm{\hat{n}} = \\
           & = \mu \left[ 2 (\bm{\hat{n}} \cdot \bm{\nabla} ) \bm{u} + \bm{\hat{n}} \times \bm{\nabla} \times \bm{u}  \right]
\end{aligned}\)\$

\sphinxAtStartPar
La geometria del problema suggerisce di utilizzare un sistema di
coordiante cilindriche.
\begin{itemize}
\item {} 
\sphinxAtStartPar
Scrittura delle equazioni di NS in coordinate cilindriche
\$\(\begin{cases}
    \rho \dfrac{\partial u_r}{\partial t}
    + \rho \left( \bm{u} \cdot \bm{\nabla}u_r - \dfrac{u_\theta^2}{r} \right)
    - \mu \left(\nabla^2 u_r 
       - \dfrac{u_r}{r^2} 
       - \dfrac{2}{r^2}\dfrac{\partial u_\theta}{\partial \theta} \right)  
       + \dfrac{\partial p}{\partial r} = f_r \\
    \rho \dfrac{\partial u_\theta}{\partial t}
    + \rho \left( \bm{u} \cdot \bm{\nabla} u_\theta + \dfrac{u_\theta u_r}{r} \right)
    - \mu \left(\nabla^2 u_\theta 
       - \dfrac{u_\theta}{r^2} 
       + \dfrac{2}{r^2}\dfrac{\partial u_r}{\partial \theta}  \right) 
    + \dfrac{1}{r} \dfrac{\partial p}{\partial \theta} = f_\theta\\
    \rho \dfrac{\partial u_z}{\partial t}
    + \rho \bm{u} \cdot \bm{\nabla} u_z
    - \mu \nabla^2 u_z
    + \dfrac{\partial p}{\partial z} = f_z \\ \\
    \dfrac{1}{r}\dfrac{\partial}{\partial r}\left( r u_r \right) 
    + \dfrac{1}{r}\dfrac{\partial u_\theta}{\partial \theta} 
    + \dfrac{\partial u_z}{\partial z} = 0
  \end{cases}\)\( con \)\(\begin{aligned}
  & \bm{a} \cdot \bm{\nabla} b = a_r \dfrac{\partial b}{\partial r} 
     + \dfrac{a_\theta}{r} \dfrac{\partial b}{\partial \theta}  
     + a_z \dfrac{\partial b}{\partial z} \\
  & \nabla^2 f = \dfrac{1}{r}\dfrac{\partial}{\partial r}
                      \left(r \dfrac{\partial f}{\partial r} \right) +
               \dfrac{1}{r^2} \dfrac{\partial^2 f}{\partial \theta^2} + 
               \dfrac{\partial^2 f}{\partial z^2} 
  \end{aligned}\)\$

\item {} 
\sphinxAtStartPar
Semplificazione delle equazioni di NS per il problema considerato.
Vengono fatte le sequenti ipotesi:
\begin{itemize}
\item {} 
\sphinxAtStartPar
problema stazionario: \(\dfrac{\partial}{\partial t} = 0\);

\item {} 
\sphinxAtStartPar
direzione z omogenea (canale infinito in direzione z):
\(\dfrac{\partial u}{\partial z} = \dfrac{\partial v}{\partial z} = 0\);
come discusso negli esercizi in geometria cartesiana, il termine
\(\dfrac{\partial P}{\partial z} = - G_P\) è costante e in
generale diverso da zero.

\item {} 
\sphinxAtStartPar
problema assialsimmetrico:
\(\dfrac{\partial}{\partial \theta} = 0\);

\item {} 
\sphinxAtStartPar
no swirl: \(u_{\theta} = 0\);

\item {} 
\sphinxAtStartPar
dall’incomprimibilità e dalle condizioni al contorno a parete,
segue che la componente radiale della velocità è identicamente
nulla, \(u_r = 0\);

\item {} 
\sphinxAtStartPar
no forze di volume: \(\bm{f} = 0\).

\end{itemize}

\sphinxAtStartPar
Grazie alle ipotesi fatte, il campo di velocità assume la forma
\(\bm{u}(\bm{r}) = u(r) \bm{\hat{z}}\). La componente radiale e
azimuthale dell’equazione della quantità di moto sono identicamente
soddisfatte, mentre la componente lungo \(z\) diventa \$\(\begin{cases}
     \mu \dfrac{1}{r} \dfrac{d}{dr}\left( r \dfrac{d}{dr} u(r)\right) = -G_P & r \in[0,R] \\
    u(0) = \)valore finito\(  \\ u(R) = 0
  \end{cases}\)\( dove la derivata ordinaria \)\textbackslash{}dfrac\{d\}\{d r\}\( è stata
utilizzata al posto della derivta parziale, poichè la componente
assiale della velocità dipende solamente dalla coordinata radiale,
\)u(r)\$. Le condizioni al contorno garantiscono che il campo di
velocità sia regolare nel dominio (in particolare che non esistano
singolarità sull’asse) e che sia soddisfatta la condizione al
contorno di adesione a parete.

\item {} 
\sphinxAtStartPar
Soluzione dell’equazione differenziale. Si integra due volte e si
ottiene: \$\(u(r) = -\dfrac{G_P}{4 \mu} r^2 + A \ln{r} + B\)\( Imponendo
le condizioni al contorno, \)A\( deve essere nullo per l'ipotesi di
valore finito in \)r=0\( (\)\textbackslash{}ln r \textbackslash{}rightarrow \sphinxhyphen{}\textbackslash{}infty\( quando
\)r \textbackslash{}rightarrow 0\(). Imponendo poi la condizione di adesione a parete
per \)r=R\(, si ottiene:
\)\(u(r) = -\dfrac{G_P}{4 \mu} (r^2 - R^2) \ .\)\$

\item {} 
\sphinxAtStartPar
Calcolo della portata: si integra la velocità sulla sezione
circolare (\sphinxstylestrong{!}) del tubo. Questa relazione lega il gradiente di
pressione \(G_P\) alla portata \(Q\) e al coefficiente di viscosità
dimanica \(\mu\),
\$\(Q = \int_{\theta=0}^{2\pi} \int_{r=0}^{R} u(r) r dr d\theta = 
  2\pi \int_{r=0}^{R} u(r) r dr =
  \dfrac{\pi}{8}\dfrac{G_P R^4}{\mu} \ .\)\$

\sphinxAtStartPar
La differenza di pressione tra i due punti A e B (separati da una
distanza \(l\)) è quindi \(P_B - P_A = -G_P l\).

\item {} 
\sphinxAtStartPar
Applicazione della legge di Stevino per ottenere il sistema
risolvente: \$\(\begin{cases}
  P_1 = P_A + \rho_f g H_0 & \text{(Stevino tra 1 e A)} \\
  P_2 = P_B + \rho_f g (H_0-\Delta h) & \text{(Stevino tra 2 e B)} \\
  P_B = P_A - G_P l & \text{(relazione trovata dalla sln di NS)} \\
  P_2 = P_1 - \rho_{Hg} g \Delta h & \text{(Stevino tra 1 e 2)}
\end{cases}\)\$

\sphinxAtStartPar
Risolvendo il sistema, si trova che:
\$\(G_P l = (\rho_{Hg} - \rho_f) g \Delta h\)\$

\sphinxAtStartPar
Esplicitando il legame tra \(G_P\) e \(\mu\), si ottiene il risultato:
\$\(\Rightarrow \quad \mu = \dfrac{\pi R^4}{8 Q l} (\rho_{Hg} - \rho_f) g \Delta h \quad \Rightarrow \quad \mu = 6.36 \cdot 10^{-5} \dfrac{kg}{m s}\)\$

\item {} 
\sphinxAtStartPar
Bisogna calcolare ora \(\tau_w\),la componente parallela alla parete
dello sforzo a parete. Usando l’espressione vettoriale della parte
viscosa del vettore sforzo agente sul fluido (aiutandosi con le
tabelle per le espressioni in coordinate cilindriche degli operatori
differenziali) con \(\bm{u} = u_z(r)\bm{\hat{z}}\) e
\(\bm{\hat{n}} = \bm{\hat{r}}\), si può scrivere \$\(\begin{aligned}
   \bm{s_n} & = \mu \left[ 2 (\bm{\hat{n}} \cdot \bm{\nabla} ) \bm{u} + \bm{\hat{n}} \times \bm{\nabla} \times \bm{u}  \right] = \\
            & = \mu \left[ 2 \dfrac{\partial u_z}{\partial r}\bm{\hat{z}} - \dfrac{\partial u_z}{\partial r} \bm{\hat{z}} \right] = \\
            & = \mu \dfrac{\partial u_z}{\partial r}\bm{\hat{z}} \ .
\end{aligned}\)\( Ricordando che lo sforzo agente sulla parete è
uguale e contrario a quello agente sul fluido e che lo sforzo dovuto
alla pressione è normale alla parete, \)\(\begin{aligned}
   \tau_w & = - \mu \dfrac{\partial u_z}{\partial r}\bigg|_{r=R} = \\
          & = \dfrac{1}{2} G_P R \ .
\end{aligned}\)\( Si ottiene quindi il valore, \)\textbackslash{}tau\_w = 31.05 N/m\textasciicircum{}2\$.

\end{itemize}

\sphinxAtStartPar
L’espressione dello sforzo tangenziale a parete
\(\tau_w = - \mu \frac{\partial u_z}{\partial r}\) per la corrente di
Poiseuille in un tubo a sezione circolare è simile a quella ottenuta per
la corrente in un canale piano, in coordinate cartesiane,
\(\tau_w = \mu \frac{\partial u}{\partial y}\). In questi due casi, la
componente tangenziale dello sforzo è proporzionale alla derivata in
direzione perpendicolare alla parete della componente di velocità
parallela alla parete. Questa \sphinxstylestrong{NON} è una formula generale per lo
sforzo tangenziale a parete, come sarà evidente nel caso della corrente
di Taylor\sphinxhyphen{}Couette.

\sphinxstepscope


\subsection{Exercise 6.6}
\label{\detokenize{polimi/fluidmechanics-ita/template/capitoli/06_slnEsatte/0601in:exercise-6-6}}\label{\detokenize{polimi/fluidmechanics-ita/template/capitoli/06_slnEsatte/0601in:fluid-mechanics-exact-solutions-ex06}}\label{\detokenize{polimi/fluidmechanics-ita/template/capitoli/06_slnEsatte/0601in::doc}}
\sphinxAtStartPar
+:———————————:+:———————————:+
| Si consideri la corrente piana    | !{[}image{]}(./fig/slnEsatte\sphinxhyphen{}taylor\sphinxhyphen{}c |
| fra due cilindri coassiali        | ouette\sphinxhyphen{}2)\{width=»90\%»\}            |
| rotanti. Si misura la velocità in |                                   |
| due punti posti rispettivamente a |                                   |
| \(1/4\) e \(3/4\) del gap fra i due   |                                   |
| cilindri:                         |                                   |
| \(u_{\theta,1/4} = 0.5\ m/s\),      |                                   |
| \(u_{\theta,3/4} = 0.8\ m/s\). Si   |                                   |
| determini la velocità di          |                                   |
| rotazione dei due cilindri nonché |                                   |
| la pressione in corrispondenza    |                                   |
| del cilindro interno sapendo che  |                                   |
| la pressione in corrispondenza    |                                   |
| del cilindro esterno vale         |                                   |
| \(5\ Pa\), che la densità del       |                                   |
| fluido è pari a \(1.225\ kg/m^3\),  |                                   |
| che il diametro del cilindro      |                                   |
| interno è \(d =2 R_1=0.1 \ m\) e    |                                   |
| che il diametro del cilindro      |                                   |
| esterno è \(D = 2 R_2 = 0.16 \ m\). |                                   |
|                                   |                                   |
| (\(\Omega_{1}=6.663\ s^{-1}\),      |                                   |
| \(\Omega_{2}=11.743\ s^{-1}\)       |                                   |
| \(P(r) = P_2 - \rho \left[ \dfrac{ |                                   |
| 1}{2} A^2 (R_2^2 - r^2) + 2 A B l |                                   |
| n \dfrac{R_2}{r} -                |                                   |
|       \dfrac{1}{2}B^2 \left( \dfr |                                   |
| ac{1}{R^2} - \dfrac{1}{r^2} \righ |                                   |
| t)  \right]\),                     |                                   |
| con \(u_{\theta}(r) = A r + B/r\) . |                                   |
| )                                 |                                   |
+———————————–+———————————–+

\sphinxAtStartPar
Soluzione esatte delle equazioni di Navier\sphinxhyphen{}Stokes in geometria
cilindrica. Corrente di Taylor\sphinxhyphen{}Couette.
\begin{equation*}
\begin{split}\begin{cases}
    \rho \dfrac{\partial u_r}{\partial t}
    + \rho \left( \bm{u} \cdot \bm{\nabla}u_r - \dfrac{u_\theta^2}{r} \right)
    - \mu \left(\nabla^2 u_r 
       - \dfrac{u_r}{r^2} 
       - \dfrac{2}{r^2}\dfrac{\partial u_\theta}{\partial \theta} \right)  
       + \dfrac{\partial p}{\partial r} = f_r \\
    \rho \dfrac{\partial u_\theta}{\partial t}
    + \rho \left( \bm{u} \cdot \bm{\nabla} u_\theta + \dfrac{u_\theta u_r}{r} \right)
    - \mu \left(\nabla^2 u_\theta 
       - \dfrac{u_\theta}{r^2} 
       + \frac{2}{r^2}\dfrac{\partial u_r}{\partial \theta}  \right) 
    + \dfrac{1}{r} \frac{\partial p}{\partial \theta} = f_\theta\\
    \rho \dfrac{\partial u_z}{\partial t}
    + \rho \bm{u} \cdot \bm{\nabla} u_z
    - \mu \nabla^2 u_z
    + \dfrac{\partial p}{\partial z} = f_z \\ \\
    \dfrac{1}{r}\dfrac{\partial}{\partial r}\left( r u_r \right) 
    + \dfrac{1}{r}\dfrac{\partial u_\theta}{\partial \theta} 
    + \dfrac{\partial u_z}{\partial z} = 0
  \end{cases}$$ con $$\begin{aligned}
  & \bm{a} \cdot \bm{\nabla} b = a_r \dfrac{\partial b}{\partial r} 
     + \dfrac{a_\theta}{r} \dfrac{\partial b}{\partial \theta}  
     + a_z \dfrac{\partial b}{\partial z} \\
  & \nabla^2 f = \dfrac{1}{r}\dfrac{\partial}{\partial r}
                      \left(r \frac{\partial f}{\partial r} \right) +
               \frac{1}{r^2} \frac{\partial^2 f}{\partial \theta^2} + 
               \frac{\partial^2 f}{\partial z^2} 
  \end{aligned}\end{split}
\end{equation*}
\sphinxAtStartPar
Il problema viene risolto calcolando prima le velocità angolari dei
cilindri e successivamente la pressione.
\begin{itemize}
\item {} 
\sphinxAtStartPar
La soluzione di Taylor\sphinxhyphen{}Couette viene ricavata dall’espressione
semplificata delle equazioni di Navier\sphinxhyphen{}Stokes, \$\(\begin{cases}
  -\rho \dfrac{u^2_\theta}{r} + \dfrac{\partial P}{\partial r} = 0 \\
  -\dfrac{1}{r}\dfrac{\partial}{\partial r} \left( r \dfrac{\partial u_\theta}{\partial r}  \right)  + \dfrac{u_\theta}{r^2}= 0 \ ,
\end{cases}\)\( ottenute imponendo che il campo di moto sia
bidimensionale
\)\textbackslash{}bm\{u\}(\textbackslash{}bm\{r\}) = u\_\{\textbackslash{}theta\}(r,\textbackslash{}theta) \textbackslash{}bm\{\textbackslash{}hat\{\textbackslash{}theta\}\} + u\_r (r,\textbackslash{}theta) \textbackslash{}bm\{\textbackslash{}hat\{r\}\}\(,
che la soluzione sia omogenea rispetto alla coordinata \)\textbackslash{}theta\( e
sfruttando le condizioni al contorno e il vincolo di
incomprimibilità per ricavare \)u\_r(\textbackslash{}theta) = 0\(. Sia il campo di
pressione sia il campo di velocità dipendono solamente dalla
coordinata radiale, \)P = P(r)\(, \)u\_\textbackslash{}theta = u\_\textbackslash{}theta (r)\(. Le
derivate parziali possono essere quindi trasformate in derivate
ordinarie. La seconda equazione è disaccoppiata dalla prima e può
essere risolta, una volta imposte le condizioni al contorno. Trovato
il campo di moto da questa equazione, la prima viene usata per
calcolare il campo di pressione. La seconda equazione può essere
riscritta come (svolgere le derivate per credere!) \)\(\begin{cases}
  -\left(\dfrac{1}{r} \left(r u_\theta\right)'\right)' = 0 \\
  u_\theta(R_1) = \Omega_1 R_1 \\
  u_\theta(R_2) = \Omega_2 R_2 \\
\end{cases}
\Rightarrow
\begin{cases}
  u_\theta(r) = A r + \dfrac{B}{r} \ \ \ \text{A,B from b.c.} \\
  u_\theta(R_1) = \Omega_1 R_1 \\
  u_\theta(R_2) = \Omega_2 R_2 \\
\end{cases}\)\( Il campo di moto tra due cilindri coassiali rotanti è
\)\(u_\theta(r) = \frac{\Omega_2 R_2^2 - \Omega_1 R_1^2}{R_2^2-R_1^2} r +
   \frac{(\Omega_1 - \Omega_2)R_1^2 R_2^2}{R_2^2-R_1^2}\frac{1}{r} \ .\)\$

\sphinxAtStartPar
La soluzione esatta di Taylor\sphinxhyphen{}Couette è facile da ricavare, se si
ricorda che è la somma di una rotazione rigida e un vortice
irrotazionale: imponendo la forma \(u_\theta (r) = A r + B/r\) e le
condizioni al contorno,
\$\(u_{\theta}(R_1) = \Omega_1 R_1 \qquad , \qquad  u_{\theta}(R_2) = \Omega_2 R_2\)\$
si ottiene la formula voluta.

\item {} 
\sphinxAtStartPar
Calcolo delle velocità angolari dei cilindri. Nota la forma del
campo di moto e le velocità in due punti a diversi raggi, è
possibile calcolare \(\Omega_1\), \(\Omega_2\) risolvendo un sistema
lineare di due equazioni nelle due incognite. Note le misure di
velocità \(u_{\theta,1/4} = u_{\theta}(r_{1/4})\),
\(u_{\theta,3/4} = u_{\theta}(r_{3/4})\), il sistema risolvente
diventa: \$\(\displaystyle\begin{bmatrix}
  -\frac{R_1^2}{R_2^2-R_1^2}r_{1/4} + \frac{R_1^2 R_2^2}{R_2^2-R_1^2}\frac{1}{r_{1/4}} & \quad
   \frac{R_2^2}{R_2^2-R_1^2}r_{1/4} - \frac{R_1^2 R_2^2}{R_2^2-R_1^2}\frac{1}{r_{1/4}} \\ 
  -\frac{R_1^2}{R_2^2-R_1^2}r_{3/4} + \frac{R_1^2 R_2^2}{R_2^2-R_1^2}\frac{1}{r_{3/4}} & \quad
   \frac{R_2^2}{R_2^2-R_1^2}r_{3/4} - \frac{R_1^2 R_2^2}{R_2^2-R_1^2}\frac{1}{r_{3/4}} \\
 \end{bmatrix}
 \displaystyle\begin{bmatrix}
  \Omega_1 \\ \Omega_2
 \end{bmatrix} =
 \displaystyle\begin{bmatrix}
  u_{\theta,1/4} \\ u_{\theta,3/4}
 \end{bmatrix}\)\$

\item {} 
\sphinxAtStartPar
Calcolo della pressione. Una volta noto il campo di moto, è
possibile calcolare il campo di pressione dalla componente radiale
dell’equazione della quantità di moto, \$\(\begin{aligned}
  P'(r) & = \rho \dfrac{u_\theta^2}{r} \quad , \quad \text{con \)P(R\_2) = P\_2\(} \\
 & \quad \rightarrow \quad
 \int_{r}^{R_2} \dfrac{dP}{dr} dr 
      = \int_r^{R_2} \rho \dfrac{1}{r} \left( A r + \dfrac{B}{r} \right)^2 dr
\end{aligned}\)\( Da questa si ricava
\)\(P(r) = P_2 - \rho \left[ \dfrac{1}{2} A^2 (R_2^2 - r^2) + 2 A B ln \dfrac{R_2}{r} - 
      \dfrac{1}{2}B^2 \left( \dfrac{1}{R^2} - \dfrac{1}{r^2} \right)  \right] \ .\)\$

\end{itemize}

\sphinxstepscope


\subsection{Exercise 6.7}
\label{\detokenize{polimi/fluidmechanics-ita/template/capitoli/06_slnEsatte/0606in:exercise-6-7}}\label{\detokenize{polimi/fluidmechanics-ita/template/capitoli/06_slnEsatte/0606in:fluid-mechanics-exact-solutions-ex07}}\label{\detokenize{polimi/fluidmechanics-ita/template/capitoli/06_slnEsatte/0606in::doc}}
\sphinxAtStartPar
+:———————————:+:———————————:+
| Si consideri la corrente piana di | !{[}image{]}(./fig/slnEsatte\sphinxhyphen{}taylor\sphinxhyphen{}c |
| un fluido di densità \(\rho\) fra   | ouette)\{width=»95\%»\}              |
| due cilindri coassiali di raggio  |                                   |
| \(R_1\) e \(R_2\). Il cilindro        |                                   |
| esterno è fermo, mentre quello    |                                   |
| interno è messo in rotazione da   |                                   |
| un motore con curva               |                                   |
| caratteristica                    |                                   |
| \(C_{disp}(\Omega) = \alpha - \bet |                                   |
| a \Omega\).                        |                                   |
| Si determini il punto di          |                                   |
| equilibrio del sistema (\(\Omega\)  |                                   |
| costante).                        |                                   |
|                                   |                                   |
| (…)                             |                                   |
+———————————–+———————————–+

\sphinxAtStartPar
Soluzione esatte delle equazioni di Navier\sphinxhyphen{}Stokes in geometria
cilindrica. Corrente di Taylor\sphinxhyphen{}Couette.
\begin{equation*}
\begin{split}\begin{cases}
    \rho \dfrac{\partial u_r}{\partial t}
    + \rho \left( \bm{u} \cdot \bm{\nabla}u_r - \dfrac{u_\theta^2}{r} \right)
    - \mu \left(\nabla^2 u_r 
       - \dfrac{u_r}{r^2} 
       - \dfrac{2}{r^2}\dfrac{\partial u_\theta}{\partial \theta} \right)  
       + \dfrac{\partial p}{\partial r} = f_r \\
    \rho \dfrac{\partial u_\theta}{\partial t}
    + \rho \left( \bm{u} \cdot \bm{\nabla} u_\theta + \dfrac{u_\theta u_r}{r} \right)
    - \mu \left(\nabla^2 u_\theta 
       - \dfrac{u_\theta}{r^2} 
       + \frac{2}{r^2}\dfrac{\partial u_r}{\partial \theta}  \right) 
    + \dfrac{1}{r} \frac{\partial p}{\partial \theta} = f_\theta\\
    \rho \dfrac{\partial u_z}{\partial t}
    + \rho \bm{u} \cdot \bm{\nabla} u_z
    - \mu \nabla^2 u_z
    + \dfrac{\partial p}{\partial z} = f_z \\ \\
    \dfrac{1}{r}\dfrac{\partial}{\partial r}\left( r u_r \right) 
    + \dfrac{1}{r}\dfrac{\partial u_\theta}{\partial \theta} 
    + \dfrac{\partial u_z}{\partial z} = 0
  \end{cases}$$ con $$\begin{aligned}
  & \bm{a} \cdot \bm{\nabla} b = a_r \dfrac{\partial b}{\partial r} 
     + \dfrac{a_\theta}{r} \dfrac{\partial b}{\partial \theta}  
     + a_z \dfrac{\partial b}{\partial z} \\
  & \nabla^2 f = \dfrac{1}{r}\dfrac{\partial}{\partial r}
                      \left(r \frac{\partial f}{\partial r} \right) +
               \frac{1}{r^2} \frac{\partial^2 f}{\partial \theta^2} + 
               \frac{\partial^2 f}{\partial z^2} 
  \end{aligned}\end{split}
\end{equation*}
\sphinxAtStartPar
Espressione vettoriale del contributo viscoso del vettore sforzo,
\$\(\begin{aligned}
  \bm{s_n} & = \mathbb{S} \cdot \bm{\hat{n}} = \\
           & = \mu [\bm{\nabla} \bm{u} + \bm{\nabla}^T \bm{u}] \cdot \bm{\hat{n}} = \\
           & = \mu \left[ 2 (\bm{\hat{n}} \cdot \bm{\nabla} ) \bm{u} + \bm{\hat{n}} \times \bm{\nabla} \times \bm{u}  \right]
\end{aligned}\)\$

\sphinxAtStartPar
Viene risolto il problema piano, nel quale i campi di velocità e di
pressione hanno la forma
\$\(\bm{u}(\bm{r}) = u_r(r,\theta) \bm{\hat{r}} + u_{\theta}(r,\theta) \bm{\hat{\theta}} \quad , \quad P(\bm{r}) = P(r,\theta) \ ,\)\$
e le azioni integrali (come la coppia fornita e quella incognita) sono
intese per unità di lunghezza, essendo la «dimensione mancante» quella
fuori dal piano del disegno.
\begin{itemize}
\item {} 
\sphinxAtStartPar
Calcolo delle velocità angolari dei cilindri. Nota la forma del
campo di moto e le velocità in due punti a diversi raggi, è
possibile calcolare \(\Omega_1\), \(\Omega_2\) risolvendo un sistema
lineare di due equazioni nelle due incognite. Il campo di moto tra
due cilindri coassiali rotanti è:
\$\(u_\theta(r) = A r + \dfrac{B}{r} = \frac{\Omega_2 R_2^2 - \Omega_1 R_1^2}{R_2^2-R_1^2} r +
   \frac{(\Omega_1 - \Omega_2)R_1^2 R_2^2}{R_2^2-R_1^2}\frac{1}{r} \ .\)\(
Se il cilindro esterno è fermo e la velocità angolare del cilindro
interno vale \)\textbackslash{}Omega\_1 = \textbackslash{}Omega\(, i coefficienti \)A\( e \)B\( valgono
\)\(A = - \frac{R_1^2}{R_2^2-R_1^2} \Omega  < 0 \qquad , \qquad B = \frac{R_1^2 R_2^2}{R_2^2-R_1^2} \Omega > 0 \ .\)\$

\sphinxAtStartPar
La soluzione esatta di Taylor\sphinxhyphen{}Couette è facile da ricavare, se si
ricorda che è la somma di una rotazione rigida e un vortice
irrotazionale: imponendo la forma \(u_\theta (r) = A r + B/r\) e le
condizioni al contorno,
\$\(u_{\theta}(R_1) = \Omega_1 R_1 \qquad , \qquad  u_{\theta}(R_2) = \Omega_2 R_2\)\$
si ottiene la formula voluta.

\item {} 
\sphinxAtStartPar
Calcolo dello sforzo tangenziale a parete per determinare il puto di
equilibrio del sistema. Si determina la componente tangenziale
(quella che contribuisce alla coppia resistente) dello sforzo sul
cilindro interno. Il contributo viscoso del vettore sforzo può
essere scritto come: \$\(\begin{aligned}
  \bm{s_n} & = \mathbb{S} \cdot \bm{\hat{n}} = \\
           & = \mu [\bm{\nabla} \bm{u} + \bm{\nabla}^T \bm{u}] \cdot \bm{\hat{n}} = \\
           & = \mu \left[ 2 (\bm{\hat{n}} \cdot \bm{\nabla} ) \bm{u} + \bm{\hat{n}} \times \bm{\nabla} \times \bm{u}  \right] = &  \text{(verificare con le tabelle)} \\
           & = \mu \left[ 2 \dfrac{\partial u_\theta}{\partial r} - \dfrac{1}{r} 
    \dfrac{\partial}{\partial r} (r u_\theta) \right] \bm{\hat{\theta}} = \\
           & = \mu \left[ \dfrac{\partial u_\theta}{\partial r} - \dfrac{u_\theta}{r} \right] \bm{\hat{\theta}} = &  \text{(\)u\_\textbackslash{}theta = A r + B / r\()}\\
           & = - 2 \mu \dfrac{B}{r^2} \bm{\hat{\theta}} 
\end{aligned}\)\$

\sphinxAtStartPar
La formula dello sforzo tangenziale a parete per la corrente di
Taylor\sphinxhyphen{}Couette è
\(\tau_w = \mu \left[ \frac{\partial u_\theta}{\partial r} - \frac{u_\theta}{r} \right]\),

\sphinxAtStartPar
La parte tangenziale dello sforzo a parete sul cilindro interno è
quindi \(\tau_w = 
2 \mu {B}/{R_1^2}\). Integrando il prodotto tra vettore sforzo e
raggio \(R_1\) sulla superficie laterale del cilindro si ottiene la
coppia resistente, \$\(\begin{aligned}
 C_{res}(\Omega) & = \int_{\theta=0}^{2\pi} \tau_w(R_1) R_1 R_1 d\theta  = \\ & = 2\pi \tau_w(R_1) R_1^2 = -4\pi \mu \dfrac{B(\Omega)}{R_1^2}R_1^2 =  -4\pi \mu  \frac{R_1^2 R_2^2}{R_2^2-R_1^2} \Omega  = - \gamma \Omega \ .
\end{aligned}\)\( All'equilibrio, la somma della coppia disponibile e
di quella resistente deve essere uguale a zero,
\)\(0 = C_{disp}(\Omega) + C_{res}(\Omega) = \alpha - \beta \Omega -  \gamma \Omega \ ,\)\(
e quindi \)\textbackslash{}Omega = \textbackslash{}dfrac\{\textbackslash{}alpha\}\{\textbackslash{}beta + \textbackslash{}gamma\}\$.

\end{itemize}

\sphinxstepscope


\subsection{Exercise 6.8}
\label{\detokenize{polimi/fluidmechanics-ita/template/capitoli/06_slnEsatte/0608in:exercise-6-8}}\label{\detokenize{polimi/fluidmechanics-ita/template/capitoli/06_slnEsatte/0608in:fluid-mechanics-exact-solutions-ex08}}\label{\detokenize{polimi/fluidmechanics-ita/template/capitoli/06_slnEsatte/0608in::doc}}
\sphinxAtStartPar
+:———————————:+:———————————:+
| Un contenitore cilindrico (raggio |                                   |
| \(R\), altezza \(H\)) è riempito fino |                                   |
| ad una quota \(h_1 =H/2\) di un     |                                   |
| liquido di densità \(\rho\). Il     |                                   |
| contenitore è messo poi in        |                                   |
| rotazione con velocità angolare   |                                   |
| costante \(\Omega\). Una volta      |                                   |
| esaurito il transitorio, viene    |                                   |
| chiesto di trovare:               |                                   |
|                                   |                                   |
| \sphinxhyphen{}   la forma che assume il        |                                   |
|     liquido all’interno del       |                                   |
|     contenitore;                  |                                   |
|                                   |                                   |
| \sphinxhyphen{}   la velocità \(\Omega_{max}\)    |                                   |
|     alla quale il liquido inizia  |                                   |
|     a uscire dal contenitore;     |                                   |
|                                   |                                   |
| \sphinxhyphen{}   il campo di pressione quando  |                                   |
|     il corpo ruota con velocità   |                                   |
|     angolare \(\Omega_{max}\).      |                                   |
|                                   |                                   |
| (R:                               |                                   |
| \(z_{free}(r) = \dfrac{\Omega^2 r^ |                                   |
| 2}{2 g} - \dfrac{\Omega^2 R^2}{4  |                                   |
| g} + \dfrac{H}{2}\)                |                                   |
| \(\Omega_{max} = \sqrt{\dfrac{2 g  |                                   |
| H}{R^2}}\)                         |                                   |
| \(P(r) = \dots\))                   |                                   |
|                                   |                                   |
| \sphinxincludegraphics{{polimi/fluidmechanics-ita/template/capitoli/06_slnEsatte/fig/vessel}}\{width=»90\% |                                   |
| «\}                                |                                   |
+———————————–+———————————–+

\sphinxAtStartPar
Soluzione esatte delle equazioni di Navier\sphinxhyphen{}Stokes. Fluido in rotazione
rigida, con superficie superiore libera.
\begin{itemize}
\item {} 
\sphinxAtStartPar
Si usano le equazioni di NS in coordinate cilindriche. Seguendo un
procedimento analogo a quello svolto per ottenere la soluzione
esatta di Taylor\sphinxhyphen{}Couette, ma senza trascurare l’effetto della
gravità, si ottiene la seguente coppia di equazioni
\$\(\label{eqn:vessel_cyl}
 \begin{cases}
  \dfrac{\partial P}{\partial z} = - \rho g \\
  \dfrac{\partial P}{\partial r} = \rho \dfrac{u_{\theta}^2}{r}
 \end{cases}\)\( Il campo di moto descrive una rotazione rigida,
poichè il termine \)1/r\( della soluzione di Taylor-Couette non è
ammissibile (l'asse appartiene al dominio, non ha senso una velocità
che tende all'infinito). La costante di proporzionalità tra
\)u\_\{\textbackslash{}theta\}\( ed \)r\( è la velocità angolare \)\textbackslash{}Omega\( per soddisfare
le condizioni al controno a parete, \)u\_\{\textbackslash{}theta\}(R) = \textbackslash{}Omega R\(.
\)\(u_{\theta}(r) = \Omega r\)\( Dall'integrazione delle due equazioni
([\[eqn:vessel\_cyl\]](#eqn:vessel_cyl){reference-type="ref"
reference="eqn:vessel_cyl"}) si ottiene il campo di pressione
\)P(r,z)\(, a meno di una costante di integrazione \)C\( \)\(\label{eqn:p}
 P(r,z) = -\rho g z + \rho \dfrac{\Omega^2 r^2}{2} + C\)\( La
condizione al contorno necessaria è \)P(r,z\_\{free\}(r)) = P\_a\(; sulla
superficie libera, la cui quota è descritta dalla funzione
\)z\_\{free\}(r)\( (ancora incognita), agisce la pressione ambiente \)P\_a\(
\)\(P(r,z_{free}(r)) = -\rho g z_{free} + \rho \dfrac{\Omega^2 r^2}{2} + C = P_a \\\)\(
\)\(\ \ \ \ \ \  \Downarrow\)\( \)\(\label{eqn:zfree}
 z_{free}(r) = \dfrac{\Omega^2 r^2}{2 g} - \dfrac{P_a - C}{\rho g}\)\$

\sphinxAtStartPar
Per determinare la costante \(C\) bisogna ricorrere alla conservazione
della massa. La massa contenuta all’interno del recipiente non varia
(fino a quando il liquido non esce). Se si considera densità
costante \(\rho\), bisogna scrivere la conservazione del volume tra
istante iniziale \(V_0 = \pi R^2 H/2\) e condizione a regime \(V\). Il
volume \(V\) viene calcolato tramite un’integrale di volume, comodo da
descrivere in coordinate cilindriche: \$\(\begin{aligned}
 V & = \int_{\theta=0}^{2\pi} \int_{r=0}^{R} \int_{z=0}^{z=z_{free}(r)} r dr dz d\theta = \\
   & = 2 \pi \int_{r=0}^{r=R} z_free(r) r dr  = \\
   & = 2 \pi \int_{r=0}^{r=R} \dfrac{\Omega^2 r^3}{2 g} - \dfrac{P_a - C}{\rho g}r dr = \\
   & = 2 \pi \left[ \dfrac{\Omega^2 R^4}{8 g} - \dfrac{(P_a - C)}{2 \rho g} R^2  \right] = \\
   & =   \pi \left[ \dfrac{\Omega^2 R^4}{4 g} - \dfrac{(P_a - C)}{  \rho g} R^2  \right] = \\
\end{aligned}\)\( Uguagliando \)V\_0\( e \)V\( si ottiene
\)\(- \dfrac{(P_a - C)R}{  \rho g} = - \dfrac{\Omega^2 R^4}{4 g} + R^2 \dfrac{H}{2}\)\(
termine che può essere sotituito in
([\[eqn:zfree\]](#eqn:zfree){reference-type="ref"
reference="eqn:zfree"}) \)\(\label{eqn:zfree2}
 z_{free}(r) = \dfrac{\Omega^2 r^2}{2 g} - \dfrac{\Omega^2 R^2}{4 g} + \dfrac{H}{2}\)\(
La superficie libera ha la forma di un parabolide. La concavità del
paraboloide è diretta verso l'alto e aumenta all'aumentare di
\)|\textbackslash{}Omega|\( (il risultato è indipendente dal verso di rotazione, e
quindi dal segno di \)\textbackslash{}Omega\(, poichè compare con potenze pari). La
quota del vertice
\)z\_v = \sphinxhyphen{} \textbackslash{}dfrac\{\textbackslash{}Omega\textasciicircum{}2 R\textasciicircum{}2\}\{4 g\} +  \textbackslash{}dfrac\{H\}\{2\}\$ invece
diminuisce.

\item {} 
\sphinxAtStartPar
Per determinare la \(\Omega_{max}\), bisogna imporre la condizione
\(z_{free}(r=R) = H\).
\$\(z_{free}(R) = \dfrac{\Omega_{max}^2 R^2}{4 g} + R^2 \dfrac{H}{2} = H 
 \Rightarrow
  \Omega_{max} = \sqrt{\dfrac{2 g H}{R^2}}\)\$

\item {} 
\sphinxAtStartPar
Per ottenere il campo di pressione, basta inserire il il valore di
\(C\) e \(\Omega_{max}\) nella formula
(\DUrole{xref,myst}{{[}eqn:p{]}}\{reference\sphinxhyphen{}type=»ref» reference=»eqn:p»\}).

\end{itemize}

\sphinxstepscope


\chapter{Similitude}
\label{\detokenize{polimi/fluidmechanics-ita/template/capitoli/07_similitudine/07teoria:similitude}}\label{\detokenize{polimi/fluidmechanics-ita/template/capitoli/07_similitudine/07teoria:fluid-mechanics-similitude}}\label{\detokenize{polimi/fluidmechanics-ita/template/capitoli/07_similitudine/07teoria::doc}}

\section{Teorema di Buckingham}
\label{\detokenize{polimi/fluidmechanics-ita/template/capitoli/07_similitudine/07teoria:teorema-di-buckingham}}
\sphinxAtStartPar
Il teorema di Buckingham afferma che un problema descritto da \(n\)
variabili fisiche, le cui dimensioni fisiche coinvolgono \(k\) grandezze
fondamentali, può essere espresso in funzione di \(n-k\) gruppi
adimensionali.


\section{Equazioni di Navier–Stokes incomprimibili in forma adimensionale}
\label{\detokenize{polimi/fluidmechanics-ita/template/capitoli/07_similitudine/07teoria:equazioni-di-navier-stokes-incomprimibili-in-forma-adimensionale}}
\sphinxAtStartPar
Nelle equazioni incomprimibili di Navier–Stokes per un fluido a densità
costante
\begin{equation*}
\begin{split}\begin{cases}
 \rho \dfrac{\partial \mathbf{u}}{\partial t} + \rho (\mathbf{u} \cdot \mathbf{\nabla}) \mathbf{u} - \mu \Delta \mathbf{u} + \mathbf{\nabla} p = \rho \mathbf{g} \\
 \mathbf{\nabla} \cdot \mathbf{u} = 0 \ ,
\end{cases}\end{split}
\end{equation*}
\sphinxAtStartPar
compaiono 7 variabili fisiche
\((\rho,\mathbf{u},\mu,p,\mathbf{g};\mathbf{r},t)\), le 2 variabili indipendenti
spaziale \(\mathbf{r}\) e temporale \(t\), e le 5 variabili dipendenti
rappresentate dalla densità \(\rho\), dal campo di velocità \(\mathbf{u}\), dal
coefficiente di viscosità dinamica \(\mu\), dal campo di pressione \(p\) e
dal campo di forze di volume \(\mathbf{g}\). Le dimensioni fisiche delle 7
variabili possono essere costruite con 3 grandezze fondamentali, la
massa \(M\), la lunghezza \(L\) e il tempo \(T\). Ad esempio, le dimensioni
fisiche della velocità sono \([\mathbf{u}] = L \ T^{-1}\) e quelle della
densità sono \([\rho] = M \ L^{-3}\). Le dimensioni delle 7 variabili
fisiche che compaiono nelle equazioni di Navier–Stokes incomprimibili
sono raccolte nella tabella
\DUrole{xref,myst}{{[}tab:adim\sphinxhyphen{}ns\sphinxhyphen{}1{]}}\{reference\sphinxhyphen{}type=»ref»
reference=»tab:adim\sphinxhyphen{}ns\sphinxhyphen{}1»\}.

\begin{sphinxVerbatim}[commandchars=\\\{\}]
   \PYGZdl{}\PYGZbs{}mathbf\PYGZob{}r\PYGZcb{}\PYGZdl{}   \PYGZdl{}t\PYGZdl{}   \PYGZdl{}\PYGZbs{}rho\PYGZdl{}   \PYGZdl{}\PYGZbs{}mathbf\PYGZob{}u\PYGZcb{}\PYGZdl{}   \PYGZdl{}\PYGZbs{}mu\PYGZdl{}   \PYGZdl{}p\PYGZdl{}   \PYGZdl{}\PYGZbs{}mathbf\PYGZob{}g\PYGZcb{}\PYGZdl{}
\end{sphinxVerbatim}


\bigskip\hrule\bigskip


\sphinxAtStartPar
M      0        0      1         0         1      1       0
L      1        0      \sphinxhyphen{}3        1        \sphinxhyphen{}1     \sphinxhyphen{}1       1
T      0        1      0         \sphinxhyphen{}1       \sphinxhyphen{}1     \sphinxhyphen{}2       \sphinxhyphen{}2

\sphinxAtStartPar
: Variabili fisiche e grandezze fondamentali.{[}{]}\{label=»tab:adim\sphinxhyphen{}ns\sphinxhyphen{}1»\}

\sphinxAtStartPar
Per poter formare i \(7-3 = 4\) gruppi adimensionali che caratterizzano il
problema, è necessario scegliere 3 variabili fisiche (o combinazione di
queste) che «contengano in maniera linearmente indipendente» tutte le 3
grandezze fondamentali del problema. Facendo riferimento alla tabella
\DUrole{xref,myst}{{[}tab:adim\sphinxhyphen{}ns\sphinxhyphen{}1{]}}\{reference\sphinxhyphen{}type=»ref»
reference=»tab:adim\sphinxhyphen{}ns\sphinxhyphen{}1»\}, le colonne relative alle variabili scelte
per l’adimensionalizzazione devono formare dei vettori linearmente
indipendenti tra di loro. Ad esempio, due scelte valide delle variabili
da usare per l’adimensionalizzazione del problema sono:
\begin{itemize}
\item {} 
\sphinxAtStartPar
\((\rho,U,L)\), una densità, una velocità e una lunghezza di
riferimento,

\item {} 
\sphinxAtStartPar
\((\mu,U,L)\), una viscosità, una velocità e una lunghezza di
riferimento,

\end{itemize}

\sphinxAtStartPar
mentre una scelta non accettabile è una terna \((T,U,L)\) formata da un
tempo, una velocità e una lunghezza di riferimento, poichè non è
possibile costruire dei gruppi adimensionali con le variabili fisiche
che contengono la massa come grandezza fisica, come la densità, la
presssione e il coefficiente di viscosità. Tutte le variabili fisiche
vengono espresse come il prodotto di una loro grandezza di riferimento,
che contiene le dimensioni fisiche e viene indicata con la tilde, e la
loro versione adimensionale, indicata con l’asterisco,
\begin{equation*}
\begin{split}\begin{aligned}
\mathbf{r} & = \tilde{L} \mathbf{r}^* \quad , \quad t = \tilde{T}\ t^* \quad , \quad \mathbf{u} = \tilde{U} \mathbf{u}^* \\
\rho & = \tilde{\rho} \ \rho^* \quad , \quad \mu = \tilde{\mu} \ \mu^* \quad , \quad 
p = \tilde{p} \ p^* \quad , \quad \mathbf{g} = \tilde{g} \ \mathbf{g}^* \ .
\end{aligned}\end{split}
\end{equation*}
\sphinxAtStartPar
Per le equazioni di Navier–Stokes incomptimibili a
proprietà costanti, è possibile scegliere il valore di riferimento della
densità e della viscosità dinamica come il valore stesso delle variabili
fisiche, \(\tilde{\rho} = \rho\), \(\tilde{\mu} = \mu\). In questo modo, il
loro valore adimensionale è uguale a 1, \(\rho^* = \mu^* = 1\). Nel caso
del campo di forze di volume dovuto alla gravità, costante e diretto
lungo la verticale, è possibile definire il valore di riferimento
\(\tilde{g} = |\mathbf{g}|\), cosicché il vettore \(\mathbf{g}^*\) è uguale e
contrario al versore \(\mathbf{\hat{z}}\) orientato in direzione verticale.
Anche l’operatore \sphinxstyleemphasis{nabla} viene adimensionalizzato,
\(\mathbf{\nabla} = \frac{1}{\tilde{L}} \mathbf{\nabla}^*\). Le equazioni di
Navier–Stokes possono essere scritte come
\begin{equation*}
\begin{split}\begin{cases}
 \dfrac{\rho \tilde{U}}{\tilde{t}} \dfrac{\partial \mathbf{u}^*}{\partial t^*} + \dfrac{\rho \tilde{U}^2}{\tilde{L}} (\mathbf{u}^* \cdot \mathbf{\nabla}^*) \mathbf{u}^* - \dfrac{\mu \tilde{U}}{\tilde{L}^2} \Delta^* \mathbf{u}^* + \dfrac{\tilde{p}}{\tilde{L}} \mathbf{\nabla}^* p^* = -\rho g \mathbf{\hat{z}} \\
 \dfrac{\tilde{U}}{\tilde{L}}\mathbf{\nabla}^* \cdot \mathbf{u}^* = 0 \ .
\end{cases}\end{split}
\end{equation*}

\section{Adimensionalizzazione «ad alti numeri di Reynolds»}
\label{\detokenize{polimi/fluidmechanics-ita/template/capitoli/07_similitudine/07teoria:adimensionalizzazione-ad-alti-numeri-di-reynolds}}
\sphinxAtStartPar
Se si scelgono \((\tilde{\rho},\tilde{U},\tilde{L})\) come grandezze di
riferimento, dividendo l’equazione della quantità di moto per
\(\tilde{\rho} \tilde{U}^2 / \tilde{L}\) e il vincolo di incomprimibilità
per \(\tilde{U} / \tilde{L}\),
\begin{equation*}
\begin{split}\begin{cases}
 \dfrac{\tilde{L}}{\tilde{U}\tilde{t}} \dfrac{\partial \mathbf{u}^*}{\partial t^*} + (\mathbf{u}^* \cdot \mathbf{\nabla}^*) \mathbf{u}^* - \dfrac{\mu }{ \rho \tilde{U} \tilde{L} } \Delta^* \mathbf{u}^* + \dfrac{\tilde{p}}{\rho \tilde{U}^2} \mathbf{\nabla}^* p^* = -\dfrac{g\tilde{L}}{\tilde{U}^2} \mathbf{\hat{z}} \\
 \mathbf{\nabla}^* \cdot \mathbf{u}^* = 0 \ ,
\end{cases}\end{split}
\end{equation*}
\sphinxAtStartPar
si possono riconoscere 4 numeri adimensionali:
\begin{itemize}
\item {} 
\sphinxAtStartPar
il numero di Strouhal, \(St = \frac{\tilde{L}}{\tilde{U}\tilde{t}}\),
che rappresenta il rapporto tra una scala dei tempi e la scala dei
tempi \(\tilde{L}/\tilde{U}\) costruita con la lunghezza e la velocità
di riferimento;

\item {} 
\sphinxAtStartPar
il numero di Reynolds, \(Re = \frac{\rho \tilde{U} \tilde{L} }{\mu}\),
che rappresenta il rapporto tra gli effetti di inerzia e quelli
viscosi;

\item {} 
\sphinxAtStartPar
il numero di Eulero, \(Eu = \frac{\tilde{p}}{\rho \tilde{U}^2}\), che
rappresenta il rapporto tra la grandezza di riferimento della
pressione e quella di un energia cinetica del fluido;

\item {} 
\sphinxAtStartPar
il numero di Froude, \(Fr = \frac{\tilde{U}^2}{g\tilde{L}}\), che
rappresenta il rapporto tra gli effetti di inerzia e quelli dovuti
al campo di forze di volume.

\end{itemize}

\sphinxAtStartPar
Quando non esiste una scala dei tempi «indipendente» dal fenomeno
fluidodinamico, è possibile scegliere il valore di riferimento del tempo
\(\tilde{t} = \tilde{L} / \tilde{U}\), in modo tale da ottenere un numero
di Strouhal unitario. Per la natura stessa della «pressione» di
moltiplicatore di Lagrange introdotto nelle equazioni di Navier–Stokes
per imporre il vincolo di incomprimibilità, è frequente che la pressione
non abbia una scala indipendente nel regime incomprimibile. É possibile
quindi scegliere una scala di pressione \(\tilde{p} = \rho \tilde{U}^2\),
in modo tale da ottenere un numero di Eulero unitario,
\begin{equation*}
\begin{split}\begin{cases}
 \dfrac{\partial \mathbf{u}^*}{\partial t^*} + (\mathbf{u}^* \cdot \mathbf{\nabla}^*) \mathbf{u}^* - \dfrac{1}{Re} \Delta^* \mathbf{u}^* + \mathbf{\nabla}^* p^* = -\dfrac{1}{Fr} \mathbf{\hat{z}} \\
 \mathbf{\nabla}^* \cdot \mathbf{u}^* = 0 \ .
\end{cases}\end{split}
\end{equation*}
\sphinxAtStartPar
Se le grandezze di riferimento sono rappresentative del
problema, in modo tale da rendere gli ordini di grandezza delle
variabili adimensionali paragonabili tra loro, il valore dei numeri
adimensionali permette di valutare l’influenza dei termini. Ad esempio,
per valori elevati del numero di Froude l’influenza delle forze di
volume è ridotta. Per valori elevati del numero di Reynolds, l’influenza
degli effetti viscosi diventa trascurabile nelle regioni del campo di
moto nelle quali le derivate spaziali del campo di velocità sono
piccole. Per applicazionii tipiche aeronautiche ad alti numeri di
Reynolds, gli effetti viscosi saranno quindi trascurabili in gran parte
del dominio, ad eccezione delle regioni di strato limite, all’interno
delle quali la componente della velocità «parallela» alla parete ha una
variazione elevata in direzione perpendicolare alla parete stessa. Se
gli effetti delle forze di volume sono trascurabili
(\(Fr \rightarrow \infty\)), le equazioni di Navier–Stokes incomprimibili
per problemi ad alti numeri di Reynolds (\(Re \rightarrow \infty\)) si
riducono alle equazioni di Eulero incomprimibili nelle regioni del
dominio in cui gli effetti viscosi sono trascurabili,
\begin{equation*}
\begin{split}\begin{cases}
 \dfrac{\partial \mathbf{u}^*}{\partial t^*} + (\mathbf{u}^* \cdot \mathbf{\nabla}^*) \mathbf{u}^* + \mathbf{\nabla}^* p^* = \mathbf{0} \\
 \mathbf{\nabla}^* \cdot \mathbf{u}^* = 0 \ .
\end{cases}\end{split}
\end{equation*}

\section{Adimensionalizzazione «a bassi numeri di Reynolds»}
\label{\detokenize{polimi/fluidmechanics-ita/template/capitoli/07_similitudine/07teoria:adimensionalizzazione-a-bassi-numeri-di-reynolds}}
\sphinxAtStartPar
Se si scelgono \((\tilde{\rho},\tilde{U},\tilde{L})\) come grandezze di
riferimento, dividendo l’equazione della quantità di moto per
\(\tilde{\mu} \tilde{U} / \tilde{L}^2\) e il vincolo di incomprimibilità
per \(\tilde{U} / \tilde{L}\), le equazioni di Navier–Stokes diventano
\begin{equation*}
\begin{split}\begin{cases}
 \dfrac{\rho\tilde{L}^2}{\mu \tilde{t}} \dfrac{\partial \mathbf{u}^*}{\partial t^*} + \dfrac{\rho \tilde{U} \tilde{L}}{\mu}(\mathbf{u}^* \cdot \mathbf{\nabla}^*) \mathbf{u}^* - \Delta^* \mathbf{u}^* + \dfrac{\tilde{p}\tilde{L}}{\mu \tilde{U}} \mathbf{\nabla}^* p^* = -\dfrac{\rho g\tilde{L}^2}{\mu \tilde{U}} \mathbf{\hat{z}} \\
 \mathbf{\nabla}^* \cdot \mathbf{u}^* = 0 \ .
\end{cases}\end{split}
\end{equation*}
\sphinxAtStartPar
Se gli effetti delle forze di volume sono trascurabili
rispetto agli effetti viscosi e non ci sono scale indipendenti di tempo
e pressione, le equazioni di Navier–Stokes in forma adimensionale
diventano
\begin{equation*}
\begin{split}\begin{cases}
 \dfrac{\partial \mathbf{u}^*}{\partial t^*} + Re (\mathbf{u}^* \cdot \mathbf{\nabla}^*) \mathbf{u}^* - \Delta^* \mathbf{u}^* + \mathbf{\nabla}^* p^* = \mathbf{0} \\
 \mathbf{\nabla}^* \cdot \mathbf{u}^* = 0 \ .
\end{cases}\end{split}
\end{equation*}
\sphinxAtStartPar
Per correnti nelle quali il numero di Reynolds
caratteristico tende a zero, note come \sphinxstyleemphasis{creeping flow}, il termine non
lineare diventa trascurabile e le equazioni di Navier–Stokes si
riducono alle equazioni di Stokes,
\begin{equation*}
\begin{split}\begin{cases}
 \dfrac{\partial \mathbf{u}^*}{\partial t^*} - \Delta^* \mathbf{u}^* + \mathbf{\nabla}^* p^* = \mathbf{0} \\
 \mathbf{\nabla}^* \cdot \mathbf{u}^* = 0 \ .
\end{cases}\end{split}
\end{equation*}

\section{Equazione di continuità e numero di Mach}
\label{\detokenize{polimi/fluidmechanics-ita/template/capitoli/07_similitudine/07teoria:equazione-di-continuita-e-numero-di-mach}}
\sphinxAtStartPar
La forma adimensionale dell’equazione di continuità permette di valutare
i limiti dell’approssimazione di corrente incomprimibile, che soddisfa
il vincolo cinematico di incomprimibilità
\(\mathbf{\nabla} \cdot \mathbf{u} = 0\). L’equazione della massa viene scritta in
forma convettiva,
\begin{equation*}
\begin{split}- \mathbf{\nabla} \cdot \mathbf{u} = \dfrac{1}{\rho}\dfrac{D \rho}{D t} \ .\end{split}
\end{equation*}
\sphinxAtStartPar
Ricordando che lo stato termodinamico di un sistema monocomponente
monofase è definito da due variabili termodinamiche, il campo di
pressione \(p\) viene espresso in funzione del campo di densità \(\rho\) e
di entropia \(s\), come \(p(\rho,s)\). Il differenziale di questa relazione,
\begin{equation*}
\begin{split}d p = \left(\dfrac{\partial p}{\partial \rho}\right)_s d \rho +
       \left(\dfrac{\partial p}{\partial s}\right)_{\rho} d s \ ,\end{split}
\end{equation*}
\sphinxAtStartPar
può
essere utilizzato per esprimere la derivata materiale della densità in
funzione delle derivate materiali di pressione ed entropia,
\begin{equation*}
\begin{split}\dfrac{D \rho}{D t} = \dfrac{1}{\left(\partial p/\partial \rho\right)_s}\dfrac{D p}{D t} - \dfrac{\left(\partial p/\partial s\right)_{\rho}}{\left(\partial p/\partial \rho\right)_s}\dfrac{D s}{D t} = \dfrac{1}{c^2}\dfrac{D p}{D t} - \dfrac{\left(\partial p/\partial s\right)_{\rho}}{c^2}\dfrac{D s}{D t} \ ,\end{split}
\end{equation*}
\sphinxAtStartPar
avendo riconosciuto il quadrato della velocità del suono
\(c^2 = \left(\frac{\partial p}{\partial \rho}\right)_s\). L’equazione
della massa diventa quindi
\begin{equation*}
\begin{split}- \mathbf{\nabla} \cdot \mathbf{u} = \dfrac{1}{\rho c^2}\dfrac{D p}{D t} - \dfrac{\left(\partial p/\partial s\right)_{\rho}}{\rho c^2}\dfrac{D s}{D t} \ .\end{split}
\end{equation*}
\sphinxAtStartPar
Per processi isentropici (o per i quali il secondo termine a destra
dell’uguale è trascurabile), l’equazione della massa si riduce a
\begin{equation*}
\begin{split}- \mathbf{\nabla} \cdot \mathbf{u} = \dfrac{1}{\rho c^2}\dfrac{D p}{D t} \ .\end{split}
\end{equation*}
\sphinxAtStartPar
Utilizzando i valori di densità \(\tilde{\rho}\), velocità \(\tilde{U}\) e
lunghezza \(\tilde{L}\) caratteristici della corrente per costruire la
scala dei tempi \(\tilde{t} = \tilde{L}/\tilde{U}\) e per la pressione
\(\tilde{p} = \tilde{\rho} \tilde{U}^2\), si ottiene l’equazione della
massa in forma adimensionale,
\begin{equation*}
\begin{split}\mathbf{\nabla}^* \cdot \mathbf{u}^* = -  \dfrac{M^2}{\rho^*} \dfrac{D p^*}{D t^*} \ ,\end{split}
\end{equation*}
\sphinxAtStartPar
nella quale si è iconosciuto il numero di Mach caratteristico della
corrente, \(M = \dfrac{\tilde{U}}{c}\), definito come il rapporto tra una
velocità caratteristica e la velocità del suono in uno stato
termodinamico di riferimento della corrente. É immediato osservare che
l’equazione di continuità della massa si riduce al vincolo di
incomprimibilità quando il numero di Mach assume valori ridotti (e il
campo di pressione non ha variazioni rapide).


\section{Equazioni di Navier–Stokes in sistemi di riferimento non inerziali}
\label{\detokenize{polimi/fluidmechanics-ita/template/capitoli/07_similitudine/07teoria:equazioni-di-navier-stokes-in-sistemi-di-riferimento-non-inerziali}}
\sphinxAtStartPar
…


\section{Equazioni di Boussinesq e numeri di Prandtl, Rayleigh e Grashof}
\label{\detokenize{polimi/fluidmechanics-ita/template/capitoli/07_similitudine/07teoria:equazioni-di-boussinesq-e-numeri-di-prandtl-rayleigh-e-grashof}}

\section{Equazioni di Boussinesq}
\label{\detokenize{polimi/fluidmechanics-ita/template/capitoli/07_similitudine/07teoria:equazioni-di-boussinesq}}
\sphinxAtStartPar
Le equazioni di Boussinesq sono un modello approssimato delle equazioni
complete del moto dei fluidi, ricavato sotto le ipotesi che:
\begin{itemize}
\item {} 
\sphinxAtStartPar
il contributo di dissipazione nell’equazione dell’energia sia
trascurabile;

\item {} 
\sphinxAtStartPar
la densità dipenda linearmente dalla temperatura nel termine di
forze di volume nell’equazione della quantità di moto.

\end{itemize}

\sphinxAtStartPar
La variazione della densità in funzione della densità diventa quindi
\begin{equation*}
\begin{split}d \rho(P, T) = \left(\dfrac{\partial \rho}{\partial P} \right)_T dP +  \left(\dfrac{\partial \rho}{\partial T} \right)_P dT \approx \left(\dfrac{\partial \rho}{\partial T} \right)_P dT = - \rho_0 \, \alpha \, dT \\end{split}
\end{equation*}\begin{equation*}
\begin{split}\rightarrow \rho = \rho_0 \left( 1 - \alpha \, (T-T_0) \right) \ ,\end{split}
\end{equation*}
\sphinxAtStartPar
dove le derivate sono calcolate nello stato termodinamico di
riferimento, \((\rho_0, \ T_0)\), ed è stato introdotto il coefficiente di
dilatazione termica
\begin{equation*}
\begin{split}\alpha = - \dfrac{1}{\rho_0} \left(\dfrac{\partial \rho}{\partial T} \right)_P \ .\end{split}
\end{equation*}
\sphinxAtStartPar
Introducendo le approssimazioni elencate, l’espressione dell’energia
interna \(e = c_v T\) e la legge di Fourier per il flusso di calore per
conduzione, \(\mathbf{q} = -k \mathbf{\nabla} T\), nelle equazioni complete per
una corrente incomprimibile di un fluido newtoniano,
\begin{equation*}
\begin{split}\begin{cases}
      \rho \dfrac{\partial \mathbf{u}}{\partial t} + \rho
      \left( \mathbf{u} \cdot \mathbf{\nabla} \right) \mathbf{u} -
      \mu \nabla^2 \mathbf{u} + \mathbf{\nabla} P = \rho \mathbf{g} \\
      \mathbf{\nabla} \cdot \mathbf{u} = 0 \\
      \rho \dfrac{\partial e}{\partial t} + \rho \mathbf{u} \cdot 
      \mathbf{\nabla} e = 2 \mu \mathbb{D}:\mathbb{D} - \mathbf{\nabla} \cdot \mathbf{q} \ ,
\end{cases}\end{split}
\end{equation*}
\sphinxAtStartPar
si ottengono le equazioni di Boussinesq
\begin{equation*}
\begin{split}\begin{cases}
      \dfrac{\partial \mathbf{u}}{\partial t} + 
      \left( \mathbf{u} \cdot \mathbf{\nabla} \right) \mathbf{u} -
      \nu \nabla^2 \mathbf{u} + \dfrac{1}{\rho_0}\mathbf{\nabla} P = \big( 1 - \alpha ( T-T_0 ) \big) \mathbf{g} \\
      \mathbf{\nabla} \cdot \mathbf{u} = 0 \\
      \dfrac{\partial T}{\partial t} + \mathbf{u} \cdot 
      \mathbf{\nabla} T =  D \nabla^2 T \ ,
\end{cases}\end{split}
\end{equation*}
\sphinxAtStartPar
avendo definito il coefficiente di diffusione termica
\(D = \dfrac{k}{\rho_0 c_v}\).


\section{Equazioni di Boussinesq: problema bidimensionale tra due superfici piane}
\label{\detokenize{polimi/fluidmechanics-ita/template/capitoli/07_similitudine/07teoria:equazioni-di-boussinesq-problema-bidimensionale-tra-due-superfici-piane}}

\subsection{Condizioni al contorno}
\label{\detokenize{polimi/fluidmechanics-ita/template/capitoli/07_similitudine/07teoria:condizioni-al-contorno}}
\sphinxAtStartPar
Si considera ora la corrente che si sviluppa tra due superfici piane
orizzontali infinite, a distanza \(h\) l’una dall’altra, mantenute a
temperatura costante: la temperatura vale \(T_w\) sulla superficie
inferiore e \(T_c\) sulla superficie superiore. Viene definita la
differenza di temperatura \(\Delta T = T_w - T_c\). Se le due superfici
considerate sono superfici solide, la velocità su di esse è nulla. Se le
due superfici sono superfici «libere» (di simmetria, a sforzo nullo) si
annulla la derivata normale della velocità. Prendendo un sistema di assi
ortogonali, con l’origine in corrispondenza della superficie inferiore,
con l’asse \(x\) parallelo e l’asse \(z\) perpendicolare alla superficie, si
possono riassumere così le condizioni al contorno,
\begin{equation*}
\begin{split}\text{wall: }
    \begin{cases}
      T(x,z=0) = T_w \\ T(x,z=h) = T_c \\
      \mathbf{u}(x,z=0) = \mathbf{0} \\ \mathbf{u}(x,z=h) = \mathbf{0}
    \end{cases}  \hspace{0.5cm}
    \text{free: } \left\{
    \begin{aligned}
      T(x,z=0) = T_w \ & \ , \ \ T(x,z=h) = T_c \\
      \dfrac{\partial u}{\partial z}(x,z=0) = 0 \ & \ , \ \ \dfrac{\partial u}{\partial z}(x,z=h) = 0 \\
      w(x,z=0) = 0 \ & \  , \  \  w(x,z=h) = 0 \ .
    \end{aligned} \right.\end{split}
\end{equation*}
\sphinxAtStartPar
Non ci sono condizioni al contorno in \(x\),
poichè la direzione è omogenea. Considereremo qui solo il problema con
le condizioni al contorno «free».


\subsection{Soluzione stazionaria non convettiva}
\label{\detokenize{polimi/fluidmechanics-ita/template/capitoli/07_similitudine/07teoria:soluzione-stazionaria-non-convettiva}}
\sphinxAtStartPar
Esiste una soluzione stazionaria (\(\partial / \partial t = 0\)) del
problema con fluido in quiete (\(\mathbf{u} = \mathbf{0}\)). Il vincolo di
incomprimibilità è soddisfatto identicamente. Sfruttando l’omogeneità
della direzione \(x\), la soluzione stazionaria indipendente dalla
coordinata \(x\) soddisfa le equazioni
\begin{equation*}
\begin{split}\begin{cases}
        \dfrac{1}{\rho_0}\dfrac{d P}{d z} = \alpha g (T-T_0)  \vspace{0.2cm} \\
        \dfrac{d^2 T}{d z^2} = 0 \ ,
    \end{cases}\end{split}
\end{equation*}
\sphinxAtStartPar
dotate delle opportune condizioni al contorno. La
soluzione stazionaria del problema è
\begin{equation*}
\begin{split}\begin{cases}
    \overline{T}(z) = T_w + (T_c-T_w) \dfrac{z}{h} =
    T_w - \Delta T \dfrac{z}{h} \\
    \overline{P}(z) = P_w + \alpha \rho_0 g \left[ (T_w-T_0) z
    - \dfrac{1}{2} \Delta T \dfrac{z^2}{h} \right] \ ,
\end{cases}\end{split}
\end{equation*}
\sphinxAtStartPar
avendo indicato con \(P_w\) il valore della pressione in
corrispondenza della superficie inferiore a \(z = 0\).


\subsection{Equazione delle fluttuazioni}
\label{\detokenize{polimi/fluidmechanics-ita/template/capitoli/07_similitudine/07teoria:equazione-delle-fluttuazioni}}
\sphinxAtStartPar
Viene definita la fluttuazione di temperatura \(\tau(x,z)\),
\begin{equation*}
\begin{split}\begin{aligned}
    \tau(x,z) = T(x,z) - \overline{T}(z) & = T(x,z) - T_w + \Delta T \dfrac{z}{h} \\
    \quad \rightarrow \quad T(x,z) - T_w & = \tau(x,z) - \Delta T \dfrac{z}{h} \ .
\end{aligned}\end{split}
\end{equation*}
\sphinxAtStartPar
Scegliendo la superficie inferiore a \(z = 0\) per
definire la condizione termodinamica di riferimento, \(T_0 = T_w\). le
equazioni di Boussinesq diventano
\begin{equation*}
\begin{split}\begin{cases}
      \dfrac{\partial \mathbf{u}}{\partial t} + 
      \left( \mathbf{u} \cdot \mathbf{\nabla} \right) \mathbf{u} -
      \nu \nabla^2 \mathbf{u} + \dfrac{1}{\rho_0}\mathbf{\nabla} P = \left( 1 - \alpha \tau + \Delta T \dfrac{z}{h} \right) \mathbf{g} \\
      \mathbf{\nabla} \cdot \mathbf{u} = 0 \\
      \dfrac{\partial \tau}{\partial t} + \mathbf{u} \cdot 
      \mathbf{\nabla} \tau + w \dfrac{\partial \overline{T}}{\partial z}=  D \nabla^2 \tau \ .
    \end{cases}\end{split}
\end{equation*}
\sphinxAtStartPar
Inoltre è possibile raccogliere il primo e il terzo
termine delle forze di galleggiamento sotto lo stesso operatore di
gradiente che opera sul campo di pressione. Infatti, è possibile
scrivere il termine di galleggiamento come
\begin{equation*}
\begin{split}\begin{aligned}
    \left( 1 - \alpha \tau + \Delta T \dfrac{z}{h} \right) \mathbf{g} & = \alpha \tau g \mathbf{\hat{z}} - \mathbf{\nabla} \left( gz + \Delta T \dfrac{z^2}{2 h} \right) \ .
\end{aligned}\end{split}
\end{equation*}
\sphinxAtStartPar
Definendo una «pressione generalizzata» \(P'\),
\begin{equation*}
\begin{split}P' = P + \rho_0 g z + \rho_0 \Delta T \dfrac{z^2}{2 h} \ ,\end{split}
\end{equation*}
\sphinxAtStartPar
le equazioni di Boussinesq diventano
\begin{equation*}
\begin{split}\label{eqn:Bouss-tau}
    \begin{cases}
      \dfrac{\partial \mathbf{u}}{\partial t} + 
      \left( \mathbf{u} \cdot \mathbf{\nabla} \right) \mathbf{u} -
      \nu \nabla^2 \mathbf{u} + \dfrac{1}{\rho_0}\mathbf{\nabla} P' = \alpha  g \tau \mathbf{\hat{z}} \\
      \mathbf{\nabla} \cdot \mathbf{u} = 0 \\
      \dfrac{\partial \tau}{\partial t} + \mathbf{u} \cdot 
      \mathbf{\nabla} \tau -\dfrac{\Delta T}{h} w =  D \nabla^2 \tau \ ,
    \end{cases}\end{split}
\end{equation*}
\sphinxAtStartPar
e le condizioni al contorno della temperatura vengono
espresse anch’esse in funzione di \(\tau\),
\begin{equation*}
\begin{split}\label{eqn:Bouss-tau-bc}
    \tau(x,z=0) = \tau(x,z=h) = 0 \ .\end{split}
\end{equation*}

\section{Equazioni di Boussinesq in forma adimensionale}
\label{\detokenize{polimi/fluidmechanics-ita/template/capitoli/07_similitudine/07teoria:equazioni-di-boussinesq-in-forma-adimensionale}}
\sphinxAtStartPar
Si ricava la forma adimensionale delle equazioni
(\DUrole{xref,myst}{{[}eqn:Bouss\sphinxhyphen{}tau{]}}\{reference\sphinxhyphen{}type=»ref»
reference=»eqn:Bouss\sphinxhyphen{}tau»\}) e delle condizioni al contorno
(\DUrole{xref,myst}{{[}eqn:Bouss\sphinxhyphen{}tau\sphinxhyphen{}bc{]}}\{reference\sphinxhyphen{}type=»ref»
reference=»eqn:Bouss\sphinxhyphen{}tau\sphinxhyphen{}bc»\}) utilizzando il teorema \(\pi\) di
Buckingham. Nel problema di Boussinesq compaiono 12 grandezze
dimensionali (13 se si volesse considerare la componente \(w\) della
velocità \(\mathbf{u}\) in maniera indipendente),
\begin{equation*}
\begin{split}\underbrace{\mathbf{x}, t}_{\text{tar. indip.}}, 
    \underbrace{\mathbf{u}, \tau, P'}_{\text{campi } f(\mathbf{x},t)},
    \underbrace{\rho_0, \nu, D, \alpha, g}_{\substack{ \text{\footnotesize{propr. del fluido}} \\ \text{\footnotesize{e del problema}} } }, 
    \underbrace{h, \Delta T}_{\substack{ \text{\footnotesize{dominio e}} \\ \text{\footnotesize{ condizioni al contorno}} } } \ ,\end{split}
\end{equation*}
\sphinxAtStartPar
e 4 grandezze fisiche fondamentali: massa \(M\), lunghezza \(L\), tempo \(T\)
e temperatura \(\Theta\). Secondo il teorema di Buckingham, il problema
può quindi essere caratterizzato da 8 numeri adimensionali. Utilizzando
la stessa scala di lunghezze per adimensionalizzare \(\mathbf{x}\) e \(h\) e la
stessa scala di temperature per adimensionalizzare \(\tau\) e \(\Delta T\),
sono sufficienti 6 numeri adimensionali. É necessario scegliere 4
grandezze fisiche di riferimento indipendenti e, possibilmente,
rappresentative del problema con le quali adimensionalizzare le altre.
Il problema della convezione non forzata descritto dalle equazioni di
Boussinesq è caratterizzato dalla differenza di temperatura \(\Delta T\) e
dalla distanza \(h\) delle superfici, dal fluido considerato e
dall’intensità delle forze di volume. I campi di velocità \(\mathbf{u}\), di
«temperatura» \(\tau\) e di «pressione» \(P'\) sono un risultato, una
conseguenza, delle condizioni al contorno e del fluido impiegato: non
esistono scale di velocità e pressione indipendenti, mentre il campo di
temperatura può essere scalato sulla differenza \(\Delta T\). Non esiste
nemmeno una scala indipendente dei tempi, poiché l’evoluzione del
sistema è determinata dalle condizioni al contorno e dal fluido
utilizzato. Come grandezze dimensionali di riferimento indipendenti e
caratteristiche del problema vengono scelte la densità del fluido, il
coefficiente di diffusione termica, la distanza tra le superfici e la
loro differenza di temperatura:
\begin{equation*}
\begin{split}\tilde{\rho}=\rho_0, \ \tilde{D} = D, \ \tilde{L} = h, \ \tilde{\Theta} = \Delta T \ .\end{split}
\end{equation*}
\begin{sphinxVerbatim}[commandchars=\\\{\}]
          \PYGZdl{}\PYGZbs{}mathbf\PYGZob{}x\PYGZcb{}\PYGZdl{}   \PYGZdl{}t\PYGZdl{}   \PYGZdl{}\PYGZbs{}mathbf\PYGZob{}u\PYGZcb{}\PYGZdl{}   \PYGZdl{}\PYGZbs{}tau\PYGZdl{}   \PYGZdl{}P\PYGZsq{}\PYGZdl{}   \PYGZdl{}\PYGZbs{}rho\PYGZus{}0\PYGZdl{}   \PYGZdl{}\PYGZbs{}nu\PYGZdl{}   \PYGZdl{}D\PYGZdl{}   \PYGZdl{}\PYGZbs{}alpha\PYGZdl{}   \PYGZdl{}g\PYGZdl{}   \PYGZdl{}h\PYGZdl{}   \PYGZdl{}\PYGZbs{}Delta T\PYGZdl{}
\end{sphinxVerbatim}


\bigskip\hrule\bigskip


\begin{sphinxVerbatim}[commandchars=\\\{\}]
  M                                             1        1                                           
  L          1                1                 \PYGZhy{}1       \PYGZhy{}3        2      2                1     1   
  T                   1       \PYGZhy{}1                \PYGZhy{}2                \PYGZhy{}1     \PYGZhy{}1               \PYGZhy{}2         
\end{sphinxVerbatim}

\sphinxAtStartPar
\(\Theta\)                                 1                                         \sphinxhyphen{}1                      1

\sphinxAtStartPar
: Teorema di Buckingham. Grandezze dimensionali e unità
fisiche.{[}{]}\{label=»tab:Bouss\sphinxhyphen{}pi\sphinxhyphen{}thm»\}

\sphinxAtStartPar
Ora è possibile scrivere ogni grandezza dimensionale come prodotto di
una grandezza omogenea di riferimento (dimensionale) e del suo valore
adimensionale. Si può quindi scrivere,
\begin{equation*}
\begin{split}\label{eqn:var-adim}
\begin{aligned}
    \mathbf{x} = \tilde{L} \mathbf{x}^*  \quad & , \quad t = \tilde{T} t^* \\
    \mathbf{u} = \tilde{U} \mathbf{u}^* \quad , \quad \tau & = \tilde{\Theta} \tau^* \quad , \quad P' = \tilde{P} P^{*'} \\
    \rho_0 = \tilde{\rho} \rho_0^* \quad  , \quad \nu = \tilde{\nu} \nu^* \quad , \quad D & = \tilde{D} D^* \quad , \quad \alpha = \tilde{\alpha} \alpha^* \quad , \quad g = \tilde{g} g^* \\
    h = \tilde{L} h^* \quad & , \quad \Delta T = \tilde{\Theta} \Delta T^* \ ,
\end{aligned}\end{split}
\end{equation*}
\sphinxAtStartPar
avendo utilizzato la stessa scala di lunghezza
\(\tilde{L}\) come riferimento per la coordinata spaziale indipendente
\(\mathbf{x}\) e la distanza \(h\) tra le due superifici, e la stessa scala di
temperatura \(\tilde{\Theta}\) come riferimento per il campo di
temperatura \(\tau\) e la differenza di temperatura tra le due superfici
\(\Delta T\), come anticipato in precedenza. Le 12 grandezze dimensionali
sono state adimensionalizzate usando 10 scale di riferimento: da queste
è possibile ricavare 6 numeri adimensionali con cui descrivere il
problema. Inserendo le espressioni
(\DUrole{xref,myst}{{[}eqn:var\sphinxhyphen{}adim{]}}\{reference\sphinxhyphen{}type=»ref»
reference=»eqn:var\sphinxhyphen{}adim»\}) nel problema di Boussinesq
(\DUrole{xref,myst}{{[}eqn:Bouss\sphinxhyphen{}tau{]}}\{reference\sphinxhyphen{}type=»ref»
reference=»eqn:Bouss\sphinxhyphen{}tau»\}), si ricava
\begin{equation*}
\begin{split}\label{eqn:Bouss-tau-adim-1}
    \begin{cases}
      \dfrac{\tilde{U}}{\tilde{T}}\dfrac{\partial \mathbf{u}^*}{\partial t^*} + \dfrac{\tilde{U}^2}{\tilde{L}}
      \left( \mathbf{u}^* \cdot \mathbf{\nabla}^* \right) \mathbf{u}^* -
      \dfrac{\tilde{\nu} \tilde{U}}{\tilde{L}^2} \nu^* \nabla^{*2} \mathbf{u}^* + \dfrac{\tilde{P}}{\tilde{\rho} \tilde{L}}\dfrac{1}{\rho_0^*}\mathbf{\nabla} P^{*'} = \tilde{\alpha} \tilde{g} \tilde{\theta} \alpha^*  g^* \tau^* \mathbf{\hat{z}} \\
      \dfrac{\tilde{U}}{\tilde{L}}\mathbf{\nabla}^* \cdot \mathbf{u}^* = 0 \\
      \dfrac{\tilde{\Theta}}{\tilde{T}}\dfrac{\partial \tau^*}{\partial t^*} + \dfrac{\tilde{U}\tilde{\Theta}}{\tilde{L}}\mathbf{u}^* \cdot 
      \mathbf{\nabla}^* \tau^* - \dfrac{\tilde{U}\tilde{\Theta}}{\tilde{L}}\dfrac{\Delta T^*}{h^*} w^* = \dfrac{\tilde{D}\tilde{\Theta}}{\tilde{L}^2} D^* \nabla^{*2} \tau^* \ ,
    \end{cases}\end{split}
\end{equation*}
\begin{sphinxVerbatim}[commandchars=\\\{\}]
con le conzioni al contorno \PYGZdq{}free\PYGZdq{}
\end{sphinxVerbatim}
\begin{equation*}
\begin{split}\text{free: } \left\{
    \begin{aligned}
      \tilde{\Theta}\tau^*(\tilde{L}x^*,\tilde{L}z^*=0) = 0 \quad & , \quad  
      \tilde{\Theta}\tau^*(\tilde{L}x^*,\tilde{L}z^*=\tilde{L}h^*) = 0 \\
      \dfrac{\tilde{U}}{\tilde{L}}\dfrac{\partial u^*}{\partial z^*}(\tilde{L}x^*,\tilde{L}z^*=0) = 0 \quad  & , \quad 
      \dfrac{\tilde{U}}{\tilde{L}}\dfrac{\partial u}{\partial z^*}(\tilde{L}x^*,\tilde{L}z^*=\tilde{L}h^*) = 0 \\
      \tilde{U} w^*(\tilde{L}x^*,\tilde{L}z^*=0) = 0 \quad  & , \quad 
      \tilde{U} w^*(\tilde{L}x^*,\tilde{L}z^*=\tilde{L}h^*) = 0
    \end{aligned} \right.\end{split}
\end{equation*}
\begin{sphinxVerbatim}[commandchars=\\\{\}]
Con un abuso di notazione, d\PYGZsq{}ora in poi si
\end{sphinxVerbatim}

\sphinxAtStartPar
indicano le grandezze adimensionali senza asterisco e i campi
adimensionali vengono definiti come funzione delle variabili
indipendenti adimensionali,
\begin{equation*}
\begin{split}\mathbf{u}(\mathbf{x},t) = \tilde{U} \mathbf{u}^*(\tilde{L} \mathbf{x}^*, \tilde{T} t^*) \quad \rightarrow \quad \tilde{U} \mathbf{u}(\mathbf{x},  t) \ .\end{split}
\end{equation*}
\sphinxAtStartPar
Le grandezze di riferimento delle grandezze costanti vengono scelte
coincidenti con la grandezza stessa, cosicché le grandezze adimensionali
relative sono uguali a 1,
\begin{equation*}
\begin{split}\label{eqn:var-adim-2}
\begin{aligned}
    \rho_0 = \tilde{\rho} \quad  , \quad \nu = \tilde{\nu} \quad , \quad D & = \tilde{D}  \quad , \quad \alpha = \tilde{\alpha}  \quad , \quad g = \tilde{g}  \\
    h = \tilde{L}  \quad & , \quad \Delta T = \tilde{\Theta}  \ .
\end{aligned}\end{split}
\end{equation*}
\sphinxAtStartPar
Dividendo l’equazione della quantità di moto per
\(\tilde{\nu}\tilde{U}/\tilde{L}^2\), il vincolo di incomprimibilità per
\(\tilde{U}/\tilde{L}\) e l’equazione dell’energia per
\(\tilde{D}\tilde{\Theta}/\tilde{L}^2\), il problema di Boussinesq diventa
\begin{equation*}
\begin{split}\label{eqn:Bouss-tau-adim-2}
    \begin{cases}
      \dfrac{\tilde{L}^2}{\tilde{\nu}\tilde{T}}\dfrac{\partial \mathbf{u}^*}{\partial t^*} + \dfrac{\tilde{U}\tilde{L}}{\tilde{\nu}}
      \left( \mathbf{u}^* \cdot \mathbf{\nabla}^* \right) \mathbf{u}^* -
      \nabla^{*2} \mathbf{u}^* + \dfrac{\tilde{P}\tilde{L}}{\tilde{\rho} \tilde{\nu} \tilde{U}}\mathbf{\nabla} P^{*'} = \dfrac{\tilde{\alpha} \tilde{g} \tilde{\Theta} \tilde{L}^2}{\tilde{\nu} \tilde{U}} \tau^* \mathbf{\hat{z}} \\
      \mathbf{\nabla}^* \cdot \mathbf{u}^* = 0 \\
      \dfrac{\tilde{L}^2}{\tilde{D}\tilde{T}}\dfrac{\partial \tau^*}{\partial t^*} + \dfrac{\tilde{U}\tilde{L}}{\tilde{D}}\mathbf{u}^* \cdot 
      \mathbf{\nabla}^* \tau^* - \dfrac{\tilde{U}\tilde{L}}{\tilde{D}} w^* = \nabla^{*2} \tau^* \ ,
    \end{cases}\end{split}
\end{equation*}
\sphinxAtStartPar
con le conzioni al contorno «free»
\begin{equation*}
\begin{split}\label{eqn:Bouss-adim-2-bc}
    \text{free: }
    \left\{
    \begin{aligned}
      \tau^*(x^*,z^*=0) = 0 \qquad  & , \qquad 
      \tau^*(x^*,z^*=1) = 0 \\
      \dfrac{\partial u^*}{\partial z^*}(x^*,z^*=0) = 0 \qquad & , \qquad 
      \dfrac{\partial u^*}{\partial z^*}(x^*,z^*=1) = 0 \\
      w^*(x^*,z^*=0) = 0 \qquad & , \qquad w^*(x^*,z^*=1) = 0 \ .
    \end{aligned} \right.\end{split}
\end{equation*}
\sphinxAtStartPar
Nel problema
(\DUrole{xref,myst}{{[}eqn:Bouss\sphinxhyphen{}tau\sphinxhyphen{}adim\sphinxhyphen{}2{]}}\{reference\sphinxhyphen{}type=»ref»
reference=»eqn:Bouss\sphinxhyphen{}tau\sphinxhyphen{}adim\sphinxhyphen{}2»\}\sphinxhyphen{}\DUrole{xref,myst}{{[}eqn:Bouss\sphinxhyphen{}adim\sphinxhyphen{}2\sphinxhyphen{}bc{]}}\{reference\sphinxhyphen{}type=»ref»
reference=»eqn:Bouss\sphinxhyphen{}adim\sphinxhyphen{}2\sphinxhyphen{}bc»\}) compaiono 6 numeri adimensionali.
Siamo arrivati al risultato previsto dal teorema di Buckingham. Prima di
andare avanti, conviene comunque fare un’osservazione. Solo 5 dei 6
numeri adimensionali trovati sono tra di loro indipendenti: in
particolare solo 3 dei 4 numeri adimensionali
\begin{equation*}
\begin{split}\pi_1 = \frac{\tilde{L}^2}{\tilde{D}\tilde{T}} \ , \quad
    \pi_2 = \frac{\tilde{U}\tilde{L}}{\tilde{D}} \ , \quad
    \pi_3 = \frac{\tilde{L}^2}{\tilde{\nu}\tilde{T}} \ , \quad 
    \hat{\pi}_4 = \frac{\tilde{U}\tilde{L}}{\tilde{\nu}} = \pi_2 \dfrac{\pi_3}{\pi_1}\end{split}
\end{equation*}
\sphinxAtStartPar
sono linearmente indipendenti. Sembra di aver commesso un errore poiché
abbiamo trovato una contraddizione del teorema di Buckingham.
L’apparente errore si nasconde nel termine adimensionale
\(\frac{\tilde{\alpha} \tilde{g} \tilde{\theta} \tilde{L}^2}{\tilde{\nu} \tilde{U}}\).
Questo termine infatti è il prodotto dei numeri adimensionali
\(\tilde{\alpha} \tilde{\theta}\) e
\(\frac{\tilde{g} \tilde{L}^2}{\tilde{\nu} \tilde{U}}\). I sei numeri
adimensionali indipendenti che caratterizzano il problema sono quindi
\begin{equation*}
\begin{split}\begin{aligned}
    \pi_1 = \frac{\tilde{L}^2}{\tilde{D}\tilde{T}} \quad ,\quad
    \pi_2 & = \frac{\tilde{U}\tilde{L}}{\tilde{D}} \quad ,\quad
    \pi_3 = \frac{\tilde{L}^2}{\tilde{\nu}\tilde{T}} \\
    \pi_4 = \frac{\tilde{P}\tilde{L}}{\tilde{\rho}\tilde{\nu}\tilde{U}} \quad ,\quad
    \pi_5 & = {\tilde{\alpha}\tilde{\Theta}} \quad ,\quad
    \pi_6 = \frac{\tilde{g} \tilde{L}^2}{\tilde{\nu} \tilde{U}} \ .
\end{aligned}\end{split}
\end{equation*}
\sphinxAtStartPar
Poiché il coefficiente di dilatazione termica \(\alpha\) e
la forza per unità di volume \(g\) comapiono sempre attraverso il loro
prodotto, questo si può considerare come un’unica variabile, \(\alpha g\).
In questo caso, i 5 numeri adimensionali che descrivono il problema
composto dalle 9 (10\sphinxhyphen{}1) scale di riferimento sono
\begin{equation*}
\begin{split}\pi'_1 = \pi_1, \  \pi'_2 = \pi_2, \ \pi'_3 = \pi_3, \ \pi'_4 = \pi_4, \ \pi'_5 = \pi_5 \pi_6 \ .\end{split}
\end{equation*}
\sphinxAtStartPar
Non essendoci scale di velocità, tempo e pressione indipendenti, è
possibile definire queste scale a partire dalle 4 grandezze fisiche di
riferimento \(\tilde{L}\), \(\Delta \tilde{T}\), \(\tilde{\rho}\),
\(\tilde{D}\), imponendo il valore unitario di alcuni parametri
adimensionali,
\begin{equation*}
\begin{split}\begin{aligned}
      \pi'_1 = 1 & \quad \rightarrow \quad \tilde{T} = \dfrac{\tilde{L}^2}{\tilde{D}} \\
      \pi'_2 = 1 & \quad \rightarrow \quad \tilde{U} = \dfrac{\tilde{D}}{\tilde{L}} \\
      \pi'_4 = 1 & \quad \rightarrow \quad \tilde{P} = \dfrac{\tilde{\rho}\tilde{\nu}\tilde{U}}{\tilde{L}} \ .
    \end{aligned}\end{split}
\end{equation*}
\sphinxAtStartPar
Gli unici due parametri adimensionali caratteristici
del problema rimangono
\begin{equation*}
\begin{split}\begin{aligned}
 \Pi_1 = \pi'_3 = \dfrac{\tilde{L^2}}{\tilde{\nu} \tilde{L}^2/\tilde{D}} \qquad \rightarrow \qquad \Pi_1  & = \dfrac{\tilde{D}}{\tilde{\nu}} = \dfrac{1}{Pr} \\
 \Pi_5 = \pi'_5 = \dfrac{\tilde{\alpha}\tilde{\Theta} \tilde{g} \tilde{L}^2}{\tilde{\nu} \tilde{D}/\tilde{L}} \qquad \rightarrow \qquad \Pi_5 & = \dfrac{\tilde{\alpha}\tilde{g}\tilde{\Theta}  \tilde{L}^3}{\tilde{\nu} \tilde{D}} = Ra = \\
 & = \dfrac{\tilde{\alpha}\tilde{g}\tilde{\Theta}  \tilde{L}^3}{\tilde{\nu}^2}\dfrac{\tilde{\nu}}{\tilde{D}} = Gr \, Pr \ ,
\end{aligned}\end{split}
\end{equation*}
\sphinxAtStartPar
nei quali si possono riconoscere i numeri di Prandtl,
\(Pr\), di Rayleigh, \(Ra\), e di Grashof, \(Gr\). La forma adimensionale del
problema di Boussinesq tra due superfici piane è quindi
\begin{equation*}
\begin{split}\label{eqn:Bouss-tau-adim-3}
    \begin{cases}
      \dfrac{1}{Pr} \left[ \dfrac{\partial \mathbf{u}}{\partial t} +
      \left( \mathbf{u} \cdot \mathbf{\nabla} \right) \mathbf{u} \right] -
      \nabla^2 \mathbf{u} + \mathbf{\nabla} P' = Ra \, \tau \mathbf{\hat{z}} \\
      \mathbf{\nabla} \cdot \mathbf{u} = 0 \\
      \dfrac{\partial \tau}{\partial t} + \mathbf{u} \cdot 
      \mathbf{\nabla} \tau -  w = \nabla^{2} \tau \ ,
\end{cases}\end{split}
\end{equation*}
\sphinxAtStartPar
con le conzioni al contorno «free»
\begin{equation*}
\begin{split}\label{eqn:Bouss-adim-3-bc}
    \text{free: }
    \left\{
    \begin{aligned}
      \tau(x,z=0) = 0 \qquad  & , \qquad 
      \tau(x,z=1) = 0 \\
      \dfrac{\partial u}{\partial z}(x,z=0) = 0 \qquad & , \qquad 
      \dfrac{\partial u}{\partial z}(x,z=1) = 0 \\
      w(x,z=0) = 0 \qquad & , \qquad w(x,z=1) = 0 \ .
\end{aligned} \right.\end{split}
\end{equation*}

\section{Equazione della vorticità e funzione di corrente nell’approssimazione di Boussinesq}
\label{\detokenize{polimi/fluidmechanics-ita/template/capitoli/07_similitudine/07teoria:equazione-della-vorticita-e-funzione-di-corrente-nell-approssimazione-di-boussinesq}}
\sphinxAtStartPar
Dall’equazione della quantità di moto in
(\DUrole{xref,myst}{{[}eqn:Bouss\sphinxhyphen{}tau\sphinxhyphen{}adim\sphinxhyphen{}3{]}}\{reference\sphinxhyphen{}type=»ref»
reference=»eqn:Bouss\sphinxhyphen{}tau\sphinxhyphen{}adim\sphinxhyphen{}3»\}) è possibile ricavare l’equazione
della vorticità, applicandole l’operatore di rotore. Poichè il problema
è piano e bidimensionale, il campo di vorticità ha componente non nulla
solo fuori dal piano \(xz\). Utilizzando un sistema di coordinate
cartesiano, il campo di vorticità

\sphinxAtStartPar
\(\mathbf{\omega}(x,z,t) = \xi(x,z,t) \mathbf{\hat{y}}\) soddisfa l’equazione
\$\(\dfrac{1}{Pr} \left[ \dfrac{\partial \xi}{\partial t} +
      \mathbf{u} \cdot \mathbf{\nabla} \xi \right] -
      \nabla^2 \xi = - Ra \, \dfrac{\partial \tau}{\partial x} \ .\)\$

\sphinxAtStartPar
Si può poi introdurre la funzione di corrente \(\psi\),
\begin{equation*}
\begin{split}u =   \dfrac{\partial \psi}{ \partial z} \quad , \quad 
    w = - \dfrac{\partial \psi}{ \partial x} \ ,\end{split}
\end{equation*}
\sphinxAtStartPar
in modo tale da soddisfare identicamente il vincolo di incomprimibilità. La componente
\(y\) del campo di vorticità diventa
\begin{equation*}
\begin{split}\xi = \dfrac{\partial u}{\partial z} - \dfrac{\partial w}{\partial x} =
    \dfrac{\partial^2 u}{\partial z^2} +
    \dfrac{\partial^2 w}{\partial x^2} = 
    \nabla^2 \psi \ ,\end{split}
\end{equation*}
\sphinxAtStartPar
e il termine advettivo di una funzione \(f\)
qualsiasi può essere scritta come un determinante,
\begin{equation*}
\begin{split}\mathbf{u} \cdot \mathbf{\nabla} f = u \dfrac{\partial f}{\partial x} + w \dfrac{\partial f}{\partial z} =
    \dfrac{\partial \psi}{ \partial z} \dfrac{\partial f}{\partial x} - \dfrac{\partial \psi}{ \partial x} \dfrac{\partial f}{\partial z} = \left| \begin{matrix} f_x & f_z \\ \psi_x & \psi_z \end{matrix} \right| =: \dfrac{\partial(f,\psi)}{\partial(x,z)} \ .\end{split}
\end{equation*}
\sphinxAtStartPar
Il sistema di equazioni del problema di Boussinesq diventa quindi
\begin{equation*}
\begin{split}\label{eqn:Bouss-vort-psi-tau}
    \begin{cases}
      \dfrac{\partial \xi}{\partial t} +
      \dfrac{\partial(\xi,\psi)}{\partial(x,z)} 
      = Pr \, \nabla^2 \xi 
      - Pr \, Ra \, \dfrac{\partial \tau}{\partial x} \\
      \dfrac{\partial \tau}{\partial t} +
      \dfrac{\partial(\tau,\psi)}{\partial(x,z)} =
      \nabla^{2} \tau + w \ .
\end{cases}\end{split}
\end{equation*}

\section{Approssimazione di Fourier–Galerkin del problema di Boussinesq}
\label{\detokenize{polimi/fluidmechanics-ita/template/capitoli/07_similitudine/07teoria:approssimazione-di-fourier-galerkin-del-problema-di-boussinesq}}
\sphinxAtStartPar
Utilizzando la geometria del dominio, è possibile espandere le funzioni
che compaiono nelle equazioni
(\DUrole{xref,myst}{{[}eqn:Bouss\sphinxhyphen{}vort\sphinxhyphen{}psi\sphinxhyphen{}tau{]}}\{reference\sphinxhyphen{}type=»ref»
reference=»eqn:Bouss\sphinxhyphen{}vort\sphinxhyphen{}psi\sphinxhyphen{}tau»\}) come somma di prodotti di funzioni
armoniche in \(x\) e \(z\), la cui ampiezza dipende dal tempo
\begin{equation*}
\begin{split}\label{eqn:harm-1}
\begin{aligned}
    \psi(x,z,t) & = \sum_m \sum_k a_{m,k}(t) \sin{(m\pi z + \phi^a_m)}\sin{(k\pi x + \phi^a_k)} \\
    \tau(x,z,t) & = \sum_m \sum_k b_{m,k}(t) \sin{(m\pi z + \phi^b_m)}\sin{(k\pi x + \phi^b_k)} \ .
\end{aligned}\end{split}
\end{equation*}
\sphinxAtStartPar
Le condizioni al contorno
(\DUrole{xref,myst}{{[}eqn:Bouss\sphinxhyphen{}adim\sphinxhyphen{}3\sphinxhyphen{}bc{]}}\{reference\sphinxhyphen{}type=»ref»
reference=»eqn:Bouss\sphinxhyphen{}adim\sphinxhyphen{}3\sphinxhyphen{}bc»\}) del problema con due superfici
infinite «free» impongono che la componente
\(w=-\partial{\psi}/\partial{x}\) e la derivata
\(\partial u/\partial z = \partial^2 \psi/\partial z^2\) siano nulle per
\(z = 0, \ 1\) per ogni istante temporale e per ogni valore di \(x\). Le
condizioni al contorno su \(w\) impongono le seguenti condizioni
sull’espanzione armonica delle funzioni,
\begin{equation*}
\begin{split}\begin{aligned}
      0 = \dfrac{\partial \psi}{\partial x}\Big|_{z=0} & = \sum_m \sum_k k \pi a_{m,k}(t) \sin{ \phi^a_m }\cos{(k\pi x + \phi^a_k)} \\
      & \hspace{4.0cm} \rightarrow \qquad \phi^a_m = 0 \ , \\
      0 = \dfrac{\partial \psi}{\partial x}\Big|_{z=1} & = \sum_m \sum_k k \pi a_{m,k}(t) \sin{ m \pi }\cos{(k\pi x + \phi^a_k)} \\
      & \hspace{4.0cm} \rightarrow \qquad m \in \mathbb{Z} \ .
\end{aligned}\end{split}
\end{equation*}
\sphinxAtStartPar
Le stesse condizioni derivano dalle condizioni al
contorno su \(\partial u/\partial z\). Le condizioni al contorno sulla
temperatura in impongono le seguenti condizioni sull’espansione armonica
della funzione \(\tau\)
\begin{equation*}
\begin{split}\begin{aligned}
      0 = \tau \Big|_{z=0} & = \sum_m \sum_k b_{m,k}(t) \sin{ \phi^b_m }\sin{(k\pi x + \phi^b_k)} \\
      & \hspace{4.0cm} \rightarrow \qquad \phi^b_m = 0 \ ,\\
      0 = \tau \Big|_{z=1} & = \sum_m \sum_k b_{m,k}(t) \sin{ m \pi }\sin{(k\pi x + \phi^b_k)} \\
      & \hspace{4.0cm} \rightarrow \qquad m \in \mathbb{Z} \ .
\end{aligned}\end{split}
\end{equation*}
\sphinxAtStartPar
A causa dell’omogeneità della direzione \(x\), nella
quale il dominio è infinito, non ci sono condizioni sul numero d’onda
\(k\), che può assumere tutti i valori \(\in \mathbb{R}\), e sulla fase
delle armoniche in \(x\). Le espansioni
(\DUrole{xref,myst}{{[}eqn:harm\sphinxhyphen{}1{]}}\{reference\sphinxhyphen{}type=»ref»
reference=»eqn:harm\sphinxhyphen{}1»\}) possono quindi essere scritte come
\begin{equation*}
\begin{split}\label{eqn:harm-2}
\begin{aligned}
    \psi(x,z,t) & = \sum_{m \in \mathbb{Z}} \sum_k a^1_{m,k}(t) \sin{(m\pi z)} \sin{(k\pi x )} +  a^2_{m,k}(t) \sin{(m\pi z)} \cos{(k \pi x)} \\
    \tau(x,z,t) & = \sum_{m \in \mathbb{Z}} \sum_k b^1_{m,k}(t) \sin{(m\pi z)} \sin{(k\pi x )} + b^2_{m,k}(t) \sin{(m\pi z)} \cos{(k \pi x)}  \ .
\end{aligned}\end{split}
\end{equation*}

\section{Dal problema di Boussinesq al modello di Lorenz}
\label{\detokenize{polimi/fluidmechanics-ita/template/capitoli/07_similitudine/07teoria:dal-problema-di-boussinesq-al-modello-di-lorenz}}
\sphinxAtStartPar
Le espansioni (\DUrole{xref,myst}{{[}eqn:harm\sphinxhyphen{}2{]}}\{reference\sphinxhyphen{}type=»ref»
reference=»eqn:harm\sphinxhyphen{}2»\}) possono essere \sphinxstyleemphasis{brutalmente} troncate per
ottenere un modello dinamico di dimensione \(N_d = 3\) partendo dal
modello continuo, che ha dimensione infinita. Le espansioni
(\DUrole{xref,myst}{{[}eqn:harm\sphinxhyphen{}2{]}}\{reference\sphinxhyphen{}type=»ref»
reference=»eqn:harm\sphinxhyphen{}2»\}) vengono troncate mantenendo solo 3 termini
\begin{equation*}
\begin{split}\label{eqn:harm-3}
\begin{aligned}
    \psi(x,z,t) & = a(t) \sin{(\pi z)} \sin{(k\pi x )}  \\
    \tau(x,z,t) & = b(t) \sin{(\pi z)} \cos{(k\pi x )} + c(t) \sin{(2 \pi z)}   \ ,
\end{aligned}\end{split}
\end{equation*}
\sphinxAtStartPar
avendo definito \(a(t) = a^1_{1,k}(t)\),
\(b(t) = b^2_{1,k}(t)\), \(c(t) = b^1_{2,0}(t)\). Usando le espansioni
(\DUrole{xref,myst}{{[}eqn:harm\sphinxhyphen{}3{]}}\{reference\sphinxhyphen{}type=»ref»
reference=»eqn:harm\sphinxhyphen{}3»\}), la componente \(y\) del campo di vorticità
\(\xi = \nabla^2 \psi\) diventa
\begin{equation*}
\begin{split}\xi = - \pi^2 (1 + k^2) \psi = - \pi^2 (1 + k^2) a(t) \sin{(\pi z)} \sin{(k\pi x )}\end{split}
\end{equation*}
\sphinxAtStartPar
I due determinanti che compaiono nelle equazioni
(\DUrole{xref,myst}{{[}eqn:Bouss\sphinxhyphen{}vort\sphinxhyphen{}psi\sphinxhyphen{}tau{]}}\{reference\sphinxhyphen{}type=»ref»
reference=»eqn:Bouss\sphinxhyphen{}vort\sphinxhyphen{}psi\sphinxhyphen{}tau»\}) valgono
\begin{equation*}
\begin{split}\begin{aligned}
     \dfrac{\partial (\xi, \psi)}{\partial (x,z)} = &
     \left[ -\pi^3 k(1+k^2) a(t) \sin{(\pi z)}  \cos{(k \pi x)} \right]
     \left[ a(t) \pi \cos{(\pi z)} \sin{(k \pi x)} \right] + \\
      - & \left[  -\pi^3 (1+k^2) a(t) \cos{(\pi z)}  \sin{(k \pi x)} \right]
      \left[ a(t) \pi k \sin{(\pi z)} \cos{(k \pi x)} \right] = 0 \ ,
\end{aligned}\end{split}
\end{equation*}
\sphinxAtStartPar
e
\begin{equation*}
\begin{split}\begin{aligned}
    \dfrac{\partial (\tau, \psi)}{\partial (x,z)} = &
    \left[ - \pi k b(t) \sin{(\pi z )} \sin{(k \pi x)} \right]
    \left[ a(t) \pi \cos{(\pi z)} \sin{(k \pi x)} \right] + \\
    - & \left[ \pi b(t) \cos{(\pi z )} \cos{(k \pi x)} + 2\pi c(t) \cos{(2\pi z)} \right]
    \left[ a(t) \pi k \sin{(\pi z)} \cos{(k \pi x)} \right] = \\
    = & - k \pi^2 a(t) b(t) \dfrac{\sin{( 2 \pi z)}}{2} -
    2 k \pi^2 a(t)c(t) \sin(\pi z) \cos(2\pi z) \cos(k \pi x) \ .
\end{aligned}\end{split}
\end{equation*}
\sphinxAtStartPar
I laplaciani che compaiono nelle equazioni
(\DUrole{xref,myst}{{[}eqn:Bouss\sphinxhyphen{}vort\sphinxhyphen{}psi\sphinxhyphen{}tau{]}}\{reference\sphinxhyphen{}type=»ref»
reference=»eqn:Bouss\sphinxhyphen{}vort\sphinxhyphen{}psi\sphinxhyphen{}tau»\}) valgono
\begin{equation*}
\begin{split}\nabla^2 \xi = -\pi^2 (1+k^2) \xi = \pi^4 (1+k^2)^2 a(t) \sin{(\pi z)} \sin{(k \pi x)} \ ,\end{split}
\end{equation*}
\sphinxAtStartPar
e
\begin{equation*}
\begin{split}\nabla^2 \tau = -\pi^2 (1+k^2) b(t) \sin{(\pi z)}\cos{(k \pi x)} - 4 \pi^2 c(t) \sin{(2\pi x)} \ .\end{split}
\end{equation*}
\sphinxAtStartPar
Il numero di Prantl viene indicato come \(Pr = \sigma\), il numero di
Rayleigh come \(Ra = R\). Inserendo le espansioni
(\DUrole{xref,myst}{{[}eqn:harm\sphinxhyphen{}3{]}}\{reference\sphinxhyphen{}type=»ref»
reference=»eqn:harm\sphinxhyphen{}3»\}) all’interno delle equazioni
(\DUrole{xref,myst}{{[}eqn:Bouss\sphinxhyphen{}vort\sphinxhyphen{}psi\sphinxhyphen{}tau{]}}\{reference\sphinxhyphen{}type=»ref»
reference=»eqn:Bouss\sphinxhyphen{}vort\sphinxhyphen{}psi\sphinxhyphen{}tau»\}), il problema troncato di Boussinesq
diventa,
\begin{equation*}
\begin{split}\begin{cases}
- \sigma \pi^2 (1+k^2) \dot{a}(t) \sin{(\pi z)} \sin{(k\pi x )} = \sigma \pi^4 (1+k^2)^2 a(t) \sin{(\pi z)} \sin{(k \pi x)} + \\
      \hspace{6.0cm} + \sigma R \, \pi k \, b(t) \sin{(\pi z)}\sin{(k \pi x)} \\
     \dot{b}(t)\sin{(\pi z)} \cos{(k\pi x )} + \dot{c}(t) \sin{(2\pi x)} + \\
     \hspace{2.0cm} - k \pi^2 a(t) b(t) \dfrac{\sin{(2 \pi z)}}{2} -
    2 k \pi^2 a(t)c(t) \sin(\pi z) \cos(2\pi z) \cos(k \pi x) = \\ 
    \hspace{1.5cm} = -\pi^2 (1+k^2) b(t) \sin{(\pi z)}\cos{(k \pi x)} - 4 \pi^2 c(t) \sin{(2\pi x)} + \\
    \hspace{2.0cm} - \pi k a(t) \sin{(\pi z)} \cos{(k \pi x)} \ .
    \end{cases}\end{split}
\end{equation*}
\sphinxAtStartPar
Raccogliendo il termine
\(\sin{(\pi z)} \sin{(k \pi x)}\) nell’equazione della vorticità si
ottiene l’equazione
\begin{equation*}
\begin{split}- \pi^2 (1+k^2) \dot{a} = \sigma \pi^4 (1+k^2)^2 a(t) +  \sigma R \, \pi k \, b(t) \ .\end{split}
\end{equation*}
\sphinxAtStartPar
L’equazione della temperatura viene «proiettata» sulle funzioni di base
\(\sin{(\pi z)} \cos{(k \pi x)}\) e \(\sin{(2 \pi x)}\) e sfruttando
l’ortogonalità delle funzioni armoniche. La proiezione consiste nella
moltiplicazione dell’equazione per le funzioni di base
\(\sin{(\pi z)} \cos{(k \pi x)}\) e nell’integrazione in
\((x,z) \in \left[0,\frac{2}{k}\right]\times\left[0,1\right]\). Usando il
valore degli integrali,
\begin{equation*}
\begin{split}\begin{aligned}
 \int_{x=0}^{2/k} \sin{(k \pi x)}^2 dx & = \dfrac{1}{2} \int_{x=0}^{2/k} \left[ 1 - \cos{(2 k \pi x)} \right] dx = \dfrac{1}{k} \\
 \int_{x=0}^{2/k} \sin{(k \pi x)}\cos{(k \pi x)} dx & = \dfrac{1}{k \pi} \int_{x=0}^{2/k} \sin{(k \pi x)} d \big( \sin{(k \pi x)} \big) = 0 \\
 \int_{x=0}^{2/k} \cos{(k \pi x)}^2 dx & = \dfrac{1}{k} \ ,
\end{aligned}\end{split}
\end{equation*}
\sphinxAtStartPar
e degli integrali
\begin{equation*}
\begin{split}\begin{aligned}
 \int_{z=0}^{1} \sin{(\pi z)}^2 dz & = \dfrac{1}{2} \int_{x=0}^{1} \left[ 1 - \cos{(2\pi z)} \right] dz = \dfrac{1}{2} \\
 \int_{z=0}^{1} \sin{(\pi z)} \sin{(\pi z)} \cos{(2\pi z)} dz  & =
 \dfrac{1}{2} \int_{z=0}^{1} \left[ 1 - \cos{(2\pi z)} \right]  \cos{(2\pi z)} dz = \\
 & = - \dfrac{1}{2} \int_{z=0}^{1} \cos^2{(2\pi z)}  dz = 
  - \dfrac{1}{4} \ .
\end{aligned}\end{split}
\end{equation*}
\sphinxAtStartPar
La proiezione dell’equazione della vorticità sulla
funzione \(\sin{(\pi z)} \cos{(k \pi x)}\) è
\begin{equation*}
\begin{split}\dfrac{1}{2}\dot{b}(t) - \dfrac{1}{4} 2 k \pi^2 a(t) c(t) =
    -\dfrac{1}{2} \pi^2 (1+k^2) b(t) - \dfrac{1}{2}\pi k a(t) \ ,\end{split}
\end{equation*}
\sphinxAtStartPar
mentre la proiezione dell’equazione della vorticità sulla funzione
\(\sin{(2 \pi z)}\) è
\begin{equation*}
\begin{split}\dfrac{1}{k}\dot{c}(t) - \dfrac{1}{k} \dfrac{\pi^2 k}{2} a(t) b(t) =
    - \dfrac{1}{k} 4 \pi^2 c(t) \ .\end{split}
\end{equation*}
\sphinxAtStartPar
Le equazioni diventano quindi
\begin{equation*}
\begin{split}\begin{cases}
   - \pi^2 (1+k^2) \dot{a} = \sigma \pi^4 (1+k^2)^2 a(t) +  \sigma R \, \pi k \, b(t) \\
    \dot{b} = -\pi^2 (1+k^2) b(t) + \pi^2 k a(t)c(t)  - \pi k a(t)  \\
    \dot{c} = \dfrac{\pi^2 k}{2} a(t) b(t) - 4 \pi^2 c(t) \ .
\end{cases}\end{split}
\end{equation*}
\sphinxAtStartPar
Partendo da queste equazioni, si introduce qualche
cambio di variabile per riportarsi all’espressione classica del sistema
di Lorenz. Viene introdotto il tempo \(t' = \pi^2 (k^2 + 1) t\), cosicché
\begin{equation*}
\begin{split}\dot{f} = \dfrac{df}{dt} = \dfrac{dt'}{dt}\dfrac{df}{dt'} =
    \pi^2 (k^2 + 1) \dfrac{df}{dt'} \ .\end{split}
\end{equation*}
\sphinxAtStartPar
Con un abuso di notazione,
d’ora in poi si indica \(\dot{(\ )}\) la derivata rispetto a \(t'\). La
stessa variabile \(t'\) viene indicata con \(t\). Le equazioni diventano
\begin{equation*}
\begin{split}\begin{cases}
    \dot{a}(t) = \sigma a(t) +  \sigma R \dfrac{k}{\pi^3 (k^2+1)^2} b(t) \\
    \dot{b}(t) = - b(t) + \dfrac{ k}{k^2+1} a(t)c(t)  - \dfrac{ k}{\pi(k^2+1)} a(t)  \\
    \dot{c}(t) = \dfrac{k}{2(k^2+1)} a(t) b(t) - \dfrac{4}{k^2+1}  c(t) \ .
\end{cases}\end{split}
\end{equation*}
\sphinxAtStartPar
Viene definito infine il cambio di variabili
\begin{equation*}
\begin{split}\begin{cases}
     X(t) = \dfrac{k}{\sqrt{2}(k^2+1)} a(t) \\
     Y(t) = \dfrac{k}{\sqrt{2}(k^2+1)}
     \left[-\dfrac{R k}{\pi^3 (k^2+1)^2}\right] b(t) \\
     Z(t) = \left[-\dfrac{R k^2}{\pi^3 (k^2+1)}\right] c(t)
\end{cases}\end{split}
\end{equation*}
\sphinxAtStartPar
che porta alla forma classica del sistema dinamico di
Lorenz
\begin{equation*}
\begin{split}\begin{cases}
      \dot{X} = - \sigma X + \sigma Y \\
      \dot{Y} = - Y + \rho X - X Z \\
      \dot{Z} = - \beta Z + X Y \ ,
    \end{cases}\end{split}
\end{equation*}
\sphinxAtStartPar
avendo definito i parameteri
\begin{equation*}
\begin{split}\rho = \dfrac{R k^2}{\pi^4 (k^2+1)^2} \qquad , \qquad
    \beta = \dfrac{4}{k^2+1} \ .\end{split}
\end{equation*}
\sphinxstepscope


\chapter{Aerodynamics}
\label{\detokenize{polimi/fluidmechanics-ita/template/capitoli/08_aerodinamica/08teoria:aerodynamics}}\label{\detokenize{polimi/fluidmechanics-ita/template/capitoli/08_aerodinamica/08teoria:fluid-mechanics-aerodynamics}}\label{\detokenize{polimi/fluidmechanics-ita/template/capitoli/08_aerodinamica/08teoria::doc}}
\sphinxAtStartPar
Per correnti irrotazionali (\(\omega = \mathbf{0}\)) in un dominio
semplicemente connesso (\(\mathbf{u} = \mathbf{\nabla} \phi\)) di fluidi
incomprimibili (\(\mathbf{\nabla} \cdot \mathbf{u} = 0\)), il potenziale cinetico
soddisfa l’equazione di Laplace \(\Delta \phi = 0\). Infatti, inserendo
nel vincolo di incomprimibilità la relazione che lega il potenziale
cinetico alla velocità si ottiene
\begin{equation*}
\begin{split}0 = \mathbf{\nabla} \cdot \mathbf{u} = \mathbf{\nabla} \cdot (\mathbf{\nabla} \phi) = \nabla^2 \phi = \Delta \phi .\end{split}
\end{equation*}
\sphinxAtStartPar
Come nel caso della seconda e della terza forma del teorema di Bernoulli
per fluidi viscosi, vedi introduzione al capitolo
§\DUrole{xref,myst}{{[}ch:bernoulli{]}}\{reference\sphinxhyphen{}type=»ref»
reference=»ch:bernoulli»\}, l’ipotesi di fluido non viscoso non è
direttamente necessaria per ottenere l’equazione di Laplace per il
potenziale. L’ipotesi di fluido non viscoso rientra però nel requisito
che la corrente sia irrotazionale. L’equazione della vorticità per
fluido incomprimibile è
\begin{equation*}
\begin{split}\frac{\partial \mathbf{\omega}}{\partial t} + (\mathbf{u} \cdot \mathbf{\nabla}) \mathbf{\omega} = (\mathbf{\omega} \cdot \mathbf{\nabla}) \mathbf{u} + \nu \Delta \mathbf{\omega} ,\end{split}
\end{equation*}
\sphinxAtStartPar
che per un fluido non viscoso, si riduce a
\begin{equation*}
\begin{split}\frac{\partial \mathbf{\omega}}{\partial t} + (\mathbf{u} \cdot \mathbf{\nabla}) \mathbf{\omega} = (\mathbf{\omega} \cdot \mathbf{\nabla}) \mathbf{u}  \qquad , \qquad
 \dfrac{D \mathbf{\omega}}{D t} = (\mathbf{\omega} \cdot \mathbf{\nabla}) \mathbf{u} ,\end{split}
\end{equation*}
\sphinxAtStartPar
dove è stata messa in evidenza la derivata materiale della vorticità,
che rappresenta la variazione della vorticità di una particella fluida,
che si muove con la velocità del fluido. Se si considera un problema in
cui un corpo aerodinamico è investito da una corrente che è uniforme
all’infinito a monte, la vorticità all’infinito a monte è nulla: si può
dimostrare facilmente allora che \(D\mathbf{\omega} / D t = \mathbf{0}\), e quindi
la vorticità si mantiene costante e nulla, sulle linee di corrente che
partono dall’infinito a monte%
\begin{footnote}[1]\sphinxAtStartFootnote
É immediato convincersi del fatto, utilizzando la descrizione
lagrangiana
(\DUrole{xref,myst}{{[}eqn:bilanci:vorticitàLagrange{]}}\{reference\sphinxhyphen{}type=»ref»
reference=»eqn:bilanci:vorticitàLagrange»\}) della vorticità per un
fluido non viscoso.
%
\end{footnote}. Per correnti ad alto numero di
Reynolds attorno a corpi affusolati, nelle quali non si verificano
separazioni, gli effetti viscosi e la vorticità sono confinati in strati
limite «sottili» attorno ai corpi solidi e in scie «sottili» che si
staccano da essi.

\sphinxAtStartPar
É quindi possibile descrivere una corrente di un fluido incomprimibile
ad alto numero di Reynolds, all”\sphinxstyleemphasis{esterno} di queste sottili regioni
vorticose, con un modello di fluido non viscoso. Partendo dalle
equazioni di Navier–Stokes che governano la dinamica di un fluido
viscoso, per le quali vale la condizione al contorno di adesione a
parete (\(\mathbf{u} = \mathbf{b}\)), si arriva a un modello che permette di
calcolare il campo di velocità dal potenziale cinetico, che soddisfa
l’equazione di Laplace \(\Delta \phi = 0\) nel dominio e la condizione al
contorno di non penetrazione
(\(\mathbf{u} \cdot \mathbf{\hat{n}} = \mathbf{b} \cdot \mathbf{\hat{n}}\)) in
corrispondenza delle pareti solide, e in seguito di calcolare la
pressione utilizzando il teorema di Bernoulli.
\begin{equation*}
\begin{split}\begin{cases}
  \frac{\partial \mathbf{u}}{\partial t} + (\mathbf{u} \cdot \mathbf{\nabla}) \mathbf{u} - \nu \Delta \mathbf{u} + \mathbf{\nabla}P = \mathbf{g} \\
  \mathbf{\nabla} \cdot \mathbf{u} = 0 \\
  \mathbf{u}\big|_{wall} = \mathbf{b}  \qquad + \textit{altre b.c}
 \end{cases}
 \xrightarrow[{\small \begin{aligned} \nu = 0 , & \ \omega = \mathbf{0} \\ \mathbf{u} & = \mathbf{\nabla}\phi \end{aligned} }]{}
  \begin{cases}
  \frac{\partial \phi}{\partial t} + \frac{|\mathbf{\nabla}\phi|^2}{2} + P + \chi = C(t) \\
  \Delta \phi = 0 \\
  \frac{\partial phi}{\partial n} = \mathbf{u}\cdot\mathbf{\hat{n}} \big|_{wall} = \mathbf{b}\cdot\mathbf{\hat{n}} \qquad +  \textit{altre b.c} \\
 \end{cases}\end{split}
\end{equation*}
\sphinxAtStartPar
Il problema di Laplace è lineare ed è quindi valido il
principio di sovrapposizione di cause ed effetti, se la geometria del
dominio è fissata. Questa considerazione può sembrare strana, ma è
determinata dalla possibile presenza di scie che si distaccano dai corpi
solidi e che possono evolvere (per problemi non stazionari) all’interno
del dominio. L’equazione di Lapalce può rappresentare anche problemi non
stazionari, nonostante non compaia esplicitamente nessuna derivata
temporale nell’equazione. La dipendenza temporale può comparire
all’interno delle condizioni al contorno e la soluzione si adatta
immediatamente ad esse. Memoria della soluzione agli istanti di tempo
precendenti è contenuta all’interno delle scie, la cui vorticità è
legata al valore di circolazione attorno al corpo (e quindi di portanza)
e la cui dinamica è determinata dalle equazioni di governo della
vorticità.


\bigskip\hrule\bigskip


\sphinxstepscope


\chapter{Boundary layer}
\label{\detokenize{polimi/fluidmechanics-ita/template/capitoli/09_bl/09teoria:boundary-layer}}\label{\detokenize{polimi/fluidmechanics-ita/template/capitoli/09_bl/09teoria:fluid-mechanics-bl}}\label{\detokenize{polimi/fluidmechanics-ita/template/capitoli/09_bl/09teoria::doc}}

\section{Equazioni di Prandtl dello strato limite}
\label{\detokenize{polimi/fluidmechanics-ita/template/capitoli/09_bl/09teoria:equazioni-di-prandtl-dello-strato-limite}}

\section{Spessori integrali dello strato limite}
\label{\detokenize{polimi/fluidmechanics-ita/template/capitoli/09_bl/09teoria:spessori-integrali-dello-strato-limite}}

\section{Equazione integrale di Von Karman}
\label{\detokenize{polimi/fluidmechanics-ita/template/capitoli/09_bl/09teoria:equazione-integrale-di-von-karman}}

\subsection{Integrazione dell’equazione integrale di Von Karman}
\label{\detokenize{polimi/fluidmechanics-ita/template/capitoli/09_bl/09teoria:integrazione-dell-equazione-integrale-di-von-karman}}

\subsection{Metodo di Thwaites}
\label{\detokenize{polimi/fluidmechanics-ita/template/capitoli/09_bl/09teoria:metodo-di-thwaites}}

\section{Strato limite laminare su lamina piana: soluzione di Blasius}
\label{\detokenize{polimi/fluidmechanics-ita/template/capitoli/09_bl/09teoria:strato-limite-laminare-su-lamina-piana-soluzione-di-blasius}}
\sphinxstepscope


\chapter{Instability and turbulence}
\label{\detokenize{polimi/fluidmechanics-ita/template/capitoli/10_turbolenza/10teoria:instability-and-turbulence}}\label{\detokenize{polimi/fluidmechanics-ita/template/capitoli/10_turbolenza/10teoria:fluid-mechanics-turbulence}}\label{\detokenize{polimi/fluidmechanics-ita/template/capitoli/10_turbolenza/10teoria::doc}}






\renewcommand{\indexname}{Indice}
\printindex
\end{document}